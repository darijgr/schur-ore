\documentclass[numbers=enddot,12pt,final,onecolumn,notitlepage]{scrartcl}%
\usepackage[headsepline,footsepline,manualmark]{scrlayer-scrpage}
\usepackage[all,cmtip]{xy}
\usepackage{amsfonts}
\usepackage{amssymb}
\usepackage{framed}
\usepackage{amsmath}
\usepackage{comment}
\usepackage{color}
\usepackage{hyperref}
\usepackage[sc]{mathpazo}
\usepackage[T1]{fontenc}
\usepackage{amsthm}
%TCIDATA{OutputFilter=latex2.dll}
%TCIDATA{Version=5.50.0.2960}
%TCIDATA{LastRevised=Friday, November 04, 2016 01:48:10}
%TCIDATA{SuppressPackageManagement}
%TCIDATA{<META NAME="GraphicsSave" CONTENT="32">}
%TCIDATA{<META NAME="SaveForMode" CONTENT="1">}
%TCIDATA{BibliographyScheme=Manual}
%BeginMSIPreambleData
\providecommand{\U}[1]{\protect\rule{.1in}{.1in}}
%EndMSIPreambleData
\theoremstyle{definition}
\newtheorem{theo}{Theorem}[section]
\newenvironment{theorem}[1][]
{\begin{theo}[#1]\begin{leftbar}}
{\end{leftbar}\end{theo}}
\newtheorem{lem}[theo]{Lemma}
\newenvironment{lemma}[1][]
{\begin{lem}[#1]\begin{leftbar}}
{\end{leftbar}\end{lem}}
\newtheorem{prop}[theo]{Proposition}
\newenvironment{proposition}[1][]
{\begin{prop}[#1]\begin{leftbar}}
{\end{leftbar}\end{prop}}
\newtheorem{defi}[theo]{Definition}
\newenvironment{definition}[1][]
{\begin{defi}[#1]\begin{leftbar}}
{\end{leftbar}\end{defi}}
\newtheorem{remk}[theo]{Remark}
\newenvironment{remark}[1][]
{\begin{remk}[#1]\begin{leftbar}}
{\end{leftbar}\end{remk}}
\newtheorem{coro}[theo]{Corollary}
\newenvironment{corollary}[1][]
{\begin{coro}[#1]\begin{leftbar}}
{\end{leftbar}\end{coro}}
\newtheorem{conv}[theo]{Convention}
\newenvironment{condition}[1][]
{\begin{conv}[#1]\begin{leftbar}}
{\end{leftbar}\end{conv}}
\newtheorem{quest}[theo]{Question}
\newenvironment{algorithm}[1][]
{\begin{quest}[#1]\begin{leftbar}}
{\end{leftbar}\end{quest}}
\newtheorem{warn}[theo]{Warning}
\newenvironment{conclusion}[1][]
{\begin{warn}[#1]\begin{leftbar}}
{\end{leftbar}\end{warn}}
\newtheorem{conj}[theo]{Conjecture}
\newenvironment{conjecture}[1][]
{\begin{conj}[#1]\begin{leftbar}}
{\end{leftbar}\end{conj}}
\newtheorem{exmp}[theo]{Example}
\newenvironment{example}[1][]
{\begin{exmp}[#1]\begin{leftbar}}
{\end{leftbar}\end{exmp}}
\newenvironment{statement}{\begin{quote}}{\end{quote}}
\iffalse
\newenvironment{proof}[1][Proof]{\noindent\textbf{#1.} }{\ \rule{0.5em}{0.5em}}
\fi
\newenvironment{verlong}{}{}
\newenvironment{vershort}{}{}
\newenvironment{noncompile}{}{}
\excludecomment{verlong}
\includecomment{vershort}
\excludecomment{noncompile}
\newcommand{\kk}{\mathbf{k}}
\newcommand{\id}{\operatorname{id}}
\newcommand{\ev}{\operatorname{ev}}
\newcommand{\Comp}{\operatorname{Comp}}
\newcommand{\bk}{\mathbf{k}}
\newcommand{\Nplus}{\mathbb{N}_{+}}
\newcommand{\NN}{\mathbb{N}}
\let\sumnonlimits\sum
\let\prodnonlimits\prod
\renewcommand{\sum}{\sumnonlimits\limits}
\renewcommand{\prod}{\prodnonlimits\limits}
\setlength\textheight{22.5cm}
\setlength\textwidth{15cm}
\ihead{Function-field analogue for symmetric functions?}
\ohead{\today}
\begin{document}

\title{Do the symmetric functions have a function-field analogue?}
\author{Darij Grinberg}
\date{draft, version 1.4,
%TCIMACRO{\TeXButton{TeX field}{\today}}%
%BeginExpansion
\today
%EndExpansion
}
\maketitle
\tableofcontents

\subsection{Introduction (Abstract?)}

This is a preliminary report on a question that is almost naive: Is there a
ring (or another structure) that has the same relation to the ring $\Lambda$
of symmetric functions as $\mathbb{F}_{q}$ has to the \textquotedblleft
mythical field $\mathbb{F}_{1}$\textquotedblright\ ?

This question allows for at least two different interpretations. One of them
is just about $q$-deforming the structure coefficients of the symmetric
functions in such a way that (some of) their combinatorial interpretations are
reinterpreted (i.e., counting sets becomes counting $\mathbb{F}_{q}$-vector
spaces). This naturally leads to Hall algebras, studied e.g. in
\cite{dyckerhoff}. A different option, however, presents itself if we are
willing to replace the bases of $\Lambda$ itself (rather than just its
structure coefficients). Namely, recall that all (or most) of the usual bases
of $\Lambda$ are indexed by integer partitions. An integer partition can be
regarded as a weakly decreasing sequence of positive integers, or,
equivalently, a conjugacy class of a permutation in a symmetric group. A
natural \textquotedblleft$\mathbb{F}_{q}$-analogue\textquotedblright\ of an
integer partition, thus, is a conjugacy class of a matrix in
$\operatorname*{GL}\nolimits_{n}\left(  \mathbb{F}_{q}\right)  $. Could we
find a ring (or anything similar -- a commutative $\mathbb{F}_{q}\left[
T\right]  $-algebra sounds like a reasonable thing to expect) which plays a
similar role to $\Lambda$ and whose bases are indexed by these $\mathbb{F}%
_{q}$-analogues?

This report is a bait-and-switch, as I do not have a good answer to this
question. Instead I recall the classical interpretation of the ring $\Lambda$
as the coordinate ring of the affine group of Witt vectors (\cite[\S 9--\S 10]%
{hw-witt1}), and construct an $\mathbb{F}_{q}$-analogue of the affine group of
Witt vectors. This analogue has a coordinate ring, which can reasonably be
called an $\mathbb{F}_{q}$-analogue of $\Lambda$. But this answer is lacking
something very important: the combinatorial bases. The most interesting
structure on the ring $\Lambda$ of symmetric functions is not so much its Hopf
algebra structure, but its various bases, such as the homogeneous symmetric
functions $\left(  h_{\lambda}\right)  _{\lambda\in\operatorname*{Par}}$, the
elementary symmetric functions $\left(  e_{\lambda}\right)  _{\lambda
\in\operatorname*{Par}}$ and the Schur functions $\left(  s_{\lambda}\right)
_{\lambda\in\operatorname*{Par}}$. I am unable to find a counterpart to any of
the bases just mentioned in the $\mathbb{F}_{q}$-analogue of $\Lambda$
suggested. All I can offer is an analogue of the power-sum functions $\left(
p_{\lambda}\right)  _{\lambda\in\operatorname*{Par}}$ (which do not even form
a basis, although with functoriality they are sufficient for many
computational purposes) and of a basis $\left(  w_{\lambda}\right)
_{\lambda\in\operatorname*{Par}}$ defined in \cite[Exercise 2.80
(c)]{reiner-hopf} (which, while having interesting properties, hardly feels at
home in combinatorics). So the $\mathbb{F}_{q}$-analogue of $\Lambda$ I find
is somewhat of an empty shell. Still, there are some surprises and my hope is
not lost that it can be made whole.

James Borger had a significant role in the studies made below. In particular,
he suggested to me to look for analogues of Theorem \ref{thm.Witt.frob.au} and
Theorem \ref{thm.Witt.AH} (which I found -- Theorem
\ref{thm.carlitz.Witt.frob.au} and Theorem \ref{thm.carlitz.Witt.AH}),
considering them as a litmus test that shows whether a functor really deserves
to be called a Witt vector functor.

The $\mathbb{F}_{q}$-analogue of the Witt vectors uses the \textit{Carlitz
polynomials}; a highly readable introduction to these polynomials appears in
\cite{kc-carlitz}.

This report is built as follows: In Section \ref{sect.nots}, we introduce
notations and present basic definitions. In Section \ref{sect.carlitzwitt}, we
remind the reader of a construction (actually, one of many constructions) of
the Witt vectors, and then introduce the $\mathbb{F}_{q}$-analogue of this
construction. In Section \ref{sect.proofs}, we shall give detailed proofs for
some of the claims made before. (This section is still under construction, so
only few of the proofs are available.) In Section \ref{sect.tinfoil}, we
speculate on how this analogue could lead to an $\mathbb{F}_{q}$-analogue of
$\Lambda$. In Section \ref{sect.log}, we prove a formula for the so-called
Carlitz logarithm which, while not having any direct relation to the rest of
this report, has emerged in my experiments in connection to it.

Being a preliminary report, this one will occasionally make for some rough
reading, although I am trying to make the more-or-less finished parts (Section
\ref{sect.carlitzwitt}) more-or-less readable. The reader is assumed to know
about Witt vectors (\cite{rabinoff-witt} or \cite{hw-witt1} or \cite[\S 1]%
{hesselholt-drw}) and a bit about Carlitz polynomials (\cite{kc-carlitz}).
Symmetric functions will only be really used in Section \ref{sect.tinfoil}.

\subsection{Remark on Borger's work}

In \cite[\S 1--\S 2]{jb-bg1}, James Borger has generalized the notion of Witt
vectors to a rather broad setting, which includes both the classical and the
\textquotedblleft nested\textquotedblright\ Witt vectors. His generalization
also includes my Carlitz-Witt functor $W_{N}$ in Theorem \ref{thm.Witt.class}
below, namely when one takes $R=\mathbb{F}_{q}\left[  T\right]  $ and
$E=\left\{  \text{all maximal ideals of }R\right\}  $. We have yet to fill in
the details, but in a nutshell, the reason why our constructions are
equivalent is that the universal property of our $W_{N}\left(  B\right)  $
given in Corollary \ref{cor.carlitz.Witt.frob.adjoint} below is the same as
the one for $W_{R,E}^{\operatorname*{fl}}\left(  A\right)  $ in
\cite[Proposition 1.9 (c)]{jb-bg1} (up to technicalities). Thus, it appears
likely that several of the results below are particular cases of results from
\cite{jb-bg1}. Nevertheless, our approach to the Carlitz-Witt functor is
different from Borger's, and somewhat more explicit.

\section{\label{sect.nots}Notations}

\subsection{General number theory}

I use the symbol $\mathbb{P}$ for the set of all primes. Further, $\mathbb{N}$
denotes the set $\left\{  0,1,2,...\right\}  $, and $\mathbb{N}_{+}$ the set
$\left\{  1,2,3,...\right\}  $.

A \textit{nest} means a nonempty subset $N$ of $\mathbb{N}_{+}$ such that for
every element $d\in N$, every divisor of $d$ lies in $N$. What I call
\textquotedblleft nest\textquotedblright\ is called a \textquotedblleft
nonempty truncation set\textquotedblright\ by some authors (e.g., by James
Borger in some of his work), and a \textquotedblleft divisor-stable
set\textquotedblright\ by others (e.g., by Joseph Rabinoff in
\cite{rabinoff-witt}).

For every prime $p$, the nest $\left\{  1,p,p^{2},p^{3},...\right\}  =\left\{
p^{i}\ \mid\ i\in\mathbb{N}\right\}  $ is called $p^{\mathbb{N}}$.

For any prime $p$ and any $n\in\mathbb{Z}$, we denote by $v_{p}\left(
n\right)  $ the largest nonnegative integer $m$ satisfying $p^{m}\mid n$; this
is set to be $+\infty$ if $n=0$.

For any $n\in\mathbb{N}_{+}$, we denote by $\operatorname{PF}n$ the set of all
prime divisors of $n$.

We let $\mu$ denote the M\"{o}bius function and $\phi$ the Euler totient
function (both are defined on $\mathbb{N}_{+}$).

For every ring $R$ and indeterminate $T$, we denote by $R\left[  T\right]
_{+}$ the set of all \textbf{monic} polynomials in the indeterminate $T$ over
$R$. (All rings are supposed to have a unity.)

We consider polynomials over fields to be analogous to integers.\footnote{This
is a well-known analogy, often taught in number theory classes.} Under this
analogy, monic polynomials correspond to positive integers; divisibility of
polynomials corresponds to divisibility of integers; monic irreducible
polynomials correspond to primes. Thus, for example, if $R$ is a field and
$M\in R\left[  T\right]  _{+}$ is a monic polynomial, then a sum like
$\sum\limits_{D\mid M}a_{D}$ is to be read as a sum over all \textbf{monic}
divisors of $M$, not over all arbitrary divisors of $M$. Moreover, if $R$ is a
field and $M\in R\left[  T\right]  _{+}$ is a monic polynomial, then
$\operatorname*{PF}M$ will denote the set of all monic irreducible divisors of
$M$ (rather than all irreducible divisors of $M$). Finally, if $\pi$ is an
irreducible polynomial in $R\left[  T\right]  _{+}$ and $f$ is any polynomial
in $R\left[  T\right]  _{+}$ (for a field $R$), then $v_{\pi}\left(  f\right)
$ means the largest nonnegative integer $m$ satisfying $\pi^{m}\mid f$; this
is set to be $+\infty$ if $f=0$.

\subsection{Algebra}

We denote by $\mathbf{CRing}$ the category of commutative rings, and by
$\mathbf{CRing}_{R}$ the category of commutative $R$-algebras for a fixed
commutative ring $R$. Also, for any ring $R$, we denote by $_{R}\mathbf{Mod}$
the category of left $R$-modules.

We denote by $\Lambda$ the ring of symmetric functions over $\mathbb{Z}$.
(This is also known as $\mathbf{Symm}$ or $Sym$. See \cite[\S 2]{reiner-hopf}
and \cite[Chapter 7]{stanley-ec2} for studies of this ring $\Lambda$.)

\subsection{Carlitz polynomials}

In discussing Carlitz polynomials, I use the notations from Keith Conrad's
\cite{kc-carlitz} (but I'm using blackboard bold instead of boldface for
labelling rings; so what Conrad calls $\mathbf{F}_{p}$ will be called
$\mathbb{F}_{p}$ here, etc.). In particular, let $q$ be a prime power. For any
$M\in\mathbb{F}_{q}\left[  T\right]  $, the Carlitz polynomial in
$\mathbb{F}_{q}\left[  T\right]  \left[  X\right]  $ corresponding to the
polynomial $M$ will be denoted by $\left[  M\right]  $. Let us recall how it
is defined:

\begin{definition}
\label{def.carlitzpoly}For every $n\in\mathbb{N}$, define a polynomial
$\left[  T^{n}\right]  \in\mathbb{F}_{q}\left[  T\right]  \left[  X\right]  $
recursively, by setting $\left[  T^{0}\right]  =X$ and $\left[  T^{n}\right]
=\left[  T^{n-1}\right]  ^{q}+T\left[  T^{n-1}\right]  $ for every $n\geq1$.
For example,%
\begin{align*}
\left[  T^{0}\right]   &  =X;\ \ \ \ \ \ \ \ \ \ \left[  T^{1}\right]
=\left[  T^{0}\right]  ^{q}+T\left[  T^{0}\right]  =X^{q}+TX;\\
\left[  T^{2}\right]   &  =\left[  T^{1}\right]  ^{q}+T\left[  T^{1}\right]
=\left(  X^{q}+TX\right)  ^{q}+T\left(  X^{q}+TX\right)  =X^{q^{2}}+\left(
T^{q}+T\right)  X^{q}+T^{2}X.
\end{align*}
(Here, we have used the fact that taking the $q$-th power is an $\mathbb{F}%
_{q}$-algebra endomorphism of $\mathbb{F}_{q}\left[  T\right]  \left[
X\right]  $.)

Now, if $M\in\mathbb{F}_{q}\left[  T\right]  $, then we define a polynomial
$\left[  M\right]  \in\mathbb{F}_{q}\left[  T\right]  \left[  X\right]  $ to
be $a_{0}\left[  T^{0}\right]  +a_{1}\left[  T^{1}\right]  +\cdots
+a_{k}\left[  T^{k}\right]  $, where the polynomial $M$ is written in the form
$M=a_{0}T^{0}+a_{1}T^{1}+\cdots+a_{k}T^{k}$. (In other words, we define a
polynomial $\left[  M\right]  \in\mathbb{F}_{q}\left[  T\right]  \left[
X\right]  $ in such a way that $\left[  M\right]  $ depends $\mathbb{F}_{q}%
$-linearly on $M$, and that our new definition of $\left[  M\right]  $ does
not conflict with our existing definition of $\left[  T^{n}\right]  $ for
$n\in\mathbb{N}$.) We call $\left[  M\right]  $ the \textit{Carlitz
polynomial} corresponding to $M$.
\end{definition}

Carlitz polynomials can be used to take the above-mentioned analogy between
$\mathbb{Z}$ and $\mathbb{F}_{q}\left[  T\right]  $ to a new level. Namely,
evaluating a Carlitz polynomial $\left[  M\right]  $ at an element $a$ of a
commutative $\mathbb{F}_{q}\left[  T\right]  $-algebra $A$ can be viewed as
the analogue of taking the $m$-th power of an element $a$ of a commutative
ring $A$.

Notice that%
\begin{equation}
\left[  \pi\right]  \left(  X\right)  \equiv X^{q^{\deg\pi}}\operatorname{mod}%
\pi\ \ \ \ \ \ \ \ \ \ \text{for any monic irreducible }\pi\in\mathbb{F}%
_{q}\left[  T\right]  . \label{carlitz-piX}%
\end{equation}
(This is proven in \cite[Theorem 2.11]{kc-carlitz} in the case when $q$ is a
prime. In the general case, the proof is analogous.)

In the Carlitz context there is an obvious analogue of the M\"{o}bius
function: it is simply the M\"{o}bius function of the lattice $\mathbb{F}%
_{q}\left[  T\right]  _{+}$ (whose partial order is the divisibility
relation). In other words, it is the function $\mu:\mathbb{F}_{q}\left[
T\right]  _{+}\rightarrow\left\{  -1,0,1\right\}  $ defined by%
\[
\mu\left(  M\right)  =%
\begin{cases}
\left(  -1\right)  ^{\left\vert \operatorname*{PF}M\right\vert }, & \text{if
}M\text{ is squarefree;}\\
0, & \text{if }M\text{ is not squarefree}%
\end{cases}
\ \ \ \ \ \ \ \ \ \ \text{for all }M\in\mathbb{F}_{q}\left[  T\right]  _{+}.
\]
Yet, in the Carlitz context, there are two reasonable analogues of the Euler
totient function. Let us give their definitions (which both are taken from
\cite{kc-carlitz}):

\textbf{1.} The first analogue is the function $\varphi_{C}:\mathbb{F}%
_{q}\left[  T\right]  _{+}\rightarrow\mathbb{F}_{q}\left[  T\right]  _{+}$
defined by%
\[
\varphi_{C}\left(  M\right)  =M\prod\limits_{\pi\in\operatorname*{PF}M}\left(
1-\dfrac{1}{\pi}\right)  =\sum\limits_{D\mid M}\mu\left(  D\right)  \dfrac
{M}{D}\ \ \ \ \ \ \ \ \ \ \text{for all }M\in\mathbb{F}_{q}\left[  T\right]
_{+}.
\]
Some properties of this $\varphi_{C}$ are shown in \cite[Theorem
4.5]{kc-carlitz}. In particular, every $M\in\mathbb{F}_{q}\left[  T\right]
_{+}$ satisfies $M=\sum\limits_{D\mid M}\varphi_{C}\left(  D\right)  $.

\textbf{2.} The second analogue is the function $\varphi:\mathbb{F}_{q}\left[
T\right]  _{+}\rightarrow\mathbb{N}_{+}$ defined by%
\[
\varphi\left(  M\right)  =q^{\deg M}\prod\limits_{\pi\in\operatorname*{PF}%
M}\left(  1-\dfrac{1}{q^{\deg\pi}}\right)  =\sum\limits_{D\mid M}\mu\left(
D\right)  q^{\deg\left(  M / D\right)  }\ \ \ \ \ \ \ \ \ \ \text{for all
}M\in\mathbb{F}_{q}\left[  T\right]  _{+}.
\]
This function appears in \cite[Section 6]{kc-carlitz}. It has the property
that $\varphi\left(  M\right)  \equiv\mu\left(  M\right)  \operatorname{mod}p$
for every $M\in\mathbb{F}_{q}\left[  T\right]  _{+}$ (where
$p=\operatorname*{char}\mathbb{F}_{q}$). Thus, $\varphi\left(  M\right)
=\mu\left(  M\right)  $ in $\mathbb{F}_{q}$. To us, this makes this function
$\varphi$ less interesting than $\varphi_{C}$.

The existence of two different analogues of the same thing is a phenomenon
that we will see a few more times in this theory.

\section{\label{sect.carlitzwitt}The Carlitz-Witt suite}

\subsection{The classical ghost-Witt equivalence theorem}

There are several approaches to the notion of Witt vectors. One of these
approaches is based on the following theorem (the ``ghost-Witt equivalence
theorem'', also known in parts as \textquotedblleft Dwork's
lemma\textquotedblright):

\begin{theorem}
\label{thm.gW}Let $N$ be a nest. Let $A$ be a commutative ring. For every
$n\in N$, let $\varphi_{n}:A\rightarrow A$ be an endomorphism of the additive
group $A$.

Further, let us make three more assumptions:

\textit{Assumption 1:} For every $n\in N$, the map $\varphi_{n}$ is an
endomorphism of the \textbf{ring} $A$.

\textit{Assumption 2:} We have $\varphi_{p}\left(  a\right)  \equiv
a^{p}\operatorname{mod}pA$ for every $a\in A$ and $p\in\mathbb{P}\cap N$.

\textit{Assumption 3:} We have $\varphi_{1}=\operatorname*{id}$, and we have
$\varphi_{n}\circ\varphi_{m}=\varphi_{nm}$ for every $n\in N$ and every $m\in
N$ satisfying $nm\in N$.

Let $\left(  b_{n}\right)  _{n\in N}\in A^{N}$ be a family of elements of $A$.
Then, the following assertions $\mathcal{C}$, $\mathcal{D}$, $\mathcal{E}$,
$\mathcal{F}$, $\mathcal{G}$, $\mathcal{H}$, and $\mathcal{J}$ are equivalent:

\textit{Assertion }$\mathcal{C}$\textit{:} Every $n\in N$ and every
$p\in\operatorname{PF}n$ satisfy%
\[
\varphi_{p}\left(  b_{n / p}\right)  \equiv b_{n}\operatorname{mod}%
p^{v_{p}\left(  n\right)  }A.
\]


\textit{Assertion }$\mathcal{D}$\textit{:} There exists a family $\left(
x_{n}\right)  _{n\in N}\in A^{N}$ of elements of $A$ such that%
\[
\left(  b_{n}=\sum_{d\mid n}dx_{d}^{n / d}\text{ for every }n\in N\right)  .
\]


\textit{Assertion }$\mathcal{E}$\textit{:} There exists a family $\left(
y_{n}\right)  _{n\in N}\in A^{N}$ of elements of $A$ such that%
\[
\left(  b_{n}=\sum_{d\mid n}d\varphi_{n / d}\left(  y_{d}\right)  \text{ for
every }n\in N\right)  .
\]


\textit{Assertion }$\mathcal{F}$\textit{:} Every $n\in N$ satisfies%
\[
\sum_{d\mid n}\mu\left(  d\right)  \varphi_{d}\left(  b_{n / d}\right)  \in
nA.
\]


\textit{Assertion }$\mathcal{G}$\textit{:} Every $n\in N$ satisfies%
\[
\sum_{d\mid n}\phi\left(  d\right)  \varphi_{d}\left(  b_{n / d}\right)  \in
nA.
\]


\textit{Assertion }$\mathcal{H}$\textit{:} Every $n\in N$ satisfies%
\[
\sum_{i=1}^{n}\varphi_{n / \gcd\left(  i,n\right)  }\left(  b_{\gcd\left(
i,n\right)  }\right)  \in nA.
\]


\textit{Assertion $\mathcal{J}$:} There exists a ring homomorphism from the
ring $\Lambda$ to $A$ which sends $p_{n}$ (the $n$-th power sum symmetric
function) to $b_{n}$ for every $n\in N$.
\end{theorem}

\begin{definition}
The families $\left(  b_{n}\right)  _{n\in N}\in A^{N}$ which satisfy the
equivalent assertions $\mathcal{C}$, $\mathcal{D}$, $\mathcal{E}$,
$\mathcal{F}$, $\mathcal{G}$, $\mathcal{H}$, and $\mathcal{J}$ of Theorem
\ref{thm.gW} will be called \textit{ghost-Witt vectors} (over $A$).
\end{definition}

There are many variations on Theorem \ref{thm.gW}. An easy way to get a more
intuitive particular case of Theorem \ref{thm.gW} is to set $\varphi
_{n}=\operatorname*{id}\nolimits_{A}$ for all $n\in N$, after which
Assumptions 1 and 3 become tautologies. However, Assumption 2 is not
guaranteed to hold in this setting; but it holds in $\mathbb{Z}$, and more
generally in binomial rings, and in some non-torsionfree rings as well.
Unfortunately, this case is in some sense too simple: it is too weak to yield
the basic properties of Witt vectors (such as the well-definedness of
addition, multiplication, Frobenius and Verschiebung). Instead one needs the
case when $A$ is a polynomial ring $\mathbb{Z}\left[  \Xi\right]  $ for some
family $\Xi$ of indeterminates, and the maps $\varphi_{n}$ are defined by
$\varphi_{n}\left(  P\right)  =P\left(  \Xi^{n}\right)  $ for every
$P\in\mathbb{Z}\left[  \Xi\right]  $ (where $P\left(  \Xi^{n}\right)  $ means
the result of $P$ upon substituting every variable by its $n$-th power). The
only part of Theorem \ref{thm.gW} which is needed for this proof is the
equivalence $\mathcal{C}\Longleftrightarrow\mathcal{D}$.

The proof of Theorem \ref{thm.gW} is everywhere and nowhere: it is a
straightforward generalization of arguments easily found in literature, but I
haven't seen it explicit in this generality anywhere. I've written it up (save
for Assertion $\mathcal{J}$) in \cite[Theorem 11]{dg-witt5}. Also, the proof
of the whole Theorem \ref{thm.gW} in the case when $N=\mathbb{N}_{+}$ appears
in \cite[Exercise 2.83]{reiner-hopf}; it is not hard to derive the general
case from it.

Some parts of Theorem \ref{thm.gW} are valid in somewhat more general
situations. The equivalence $\mathcal{C}\Longleftrightarrow\mathcal{D}$ needs
Assumptions 1 and 2 but not 3 (unsurprisingly), and the equivalence
$\mathcal{C}\Longleftrightarrow\mathcal{E}\Longleftrightarrow\mathcal{F}%
\Longleftrightarrow\mathcal{G}\Longleftrightarrow\mathcal{H}$ needs only
Assumption 3 (not 1 and 2; actually, $A$ can be any additive group rather than
a ring for this equivalence). The equivalence $\mathcal{D}\Longleftrightarrow
\mathcal{J}$ needs nothing. This is all old news.

\subsection{Classical Witt vectors}

We recall a way to define the classical notion of Witt vectors. We work with a
nest $N$, so that both $p$-typical and big Witt vectors are provided for.

\begin{definition}
\label{def.Witt.ghostmap}Let $N$ be a nest. Let $A$ be a commutative ring. The
\textit{ghost ring} of $A$ will mean the ring $A^{N}$ with componentwise ring
structure (i. e., a direct product of rings $A$ indexed over $N$). The
$N$\textit{-ghost map} $w_{N}:A^{N}\rightarrow A^{N}$ is the map defined by%
\[
w_{N}\left(  \left(  x_{n}\right)  _{n\in N}\right)  =\left(  \sum
\limits_{d\mid n}dx_{d}^{n / d}\right)  _{n\in N}\ \ \ \ \ \ \ \ \ \ \text{for
all }\left(  x_{n}\right)  _{n\in N}\in A^{N}.
\]
This $N$-ghost map is (generally) neither additive nor multiplicative.
\end{definition}

The following theorem is easily derived from Theorem \ref{thm.gW} (more
precisely, the equivalence $\mathcal{C}\Longleftrightarrow\mathcal{D}$)
applied to the case $A=\mathbb{Z}\left[  \Xi\right]  $ and $\varphi_{n}\left(
P\right)  =P\left(  \Xi^{n}\right)  $:

\begin{theorem}
\label{thm.Witt.class}Let $N$ be a nest. There exists a unique functor
$W_{N}:\mathbf{CRing}\rightarrow\mathbf{CRing}$ with the following two properties:

-- We have $W_{N}\left(  A\right)  =A^{N}$ \textbf{as a set} for every
commutative ring $A$.

-- The map $w_{N}:A^{N}\rightarrow A^{N}$ \textbf{regarded as a map }%
$W_{N}\left(  A\right)  \rightarrow A^{N}$ is a ring homomorphism for every
commutative ring $A$.

This functor $W_{N}$ is called the $N$\textit{-Witt vector functor}. For every
commutative ring $A$, we call the commutative ring $W_{N}\left(  A\right)  $
the $N$\textit{-Witt vector ring over }$A$. Its zero is the family $\left(
0\right)  _{n\in N}$, and its unity is the family $\left(  \delta
_{n,1}\right)  _{n\in N}$ (where $\delta_{u,v}$ is defined to be $%
\begin{cases}
1, & \text{if }u=v;\\
0, & \text{if }u\neq v
\end{cases}
$ for any two objects $u$ and $v$).

The map $w_{N}:W_{N}\left(  A\right)  \rightarrow A^{N}$ itself becomes a
natural transformation from the functor $W_{N}$ to the functor $\mathbf{CRing}%
\rightarrow\mathbf{CRing},\ A\mapsto A^{N}$. We will call this natural
transformation $w_{N}$ as well.
\end{theorem}

Theorem \ref{thm.Witt.class} appears in \cite[Theorem 2.6]{rabinoff-witt}.
Note that a consequence of Theorem \ref{thm.Witt.class} is that the sum and
the product of two ghost-Witt vectors \textbf{over any commutative ring }$A$
are again ghost-Witt vectors. This is not an immediate consequence of Theorem
\ref{thm.gW} (because it is not clear how we could construct maps $\varphi
_{n}$ satisfying Assumptions 1, 2 and 3 over any commutative ring $A$), but
rather requires a detour via $\mathbb{Z}\left[  \Xi\right]  $.

The following theorem (\cite[Remark 2.9, part 3]{rabinoff-witt}) allows us to
prove functorial identities by working with ghost components:

\begin{theorem}
\label{thm.Witt.iso}Let $N$ be a nest. For any commutative $\mathbb{Q}%
$-algebra $A$, the map $w_{N}:W_{N}\left(  A\right)  \rightarrow A^{N}$ is a
ring isomorphism.
\end{theorem}

The Witt vector rings allow for an ``almost-universal property'' \cite[Theorem
6.1]{rabinoff-witt}:

\begin{theorem}
\label{thm.Witt.frob.au}Let $N$ be a nest. Let $A$ be a commutative ring such
that no element of $N$ is a zero-divisor in $A$. For every $n\in N$, let
$\sigma_{n}$ be a ring endomorphism of $A$. Assume that $\sigma_{n}\circ
\sigma_{m}=\sigma_{nm}$ for any $n\in N$ and $m\in N$ satisfying $nm\in N$.
Also assume that $\sigma_{1}=\operatorname*{id}$. Finally, assume that
$\sigma_{p}\left(  a\right)  \equiv a^{p}\operatorname{mod}pA$ for every prime
$p\in N$ and every $a\in A$. Then, there exists a unique ring homomorphism
$\varphi:A\rightarrow W_{N}\left(  A\right)  $ satisfying%
\[
\left(  w_{N}\circ\varphi\right)  \left(  a\right)  =\left(  \sigma_{n}\left(
a\right)  \right)  _{n\in N}\ \ \ \ \ \ \ \ \ \ \text{for every }a\in A.
\]

\end{theorem}

Now let us describe some known functorial operations on $W_{N}\left(
A\right)  $. I will follow \cite{rabinoff-witt} most of the time.

\begin{theorem}
\label{thm.Witt.frob}Let $N$ be a nest.

\textbf{(a)} Let $m$ be a positive integer such that every $n\in N$ satisfies
$mn\in N$. Then, there exists a unique natural transformation $\mathbf{f}%
_{m}:W_{N}\rightarrow W_{N}$ of \textbf{set-valued} (not ring-valued) functors
such that any commutative ring $A$ and any $\mathbf{x}\in W_{N}\left(
A\right)  $ satisfy%
\[
w_{N}\left(  \mathbf{f}_{m}\left(  \mathbf{x}\right)  \right)  =\left(
mn\text{-th coordinate of }w_{N}\left(  \mathbf{x}\right)  \right)  _{n\in
N},
\]
where $\mathbf{f}_{m}$ is short for $\mathbf{f}_{m}\left(  A\right)  $.

\textbf{(b)} This natural transformation $\mathbf{f}_{m}$ is actually a
natural transformation $W_{N}\rightarrow W_{N}$ of \textbf{ring-valued}
functors as well. That is, $\mathbf{f}_{m}:W_{N}\left(  A\right)  \rightarrow
W_{N}\left(  A\right)  $ is a ring homomorphism for every commutative ring
$A$. (Here, again, $\mathbf{f}_{m}$ stands short for $\mathbf{f}_{m}\left(
A\right)  $.) We call $\mathbf{f}_{m}$ the $m$\textit{-th Frobenius} on
$W_{N}$.

\textbf{(c)} We have $\mathbf{f}_{1}=\operatorname*{id}$. Any two positive
integers $n$ and $m$ such that $\mathbf{f}_{n}$ and $\mathbf{f}_{m}$ are
well-defined satisfy $\mathbf{f}_{n}\circ\mathbf{f}_{m}=\mathbf{f}_{nm}$.

\textbf{(d)} Let $p$ be a prime such that every $n\in N$ satisfies $pn\in N$.
We have $\mathbf{f}_{p}\left(  \mathbf{x}\right)  \equiv\mathbf{x}%
^{p}\operatorname{mod}p$ (in $W_{N}\left(  A\right)  $) for every commutative
ring $A$ and every $\mathbf{x}\in W_{N}\left(  A\right)  $.
\end{theorem}

In one or the other form, Theorem \ref{thm.Witt.frob} appears in most sources
on Witt vectors; for example, it can be pieced together from parts of
\cite[Theorem 5.7, Proposition 5.9 and Proposition 5.12]{rabinoff-witt}.

\begin{noncompile}
[By the way, what if we loosen the \textquotedblleft$n\in N\Longrightarrow
mn\in N$\textquotedblright\ condition in Theorem \ref{thm.Witt.frob}? I feel
we should get something like partial Frobenii $\mathbf{f}_{m}:W_{N}\left(
A\right)  \rightarrow W_{N/m}\left(  A\right)  $, where $N/m=\left\{
k\in\mathbb{N}_{+}\ \mid\ mk\in N\right\}  $.]
\end{noncompile}

\begin{noncompile}
[(Here I'm talking to Jim Borger:) Let me use this occasion to explain
something I wrote in an old email: I claimed that the maps $\varphi_{n}$ of
Theorem \ref{thm.gW} \textquotedblleft are used to prove the existence of Witt
vectors though weirdly enough don't actually matter in the
end\textquotedblright. What I meant is that these maps $\varphi_{n}$ (which
are Frobenius lifts on $A$) appear neither in Theorem \ref{thm.Witt.class}
(although its proof uses Theorem \ref{thm.gW}) nor in Theorem
\ref{thm.Witt.frob} (which constructs Frobenius lifts on $W_{N}\left(
A\right)  $). In particular, the Frobenii $\mathbf{f}_{n}$ on the Witt vector
ring $W_{N}\left(  A\right)  $ don't depend on these maps $\varphi_{n}$. That
said, I wouldn't be surprised if there is a deformation of the Witt vector
ring which \textit{does} take these maps into account, in the same way as, e.
g., the shuffle algebra over a vector space which happens to have an algebra
structure can be deformed to a quasi-shuffle algebra. In \cite[\S 17.35]%
{hw-witt1} (or rather in the paper by Oh cited there), this is done for
necklace rings.]
\end{noncompile}

Here is the definition of Verschiebung (\cite[Theorem 5.5 and Proposition
5.9]{rabinoff-witt}):

\begin{theorem}
\label{thm.Witt.ver}Let $N$ be a nest.

\textbf{(a)} Let $m$ be a positive integer. Then, there exists a unique
natural transformation $\mathbf{V}_{m}:W_{N}\rightarrow W_{N}$ of
\textbf{set-valued} (not ring-valued) functors such that any commutative ring
$A$ and any $\mathbf{x}\in W_{N}\left(  A\right)  $ satisfy%
\[
w_{N}\left(  \mathbf{V}_{m}\left(  \mathbf{x}\right)  \right)  =\left(
\begin{cases}
m\cdot\left(  \dfrac{n}{m}\text{-th coordinate of }w_{N}\left(  \mathbf{x}%
\right)  \right)  , & \text{if }m\mid n;\\
0, & \text{if }m\nmid n
\end{cases}
\right)  _{n\in N},
\]
where $\mathbf{V}_{m}$ is short for $\mathbf{V}_{m}\left(  A\right)  $.

\textbf{(b)} This natural transformation $\mathbf{V}_{m}$ is actually a
natural transformation $W_{N}\rightarrow W_{N}$ of
\textbf{abelian-group-valued} functors as well. More precisely, $\mathbf{V}%
_{m}:W_{N}\left(  A\right)  \rightarrow W_{N}\left(  A\right)  $ is a
homomorphism of additive groups for every commutative ring $A$. (Here, again,
$\mathbf{V}_{m}$ stands short for $\mathbf{V}_{m}\left(  A\right)  $.) We call
$\mathbf{V}_{m}$ the $m$\textit{-th Verschiebung} on $W_{N}$.

\textbf{(c)} We have $\mathbf{V}_{1}=\operatorname*{id}$. Any two positive
integers $n$ and $m$ satisfy $\mathbf{V}_{n}\circ\mathbf{V}_{m}=\mathbf{V}%
_{nm}$.

\textbf{(d)} Actually, $\mathbf{V}_{m}\left(  \left(  x_{n}\right)  _{n\in
N}\right)  =\left(
\begin{cases}
x_{n/m}, & \text{if }m\mid n;\\
0, & \text{if }m\nmid n
\end{cases}
\right)  _{n\in N}$ for any positive integer $m$, any commutative ring $A$ and
any $\left(  x_{n}\right)  _{n\in N}\in W_{N}\left(  A\right)  $.
\end{theorem}

There are some equalities involving $\mathbf{V}_{m}$ and $\mathbf{f}_{m}$
which should be here, but I don't have the time to write them down. They
definitely need to be checked for Carlitz analogues.

Finally, here is one possible definition of the comonadic Artin-Hasse
exponential\footnote{This is something Hazewinkel, in \cite[\S 16.45]%
{hw-witt1}, calls Artin-Hasse exponential. I am not sure if I completely
understand its relation to the usual Artin-Hasse exponential...}
(\cite[Corollary 6.3]{rabinoff-witt}):

\begin{theorem}
\label{thm.Witt.AH}Let $N$ be a nest. Assume that $nm\in N$ for all $n\in N$
and $m\in N$.

\textbf{(a)} There exists a unique natural transformation $\operatorname*{AH}%
:W_{N}\rightarrow W_{N}\circ W_{N}$ (of functors $\mathbf{CRing}%
\rightarrow\mathbf{CRing}$) such that every commutative ring $A$, every $n\in
N$ and every $\mathbf{x}\in W_{N}\left(  A\right)  $ satisfy%
\[
\left(  n\text{-th coordinate of }w_{N}\left(  \operatorname*{AH}\left(
\mathbf{x}\right)  \right)  \right)  =\mathbf{f}_{n}\left(  \mathbf{x}\right)
\]
(where $w_{N}$ this time stands for the natural transformation $w_{N}$
evaluated at the ring $W_{N}\left(  A\right)  $; thus, $w_{N}\left(
\operatorname*{AH}\left(  \mathbf{x}\right)  \right)  $ is an element of
$\left(  W_{N}\left(  A\right)  \right)  ^{N}$).

\textbf{(b)} Let $n\in N$, and let $A$ be a commutative ring. Let $w_{n}%
:W_{N}\left(  A\right)  \rightarrow A$ be the map sending each $\mathbf{x}\in
W_{N}\left(  A\right)  $ to the $n$-th coordinate of $w_{N}\left(
\mathbf{x}\right)  $. Then, $W_{N}\left(  w_{n}\right)  \circ
\operatorname*{AH}=\mathbf{f}_{n}$.
\end{theorem}

\subsection{The Carlitz ghost-Witt equivalence theorem}

Now, let us move to the Carlitz case.

\begin{condition}
From now on until the rest of Section \ref{sect.carlitzwitt}, we let $q$
denote an arbitrary prime power ($\neq1$, that is), and let $p$ be the prime
whose power $q$ is.
\end{condition}

\begin{definition}
\label{def.q-nest}A $q$-\textit{nest} means a nonempty subset $N$ of
$\mathbb{F}_{q}\left[  T\right]  _{+}$ such that for every element $P\in N$,
every monic divisor of $P$ lies in $N$.
\end{definition}

Notice that any $q$-nest is a subset of $\mathbb{F}_{q}\left[  T\right]  _{+}%
$. Thus, any element of a $q$-nest must be a monic polynomial. Also, every
$q$-nest contains $1$\ \ \ \ \footnote{\textit{Proof.} Let $N$ be a $q$-nest.
We must prove that $N$ contains $1$.
\par
Any $q$-nest is nonempty (by definition). Thus, $N$ is nonempty (since $N$ is
a $q$-nest). In other words, there exists some $P\in N$. Consider this $P$.
Now, $1$ is a monic divisor of $P\in N$, and thus must itself belong to $N$
(since $N$ is a $q$-nest). In other words, $N$ contains $1$. Qed.}. We shall
use these facts without mention.

\begin{definition}
\label{def.PF(q)}Let $P\in\mathbb{F}_{q}\left[  T\right]  _{+}$. Then,
$\operatorname{PF}P$ denotes the set of all monic irreducible divisors of $P$
in $\mathbb{F}_{q}\left[  T\right]  _{+}$.
\end{definition}

\begin{theorem}
\label{thm.carlitz.gW}Let $N$ be a $q$-nest. Let $A$ be a commutative
$\mathbb{F}_{q}\left[  T\right]  $-algebra. For every $P\in N$, let
$\varphi_{P}:A\rightarrow A$ be an endomorphism of the $\mathbb{F}_{q}\left[
T\right]  $-module $A$.

Further, let us make three more assumptions:

\textit{Assumption 1:} For every $P\in N$, the map $\varphi_{P}$ is an
endomorphism of the $\mathbb{F}_{q}\left[  T\right]  $\textbf{-algebra} $A$.

\textit{Assumption 2:} We have $\varphi_{\pi}\left(  a\right)  \equiv\left[
\pi\right]  \left(  a\right)  \operatorname{mod}\pi A$ for every $a\in A$ and
every monic irreducible $\pi\in N$. (This rewrites as follows: We have
$\varphi_{\pi}\left(  a\right)  \equiv a^{q^{\deg\pi}}\operatorname{mod}\pi A$
for every $a\in A$ and every monic irreducible $\pi\in N$.)

\textit{Assumption 3:} We have $\varphi_{1}=\operatorname*{id}$, and we have
$\varphi_{P}\circ\varphi_{Q}=\varphi_{PQ}$ for every $P\in N$ and every $Q\in
N$ satisfying $PQ\in N$.

Let $\left(  b_{P}\right)  _{P\in N}\in A^{N}$ be a family of elements of $A$.
Then, the following assertions $\mathcal{C}_{1}$, $\mathcal{D}_{1}$,
$\mathcal{D}_{2}$, $\mathcal{E}_{1}$, $\mathcal{F}_{1}$, $\mathcal{G}_{1}$,
and $\mathcal{G}_{2}$ are equivalent:

\textit{Assertion }$\mathcal{C}_{1}$\textit{:} Every $P\in N$ and every
$\pi\in\operatorname{PF}P$ satisfy%
\[
\varphi_{\pi}\left(  b_{P/\pi}\right)  \equiv b_{P}\operatorname{mod}%
\pi^{v_{\pi}\left(  P\right)  }A.
\]


\textit{Assertion }$\mathcal{D}_{1}$\textit{:} There exists a family $\left(
x_{P}\right)  _{P\in N}\in A^{N}$ of elements of $A$ such that%
\[
\left(  b_{P}=\sum_{D\mid P}D\left[  \dfrac{P}{D}\right]  \left(
x_{D}\right)  \text{ for every }P\in N\right)  .
\]


\textit{Assertion }$\mathcal{D}_{2}$\textit{:} There exists a family $\left(
\widetilde{x}_{P}\right)  _{P\in N}\in A^{N}$ of elements of $A$ such that%
\[
\left(  b_{P}=\sum_{D\mid P}D\widetilde{x}_{D}^{q^{\deg\left(  P/D\right)  }%
}\text{ for every }P\in N\right)  .
\]


\textit{Assertion }$\mathcal{E}_{1}$\textit{:} There exists a family $\left(
y_{P}\right)  _{P\in N}\in A^{N}$ of elements of $A$ such that%
\[
\left(  b_{P}=\sum_{D\mid P}D\varphi_{P / D}\left(  y_{D}\right)  \text{ for
every }P\in N\right)  .
\]


\textit{Assertion }$\mathcal{F}_{1}$\textit{:} Every $P\in N$ satisfies%
\[
\sum_{D\mid P}\mu\left(  D\right)  \varphi_{D}\left(  b_{P / D}\right)  \in
PA.
\]


\textit{Assertion }$\mathcal{G}_{1}$\textit{:} Every $P\in N$ satisfies%
\[
\sum_{D\mid P}\varphi_{C}\left(  D\right)  \varphi_{D}\left(  b_{P /
D}\right)  \in PA.
\]


\textit{Assertion }$\mathcal{G}_{2}$\textit{:} Every $P\in N$ satisfies%
\[
\sum_{D\mid P}\varphi\left(  D\right)  \varphi_{D}\left(  b_{P / D}\right)
\in PA.
\]

\end{theorem}

For this Theorem \ref{thm.carlitz.gW} to be a complete analogue of Theorem
\ref{thm.gW}, two assertions are missing: $\mathcal{H}$ and $\mathcal{J}$.
Finding an analogue of $\mathcal{J}$ requires finding an analogue of $\Lambda
$, which is the question that I have started this report with; approaches to
it will be discussed in Section \ref{sect.tinfoil}. Two other assertions
($\mathcal{D}$ and $\mathcal{G}$) have two analogues each. However, Assertion
$\mathcal{G}_{2}$ is clearly equivalent to Assertion $\mathcal{F}_{1}$ because
of $\varphi\left(  M\right)  \equiv\mu\left(  M\right)  \operatorname{mod}p$
for every $M\in\mathbb{F}_{q}\left[  T\right]  _{+}$. I have written out the
former assertion merely to produce a clearer view of the analogy.

The proof of Theorem \ref{thm.carlitz.gW} is analogous to that of (the
respective parts of) Theorem \ref{thm.gW}, and finding it should not be
difficult. (One of the easier ways to proceed is showing $\mathcal{D}%
_{1}\Longleftrightarrow\mathcal{C}_{1}\Longleftrightarrow\mathcal{D}_{2}$,
$\mathcal{C}_{1}\Longrightarrow\mathcal{F}_{1}\Longrightarrow\mathcal{E}%
_{1}\Longrightarrow\mathcal{C}_{1}$, $\mathcal{F}_{1}\Longleftrightarrow
\mathcal{G}_{2}$ and $\mathcal{E}_{1}\Longleftrightarrow\mathcal{G}_{1}$. Two
different analogues of Hensel's exponent lifting are used in proving
$\mathcal{C}_{1}\Longleftrightarrow\mathcal{D}_{1}$ and $\mathcal{C}%
_{1}\Longleftrightarrow\mathcal{D}_{2}$.)

\begin{definition}
The families $\left(  b_{n}\right)  _{n\in N}\in A^{N}$ which satisfy the
equivalent assertions $\mathcal{C}_{1}$, $\mathcal{D}_{1}$, $\mathcal{D}_{2}$,
$\mathcal{E}_{1}$, $\mathcal{F}_{1}$, $\mathcal{G}_{1}$, and $\mathcal{G}_{2}$
of Theorem \ref{thm.carlitz.gW} will be called \textit{Carlitz ghost-Witt
vectors} (over $A$).
\end{definition}

What is more interesting is the following observation:

\begin{remark}
\label{rmk.carlitz.gW.1'}Assumption 1 in Theorem \ref{thm.carlitz.gW} can be
replaced by the following weaker one:

\textit{Assumption 1':} For every $P\in N$, the map $\varphi_{P}$ is an
endomorphism of the $\mathbb{F}_{q}\left[  T\right]  $-module $A$ and commutes
with the Frobenius endomorphism $A\rightarrow A,\ a\mapsto a^{q}$.

Moreover, instead of assuming that $A$ be a commutative $\mathbb{F}_{q}\left[
T\right]  $-algebra, it is enough to assume that $A$ is an $\mathbb{F}%
_{q}\left[  T\right]  $-module with an $\mathbb{F}_{q}$-linear Frobenius map
$F:A\rightarrow A$ which satisfies%
\begin{equation}
F\left(  \lambda a\right)  =\lambda^{q}F\left(  a\right)
\ \ \ \ \ \ \ \ \ \ \text{for every }\lambda\in\mathbb{F}_{q}\left[  T\right]
\text{ and }a\in A. \label{eq.frobcond}%
\end{equation}
Of course, in this general setup, one has to \textbf{define} $a^{q}$ to mean
$F\left(  a\right)  $ for every $a\in A$. (Once this definition is made, the
classical definition of $\left[  P\right]  \left(  a\right)  $ for any
$P\in\mathbb{F}_{q}\left[  T\right]  $ and any $a\in A$ should work perfectly.)

More about this in Subsection \ref{subsect.F}.
\end{remark}

Here is why this is strange. One could wonder whether similar things hold in
the classical case (Theorem \ref{thm.gW}): what if $A$ is not a commutative
ring but just an (additive) abelian group with \textquotedblleft power
operations\textquotedblright\ satisfying rules like $\left(  a^{n}\right)
^{m}=a^{nm}$ ? After all, the only way multiplication in $A$ appears in
Theorem \ref{thm.gW} is through taking powers. However, the proof of Theorem
\ref{thm.gW} depends on exponent lifting, which uses multiplication and its
commutativity in a nontrivial way. In contrast, the two exponent lifting
lemmata used in the proof of Theorem \ref{thm.carlitz.gW} are both extremely
simple and \textbf{do not} use multiplication in $A$. It seems that $A$ being
a ring is a red herring in Theorem \ref{thm.carlitz.gW}.

I am wondering what use this generality can be put to. One possible field of
application would be restricted Lie algebras. What is a good example of a
restricted Lie algebra with an $\mathbb{F}_{q}\left[  T\right]  $-module
structure?\footnote{Non-rhetorical question. Please let me know!
(darijgrinberg[at]gmail.com)}

\subsection{\label{subsect.carlitz-Witt}Carlitz-Witt vectors}

Parroting Definition \ref{def.Witt.ghostmap}, we define:

\begin{definition}
\label{def.carlitz.Witt.ghostmap}Let $N$ be a $q$-nest. Let $A$ be a
commutative $\mathbb{F}_{q}\left[  T\right]  $-algebra. The \textit{Carlitz
ghost ring} of $A$ will mean the $\mathbb{F}_{q}\left[  T\right]  $-algebra
$A^{N}$ with componentwise $\mathbb{F}_{q}\left[  T\right]  $-algebra
structure (i. e., a direct product of $\mathbb{F}_{q}\left[  T\right]
$-algebras $A$ indexed over $N$). The \textit{Carlitz }$N$\textit{-ghost map}
$w_{N}:A^{N}\rightarrow A^{N}$ is the map defined by%
\[
w_{N}\left(  \left(  x_{P}\right)  _{P\in N}\right)  =\left(  \sum
\limits_{D\mid P}D\left[  \dfrac{P}{D}\right]  \left(  x_{D}\right)  \right)
_{P\in N}\ \ \ \ \ \ \ \ \ \ \text{for all }\left(  x_{P}\right)  _{P\in N}\in
A^{N}.
\]
This $N$-ghost map is $\mathbb{F}_{q}$-linear but (generally) neither
multiplicative nor $\mathbb{F}_{q}\left[  T\right]  $-linear.
\end{definition}

From the equivalence $\mathcal{C}_{1}\Longleftrightarrow\mathcal{D}_{1}$ in
Theorem \ref{thm.carlitz.gW}, we can obtain:\footnote{I'm not going to show
the proof, as I don't think you will have any trouble reconstructing it. One
has to set $A=\mathbb{F}_{q}\left[  T\right]  \left[  \Xi\right]  $, where
$\Xi$ is a family of indeterminates, and define morphisms $\varphi_{P}$ by
$\varphi_{P}\left(  Q\right)  =Q\left(  \left[  P\right]  \left(  \Xi\right)
\right)  $, where $\left[  P\right]  \left(  \Xi\right)  $ means the family
obtained by applying $\left[  P\right]  $ to each variable in the family $\Xi
$. Alternatively, one could define morphisms $\varphi_{P}$ by $\varphi
_{P}\left(  Q\right)  =Q\left(  \Xi^{q^{\deg P}}\right)  $; these are
different morphisms but they also work here.}

\begin{theorem}
\label{thm.carlitz.Witt.class}Let $N$ be a $q$-nest. There exists a unique
functor $W_{N}:\mathbf{CRing}_{\mathbb{F}_{q}\left[  T\right]  }%
\rightarrow\mathbf{CRing}_{\mathbb{F}_{q}\left[  T\right]  }$ with the
following two properties:

-- We have $W_{N}\left(  A\right)  =A^{N}$ \textbf{as a set} for every
commutative $\mathbb{F}_{q}\left[  T\right]  $-algebra $A$.

-- The map $w_{N}:A^{N}\rightarrow A^{N}$ \textbf{regarded as a map }%
$W_{N}\left(  A\right)  \rightarrow A^{N}$ is an $\mathbb{F}_{q}\left[
T\right]  $-algebra homomorphism for every commutative $\mathbb{F}_{q}\left[
T\right]  $-algebra $A$.

This functor $W_{N}$ is called the \textit{Carlitz }$N$\textit{-Witt vector
functor}. For every $\mathbb{F}_{q}\left[  T\right]  $-algebra $A$, we call
the $\mathbb{F}_{q}\left[  T\right]  $-algebra $W_{N}\left(  A\right)  $ the
\textit{Carlitz }$N$\textit{-Witt vector ring over }$A$.

The map $w_{N}:W_{N}\left(  A\right)  \rightarrow A^{N}$ itself becomes a
natural transformation from the functor $W_{N}$ to the functor $\mathbf{CRing}%
_{\mathbb{F}_{q}\left[  T\right]  }\rightarrow\mathbf{CRing}_{\mathbb{F}%
_{q}\left[  T\right]  },\ A\mapsto A^{N}$. We will call this natural
transformation $w_{N}$ as well.
\end{theorem}

This theorem, of course, yields that the sum and the product of two Carlitz
ghost-Witt vectors \textbf{over any commutative }$\mathbb{F}_{q}\left[
T\right]  $\textbf{-algebra} is a Carlitz ghost-Witt vector, and that any
$\mathbb{F}_{q}\left[  T\right]  $-multiple of a Carlitz ghost-Witt vector is
a Carlitz ghost-Witt vector.

But this result is not optimal. In fact, it still holds in the more general
setup of Remark \ref{rmk.carlitz.gW.1'}. This can no longer be proven using
Theorem \ref{thm.carlitz.Witt.class}, since the polynomial ring $\mathbb{F}%
_{q}\left[  T\right]  \left[  \Xi\right]  $ is a free commutative
$\mathbb{F}_{q}\left[  T\right]  $-algebra but not (in a reasonable way) a
free object in the category of $\mathbb{F}_{q}\left[  T\right]  $-modules $A$
with an $\mathbb{F}_{q}$-linear Frobenius map $F:A\rightarrow A$ which
satisfies (\ref{eq.frobcond}). I will lose some more words on this in
Subsection \ref{subsect.F}.

\begin{remark}
Let $N$ be a $q$-nest. The $\mathbb{F}_{q}$-vector space structure on the
$\mathbb{F}_{q}\left[  T\right]  $-algebra $W_{N}\left(  A\right)  $ is just
componentwise. Thus, $w_{N}$ is an $\mathbb{F}_{q}$-vector space homomorphism
when considered as a map $A^{N}\rightarrow A^{N}$. As a consequence, the zero
of the $\mathbb{F}_{q}\left[  T\right]  $-algebra $W_{N}\left(  A\right)  $ is
the family $\left(  0\right)  _{P\in N}$.
\end{remark}

The unity of the $\mathbb{F}_{q}\left[  T\right]  $-algebra $W_{N}\left(
A\right)  $ is not as simple as it was in Theorem \ref{thm.Witt.class}.

We have only used $\mathcal{C}_{1}\Longleftrightarrow\mathcal{D}_{1}$ so far.
What about $\mathcal{C}_{1}\Longleftrightarrow\mathcal{D}_{2}$ ?

\begin{definition}
\label{def.carlitz.Witt.ghostmap.tilde}Let $N$ be a $q$-nest. Let $A$ be a
commutative $\mathbb{F}_{q}\left[  T\right]  $-algebra. The \textit{Carlitz
tilde }$N$\textit{-ghost map} $\widetilde{w}_{N}:A^{N}\rightarrow A^{N}$ is
the map defined by%
\[
\widetilde{w}_{N}\left(  \left(  x_{P}\right)  _{P\in N}\right)  =\left(
\sum\limits_{D\mid P}Dx_{D}^{q^{\deg\left(  P / D\right)  }}\right)  _{P\in
N}\ \ \ \ \ \ \ \ \ \ \text{for all }\left(  x_{P}\right)  _{P\in N}\in
A^{N}.
\]
This tilde $N$-ghost map is $\mathbb{F}_{q}$-linear but (generally) neither
multiplicative nor $\mathbb{F}_{q}\left[  T\right]  $-linear.
\end{definition}

From the equivalence $\mathcal{C}_{1}\Longleftrightarrow\mathcal{D}_{2}$ in
Theorem \ref{thm.carlitz.gW}, we get:

\begin{theorem}
\label{thm.carlitz.Witt.class.tilde}Let $N$ be a $q$-nest. There exists a
unique functor $\widetilde{W}_{N}:\mathbf{CRing}_{\mathbb{F}_{q}\left[
T\right]  }\rightarrow\mathbf{CRing}_{\mathbb{F}_{q}\left[  T\right]  }$ with
the following two properties:

-- We have $\widetilde{W}_{N}\left(  A\right)  =A^{N}$ \textbf{as a set} for
every commutative $\mathbb{F}_{q}\left[  T\right]  $-algebra $A$.

-- The map $\widetilde{w}_{N}:A^{N}\rightarrow A^{N}$ \textbf{regarded as a
map }$\widetilde{W}_{N}\left(  A\right)  \rightarrow A^{N}$ is an
$\mathbb{F}_{q}\left[  T\right]  $-algebra homomorphism for every commutative
$\mathbb{F}_{q}\left[  T\right]  $-algebra $A$.

This functor $\widetilde{W}_{N}$ is called the \textit{Carlitz tilde }%
$N$\textit{-Witt vector functor}. For every $\mathbb{F}_{q}\left[  T\right]
$-algebra $A$, we call the $\mathbb{F}_{q}\left[  T\right]  $-algebra
$\widetilde{W}_{N}\left(  A\right)  $ the \textit{Carlitz tilde }%
$N$\textit{-Witt vector ring over }$A$. The zero of this $\mathbb{F}%
_{q}\left[  T\right]  $-algebra $\widetilde{W}_{N}\left(  A\right)  $ is the
family $\left(  0\right)  _{P\in N}$, and its unity is the family $\left(
\delta_{P,1}\right)  _{P\in N}$ (where $\delta_{u,v}$ is defined to be $%
\begin{cases}
1, & \text{if }u=v;\\
0, & \text{if }u\neq v
\end{cases}
$ for any two objects $u$ and $v$).

The map $\widetilde{w}_{N}:\widetilde{W}_{N}\left(  A\right)  \rightarrow
A^{N}$ itself becomes a natural transformation from the functor $\widetilde{W}%
_{N}$ to the functor $\mathbf{CRing}_{\mathbb{F}_{q}\left[  T\right]
}\rightarrow\mathbf{CRing}_{\mathbb{F}_{q}\left[  T\right]  },\ A\mapsto
A^{N}$. We will call this natural transformation $\widetilde{w}_{N}$ as well.
\end{theorem}

But we have not really found two really different functors...

\begin{theorem}
\label{thm.carlitz.Witt.W=W}Let $N$ be a $q$-nest. The functors $W_{N}$ and
$\widetilde{W}_{N}$ are isomorphic by an isomorphism which forms a commutative
triangle with $w_{N}$ and $\widetilde{w}_{N}$.
\end{theorem}

This is again proven using Theorem \ref{thm.carlitz.gW} and universal polynomials.

The following theorem allows us to prove functorial identities by working with
ghost components:

\begin{theorem}
\label{thm.carlitz.Witt.iso}Let $N$ be a $q$-nest. For any commutative
$\mathbb{F}_{q}\left(  T\right)  $-algebra $A$, the maps $w_{N}:W_{N}\left(
A\right)  \rightarrow A^{N}$ and $\widetilde{w}_{N}:\widetilde{W}_{N}\left(
A\right)  \rightarrow A^{N}$ are $\mathbb{F}_{q}\left[  T\right]  $-algebra isomorphisms.
\end{theorem}

We have an ``almost-universal property'' again, following from exponent
lifting and the implication $\mathcal{C}_{1}\Longrightarrow\mathcal{D}_{1}$ in
Theorem \ref{thm.carlitz.gW}:

\begin{theorem}
\label{thm.carlitz.Witt.frob.au}Let $N$ be a $q$-nest. Let $A$ be a
commutative $\mathbb{F}_{q}\left[  T\right]  $-algebra such that no element of
$N$ is a zero-divisor in $A$. For every $P\in N$, let $\sigma_{P}$ be an
$\mathbb{F}_{q}\left[  T\right]  $-algebra endomorphism of $A$. Assume that
$\sigma_{P}\circ\sigma_{Q}=\sigma_{PQ}$ for any $P\in N$ and $Q\in N$
satisfying $PQ\in N$. Also assume that $\sigma_{1}=\operatorname*{id}$.
Finally, assume that $\sigma_{\pi}\left(  a\right)  \equiv\left[  \pi\right]
\left(  a\right)  \operatorname{mod}\pi A$ (or, equivalently, $\sigma_{\pi
}\left(  a\right)  \equiv a^{q^{\deg\pi}}\operatorname{mod}\pi A$) for every
monic irreducible $\pi\in N$ and every $a\in A$. Then, there exists a unique
$\mathbb{F}_{q}\left[  T\right]  $-algebra homomorphism $\varphi:A\rightarrow
W_{N}\left(  A\right)  $ satisfying%
\begin{equation}
\left(  w_{N}\circ\varphi\right)  \left(  a\right)  =\left(  \sigma_{P}\left(
a\right)  \right)  _{P\in N}\ \ \ \ \ \ \ \ \ \ \text{for every }a\in A.
\label{eq.thm.carlitz.Witt.frob.au.phi}%
\end{equation}

\end{theorem}

A similar result holds for $\widetilde{W}_{N}$ and $\widetilde{w}_{N}$.

What about Frobenius operations?

\begin{theorem}
\label{thm.carlitz.Witt.frob}Let $N$ be a $q$-nest.

\textbf{(a)} Let $M\in\mathbb{F}_{q}\left[  T\right]  _{+}$ be such that every
$P\in N$ satisfies $MP\in N$. Then, there exists a unique natural
transformation $\mathbf{f}_{M}:W_{N}\rightarrow W_{N}$ of \textbf{set-valued}
(not $\mathbb{F}_{q}\left[  T\right]  $-algebra-valued) functors such that any
commutative $\mathbb{F}_{q}\left[  T\right]  $-algebra $A$ and any
$\mathbf{x}\in W_{N}\left(  A\right)  $ satisfy%
\[
w_{N}\left(  \mathbf{f}_{M}\left(  \mathbf{x}\right)  \right)  =\left(
MP\text{-th coordinate of }w_{N}\left(  \mathbf{x}\right)  \right)  _{P\in
N},
\]
where $\mathbf{f}_{M}$ is short for $\mathbf{f}_{M}\left(  A\right)  $.

\textbf{(b)} This natural transformation $\mathbf{f}_{M}$ is actually a
natural transformation $W_{N}\rightarrow W_{N}$ of $\mathbb{F}_{q}\left[
T\right]  $\textbf{-algebra-valued} functors as well. That is, $\mathbf{f}%
_{M}:W_{N}\left(  A\right)  \rightarrow W_{N}\left(  A\right)  $ is an
$\mathbb{F}_{q}\left[  T\right]  $-algebra homomorphism for every commutative
$\mathbb{F}_{q}\left[  T\right]  $-algebra $A$. (Here, again, $\mathbf{f}_{M}$
stands short for $\mathbf{f}_{M}\left(  A\right)  $.) We call $\mathbf{f}_{M}$
the $M$\textit{-th Frobenius} on $W_{N}$.

\textbf{(c)} We have $\mathbf{f}_{1}=\operatorname*{id}$. Any $P\in
\mathbb{F}_{q}\left[  T\right]  _{+}$ and $Q\in\mathbb{F}_{q}\left[  T\right]
_{+}$ such that $\mathbf{f}_{P}$ and $\mathbf{f}_{Q}$ are well-defined satisfy
$\mathbf{f}_{P}\circ\mathbf{f}_{Q}=\mathbf{f}_{PQ}$.

\textbf{(d)} Let $\pi\in\mathbb{F}_{q}\left[  T\right]  $ be a monic
irreducible such that every $P\in N$ satisfies $\pi P\in N$. We have
$\mathbf{f}_{\pi}\left(  \mathbf{x}\right)  \equiv\left[  \pi\right]  \left(
\mathbf{x}\right)  \operatorname{mod}\pi W_{N}\left(  A\right)  $ (in
$W_{N}\left(  A\right)  $) for every commutative $\mathbb{F}_{q}\left[
T\right]  $-algebra $A$ and every $\mathbf{x}\in W_{N}\left(  A\right)  $.
\end{theorem}

\begin{corollary}
\label{cor.carlitz.Witt.frob.au.preserves-frob}Consider the setting of Theorem
\ref{thm.carlitz.Witt.frob.au}. Then (from Theorem
\ref{thm.carlitz.Witt.frob.au}) we know that there exists a unique
$\mathbb{F}_{q}\left[  T\right]  $-algebra homomorphism $\varphi:A\rightarrow
W_{N}\left(  A\right)  $ satisfying (\ref{eq.thm.carlitz.Witt.frob.au.phi}).
Consider this $\varphi$. Let $M\in N$ be such that every $P\in N$ satisfies
$MP\in N$. Then,%
\[
\varphi\circ\sigma_{M}=\mathbf{f}_{M}\circ\varphi\ \ \ \ \ \ \ \ \ \ \text{for
every }M\in N.
\]

\end{corollary}

\begin{corollary}
\label{cor.carlitz.Witt.frob.adjoint}Consider the setting of Theorem
\ref{thm.carlitz.Witt.frob.au}. Assume that $N$ is closed under multiplication
(i.e., we have $MP\in N$ for every $M\in N$ and $P\in N$). Furthermore, let
$B$ be a commutative $\mathbb{F}_{q}\left[  T\right]  $-algebra such that no
element of $N$ is a zero-divisor in $B$. Let $\operatorname*{proj}%
\nolimits_{B}:W_{N}\left(  B\right)  \rightarrow B$ be the map sending every
$u\in W_{N}\left(  B\right)  $ to the $1$-st coordinate of $w_{N}\left(
u\right)  \in B^{N}$. This $\operatorname*{proj}\nolimits_{B}$ is an
$\mathbb{F}_{q}\left[  T\right]  $-algebra homomorphism (since $w_{N}$ is an
$\mathbb{F}_{q}\left[  T\right]  $-algebra homomorphism).

Let $g:A\rightarrow B$ be an $\mathbb{F}_{q}\left[  T\right]  $-algebra
homomorphism. Then, there exists a unique $\mathbb{F}_{q}\left[  T\right]
$-algebra homomorphism $G:A\rightarrow W_{N}\left(  B\right)  $ with the
properties that $w_{1}\circ G=g$ and that%
\[
G\circ\sigma_{M}=\mathbf{f}_{M}\circ g\ \ \ \ \ \ \ \ \ \ \text{for every
}M\in N.
\]
This $G$ can be constructed as follows: Theorem \ref{thm.carlitz.Witt.frob.au}
shows that there exists a unique $\mathbb{F}_{q}\left[  T\right]  $-algebra
homomorphism $\varphi:A\rightarrow W_{N}\left(  A\right)  $ satisfying
(\ref{eq.thm.carlitz.Witt.frob.au.phi}). Consider this $\varphi$. Since
$W_{N}$ is a functor, the $\mathbb{F}_{q}\left[  T\right]  $-algebra
homomorphism $g:A\rightarrow B$ gives rise to an $\mathbb{F}_{q}\left[
T\right]  $-algebra homomorphism $W_{N}\left(  g\right)  :W_{N}\left(
A\right)  \rightarrow W_{N}\left(  B\right)  $. Now, the $G$ is constructed as
the composition $W_{N}\left(  g\right)  \circ\varphi$.
\end{corollary}

A Verschiebung exists too:

\begin{theorem}
\label{thm.carlitz.Witt.ver}Let $N$ be a $q$-nest.

\textbf{(a)} Let $M\in\mathbb{F}_{q}\left[  T\right]  _{+}$. Then, there
exists a unique natural transformation $\mathbf{V}_{M}:W_{N}\rightarrow W_{N}$
of \textbf{set-valued} (not $\mathbb{F}_{q}\left[  T\right]  $-algebra-valued)
functors such that any commutative $\mathbb{F}_{q}\left[  T\right]  $-algebra
$A$ and any $\mathbf{x}\in W_{N}\left(  A\right)  $ satisfy%
\[
w_{N}\left(  \mathbf{V}_{M}\left(  \mathbf{x}\right)  \right)  =\left(
\begin{cases}
M\cdot\left(  \dfrac{P}{M}\text{-th coordinate of }w_{N}\left(  \mathbf{x}%
\right)  \right)  , & \text{if }M\mid P;\\
0, & \text{if }M\nmid P
\end{cases}
\right)  _{P\in N},
\]
where $\mathbf{V}_{M}$ is short for $\mathbf{V}_{M}\left(  A\right)  $.

\textbf{(b)} This natural transformation $\mathbf{V}_{M}$ is actually a
natural transformation $W_{N}\rightarrow W_{N}$ of
\textbf{abelian-group-valued} functors as well. More precisely, $\mathbf{V}%
_{M}:W_{N}\left(  A\right)  \rightarrow W_{N}\left(  A\right)  $ is a
homomorphism of additive groups for every commutative $\mathbb{F}_{q}\left[
T\right]  $-algebra $A$. (Here, again, $\mathbf{V}_{M}$ stands short for
$\mathbf{V}_{M}\left(  A\right)  $.) We call $\mathbf{V}_{M}$ the
$M$\textit{-th Verschiebung} on $W_{N}$.

\textbf{(c)} We have $\mathbf{V}_{1}=\operatorname*{id}$. Any two
$P\in\mathbb{F}_{q}\left[  T\right]  _{+}$ and $Q\in\mathbb{F}_{q}\left[
T\right]  _{+}$ satisfy $\mathbf{V}_{P}\circ\mathbf{V}_{Q}=\mathbf{V}_{PQ}$.

\textbf{(d)} Actually, $\mathbf{V}_{M}\left(  \left(  x_{P}\right)  _{P\in
N}\right)  =\left(
\begin{cases}
x_{P/M}, & \text{if }M\mid P;\\
0, & \text{if }M\nmid P
\end{cases}
\right)  _{P\in N}$ for any $P\in\mathbb{F}_{q}\left[  T\right]  _{+}$, any
commutative $\mathbb{F}_{q}\left[  T\right]  $-algebra $A$ and any $\left(
x_{P}\right)  _{P\in N}\in W_{N}\left(  A\right)  $.
\end{theorem}

And here is a Carlitz analogue of the Artin-Hasse exponential:

\begin{theorem}
\label{thm.carlitz.Witt.AH}Let $N$ be a $q$-nest. Assume that $PQ\in N$ for
all $P\in N$ and $Q\in N$.

\textbf{(a)} There exists a unique natural transformation $\operatorname*{AH}%
:W_{N}\rightarrow W_{N}\circ W_{N}$ (of functors $\mathbf{CRing}%
_{\mathbb{F}_{q}\left[  T\right]  }\rightarrow\mathbf{CRing}_{\mathbb{F}%
_{q}\left[  T\right]  }$) such that every commutative $\mathbb{F}_{q}\left[
T\right]  $-algebra $A$, every $P\in N$ and every $\mathbf{x}\in W_{N}\left(
A\right)  $ satisfy%
\[
\left(  P\text{-th coordinate of }w_{N}\left(  \operatorname*{AH}\left(
\mathbf{x}\right)  \right)  \right)  =\mathbf{f}_{P}\left(  \mathbf{x}\right)
\]
(where $w_{N}$ this time stands for the natural transformation $w_{N}$
evaluated at the $\mathbb{F}_{q}\left[  T\right]  $-algebra $W_{N}\left(
A\right)  $; thus, $w_{N}\left(  \operatorname*{AH}\left(  \mathbf{x}\right)
\right)  $ is an element of $\left(  W_{N}\left(  A\right)  \right)  ^{N}$).

\textbf{(b)} Let $P\in N$, and let $A$ be a commutative $\mathbb{F}_{q}\left[
T\right]  $-algebra. Let $w_{P}:W_{N}\left(  A\right)  \rightarrow A$ be the
map sending each $\mathbf{x}\in W_{N}\left(  A\right)  $ to the $P$-th
coordinate of $w_{N}\left(  \mathbf{x}\right)  $. Then, $W_{N}\left(
w_{P}\right)  \circ\operatorname*{AH}=\mathbf{f}_{P}$.
\end{theorem}

\subsection{\label{subsect.F}$\mathcal{F}$-modules}

The classical $N$-Witt vector functor for $N\subseteq\mathbb{N}_{+}$ being a
nest is a functor $\mathbf{CRing}\rightarrow\mathbf{CRing}$, and I don't see
how to extend it to any broader category than $\mathbf{CRing}$. The proof of
its well-definedness, at least, uses the whole ring structure, not just the
power maps. The situation with $q$-nests and their Carlitz $N$-Witt vector
functors is different, as mentioned in Remark \ref{rmk.carlitz.gW.1'}. Let me
develop this a bit further, although I don't really understand where this all
is headed.

Let $\mathcal{F}$ be the $\mathbb{F}_{q}$-algebra $\mathbb{F}_{q}\left\langle
F,T\ \mid\ FT=T^{q}F\right\rangle $. This $\mathcal{F}$ can be considered as a
skew polynomial ring $\mathbb{F}_{q}\left[  T\right]  \left[
F;\ \operatorname*{Frob}\right]  $ over the polynomial ring $\mathbb{F}%
_{q}\left[  T\right]  $, where $\operatorname*{Frob}:\mathbb{F}_{q}\left[
T\right]  \rightarrow\mathbb{F}_{q}\left[  T\right]  $ is the Frobenius
endomorphism which sends every $a\in\mathbb{F}_{q}\left[  T\right]  $ to
$a^{q}$.

Note that $\mathcal{F}$ is neither an $\mathbb{F}_{q}\left[  T\right]
$-algebra nor an $\mathbb{F}_{q}\left[  F\right]  $-algebra in the way I
understand these words, since the center of $\mathcal{F}$ is $\mathbb{F}_{q}$.
But we have well-defined $\mathbb{F}_{q}$-algebra homomorphisms $\mathbb{F}%
_{q}\left[  T\right]  \rightarrow\mathcal{F}$ and $\mathbb{F}_{q}\left[
F\right]  \rightarrow\mathcal{F}$, which make $\mathcal{F}$ into a left
$\mathbb{F}_{q}\left[  T\right]  $-module, a right $\mathbb{F}_{q}\left[
T\right]  $-module, a left $\mathbb{F}_{q}\left[  F\right]  $-module, and a
right $\mathbb{F}_{q}\left[  F\right]  $-module. The left $\mathbb{F}%
_{q}\left[  T\right]  $-module structure on $\mathcal{F}$ is probably the most
useful one.

\begin{itemize}
\item As left $\mathbb{F}_{q}\left[  T\right]  $-module, $\mathcal{F}$ is free
with basis $\left(  F^{i}\right)  _{i\geq0}$ and thus torsionfree (this will
be useful).

\item As right $\mathbb{F}_{q}\left[  T\right]  $-module, $\mathcal{F}$ is
free with basis $\left(  T^{j}F^{i}\right)  _{i\geq0,\ 0\leq j<q^{i}}$.

\item As right $\mathbb{F}_{q}\left[  F\right]  $-module, $\mathcal{F}$ is
free with basis $\left(  T^{j}\right)  _{j\geq0}$.

\item As left $\mathbb{F}_{q}\left[  F\right]  $-module, $\mathcal{F}$ is free
with basis $\left(  T^{j}F^{i}\right)  _{i=0\text{ or }q\nmid j}$. As a
consequence, it is torsionfree (but this also follows from the isomorphism
$\mathcal{F}\rightarrow\mathbb{F}_{q}\left[  T\right]  \left[  X\right]
_{q-\operatorname*{lin}}$ introduced below).

\item As $\mathbb{F}_{q}\left[  F\right]  $-$\mathbb{F}_{q}\left[  T\right]
$-bimodule, $\mathcal{F}$ is free with basis $\left(  T^{j}F^{i}\right)
_{\left(  i=0\text{ or }q\nmid j\right)  \text{ and }0\leq j<q^{i}}$ (that is,
$\mathcal{F}=\bigoplus\limits_{\substack{\left(  i,j\right)  \in\mathbb{N}%
^{2};\\\left(  i=0\text{ or }q\nmid j\right)  \text{ and }0\leq j<q^{i}%
}}\mathbb{F}_{q}\left[  F\right]  \cdot\left(  T^{j}F^{i}\right)
\cdot\mathbb{F}_{q}\left[  T\right]  $, and each $\mathbb{F}_{q}\left[
F\right]  \cdot\left(  T^{j}F^{i}\right)  \cdot\mathbb{F}_{q}\left[  T\right]
$ is isomorphic to $\mathbb{F}_{q}\left[  F\right]  \otimes\mathbb{F}%
_{q}\left[  T\right]  $ as an $\mathbb{F}_{q}\left[  F\right]  $%
-$\mathbb{F}_{q}\left[  T\right]  $-bimodule).
\end{itemize}

These freeness statements actually have little to do with $\mathbb{F}_{q}$ or
the fact that $q$ is a prime power. They are combinatorial consequences of the
fact that $\mathcal{F}$ is the monoid algebra (over $\mathbb{F}_{q}$) of the
monoid $\left\langle F,T\ \mid\ FT=T^{q}F\right\rangle $, which monoid is
cancellative and whose elements can be uniquely written in the form
$T^{j}F^{i}$ with $\left(  i,j\right)  \in\mathbb{N}^{2}$. Actually, this
monoid is $\mathcal{J}$-trivial. Finite $\mathcal{J}$-trivial monoids have a
very nice representation theory \cite{dhns}; does ours?\footnote{I wouldn't
hope for much; the representation theory of $\left\langle F,T\ \mid
\ FT=TF\right\rangle $ is supposedly ugly.}

Every commutative $\mathbb{F}_{q}\left[  T\right]  $-algebra is canonically an
$\mathcal{F}$-module, by letting $T$ act as left multiplication with $T$, and
letting $F$ act as taking the $q$-th power in the algebra.

Let us notice that $FP=P^{q}F$ in $\mathcal{F}$ for every $P\in\mathbb{F}%
_{q}\left[  T\right]  $. This is rather important; it yields that
$\mathcal{F}\cdot P\cdot\mathcal{F}\subseteq P\cdot\mathcal{F}$ for every
$P\in\mathbb{F}_{q}\left[  T\right]  $.

By the universal property of the polynomial ring, there exists a unique
$\mathbb{F}_{q}$-algebra homomorphism $\operatorname*{Carl}:\mathbb{F}%
_{q}\left[  T\right]  \rightarrow\mathcal{F}$ which sends $T$ to $F+T$. This
$\operatorname*{Carl}$ is a very important homomorphism.

There is another interesting, and important, map around here. Let
$\mathbb{F}_{q}\left[  T\right]  \left[  X\right]  _{q-\operatorname*{lin}}$
be the $\mathbb{F}_{q}\left[  T\right]  $-submodule of the polynomial ring
$\mathbb{F}_{q}\left[  T\right]  \left[  X\right]  $ consisting of all
$q$\textbf{-polynomials}, i. e., polynomials in which only the monomials
$X^{q^{0}}$, $X^{q^{1}}$, $X^{q^{2}}$, $...$ appear (we consider $T$ as a
constant here). Then, $\mathbb{F}_{q}\left[  T\right]  \left[  X\right]
_{q-\operatorname*{lin}}$ is not an algebra under usual multiplication, but a
(noncommutative) algebra under composition (where again $X$ is the variable
and $T$ a constant). It turns out that%
\begin{align*}
\mathcal{F}  &  \rightarrow\mathbb{F}_{q}\left[  T\right]  \left[  X\right]
_{q-\operatorname*{lin}},\\
F  &  \mapsto X^{q},\\
T  &  \mapsto TX
\end{align*}
yields a well-defined $\mathbb{F}_{q}$-algebra isomorphism $\mathcal{F}%
\rightarrow\mathbb{F}_{q}\left[  T\right]  \left[  X\right]
_{q-\operatorname*{lin}}$. This is easy to check. This isomorphism allows
transferring some results from $\mathbb{F}_{q}\left[  T\right]  \left[
X\right]  $ to $\mathcal{F}$ (this is, for example, how I show that
$\mathcal{F}$ is a torsionfree right $\mathbb{F}_{q}\left[  T\right]  $-module).

It can be shown that for every monic irreducible $\pi\in\mathbb{F}_{q}\left[
T\right]  $,%
\begin{equation}
\text{there exists a unique }u\left(  \pi\right)  \in\mathcal{F}\text{ such
that }\operatorname*{Carl}\pi=F^{\deg\pi}+\pi\cdot u\left(  \pi\right)  .
\label{carl.pi}%
\end{equation}
\footnote{The notation $u\left(  \pi\right)  $ means that $u$ depends on $\pi
$; it is not meant to imply that $u\left(  \pi\right)  $ is a polynomial in
$\pi$.} Indeed, this follows easily from the fact that $\left[  \pi\right]
\left(  X\right)  \equiv X^{q^{\deg\pi}}\operatorname{mod}\pi$ in
$\mathbb{F}_{q}\left[  T\right]  \left[  X\right]  $ using the isomorphism
$\mathcal{F}\rightarrow\mathbb{F}_{q}\left[  T\right]  \left[  X\right]
_{q-\operatorname*{lin}}$.

Now, what is a left $\mathcal{F}$-module? One way to see a left $\mathcal{F}%
$-module is as a left $\mathbb{F}_{q}\left[  T\right]  $-module $A$ with an
$\mathbb{F}_{q}$-linear map $F:A\rightarrow A$ which satisfies $F\left(
Ta\right)  =T^{q}F\left(  a\right)  $ for every $a\in A$. This is easily seen
to be equivalent to a left $\mathbb{F}_{q}\left[  T\right]  $-module $A$ with
an $\mathbb{F}_{q}$-linear map $F:A\rightarrow A$ which satisfies $F\left(
\lambda a\right)  =\lambda^{q}F\left(  a\right)  $ for every $\lambda
\in\mathbb{F}_{q}\left[  T\right]  $ and $a\in A$. In every left $\mathcal{F}%
$-module $A$, we can \textbf{define} the operation of ``taking the $q$-th
power'' by $a^{q}=F\left(  a\right)  $ for every $a\in A$. Hence, we can
define an operation of ``taking the $q^{i}$-th power'' for every $i\geq0$.
This allows us to evaluate any Carlitz polynomial at elements of $A$; that is,
for any $P\in\mathbb{F}_{q}\left[  T\right]  $ and $a\in A$ we can define
$\left[  P\right]  \left(  a\right)  \in A$ (in the same way as this is
usually defined for $A$ being a commutative algebra). It is easily seen that%
\[
\left[  P\right]  \left(  a\right)  =\left(  \operatorname*{Carl}\left(
P\right)  \right)  \left(  a\right)  \ \ \ \ \ \ \ \ \ \ \text{for any }%
P\in\mathbb{F}_{q}\left[  T\right]  \text{ and }a\in A.
\]


Now, the situation described in Remark \ref{rmk.carlitz.gW.1'} is simply
understood as having a left $\mathcal{F}$-module $A$, and for every $P\in N$,
an $\mathcal{F}$-module endomorphism $\varphi_{P}$ of $A$.

The category of left $\mathcal{F}$-modules has its free objects, which simply
are free left $\mathcal{F}$-modules. If $\Xi$ is a set (to be viewed as a set
of ``indeterminates''), then a family of $\mathcal{F}$-module endomorphisms
$\varphi_{P}$ of the free $\mathcal{F}$-module $\mathcal{F}\Xi$ satisfying
Assumptions 1', 2 and 3 can be easily constructed (namely, $\varphi_{P}$ is
the unique $\mathcal{F}$-module homomorphism $\mathcal{F}\Xi\rightarrow
\mathcal{F}\Xi$ satisfying $\varphi_{P}\left(  \xi\right)  =\left[  P\right]
\left(  \xi\right)  $ for every $\xi\in\Xi$), although it took me a while to
show that they actually satisfy Assumption 2 (here I used (\ref{carl.pi})).

If I haven't done any mistakes, all results of Subsection
\ref{subsect.carlitz-Witt} carry over to the category of $\mathcal{F}%
$-modules; of course, $W_{N}$ and $\widetilde{W}_{N}$ will then be functors
from $_{\mathcal{F}}\mathbf{Mod}$ to $_{\mathcal{F}}\mathbf{Mod}$. One has to
be somewhat careful in the proofs because $\mathcal{F}$ is noncommutative and
it needs to be used that every $P\in\mathbb{F}_{q}\left[  T\right]  $
satisfies $\mathcal{F}\cdot P\cdot\mathcal{F}\subseteq P\cdot\mathcal{F}$.

\section{\label{sect.proofs}Proofs}

In this (so far unfinished) Section, I am going to prove most of the
statements made in Section \ref{sect.carlitzwitt}. I shall start from scratch
and forget about all the notation introduced in Section \ref{sect.carlitzwitt}%
; this notation will be reintroduced when the need for it arises.

In Section \ref{sect.carlitzwitt}, I presented the results for the case of
commutative $\mathbb{F}_{q}\left[  T\right]  $-algebras first, and then
pointed out how they can be generalized to $\mathcal{F}$-modules. In the
present Section \ref{sect.proofs}, however, I will proceed the other way
round, starting with the properties of $\mathcal{F}$. The latter properties
are unlikely to be new, as they are elementary and concern a well-studied
object ($\mathcal{F}$ is one of the most basic examples of an Ore extension);
in particular I suspect that some of them appear in \cite{ore-pp1} and
\cite{ore-pp2} (two references I regrettably have not had the time to read).

\subsection{The skew polynomial ring $\mathcal{M}$}

Let us first show a general fact:

\begin{proposition}
\label{prop.F-gen.bases}Let $\mathbb{K}$ be a commutative ring. Let $r$ be a
positive integer. Let $\mathcal{M}$ be the $\mathbb{K}$-algebra $\mathbb{K}%
\left\langle F,T\ \mid\ FT=T^{r}F\right\rangle $. There are well-defined
$\mathbb{K}$-algebra homomorphisms $\mathbb{K}\left[  T\right]  \rightarrow
\mathcal{M}$ (sending $T$ to $T$) and $\mathbb{K}\left[  F\right]
\rightarrow\mathcal{M}$ (sending $F$ to $F$). These homomorphisms make
$\mathcal{M}$ into a left $\mathbb{K}\left[  T\right]  $-module, a right
$\mathbb{K}\left[  T\right]  $-module, a left $\mathbb{K}\left[  F\right]
$-module, and a right $\mathbb{K}\left[  F\right]  $-module. Any of these two
left module structures can be combined with any of these two right module
structures to form a bimodule structure on $\mathcal{M}$ (for example, the
left $\mathbb{K}\left[  T\right]  $-module structure and the right
$\mathbb{K}\left[  F\right]  $-module structure on $\mathcal{M}$ can be
combined to form an $\mathbb{K}\left[  T\right]  $-$\mathbb{K}\left[
F\right]  $-bimodule structure on $\mathcal{M}$). (However, in general,
$\mathcal{M}$ is neither a $\mathbb{K}\left[  T\right]  $-algebra nor a
$\mathbb{K}\left[  F\right]  $-algebra.)

\textbf{(a)} We have $F^{a}T^{b}=T^{r^{a}b}F^{a}$ in $\mathcal{M}$ for every
$a\in\mathbb{N}$ and $b\in\mathbb{N}$.

\textbf{(b)} The $\mathbb{K}$-module $\mathcal{M}$ is free with basis $\left(
T^{j}F^{i}\right)  _{i\geq0,\ j\geq0}$.

\textbf{(c)} As left $\mathbb{K}\left[  T\right]  $-module, $\mathcal{M}$ is
free with basis $\left(  F^{i}\right)  _{i\geq0}$.

\textbf{(d)} As right $\mathbb{K}\left[  T\right]  $-module, $\mathcal{M}$ is
free with basis $\left(  T^{j}F^{i}\right)  _{i\geq0,\ 0\leq j<r^{i}}$.

\textbf{(e)} As right $\mathbb{K}\left[  F\right]  $-module, $\mathcal{M}$ is
free with basis $\left(  T^{j}\right)  _{j\geq0}$.

\textbf{(f)} As left $\mathbb{K}\left[  F\right]  $-module, $\mathcal{M}$ is
free with basis $\left(  T^{j}F^{i}\right)  _{i=0\text{ or }r\nmid j}$.

\textbf{(g)} As $\mathbb{K}\left[  F\right]  $-$\mathbb{K}\left[  T\right]
$-bimodule, $\mathcal{M}$ is free with basis $\left(  T^{j}F^{i}\right)
_{\left(  i=0\text{ or }r\nmid j\right)  \text{ and }0\leq j<r^{i}}$ (that is,
we have $\mathcal{M}=\bigoplus\limits_{\substack{\left(  i,j\right)
\in\mathbb{N}^{2};\\\left(  i=0\text{ or }r\nmid j\right)  \text{ and }0\leq
j<r^{i}}}\mathbb{K}\left[  F\right]  \cdot\left(  T^{j}F^{i}\right)
\cdot\mathbb{K}\left[  T\right]  $, and each $\mathbb{K}\left[  F\right]
\cdot\left(  T^{j}F^{i}\right)  \cdot\mathbb{K}\left[  T\right]  $ is
isomorphic to $\mathbb{K}\left[  F\right]  \otimes\mathbb{K}\left[  T\right]
$ as an $\mathbb{K}\left[  F\right]  $-$\mathbb{K}\left[  T\right]
$-bimodule, where the tensor product is taken over $\mathbb{K}$).
\end{proposition}

We notice that the $\mathbb{K}$-algebra $\mathcal{M}$ in Proposition
\ref{prop.F-gen.bases} is actually the monoid algebra (over $\mathbb{K}$) of
the monoid with generators $F,T$ and relation $FT=T^{r}F$. From this
viewpoint, all of Proposition \ref{prop.F-gen.bases} is easily revealed to be
a monoid-theoretical statement (with $\mathbb{K}$ being merely a distraction).
However, we shall work with $\mathbb{K}$-algebras rather than monoids for the
whole proof, if only for the sake of habitualness.

The only parts of Proposition \ref{prop.F-gen.bases} that will be used in the
following are parts \textbf{(a)}, \textbf{(b)}, \textbf{(c)} and \textbf{(e)}.
These are also the easiest ones to prove, so we advise the reader to skip most
of the following technical proof.

The following lemma will be used in our proof of Proposition
\ref{prop.F-gen.bases} \textbf{(f)}:

\begin{lemma}
\label{lem.F-gen.bases.f.1}Let $S$ be a set. Let $\phi:S\rightarrow S$ be an
injective map. Let $\ell:S\rightarrow\mathbb{N}$ be a map. Assume that%
\begin{equation}
\ell\left(  \phi\left(  s\right)  \right)  >\ell\left(  s\right)
\ \ \ \ \ \ \ \ \ \ \text{for every }s\in S.
\label{eq.lem.F-gen.bases.f.1.ass}%
\end{equation}
Let $B=S\setminus\phi\left(  S\right)  $. Define a map $\rho:B\times
\mathbb{N}\rightarrow S$ by%
\[
\rho\left(  s,k\right)  =\phi^{k}\left(  s\right)
\ \ \ \ \ \ \ \ \ \ \text{for every }\left(  s,k\right)  \in B\times
\mathbb{N}.
\]
Then, $\rho$ is a bijection.
\end{lemma}

(If we want to interpret Lemma \ref{lem.F-gen.bases.f.1} constructively, then
we should also require that there is an algorithm which, given an $s\in S$,
either reveals that $s\notin\phi\left(  S\right)  $ or computes a preimage of
$s$ under $\phi$.)

\begin{proof}
[Proof of Lemma \ref{lem.F-gen.bases.f.1}.]Let us first prove that the map
$\rho$ is injective.

Indeed, let $\left(  s,k\right)  $ and $\left(  s^{\prime},k^{\prime}\right)
$ be two elements of $B\times\mathbb{N}$ such that $\rho\left(  s,k\right)
=\rho\left(  s^{\prime},k^{\prime}\right)  $. We are going to prove that
$\left(  s,k\right)  =\left(  s^{\prime},k^{\prime}\right)  $.

The definition of $\rho$ yields $\rho\left(  s,k\right)  =\phi^{k}\left(
s\right)  $. Thus, $\phi^{k}\left(  s\right)  =\rho\left(  s,k\right)
=\rho\left(  s^{\prime},k^{\prime}\right)  =\phi^{k^{\prime}}\left(
s^{\prime}\right)  $ (by the definition of $\rho$).

The map $\phi^{k^{\prime}}$ is injective (since $\phi$ is injective).

We have $s^{\prime}\in B=S\setminus\phi\left(  S\right)  $. Thus, $s^{\prime
}\notin\phi\left(  S\right)  $.

Now, assume (for the sake of contradiction) that $k>k^{\prime}$. Hence,
$\phi^{k}\left(  s\right)  =\phi^{k^{\prime}+\left(  k-k^{\prime}\right)
}\left(  s\right)  =\phi^{k^{\prime}}\left(  \phi^{k-k^{\prime}}\left(
s\right)  \right)  $. But the map $\phi^{k^{\prime}}$ is injective. Therefore,
from $\phi^{k^{\prime}}\left(  \phi^{k-k^{\prime}}\left(  s\right)  \right)
=\phi^{k}\left(  s\right)  =\phi^{k^{\prime}}\left(  s^{\prime}\right)  $, we
obtain $\phi^{k-k^{\prime}}\left(  s\right)  =s^{\prime}$. Hence, $s^{\prime
}=\phi^{k-k^{\prime}}\left(  s\right)  \in\phi^{k-k^{\prime}}\left(  S\right)
\subseteq\phi\left(  S\right)  $ (since $k-k^{\prime}\geq1$ (since
$k>k^{\prime}$)). This contradicts $s^{\prime}\notin\phi\left(  S\right)  $.
This contradiction proves that our assumption (that $k>k^{\prime}$) was false.
Hence, we cannot have $k>k^{\prime}$. In other words, we must have $k\leq
k^{\prime}$. An analogous argument shows that $k^{\prime}\leq k$. Combining
this with $k\leq k^{\prime}$, we obtain $k=k^{\prime}$. Thus, $\phi^{k}\left(
s\right)  =\phi^{k^{\prime}}\left(  s\right)  $, so that $\phi^{k^{\prime}%
}\left(  s\right)  =\phi^{k}\left(  s\right)  =\phi^{k^{\prime}}\left(
s^{\prime}\right)  $. This yields $s=s^{\prime}$ (since the map $\phi
^{k^{\prime}}$ is injective). Combining this with $k=k^{\prime}$, we obtain
$\left(  s,k\right)  =\left(  s^{\prime},k^{\prime}\right)  $.

Let us now forget that we fixed $\left(  s,k\right)  $ and $\left(  s^{\prime
},k^{\prime}\right)  $. We thus have shown that if $\left(  s,k\right)  $ and
$\left(  s^{\prime},k^{\prime}\right)  $ are two elements of $B\times
\mathbb{N}$ such that $\rho\left(  s,k\right)  =\rho\left(  s^{\prime
},k^{\prime}\right)  $, then $\left(  s,k\right)  =\left(  s^{\prime
},k^{\prime}\right)  $. In other words, the map $\rho$ is injective.

Let us now show that the map $\rho$ is surjective. Indeed, we shall prove that%
\begin{equation}
\ell^{-1}\left(  n\right)  \subseteq\rho\left(  B\times\mathbb{N}\right)
\ \ \ \ \ \ \ \ \ \ \text{for every }n\in\mathbb{N}.
\label{pf.lem.F-gen.bases.f.1}%
\end{equation}


\textit{Proof of (\ref{pf.lem.F-gen.bases.f.1}):} We shall prove
(\ref{pf.lem.F-gen.bases.f.1}) by strong induction over $n$. Thus, we fix an
$N\in\mathbb{N}$, and we assume (as the induction hypothesis) that
(\ref{pf.lem.F-gen.bases.f.1}) holds for every $n<N$. Now we must prove that
(\ref{pf.lem.F-gen.bases.f.1}) holds for $n=N$. In other words, we must prove
that $\ell^{-1}\left(  N\right)  \subseteq\rho\left(  B\times\mathbb{N}%
\right)  $.

Let $x\in\ell^{-1}\left(  N\right)  $. Thus, $x\in S$ and $\ell\left(
x\right)  =N$. We shall prove that $x\in\rho\left(  B\times\mathbb{N}\right)
$.

If $x\notin\phi\left(  S\right)  $, then $x\in\rho\left(  B\times
\mathbb{N}\right)  $ holds\footnote{\textit{Proof.} Assume that $x\notin%
\phi\left(  S\right)  $. Thus, $x\in S\setminus\phi\left(  S\right)  =B$, so
that $\left(  x,0\right)  \in B\times\mathbb{N}$. Clearly, $\rho\left(
x,0\right)  =\phi^{0}\left(  x\right)  =x$, so that $x=\rho\left(  x,0\right)
\in\rho\left(  B\times\mathbb{N}\right)  $, qed.}. Hence, for the rest of the
proof of $x\subseteq\rho\left(  B\times\mathbb{N}\right)  $, we can WLOG
assume that $x\in\phi\left(  S\right)  $. Assume this. Thus, there exists an
$s\in S$ such that $x=\phi\left(  s\right)  $. Consider this $s$. From
$x=\phi\left(  s\right)  $, we obtain $\ell\left(  x\right)  =\ell\left(
\phi\left(  s\right)  \right)  >\ell\left(  s\right)  $ (by
(\ref{eq.lem.F-gen.bases.f.1.ass})). Hence, $\ell\left(  s\right)
<\ell\left(  x\right)  =N$. Therefore, the induction hypothesis shows that
(\ref{pf.lem.F-gen.bases.f.1}) holds for $n=\ell\left(  s\right)  $. In other
words, $\ell^{-1}\left(  \ell\left(  s\right)  \right)  \subseteq\rho\left(
B\times\mathbb{N}\right)  $. But $s\in\ell^{-1}\left(  \ell\left(  s\right)
\right)  \subseteq\rho\left(  B\times\mathbb{N}\right)  $. In other words,
there exists a $\left(  t,k\right)  \in B\times\mathbb{N}$ such that
$s=\rho\left(  t,k\right)  $. Consider this $\left(  t,k\right)  $. We have
$s=\rho\left(  t,k\right)  =\phi^{k}\left(  t\right)  $ (by the definition of
$\rho$), and $x=\phi\left(  \underbrace{s}_{=\phi^{k}\left(  t\right)
}\right)  =\phi\left(  \phi^{k}\left(  t\right)  \right)  =\phi^{k+1}\left(
t\right)  $. Comparing this with $\rho\left(  t,k+1\right)  =\phi^{k+1}\left(
t\right)  $ (by the definition of $\rho$), we obtain $x=\rho\left(
t,k+1\right)  \in\rho\left(  B\times\mathbb{N}\right)  $. Hence, $x\in
\rho\left(  B\times\mathbb{N}\right)  $ is proven.

Let us now forget that we fixed $x$. We thus have shown that $x\in\rho\left(
B\times\mathbb{N}\right)  $ for every $x\in\ell^{-1}\left(  N\right)  $. In
other words, $\ell^{-1}\left(  N\right)  \subseteq\rho\left(  B\times
\mathbb{N}\right)  $. In other words, (\ref{pf.lem.F-gen.bases.f.1}) holds for
$n=N$. This completes the induction proof of (\ref{pf.lem.F-gen.bases.f.1}).

Now, $\ell$ is a map $S\rightarrow\mathbb{N}$. Hence, $S=\bigcup
_{n\in\mathbb{N}}\underbrace{\ell^{-1}\left(  n\right)  }_{\substack{\subseteq
\rho\left(  B\times\mathbb{N}\right)  \\\text{(by
(\ref{pf.lem.F-gen.bases.f.1}))}}}\subseteq\bigcup_{n\in\mathbb{N}}\rho\left(
B\times\mathbb{N}\right)  \subseteq\rho\left(  B\times\mathbb{N}\right)  $. In
other words, the map $\rho$ is surjective. Hence, the map $\rho$ is bijective
(since we already know that $\rho$ is injective). This proves Lemma
\ref{lem.F-gen.bases.f.1}.
\end{proof}

We record two corollaries of Lemma \ref{lem.F-gen.bases.f.1}:

\begin{corollary}
\label{cor.F-gen.bases.f.1.cor1}Define a subset $B$ of $\mathbb{N}^{2}$ by%
\begin{equation}
B=\left\{  \left(  i,j\right)  \in\mathbb{N}^{2}\ \mid\ i=0\text{ or }r\nmid
j\right\}  . \label{eq.cor.F-gen.bases.f.1.cor1.def-B}%
\end{equation}
Define a map $\rho:B\times\mathbb{N}\rightarrow\mathbb{N}^{2}$ by%
\begin{equation}
\rho\left(  \left(  i,j\right)  ,k\right)  =\left(  i+k,r^{k}j\right)
\ \ \ \ \ \ \ \ \ \ \text{for every }\left(  \left(  i,j\right)  ,k\right)
\in B\times\mathbb{N}. \label{eq.cor.F-gen.bases.f.1.cor1.def-rho}%
\end{equation}
Then, the map $\rho$ is a bijection.
\end{corollary}

\begin{proof}
[Proof of Corollary \ref{cor.F-gen.bases.f.1.cor1}.]Let $\phi:\mathbb{N}%
^{2}\rightarrow\mathbb{N}^{2}$ be the map defined by%
\[
\phi\left(  i,j\right)  =\left(  i+1,rj\right)  \ \ \ \ \ \ \ \ \ \ \text{for
every }\left(  i,j\right)  \in\mathbb{N}^{2}.
\]
It is clear that this map $\phi$ is injective (since $r>0$). Moreover,
$B=\mathbb{N}^{2}\setminus\phi\left(  \mathbb{N}^{2}\right)  $%
\ \ \ \ \footnote{\textit{Proof.} We have
\begin{align*}
&  \mathbb{N}^{2}\setminus\phi\left(  \mathbb{N}^{2}\right) \\
&  =\left\{  \left(  i,j\right)  \in\mathbb{N}^{2}\ \mid\ \text{there exists
no }\left(  u,v\right)  \in\mathbb{N}^{2}\text{ such that }\left(  i,j\right)
=\underbrace{\phi\left(  u,v\right)  }_{\substack{=\left(  u+1,rv\right)
\\\text{(by the definition of }\phi\text{)}}}\right\} \\
&  =\left\{  \left(  i,j\right)  \in\mathbb{N}^{2}\ \mid
\ \underbrace{\text{there exists no }\left(  u,v\right)  \in\mathbb{N}%
^{2}\text{ such that }\left(  i,j\right)  =\left(  u+1,rv\right)
}_{\substack{\Longleftrightarrow\ \left(  \left(  i-1,j/r\right)
\notin\mathbb{N}^{2}\right)  \\\Longleftrightarrow\ \left(  i-1\notin%
\mathbb{N}\text{ or }j/r\notin\mathbb{N}\right)  }}\right\} \\
&  =\left\{  \left(  i,j\right)  \in\mathbb{N}^{2}\ \mid
\ \underbrace{i-1\notin\mathbb{N}}_{\Longleftrightarrow\ \left(  i=0\right)
}\text{ or }\underbrace{j/r\notin\mathbb{N}}_{\Longleftrightarrow\ \left(
r\nmid j\right)  }\right\} \\
&  =\left\{  \left(  i,j\right)  \in\mathbb{N}^{2}\ \mid\ i=0\text{ or }r\nmid
j\right\}  =B,
\end{align*}
qed.}. Given an $s\in S$, it is easy to algorithmically check whether
$s\notin\phi\left(  \mathbb{N}^{2}\right)  $ (because of the equivalence
$s\notin\phi\left(  \mathbb{N}^{2}\right)  \ \Longleftrightarrow
\ s\in\underbrace{\mathbb{N}^{2}\setminus\phi\left(  \mathbb{N}^{2}\right)
}_{=B}\ \Longleftrightarrow\ s\in B$), and if $s\in\phi\left(  \mathbb{N}%
^{2}\right)  $, then it is easy to compute a preimage of $s$ under $\phi$
(indeed, if $s=\left(  i,j\right)  \in\phi\left(  \mathbb{N}^{2}\right)  $,
then $\phi^{-1}\left(  s\right)  =\left(  i-1,j/r\right)  $).

Every $\left(  i,j\right)  \in\mathbb{N}^{2}$ and $k\in\mathbb{N}$ satisfy%
\begin{equation}
\phi^{k}\left(  i,j\right)  =\left(  i+k,r^{k}j\right)  .
\label{pf.cor.F-gen.bases.f.1.cor1.1}%
\end{equation}
(Indeed, this follows easily by induction on $k$.) Thus,%
\begin{equation}
\rho\left(  s,k\right)  =\phi^{k}\left(  s\right)
\ \ \ \ \ \ \ \ \ \ \text{for every }\left(  s,k\right)  \in B\times\mathbb{N}
\label{pf.cor.F-gen.bases.f.1.cor1.3}%
\end{equation}
\footnote{\textit{Proof of (\ref{pf.cor.F-gen.bases.f.1.cor1.3}):} Let
$\left(  s,k\right)  \in B\times\mathbb{N}$. Then, $s\in B\subseteq
\mathbb{N}^{2}$. Hence, $s$ can be written in the form $\left(  i,j\right)  $
for some $i,j\in\mathbb{N}$. Consider these $i,j$. We have%
\begin{align*}
\phi^{k}\left(  \underbrace{s}_{=\left(  i,j\right)  }\right)   &  =\phi
^{k}\left(  i,j\right)  =\left(  i+k,r^{k}j\right)
\ \ \ \ \ \ \ \ \ \ \left(  \text{by (\ref{pf.cor.F-gen.bases.f.1.cor1.1}%
)}\right) \\
&  =\rho\left(  \underbrace{\left(  i,j\right)  }_{=s},k\right)
\ \ \ \ \ \ \ \ \ \ \left(  \text{by
(\ref{eq.cor.F-gen.bases.f.1.cor1.def-rho})}\right) \\
&  =\rho\left(  s,k\right)  .
\end{align*}
This proves (\ref{pf.cor.F-gen.bases.f.1.cor1.3}).}.

Furthermore, define a map $\ell:\mathbb{N}^{2}\rightarrow\mathbb{N}$ by
\[
\ell\left(  i,j\right)  =i\ \ \ \ \ \ \ \ \ \ \text{for every }\left(
i,j\right)  \in\mathbb{N}^{2}.
\]
It is easy to see that for every $s\in\mathbb{N}^{2}$, we have $\ell\left(
\phi\left(  s\right)  \right)  =\ell\left(  s\right)  +1>\ell\left(  s\right)
$. Thus, we can apply Lemma \ref{lem.F-gen.bases.f.1} to $S=\mathbb{N}^{2}$
(indeed, the equality (\ref{pf.cor.F-gen.bases.f.1.cor1.3}) shows that our map
$\rho:B\times\mathbb{N}\rightarrow\mathbb{N}^{2}$ is identical with the map
$\rho:B\times\mathbb{N}\rightarrow S$ in Lemma \ref{lem.F-gen.bases.f.1}). As
a result, we conclude that $\rho$ is a bijection. This proves Corollary
\ref{cor.F-gen.bases.f.1.cor1}.
\end{proof}

\begin{corollary}
\label{cor.F-gen.bases.f.1.cor2}Define a subset $C$ of $\mathbb{N}^{2}$ by%
\begin{equation}
C=\left\{  \left(  i,j\right)  \in\mathbb{N}^{2}\ \mid\ \left(  i=0\text{ or
}r\nmid j\right)  \text{ and }0\leq j<r^{i}\right\}  .
\label{eq.cor.F-gen.bases.f.1.cor2.def-C}%
\end{equation}
Define a map $\zeta:C\times\mathbb{N}\times\mathbb{N}\rightarrow\mathbb{N}%
^{2}$ by%
\begin{equation}
\zeta\left(  \left(  i,j\right)  ,\ell,k\right)  =\left(  i+k,r^{k}\left(
j+r^{i}\ell\right)  \right)  \ \ \ \ \ \ \ \ \ \ \text{for every }\left(
\left(  i,j\right)  ,k,\ell\right)  \in C\times\mathbb{N}\times\mathbb{N}.
\label{eq.cor.F-gen.bases.f.1.cor2.def-zeta}%
\end{equation}
Then, the map $\zeta$ is a bijection.
\end{corollary}

\begin{proof}
[Proof of Corollary \ref{cor.F-gen.bases.f.1.cor2}.]Define a subset $B$ of
$\mathbb{N}^{2}$ by (\ref{eq.cor.F-gen.bases.f.1.cor1.def-B}). Clearly,
$C\subseteq B$.

Define a map $\tau:C\times\mathbb{N}\rightarrow B$ by
\[
\tau\left(  \left(  i,j\right)  ,\ell\right)  =\left(  i,j+r^{i}\ell\right)
\ \ \ \ \ \ \ \ \ \ \text{for every }\left(  \left(  i,j\right)  ,\ell\right)
\in C\times\mathbb{N}.
\]
It is easy to see that this map $\tau$ is well-defined (i.e., that $\left(
i,j+r^{i}\ell\right)  \in B$ for every $\left(  \left(  i,j\right)
,\ell\right)  \in C\times\mathbb{N}$).

For every integer $u$ and every positive integer $v$, we let $u\%v$ denote the
remainder of $u$ when divided by $v$, and we let $u//v$ denote the quotient of
$u$ when divided by $v$ with remainder. Thus, $u//v\in\mathbb{Z}$,
$u\%v\in\left\{  0,1,\ldots,v-1\right\}  $ and $u=\left(  u//v\right)  v+u\%v$.

Define a map $\gamma:B\rightarrow C\times\mathbb{N}$ by%
\[
\gamma\left(  i,j\right)  =\left(  \left(  i,j\%r^{i}\right)  ,j//r^{i}%
\right)  \ \ \ \ \ \ \ \ \ \ \text{for every }\left(  i,j\right)  \in B.
\]
Again, it is easy to see that this map $\gamma$ is well-defined (i.e., that
$\left(  \left(  i,j\%r^{i}\right)  ,j//r^{i}\right)  \in C\times\mathbb{N}$
for every $\left(  i,j\right)  \in B$).

Furthermore, it is easy to see that the maps $\tau$ and $\gamma$ are mutually
inverse\footnote{\textit{Proof.} Let us first show that $\tau\circ
\gamma=\operatorname*{id}$.
\par
Indeed, every $\left(  i,j\right)  \in B$ satisfies%
\begin{align*}
\left(  \tau\circ\gamma\right)  \left(  i,j\right)   &  =\tau\left(
\underbrace{\gamma\left(  i,j\right)  }_{=\left(  \left(  i,j\%r^{i}\right)
,j//r^{i}\right)  }\right)  =\tau\left(  \left(  i,j\%r^{i}\right)
,j//r^{i}\right)  =\left(  i,\underbrace{j\%r^{i}+r^{i}\left(  j//r^{i}%
\right)  }_{=j}\right) \\
&  \ \ \ \ \ \ \ \ \ \ \left(  \text{by the definition of }\tau\right) \\
&  =\left(  i,j\right)  .
\end{align*}
Thus, $\tau\circ\gamma=\operatorname*{id}$.
\par
On the other hand, let us prove that $\gamma\circ\tau=\operatorname*{id}$.
Indeed, fix $\left(  \left(  i,j\right)  ,\ell\right)  \in C\times\mathbb{N}$.
Then, $\left(  i,j\right)  \in C$. Thus, $\left(  i=0\text{ or }r\nmid
j\right)  $ and $0\leq j<r^{i}$. Now,%
\begin{align*}
\left(  \gamma\circ\tau\right)  \left(  \left(  i,j\right)  ,\ell\right)   &
=\gamma\left(  \underbrace{\tau\left(  \left(  i,j\right)  ,\ell\right)
}_{=\left(  i,j+r^{i}\ell\right)  }\right)  =\gamma\left(  i,j+r^{i}%
\ell\right) \\
&  =\left(  \left(  i,\underbrace{\left(  j+r^{i}\ell\right)  \%r^{i}%
}_{\substack{=j\\\text{(since }0\leq j<r^{i}\text{)}}}\right)
,\underbrace{\left(  j+r^{i}\ell\right)  //r^{i}}_{\substack{=\ell
\\\text{(since }0\leq j<r^{i}\text{)}}}\right)  =\left(  \left(  i,j\right)
,\ell\right)  .
\end{align*}
This proves that $\gamma\circ\tau=\operatorname*{id}$. Combining this with
$\tau\circ\gamma=\operatorname*{id}$, we obtain that the maps $\tau$ and
$\gamma$ are mutually inverse, qed.}. Hence, the map $\tau$ is a bijection.

We shall identify the set $C\times\mathbb{N}\times\mathbb{N}$ with $\left(
C\times\mathbb{N}\right)  \times\mathbb{N}$. Then, the map $\tau
\times\operatorname*{id}\nolimits_{\mathbb{N}}:\left(  C\times\mathbb{N}%
\right)  \times\mathbb{N}\rightarrow B\times\mathbb{N}$ can be viewed as a map
$C\times\mathbb{N}\times\mathbb{N}\rightarrow B\times\mathbb{N}$. This map
$\tau\times\operatorname*{id}\nolimits_{\mathbb{N}}$ sends every $\left(
\left(  i,j\right)  ,\ell,k\right)  \in C\times\mathbb{N}\times\mathbb{N}$ to
$\left(  \tau\left(  \left(  i,j\right)  ,\ell\right)  ,k\right)  $. Clearly,
the map $\tau\times\operatorname*{id}\nolimits_{\mathbb{N}}$ is a bijection
(since $\tau$ is a bijection).

On the other hand, define a map $\rho$ as in Corollary
\ref{cor.F-gen.bases.f.1.cor1}. Then, Corollary \ref{cor.F-gen.bases.f.1.cor1}
shows that the map $\rho$ is a bijection. But every $\left(  \left(
i,j\right)  ,\ell,k\right)  \in C\times\mathbb{N}\times\mathbb{N}$ satisfies%
\begin{align*}
&  \left(  \rho\circ\left(  \tau\times\operatorname*{id}\nolimits_{\mathbb{N}%
}\right)  \right)  \left(  \left(  i,j\right)  ,\ell,k\right) \\
&  =\rho\left(  \underbrace{\left(  \tau\times\operatorname*{id}%
\nolimits_{\mathbb{N}}\right)  \left(  \left(  i,j\right)  ,\ell,k\right)
}_{=\left(  \tau\left(  \left(  i,j\right)  ,\ell\right)  ,k\right)  }\right)
=\rho\left(  \left(  \underbrace{\tau\left(  \left(  i,j\right)  ,\ell\right)
}_{=\left(  i,j+r^{i}\ell\right)  },k\right)  \right) \\
&  =\rho\left(  \left(  i,j+r^{i}\ell\right)  ,k\right)  =\left(
i+k,r^{k}\left(  j+r^{i}\ell\right)  \right)  \ \ \ \ \ \ \ \ \ \ \left(
\text{by the definition of }\rho\right) \\
&  =\zeta\left(  \left(  i,j\right)  ,\ell,k\right)
\ \ \ \ \ \ \ \ \ \ \left(  \text{by
(\ref{eq.cor.F-gen.bases.f.1.cor2.def-zeta})}\right)  .
\end{align*}
Hence, $\rho\circ\left(  \tau\times\operatorname*{id}\nolimits_{\mathbb{N}%
}\right)  =\zeta$. Since the map $\rho\circ\left(  \tau\times
\operatorname*{id}\nolimits_{\mathbb{N}}\right)  $ is a bijection (because
both $\rho$ and $\tau\times\operatorname*{id}\nolimits_{\mathbb{N}}$ are
bijections), this shows that the map $\zeta$ is a bijection. This proves
Corollary \ref{cor.F-gen.bases.f.1.cor2}.
\end{proof}

\begin{proof}
[Proof of Proposition \ref{prop.F-gen.bases}.]\textbf{(a)} First, we have the
equality
\begin{equation}
FT^{b}=T^{rb}F \label{pf.prop.F-gen.bases.a.1}%
\end{equation}
in $\mathcal{M}$ for every $b\in\mathbb{N}$ (this can be proven by
straightforward induction over $b$). Using this equality, Proposition
\ref{prop.F-gen.bases} \textbf{(a)} can be proven by straightforward induction
over $a$.

\textbf{(b)} Let $\mathcal{N}$ be the free $\mathbb{K}$-module with basis
$\left(  a_{i,j}\right)  _{i\geq0,\ j\geq0}$. We let $\mathfrak{f}$ be the
$\mathbb{K}$-linear map $\mathcal{N}\rightarrow\mathcal{N}$ which sends every
$a_{i,j}$ to $a_{i+1,rj}$. We let $\mathfrak{t}$ be the $\mathbb{K}$-linear
map $\mathcal{N}\rightarrow\mathcal{N}$ which sends every $a_{i,j}$ to
$a_{i,j+1}$. Every $i,j,k\in\mathbb{N}$ satisfy%
\begin{equation}
\mathfrak{f}^{k}\left(  a_{i,j}\right)  =a_{i+k,r^{k}j}
\label{pf.prop.F-gen.bases.b.1}%
\end{equation}
and
\begin{equation}
\mathfrak{t}^{k}\left(  a_{i,j}\right)  =a_{i,j+k}.
\label{pf.prop.F-gen.bases.b.2}%
\end{equation}
(Both of these equalities are easily proven by induction over $k$.) Using
(\ref{pf.prop.F-gen.bases.b.2}), it is easy to see that $\mathfrak{f}%
\circ\mathfrak{t}=\mathfrak{t}^{r}\circ\mathfrak{f}$. Thus, we can define a
$\mathbb{K}$-algebra homomorphism $\Phi:\mathcal{M}\rightarrow
\operatorname*{End}\mathcal{N}$ by setting%
\begin{equation}
\Phi\left(  F\right)  =\mathfrak{f}\ \ \ \ \ \ \ \ \ \ \text{and}%
\ \ \ \ \ \ \ \ \ \ \Phi\left(  T\right)  =\mathfrak{t}
\label{pf.prop.F-gen.bases.b.3}%
\end{equation}
(where $\operatorname*{End}\mathcal{N}$ denotes the $\mathbb{K}$-algebra of
all $\mathbb{K}$-module endomorphisms of $\mathcal{N}$). Consider this $\Phi$.
For every $i,j\in\mathbb{N}$, we have%
\[
\Phi\left(  T^{j}F^{i}\right)  =\Phi\left(  T\right)  ^{j}\circ\Phi\left(
F\right)  ^{i}=\mathfrak{t}^{j}\circ\mathfrak{f}^{i}%
\ \ \ \ \ \ \ \ \ \ \left(  \text{by (\ref{pf.prop.F-gen.bases.b.3})}\right)
\]
and thus%
\begin{align}
\underbrace{\left(  \Phi\left(  T^{j}F^{i}\right)  \right)  }_{=\mathfrak{t}%
^{j}\circ\mathfrak{f}^{i}}\left(  a_{0,0}\right)   &  =\left(  \mathfrak{t}%
^{j}\circ\mathfrak{f}^{i}\right)  \left(  a_{0,0}\right)  =\mathfrak{t}%
^{j}\left(  \underbrace{\mathfrak{f}^{i}\left(  a_{0,0}\right)  }%
_{\substack{=a_{i,0}\\\text{(by (\ref{pf.prop.F-gen.bases.b.1}))}}}\right)
\nonumber\\
&  =\mathfrak{t}^{j}\left(  a_{i,0}\right)  =a_{i,j}
\label{pf.prop.F-gen.bases.b.4}%
\end{align}
(by (\ref{pf.prop.F-gen.bases.b.2})). Hence, the family $\left(  T^{j}%
F^{i}\right)  _{i\geq0,\ j\geq0}$ of elements of $\mathcal{M}$ is $\mathbb{K}%
$-linearly independent\footnote{because any linear dependence relation
$\sum_{i\geq0,\ j\geq0}\lambda_{i,j}T^{j}F^{i}=0$ would yield
\begin{align*}
\sum_{i\geq0,\ j\geq0}\lambda_{i,j}\underbrace{a_{i,j}}_{\substack{=\left(
\Phi\left(  T^{j}F^{i}\right)  \right)  \left(  a_{0,0}\right)  \\\text{(by
(\ref{pf.prop.F-gen.bases.b.4}))}}}  &  =\sum_{i\geq0,\ j\geq0}\lambda
_{i,j}\left(  \Phi\left(  T^{j}F^{i}\right)  \right)  \left(  a_{0,0}\right)
\\
&  =\left(  \Phi\left(  \underbrace{\sum_{i\geq0,\ j\geq0}\lambda_{i,j}%
T^{j}F^{i}}_{=0}\right)  \right)  \left(  a_{0,0}\right)  =0,
\end{align*}
which would lead to $\left(  \lambda_{i,j}\right)  _{i\geq0,\ j\geq0}=\left(
0\right)  _{i\geq0,\ j\geq0}$ since the family $\left(  a_{i,j}\right)
_{i\geq0,\ j\geq0}$ is linearly independent}.

Let us now show that this family spans $\mathcal{M}$. Indeed, let
$\mathcal{M}^{\prime}$ be the $\mathbb{K}$-submodule of $\mathcal{M}$ spanned
by the family $\left(  T^{j}F^{i}\right)  _{i\geq0,\ j\geq0}$. Then,
$1=T^{0}F^{0}\in\mathcal{M}^{\prime}$. Moreover, the $\mathbb{K}$-submodule
$\mathcal{M}^{\prime}$ satisfies $T\mathcal{M}^{\prime}\subseteq
\mathcal{M}^{\prime}$ (since $T\cdot T^{j}F^{i}=T^{j+1}F^{i}$ for every
$i,j\in\mathbb{N}$) and $F\mathcal{M}^{\prime}\subseteq\mathcal{M}^{\prime}$
(since $F\cdot T^{j}F^{i}=\underbrace{FT^{j}}_{\substack{=T^{rj}F\\\text{(by
(\ref{pf.prop.F-gen.bases.a.1}))}}}F^{i}=T^{rj}FF^{i}=T^{rj}F^{i+1}$ for every
$i,j\in\mathbb{N}$). Hence, $\mathcal{M}^{\prime}$ is a left $\mathcal{M}%
$-submodule of $\mathcal{M}$ (since the $\mathbb{K}$-algebra $\mathcal{M}$ is
generated by $F$ and $T$)\ \ \ \ \footnote{This argument in more detail:
\par
The $\mathbb{K}$-algebra $\mathcal{M}$ is generated by $F$ and $T$. From this,
it is easy to derive the following fact: If $\mathcal{V}$ is an $\mathbb{K}%
$-vector subspace of some left $\mathcal{M}$-module $\mathcal{U}$ satisfying
$F\mathcal{V}\subseteq\mathcal{V}$ and $T\mathcal{V}\subseteq\mathcal{V}$,
then $\mathcal{V}$ is a left $\mathcal{M}$-submodule of $\mathcal{U}$.
Applying this to $\mathcal{U}=\mathcal{M}$ and $\mathcal{V}=\mathcal{M}%
^{\prime}$, we conclude that $\mathcal{M}^{\prime}$ is a left $\mathcal{M}%
$-submodule of $\mathcal{M}$ (since $F\mathcal{M}^{\prime}\subseteq
\mathcal{M}^{\prime}$ and $T\mathcal{M}^{\prime}\subseteq\mathcal{M}^{\prime}%
$).}. Therefore, $\mathcal{M}\cdot\mathcal{M}^{\prime}\subseteq\mathcal{M}%
^{\prime}$. But $\mathcal{M}=\mathcal{M}\cdot\underbrace{1}_{\in
\mathcal{M}^{\prime}}\subseteq\mathcal{M}\cdot\mathcal{M}^{\prime}%
\subseteq\mathcal{M}^{\prime}$. This shows that the family $\left(  T^{j}%
F^{i}\right)  _{i\geq0,\ j\geq0}$ spans the $\mathbb{K}$-module $\mathcal{M}$
(since the $\mathbb{K}$-linear span of this family is $\mathcal{M}^{\prime}$).
Since we already know that this family is $\mathbb{K}$-linearly independent,
we can thus conclude that this family is a basis of the $\mathbb{K}$-module
$\mathcal{M}$. This proves Proposition \ref{prop.F-gen.bases} \textbf{(b)}.

\textbf{(c)} Let $\left(  e_{0},e_{1},e_{2},\ldots\right)  $ be the standard
basis of the left $\mathbb{K}\left[  T\right]  $-module $\mathbb{K}\left[
T\right]  ^{\left(  \mathbb{N}\right)  }$. Define a left $\mathbb{K}\left[
T\right]  $-module homomorphism $\alpha:\mathbb{K}\left[  T\right]  ^{\left(
\mathbb{N}\right)  }\rightarrow\mathcal{M}$ by sending each $e_{i}$ to $F^{i}%
$. Define a $\mathbb{K}$-module homomorphism $\beta:\mathcal{M}\rightarrow
\mathbb{K}\left[  T\right]  ^{\left(  \mathbb{N}\right)  }$ by sending each
$T^{j}F^{i}$ to $T^{j}e_{i}$. (This $\beta$ is well-defined, since Proposition
\ref{prop.F-gen.bases} \textbf{(b)} shows that $\left(  T^{j}F^{i}\right)
_{i\geq0,\ j\geq0}$ is a basis of the $\mathbb{K}$-module $\mathcal{M}$.) It
is easy to see that $\beta$ is a left $\mathbb{K}\left[  T\right]  $-module
homomorphism. It is straightforward to see that the homomorphisms $\alpha$ and
$\beta$ are mutually inverse. Thus, $\alpha$ is a left $\mathbb{K}\left[
T\right]  $-module isomorphism. As a consequence, the left $\mathbb{K}\left[
T\right]  $-module $\mathcal{M}$ has a basis $\left(  \underbrace{\alpha
\left(  e_{i}\right)  }_{=F^{i}}\right)  _{i\geq0}=\left(  F^{i}\right)
_{i\geq0}$. This proves Proposition \ref{prop.F-gen.bases} \textbf{(c)}.

\textbf{(d)} For every integer $u$ and every positive integer $v$, we let
$u\%v$ denote the remainder of $u$ when divided by $v$, and we let $u//v$
denote the quotient of $u$ when divided by $v$ with remainder. Thus,
$u//v\in\mathbb{Z}$, $u\%v\in\left\{  0,1,\ldots,v-1\right\}  $ and $u=\left(
u//v\right)  v+u\%v$.

Let $\mathcal{G}$ be the free right $\mathbb{K}\left[  T\right]  $-module with
basis $\left(  g_{i,j}\right)  _{i\geq0,\ 0\leq j<r^{i}}$. Define a right
$\mathbb{K}\left[  T\right]  $-module homomorphism $\alpha:\mathcal{G}%
\rightarrow\mathcal{M}$ by sending each $g_{i,j}$ to $T^{j}F^{i}$. Define a
$\mathbb{K}$-module homomorphism $\beta:\mathcal{M}\rightarrow\mathcal{G}$ by
sending each $T^{j}F^{i}$ to $g_{i,j\%r^{i}}T^{j//r^{i}}$. (This $\beta$ is
well-defined, since Proposition \ref{prop.F-gen.bases} \textbf{(b)} shows that
$\left(  T^{j}F^{i}\right)  _{i\geq0,\ j\geq0}$ is a basis of the $\mathbb{K}%
$-module $\mathcal{M}$.) It is easy to see that the homomorphisms $\alpha$ and
$\beta$ are mutually inverse\footnote{\textit{Proof.} We need to show that
$\alpha\circ\beta=\operatorname*{id}$ and $\beta\circ\alpha=\operatorname*{id}%
$.
\par
To prove that $\alpha\circ\beta=\operatorname*{id}$, we need to show that
$\left(  \alpha\circ\beta\right)  \left(  T^{j}F^{i}\right)  =T^{j}F^{i}$ for
every $i,j\in\mathbb{N}$. So let us fix $i,j\in\mathbb{N}$. Then,%
\begin{align*}
\left(  \alpha\circ\beta\right)  \left(  T^{j}F^{i}\right)   &  =\alpha\left(
\underbrace{\beta\left(  T^{j}F^{i}\right)  }_{=g_{i,j\%r^{i}}T^{j//r^{i}}%
}\right)  =\alpha\left(  g_{i,j\%r^{i}}T^{j//r^{i}}\right)
=\underbrace{\alpha\left(  g_{i,j\%r^{i}}\right)  }_{\substack{=T^{j\%r^{i}%
}F^{i}\\\text{(by the definition of }\alpha\text{)}}}T^{j//r^{i}}\\
&  \ \ \ \ \ \ \ \ \ \ \left(  \text{since }\alpha\text{ is a right
}\mathbb{K}\left[  T\right]  \text{-module homomorphism}\right) \\
&  =T^{j\%r^{i}}\underbrace{F^{i}T^{j//r^{i}}}_{\substack{=T^{r^{i}\left(
j//r^{i}\right)  }F^{i}\\\text{(by Proposition \ref{prop.F-gen.bases}
\textbf{(a),}}\\\text{applied to }a=i\text{ and }b=j//r^{i}\text{)}%
}}=\underbrace{T^{j\%r^{i}}T^{r^{i}\left(  j//r^{i}\right)  }}%
_{\substack{=T^{j\%r^{i}+r^{i}\left(  j//r^{i}\right)  }=T^{j}\\\text{(since
}j\%r^{i}+r^{i}\left(  j//r^{i}\right)  =\left(  j//r^{i}\right)
r^{i}+j\%r^{i}=j\text{)}}}F^{i}=T^{j}F^{i},
\end{align*}
which is what we wanted to prove.
\par
Thus, $\alpha\circ\beta=\operatorname*{id}$ is proven. It remains to prove
that $\beta\circ\alpha=\operatorname*{id}$.
\par
We know that $\mathcal{G}$ is spanned by $\left(  g_{i,j}\right)
_{i\geq0,\ 0\leq j<r^{i}}$ as a right $\mathbb{K}\left[  T\right]  $-module
(by the definition of $\mathcal{G}$). Hence, $\mathcal{G}$ is spanned by
$\left(  g_{i,j}T^{k}\right)  _{i\geq0,\ 0\leq j<r^{i},\ k\geq0}$ as a
$\mathbb{K}$-module. Hence, in order to prove that $\beta\circ\alpha
=\operatorname*{id}$, it suffices to show that $\left(  \beta\circ
\alpha\right)  \left(  g_{i,j}T^{k}\right)  =g_{i,j}T^{k}$ for every $i\geq0$,
$0\leq j<r^{i}$ and $k\geq0$.
\par
So let us fix $i\geq0$, $0\leq j<r^{i}$ and $k\geq0$. The definition of
$\alpha$ yields $\alpha\left(  g_{i,j}\right)  =T^{j}F^{i}$. But since
$\alpha$ is a right $\mathbb{K}\left[  T\right]  $-module homomorphism, we
have%
\[
\alpha\left(  g_{i,j}T^{k}\right)  =\underbrace{\alpha\left(  g_{i,j}\right)
}_{=T^{j}F^{i}}T^{k}=T^{j}\underbrace{F^{i}T^{k}}_{\substack{=T^{r^{i}k}%
F^{i}\\\text{(by Proposition \ref{prop.F-gen.bases} \textbf{(a)}%
,}\\\text{applied to }a=i\text{ and }b=k\text{)}}}=\underbrace{T^{j}T^{r^{i}%
k}}_{=T^{j+r^{i}k}}F^{i}=T^{j+r^{i}k}F^{i}.
\]
Now,%
\begin{equation}
\left(  \beta\circ\alpha\right)  \left(  g_{i,j}T^{k}\right)  =\beta\left(
\underbrace{\alpha\left(  g_{i,j}T^{k}\right)  }_{=T^{j+r^{i}k}F^{i}}\right)
=\beta\left(  T^{j+r^{i}k}F^{i}\right)  =g_{i,\left(  j+r^{i}k\right)
\%r^{i}}T^{\left(  j+r^{i}k\right)  //r^{i}}.
\label{pf.prop.F-gen.bases.d.fn1.3}%
\end{equation}
\par
But $0\leq j<r^{i}$. Hence, $\left(  j+r^{i}k\right)  \%r^{i}=j$ and $\left(
j+r^{i}k\right)  //r^{i}=k$. In view of these two equalities,
(\ref{pf.prop.F-gen.bases.d.fn1.3}) rewrites as $\left(  \beta\circ
\alpha\right)  \left(  g_{i,j}T^{k}\right)  =g_{i,j}T^{k}$. This completes our
proof of $\beta\circ\alpha=\operatorname*{id}$. Thus, we have shown that
$\alpha$ and $\beta$ are mutually inverse.}. Thus, $\alpha$ is a right
$\mathbb{K}\left[  T\right]  $-module isomorphism. Since the right
$\mathbb{K}\left[  T\right]  $-module $\mathcal{G}$ has a basis $\left(
g_{i,j}\right)  _{i\geq0,\ 0\leq j<r^{i}}$, this shows that the right
$\mathbb{K}\left[  T\right]  $-module $\mathcal{M}$ has a basis $\left(
\underbrace{\alpha\left(  g_{i,j}\right)  }_{=T^{j}F^{i}}\right)
_{i\geq0,\ 0\leq j<r^{i}}=\left(  T^{j}F^{i}\right)  _{i\geq0,\ 0\leq j<r^{i}%
}$. This proves Proposition \ref{prop.F-gen.bases} \textbf{(d)}.

\textbf{(e)} Let $\left(  e_{0},e_{1},e_{2},\ldots\right)  $ be the standard
basis of the right $\mathbb{K}\left[  F\right]  $-module $\mathbb{K}\left[
F\right]  ^{\left(  \mathbb{N}\right)  }$. Define a right $\mathbb{K}\left[
F\right]  $-module homomorphism $\alpha:\mathbb{K}\left[  F\right]  ^{\left(
\mathbb{N}\right)  }\rightarrow\mathcal{M}$ by sending each $e_{j}$ to $T^{j}%
$. Define a $\mathbb{K}$-module homomorphism $\beta:\mathcal{M}\rightarrow
\mathbb{K}\left[  T\right]  ^{\left(  \mathbb{N}\right)  }$ by sending each
$T^{j}F^{i}$ to $e_{j}F^{i}$. (This $\beta$ is well-defined, since Proposition
\ref{prop.F-gen.bases} \textbf{(b)} shows that $\left(  T^{j}F^{i}\right)
_{i\geq0,\ j\geq0}$ is a basis of the $\mathbb{K}$-module $\mathcal{M}$.) It
is easy to see that $\beta$ is a right $\mathbb{K}\left[  F\right]  $-module
homomorphism. It is straightforward to see that the homomorphisms $\alpha$ and
$\beta$ are mutually inverse. Thus, $\alpha$ is a right $\mathbb{K}\left[
F\right]  $-module isomorphism. As a consequence, the right $\mathbb{K}\left[
F\right]  $-module $\mathcal{M}$ has a basis $\left(  \underbrace{\alpha
\left(  e_{j}\right)  }_{=T^{j}}\right)  _{j\geq0}=\left(  T^{j}\right)
_{j\geq0}$. This proves Proposition \ref{prop.F-gen.bases} \textbf{(e)}.

\textbf{(f)} Define a subset $B$ of $\mathbb{N}^{2}$ by
(\ref{eq.cor.F-gen.bases.f.1.cor1.def-B}). Define a map $\rho:B\times
\mathbb{N}\rightarrow\mathbb{N}^{2}$ by
(\ref{eq.cor.F-gen.bases.f.1.cor1.def-rho}). Corollary
\ref{cor.F-gen.bases.f.1.cor1} shows that $\rho$ is a bijection. Hence, its
inverse $\rho^{-1}:\mathbb{N}^{2}\rightarrow B\times\mathbb{N}$ is well-defined.

Now, let $\mathcal{H}$ be the free left $\mathbb{K}\left[  F\right]  $-module
with basis $\left(  h_{\left(  i,j\right)  }\right)  _{\left(  i,j\right)  \in
B}$. Define a left $\mathbb{K}\left[  F\right]  $-module homomorphism
$\alpha:\mathcal{H}\rightarrow\mathcal{M}$ by sending each $h_{\left(
i,j\right)  }$ to $T^{j}F^{i}$. Define a $\mathbb{K}$-module homomorphism
$\beta:\mathcal{M}\rightarrow\mathcal{H}$ by sending each $T^{j}F^{i}$ to
$F^{k}h_{\left(  u,v\right)  }$, where $\left(  \left(  u,v\right)  ,k\right)
=\rho^{-1}\left(  i,j\right)  $. (This $\beta$ is well-defined, since
Proposition \ref{prop.F-gen.bases} \textbf{(b)} shows that $\left(  T^{j}%
F^{i}\right)  _{i\geq0,\ j\geq0}$ is a basis of the $\mathbb{K}$-module
$\mathcal{M}$.) It is straightforward to see that the homomorphisms $\alpha$
and $\beta$ are mutually inverse\footnote{\textit{Proof.} We need to show that
$\alpha\circ\beta=\operatorname*{id}$ and $\beta\circ\alpha=\operatorname*{id}%
$.
\par
To prove that $\alpha\circ\beta=\operatorname*{id}$, we need to show that
$\left(  \alpha\circ\beta\right)  \left(  T^{j}F^{i}\right)  =T^{j}F^{i}$ for
every $i,j\in\mathbb{N}$. So let us fix $i,j\in\mathbb{N}$. Set $\left(
\left(  u,v\right)  ,k\right)  =\rho^{-1}\left(  i,j\right)  $. Then, $\left(
i,j\right)  =\rho\left(  \left(  u,v\right)  ,k\right)  =\left(
u+k,r^{k}v\right)  $ (by the definition of $\rho$). In other words, $i=u+k$
and $j=r^{k}v$.
\par
The definition of $\beta$ shows that $\beta\left(  T^{j}F^{i}\right)
=F^{k}h_{\left(  u,v\right)  }$. Now,%
\begin{align*}
\left(  \alpha\circ\beta\right)  \left(  T^{j}F^{i}\right)   &  =\alpha\left(
\underbrace{\beta\left(  T^{j}F^{i}\right)  }_{=F^{k}h_{\left(  u,v\right)  }%
}\right)  =\alpha\left(  F^{k}h_{\left(  u,v\right)  }\right)  =F^{k}%
\underbrace{\alpha\left(  h_{\left(  u,v\right)  }\right)  }_{\substack{=T^{v}%
F^{u}\\\text{(by the definition of }\alpha\text{)}}}\\
&  \ \ \ \ \ \ \ \ \ \ \left(  \text{since }\alpha\text{ is a left }%
\mathbb{K}\left[  F\right]  \text{-module homomorphism}\right) \\
&  =\underbrace{F^{k}T^{v}}_{\substack{=T^{r^{k}v}F^{k}\\\text{(by Proposition
\ref{prop.F-gen.bases} \textbf{(a)},}\\\text{applied to }a=k\text{ and
}b=v\text{)}}}F^{u}=\underbrace{T^{r^{k}v}}_{\substack{=T^{j}\\\text{(since
}r^{k}v=j\text{)}}}\underbrace{F^{k}F^{u}}_{\substack{=F^{u+k}=F^{i}%
\\\text{(since }u+k=i\text{)}}}=T^{j}F^{i},
\end{align*}
which is what we wanted to prove.
\par
Thus, $\alpha\circ\beta=\operatorname*{id}$ is proven. It thus remains to
prove that $\beta\circ\alpha=\operatorname*{id}$.
\par
We know that $\mathcal{H}$ is spanned by $\left(  h_{\left(  i,j\right)
}\right)  _{\left(  i,j\right)  \in B}$ as a left $\mathbb{K}\left[  F\right]
$-module (by the definition of $\mathcal{H}$). Hence, $\mathcal{H}$ is spanned
by $\left(  F^{k}h_{\left(  i,j\right)  }\right)  _{\left(  \left(
i,j\right)  ,k\right)  \in B\times\mathbb{N}}$ as a $\mathbb{K}$-module.
Hence, in order to prove that $\beta\circ\alpha=\operatorname*{id}$, it
suffices to show that $\left(  \beta\circ\alpha\right)  \left(  F^{k}%
h_{\left(  i,j\right)  }\right)  =F^{k}h_{\left(  i,j\right)  }$ for every
$\left(  \left(  i,j\right)  ,k\right)  \in B\times\mathbb{N}$.
\par
So let us fix $\left(  \left(  i,j\right)  ,k\right)  \in B\times\mathbb{N}$.
The definition of $\alpha$ yields $\alpha\left(  h_{\left(  i,j\right)
}\right)  =T^{j}F^{i}$. But since $\alpha$ is a left $\mathbb{K}\left[
F\right]  $-module homomorphism, we have%
\[
\alpha\left(  F^{k}h_{\left(  i,j\right)  }\right)  =F^{k}\underbrace{\alpha
\left(  h_{\left(  i,j\right)  }\right)  }_{=T^{j}F^{i}}=\underbrace{F^{k}%
T^{j}}_{\substack{=T^{r^{k}j}F^{k}\\\text{(by Proposition
\ref{prop.F-gen.bases} \textbf{(a)},}\\\text{applied to }a=k\text{ and
}b=j\text{)}}}F^{i}=T^{r^{k}j}\underbrace{F^{k}F^{i}}_{=F^{k+i}}=T^{r^{k}%
j}F^{k+i}.
\]
\par
On the other hand, the definition of $\rho$ yields $\rho\left(  \left(
i,j\right)  ,k\right)  =\left(  \underbrace{i+k}_{=k+i},r^{k}j\right)
=\left(  k+i,r^{k}j\right)  $, so that $\left(  \left(  i,j\right)  ,k\right)
=\rho^{-1}\left(  k+i,r^{k}j\right)  $. Hence, the definition of $\beta$
yields $\beta\left(  T^{r^{k}j}F^{k+i}\right)  =F^{k}h_{\left(  i,j\right)  }%
$. Now,%
\[
\left(  \beta\circ\alpha\right)  \left(  F^{k}h_{\left(  i,j\right)  }\right)
=\beta\left(  \underbrace{\alpha\left(  F^{k}h_{\left(  i,j\right)  }\right)
}_{=T^{r^{k}j}F^{k+i}}\right)  =\beta\left(  T^{r^{k}j}F^{k+i}\right)
=F^{k}h_{\left(  i,j\right)  }.
\]
This completes our proof of $\beta\circ\alpha=\operatorname*{id}$. Thus, we
have shown that $\alpha$ and $\beta$ are mutually inverse.}. Thus, $\alpha$ is
a left $\mathbb{K}\left[  F\right]  $-module isomorphism. As a consequence,
the left $\mathbb{K}\left[  F\right]  $-module $\mathcal{M}$ has a basis
\[
\left(  \underbrace{\alpha\left(  h_{\left(  i,j\right)  }\right)  }%
_{=T^{j}F^{i}}\right)  _{\left(  i,j\right)  \in B}=\left(  T^{j}F^{i}\right)
_{\left(  i,j\right)  \in B}=\left(  T^{j}F^{i}\right)  _{i=0\text{ or }r\nmid
j}%
\]
(since $B=\left\{  \left(  i,j\right)  \in\mathbb{N}^{2}\ \mid\ i=0\text{ or
}r\nmid j\right\}  $). This proves Proposition \ref{prop.F-gen.bases}
\textbf{(f)}.

\textbf{(g)} Define $C$ and $\zeta$ as in Corollary
\ref{cor.F-gen.bases.f.1.cor2}. In this proof, the $\otimes$ sign always shall
mean tensor products over $\mathbb{K}$.

Corollary \ref{cor.F-gen.bases.f.1.cor2} shows that the map $\zeta$ is a
bijection. In other words, the map
\begin{equation}
C\times\mathbb{N}\times\mathbb{N}\rightarrow\mathbb{N}^{2}%
,\ \ \ \ \ \ \ \ \ \ \left(  \left(  i,j\right)  ,\ell,k\right)
\mapsto\left(  i+k,r^{k}\left(  j+r^{i}\ell\right)  \right)
\label{pf.prop.F-gen.bases.g.bij}%
\end{equation}
is a bijection (since this map is the map $\zeta$).

Proposition \ref{prop.F-gen.bases} \textbf{(b)} shows that $\left(  T^{j}%
F^{i}\right)  _{i\geq0,\ j\geq0}$ is a basis of the $\mathbb{K}$-module
$\mathcal{M}$. We can reindex this basis using the bijection
(\ref{pf.prop.F-gen.bases.g.bij}); thus, we conclude that \newline$\left(
T^{r^{k}\left(  j+r^{i}\ell\right)  }F^{i+k}\right)  _{\left(  \left(
i,j\right)  ,\ell,k\right)  \in C\times\mathbb{N}\times\mathbb{N}}$ is a basis
of the $\mathbb{K}$-module $\mathcal{M}$.

Let $\mathcal{R}$ be the free $\mathbb{K}$-module with basis $\left(
r_{\left(  i,j\right)  }\right)  _{\left(  i,j\right)  \in C}$. Then,
\newline$\left(  r_{\left(  i,j\right)  }\otimes F^{k}\otimes T^{\ell}\right)
_{\left(  \left(  i,j\right)  ,\ell,k\right)  \in C\times\mathbb{N}%
\times\mathbb{N}}$ is a basis of the $\mathbb{K}$-module $\mathcal{R}%
\otimes\mathbb{K}\left[  F\right]  \otimes\mathbb{K}\left[  T\right]  $ (since
$\left(  F^{k}\right)  _{k\in\mathbb{N}}$ is a basis of $\mathbb{K}\left[
F\right]  $, and since $\left(  T^{\ell}\right)  _{\ell\in\mathbb{N}}$ is a
basis of $\mathbb{K}\left[  T\right]  $). Hence, we can define a $\mathbb{K}%
$-linear map $\eta:\mathcal{R}\otimes\mathbb{K}\left[  F\right]
\otimes\mathbb{K}\left[  T\right]  \rightarrow\mathcal{M}$ by%
\[
\eta\left(  r_{\left(  i,j\right)  }\otimes F^{k}\otimes T^{\ell}\right)
=T^{r^{k}\left(  j+r^{i}\ell\right)  }F^{i+k}.
\]
Consider this map $\eta$. It sends the basis $\left(  r_{\left(  i,j\right)
}\otimes F^{k}\otimes T^{\ell}\right)  _{\left(  \left(  i,j\right)
,\ell,k\right)  \in C\times\mathbb{N}\times\mathbb{N}}$ of $\mathcal{R}%
\otimes\mathbb{K}\left[  F\right]  \otimes\mathbb{K}\left[  T\right]  $ to the
basis $\left(  T^{r^{k}\left(  j+r^{i}\ell\right)  }F^{i+k}\right)  _{\left(
\left(  i,j\right)  ,\ell,k\right)  \in C\times\mathbb{N}\times\mathbb{N}}$ of
$\mathcal{M}$. Thus, $\eta$ is an isomorphism of $\mathbb{K}$-modules.

Now, $\mathcal{R}\otimes\mathbb{K}\left[  F\right]  \otimes\mathbb{K}\left[
T\right]  $ becomes a left $\mathbb{K}\left[  F\right]  $-module (by having
$\mathbb{K}\left[  F\right]  $ act on the tensorand $\mathbb{K}\left[
F\right]  $) and a right $\mathbb{K}\left[  T\right]  $-module (by having
$\mathbb{K}\left[  T\right]  $ act on the tensorand $\mathbb{K}\left[
T\right]  $). The map $\eta$ is a left $\mathbb{K}\left[  F\right]  $-module
homomorphism\footnote{\textit{Proof.} It suffices to show that $\eta\left(
fz\right)  =f\eta\left(  z\right)  $ for every $f\in\mathbb{K}\left[
F\right]  $ and $z\in\mathcal{R}\otimes\mathbb{K}\left[  F\right]
\otimes\mathbb{K}\left[  T\right]  $. So let us prove this.
\par
Fix $f\in\mathbb{K}\left[  F\right]  $ and $z\in\mathcal{R}\otimes
\mathbb{K}\left[  F\right]  \otimes\mathbb{K}\left[  T\right]  $. We need to
show the equality $\eta\left(  fz\right)  =f\eta\left(  z\right)  $. Since
this equality is $\mathbb{K}$-linear in each of $f$ and $z$, we can WLOG
assume that $f$ belongs to the basis $\left(  F^{k}\right)  _{k\in\mathbb{N}}$
of $\mathbb{K}\left[  F\right]  $, and that $z$ belongs to the basis $\left(
r_{\left(  i,j\right)  }\otimes F^{k}\otimes T^{\ell}\right)  _{\left(
\left(  i,j\right)  ,\ell,k\right)  \in C\times\mathbb{N}\times\mathbb{N}}$ of
$\mathcal{R}\otimes\mathbb{K}\left[  F\right]  \otimes\mathbb{K}\left[
T\right]  $. Assume this. Thus, $f=F^{p}$ for some $p\in\mathbb{N}$, and
$z=r_{\left(  i,j\right)  }\otimes F^{k}\otimes T^{\ell}$ for some $\left(
\left(  i,j\right)  ,\ell,k\right)  \in C\times\mathbb{N}\times\mathbb{N}$.
Consider these $p$ and $\left(  \left(  i,j\right)  ,\ell,k\right)  $.
\par
From $f=F^{p}$ and $z=r_{\left(  i,j\right)  }\otimes F^{k}\otimes T^{\ell}$,
we obtain $fz=F^{p}\left(  r_{\left(  i,j\right)  }\otimes F^{k}\otimes
T^{\ell}\right)  =r_{\left(  i,j\right)  }\otimes\underbrace{F^{p}F^{k}%
}_{=F^{p+k}}\otimes T^{\ell}=r_{\left(  i,j\right)  }\otimes F^{p+k}\otimes
T^{\ell}$. Hence,%
\[
\eta\left(  fz\right)  =\eta\left(  r_{\left(  i,j\right)  }\otimes
F^{p+k}\otimes T^{\ell}\right)  =T^{r^{p+k}\left(  j+r^{i}\ell\right)
}F^{i+p+k}%
\]
(by the definition of $\eta$). On the other hand, from $z=r_{\left(
i,j\right)  }\otimes F^{k}\otimes T^{\ell}$, we obtain $\eta\left(  z\right)
=\eta\left(  r_{\left(  i,j\right)  }\otimes F^{k}\otimes T^{\ell}\right)
=T^{r^{k}\left(  j+r^{i}\ell\right)  }F^{i+k}$, so that%
\begin{align*}
\underbrace{f}_{=F^{p}}\underbrace{\eta\left(  z\right)  }_{=T^{r^{k}\left(
j+r^{i}\ell\right)  }F^{i+k}}  &  =\underbrace{F^{p}T^{r^{k}\left(
j+r^{i}\ell\right)  }}_{\substack{=T^{r^{p}r^{k}\left(  j+r^{i}\ell\right)
}F^{p}\\\text{(by Proposition \ref{prop.F-gen.bases} \textbf{(a)}%
,}\\\text{applied to }a=p\text{ and }b=r^{k}\left(  j+r^{i}\ell\right)
\text{)}}}F^{i+k}=\underbrace{T^{r^{p}r^{k}\left(  j+r^{i}\ell\right)  }%
}_{=T^{r^{p+k}\left(  j+r^{i}\ell\right)  }}\underbrace{F^{p}F^{i+k}%
}_{=F^{p+i+k}=F^{i+p+k}}\\
&  =T^{r^{p+k}\left(  j+r^{i}\ell\right)  }F^{i+p+k}.
\end{align*}
Comparing this with $\eta\left(  fz\right)  =T^{r^{p+k}\left(  j+r^{i}%
\ell\right)  }F^{i+p+k}$, we obtain $\eta\left(  fz\right)  =f\eta\left(
z\right)  $, qed.} and a right $\mathbb{K}\left[  T\right]  $-module
homomorphism\footnote{\textit{Proof.} It suffices to show that $\eta\left(
zt\right)  =\eta\left(  z\right)  t$ for every $t\in\mathbb{K}\left[
T\right]  $ and $z\in\mathcal{R}\otimes\mathbb{K}\left[  F\right]
\otimes\mathbb{K}\left[  T\right]  $. So let us prove this.
\par
Fix $t\in\mathbb{K}\left[  T\right]  $ and $z\in\mathcal{R}\otimes
\mathbb{K}\left[  F\right]  \otimes\mathbb{K}\left[  T\right]  $. We need to
show the equality $\eta\left(  zt\right)  =\eta\left(  z\right)  t$. Since
this equality is $\mathbb{K}$-linear in each of $t$ and $z$, we can WLOG
assume that $t$ belongs to the basis $\left(  T^{\ell}\right)  _{\ell
\in\mathbb{N}}$ of $\mathbb{K}\left[  T\right]  $, and that $z$ belongs to the
basis $\left(  r_{\left(  i,j\right)  }\otimes F^{k}\otimes T^{\ell}\right)
_{\left(  \left(  i,j\right)  ,\ell,k\right)  \in C\times\mathbb{N}%
\times\mathbb{N}}$ of $\mathcal{R}\otimes\mathbb{K}\left[  F\right]
\otimes\mathbb{K}\left[  T\right]  $. Assume this. Thus, $t=T^{p}$ for some
$p\in\mathbb{N}$, and $z=r_{\left(  i,j\right)  }\otimes F^{k}\otimes T^{\ell
}$ for some $\left(  \left(  i,j\right)  ,\ell,k\right)  \in C\times
\mathbb{N}\times\mathbb{N}$. Consider these $p$ and $\left(  \left(
i,j\right)  ,\ell,k\right)  $.
\par
From $t=T^{p}$ and $z=r_{\left(  i,j\right)  }\otimes F^{k}\otimes T^{\ell}$,
we obtain $zt=\left(  r_{\left(  i,j\right)  }\otimes F^{k}\otimes T^{\ell
}\right)  T^{p}=r_{\left(  i,j\right)  }\otimes F^{k}\otimes
\underbrace{T^{\ell}T^{p}}_{=T^{\ell+p}}=r_{\left(  i,j\right)  }\otimes
F^{k}\otimes T^{\ell+p}$. Hence,%
\[
\eta\left(  zt\right)  =\eta\left(  r_{\left(  i,j\right)  }\otimes
F^{k}\otimes T^{\ell+p}\right)  =T^{r^{k}\left(  j+r^{i}\left(  \ell+p\right)
\right)  }F^{i+k}%
\]
(by the definition of $\eta$). On the other hand, from $z=r_{\left(
i,j\right)  }\otimes F^{k}\otimes T^{\ell}$, we obtain $\eta\left(  z\right)
=\eta\left(  r_{\left(  i,j\right)  }\otimes F^{k}\otimes T^{\ell}\right)
=T^{r^{k}\left(  j+r^{i}\ell\right)  }F^{i+k}$, so that%
\begin{align*}
\underbrace{\eta\left(  z\right)  }_{=T^{r^{k}\left(  j+r^{i}\ell\right)
}F^{i+k}}\underbrace{t}_{=T^{p}}  &  =T^{r^{k}\left(  j+r^{i}\ell\right)
}\underbrace{F^{i+k}T^{p}}_{\substack{=T^{r^{i+k}p}F^{i+k}\\\text{(by
Proposition \ref{prop.F-gen.bases} \textbf{(a)},}\\\text{applied to
}a=i+k\text{ and }b=p\text{)}}}=\underbrace{T^{r^{k}\left(  j+r^{i}%
\ell\right)  }T^{r^{i+k}p}}_{\substack{=T^{r^{k}\left(  j+r^{i}\ell\right)
+r^{i+k}p}\\=T^{r^{k}\left(  j+r^{i}\left(  \ell+p\right)  \right)
}\\\text{(since}\\r^{k}\left(  j+r^{i}\ell\right)  +r^{i+k}p=r^{k}\left(
j+r^{i}\left(  \ell+p\right)  \right)  \text{)}}}F^{i+k}\\
&  =T^{r^{k}\left(  j+r^{i}\left(  \ell+p\right)  \right)  }F^{i+k}.
\end{align*}
Comparing this with $\eta\left(  zt\right)  =T^{r^{k}\left(  j+r^{i}\left(
\ell+p\right)  \right)  }F^{i+k}$, we obtain $\eta\left(  zt\right)
=\eta\left(  z\right)  t$, qed.}. Thus, $\eta$ is a $\mathbb{K}\left[
F\right]  $-$\mathbb{K}\left[  T\right]  $-bimodule homomorphism.

Now, recall that $\left(  r_{\left(  i,j\right)  }\right)  _{\left(
i,j\right)  \in C}$ is a basis of the free $\mathbb{K}$-module $\mathcal{R}$.
Hence, $\mathcal{R}=\bigoplus_{\left(  i,j\right)  \in C}r_{\left(
i,j\right)  }\mathbb{K}$. Since direct sums commute with tensor products, this
yields%
\begin{align*}
\mathcal{R}\otimes\mathbb{K}\left[  F\right]  \otimes\mathbb{K}\left[
T\right]   &  =\bigoplus_{\left(  i,j\right)  \in C}\underbrace{r_{\left(
i,j\right)  }\mathbb{K}\otimes\mathbb{K}\left[  F\right]  \otimes
\mathbb{K}\left[  T\right]  }_{\substack{=\mathbb{K}\left[  F\right]
\cdot\left(  r_{\left(  i,j\right)  }\otimes F^{0}\otimes T^{0}\right)
\cdot\mathbb{K}\left[  T\right]  \\\text{(this follows easily from the
definition of the}\\\mathbb{K}\left[  F\right]  \text{-}\mathbb{K}\left[
T\right]  \text{-bimodule structure on }\mathcal{R}\otimes\mathbb{K}\left[
F\right]  \otimes\mathbb{K}\left[  T\right]  \text{)}}}\\
&  =\bigoplus_{\left(  i,j\right)  \in C}\mathbb{K}\left[  F\right]
\cdot\left(  r_{\left(  i,j\right)  }\otimes F^{0}\otimes T^{0}\right)
\cdot\mathbb{K}\left[  T\right]  .
\end{align*}
We can apply the map $\eta$ to this equality. The left hand side becomes
$\mathcal{M}$ (since $\eta$ is an isomorphism of $\mathbb{K}$-modules), and
the direct sum on the right hand side remains direct (for the same reason).
Hence, we obtain%
\begin{align*}
\mathcal{M}  &  =\bigoplus_{\left(  i,j\right)  \in C}\underbrace{\eta\left(
\mathbb{K}\left[  F\right]  \cdot\left(  r_{\left(  i,j\right)  }\otimes
F^{0}\otimes T^{0}\right)  \cdot\mathbb{K}\left[  T\right]  \right)
}_{\substack{=\mathbb{K}\left[  F\right]  \cdot\eta\left(  r_{\left(
i,j\right)  }\otimes F^{0}\otimes T^{0}\right)  \cdot\mathbb{K}\left[
T\right]  \\\text{(since }\eta\text{ is a }\mathbb{K}\left[  F\right]
\text{-}\mathbb{K}\left[  T\right]  \text{-bimodule homomorphism)}}}\\
&  =\bigoplus_{\left(  i,j\right)  \in C}\mathbb{K}\left[  F\right]
\cdot\underbrace{\eta\left(  r_{\left(  i,j\right)  }\otimes F^{0}\otimes
T^{0}\right)  }_{\substack{=T^{r^{0}\left(  j+r^{i}0\right)  }F^{i+0}%
\\\text{(by the definition of }\eta\text{)}}}\cdot\mathbb{K}\left[  T\right]
\\
&  =\underbrace{\bigoplus_{\left(  i,j\right)  \in C}}_{=\bigoplus
\limits_{\substack{\left(  i,j\right)  \in\mathbb{N}^{2};\\\left(  i=0\text{
or }r\nmid j\right)  \text{ and }0\leq j<r^{i}}}}\mathbb{K}\left[  F\right]
\cdot\underbrace{T^{r^{0}\left(  j+r^{i}0\right)  }}_{=T^{j}}%
\underbrace{F^{i+0}}_{=F^{i}}\cdot\mathbb{K}\left[  T\right] \\
&  =\bigoplus\limits_{\substack{\left(  i,j\right)  \in\mathbb{N}%
^{2};\\\left(  i=0\text{ or }r\nmid j\right)  \text{ and }0\leq j<r^{i}%
}}\mathbb{K}\left[  F\right]  \cdot\left(  T^{j}F^{i}\right)  \cdot
\mathbb{K}\left[  T\right]  .
\end{align*}
It remains to show that each $\mathbb{K}\left[  F\right]  \cdot\left(
T^{j}F^{i}\right)  \cdot\mathbb{K}\left[  T\right]  $ is isomorphic to
$\mathbb{K}\left[  F\right]  \otimes\mathbb{K}\left[  T\right]  $ as an
$\mathbb{K}\left[  F\right]  $-$\mathbb{K}\left[  T\right]  $-bimodule. This
follows from $\eta$ being an isomorphism (the details are left to the reader).
Thus, Proposition \ref{prop.F-gen.bases} \textbf{(g)} is proven.
\end{proof}

\subsection{The skew polynomial ring $\mathcal{F}$}

Now, let us return to the setup of polynomials over $\mathbb{F}_{q}$.

We are still using the notations of Section \ref{sect.nots}. In particular,
$q$ is a (nontrivial) power of a prime $p$.

For every commutative $\mathbb{F}_{q}$-algebra $A$, we let
$\operatorname*{Frob}\nolimits_{A}:A\rightarrow A$ be the map which sends
every $a\in A$ to $a^{q}$. This map $\operatorname*{Frob}\nolimits_{A}$ is
called the \textit{Frobenius endomorphism} of $A$. It is well-known that
$\operatorname*{Frob}\nolimits_{A}$ is an $\mathbb{F}_{q}$-algebra
homomorphism\footnote{This follows from the fact that $\left(  \lambda
a\right)  ^{q}=\underbrace{\lambda^{q}}_{\substack{=\lambda\\\text{(since
}\lambda\in\mathbb{F}_{q}\text{)}}}a^{q}=\lambda a^{q}$ for every $a\in A$ and
$\lambda\in\mathbb{F}_{q}$, and the fact that $\left(  a+b\right)  ^{q}%
=a^{q}+b^{q}$ for every $a,b\in A$.}. We will often denote the $\mathbb{F}%
_{q}$-algebra homomorphism $\operatorname*{Frob}\nolimits_{A}$ by
$\operatorname*{Frob}$ when no confusion can arise from the omission of $A$. A
rather important particular case is the endomorphism $\operatorname*{Frob}%
=\operatorname*{Frob}\nolimits_{\mathbb{F}_{q}\left[  T\right]  }$ of the
commutative $\mathbb{F}_{q}$-algebra $\mathbb{F}_{q}\left[  T\right]  $.

We let $\mathcal{F}$ be the $\mathbb{F}_{q}$-algebra $\mathbb{F}%
_{q}\left\langle F,T\ \mid\ FT=T^{q}F\right\rangle $. We can immediately
define the following $\mathbb{F}_{q}$-algebra homomorphisms (whose
well-definedness is easy to check using the universal properties of their domains):

\begin{itemize}
\item We define an $\mathbb{F}_{q}$-algebra homomorphism $\operatorname*{Finc}%
\nolimits_{F}:\mathbb{F}_{q}\left[  F\right]  \rightarrow\mathcal{F}$ by
$\operatorname*{Finc}\nolimits_{F}\left(  F\right)  =F$. Thus,
$\operatorname*{Finc}\nolimits_{F}\left(  p\right)  =p\left(  F\right)  $ for
every $p\in\mathbb{F}_{q}\left[  F\right]  $ (where $p\left(  F\right)  $
means the result of substituting $F$ into the polynomial $p$).

\item We define an $\mathbb{F}_{q}$-algebra homomorphism $\operatorname*{Finc}%
\nolimits_{T}:\mathbb{F}_{q}\left[  T\right]  \rightarrow\mathcal{F}$ by
$\operatorname*{Finc}\nolimits_{T}\left(  T\right)  =T$. Thus,
$\operatorname*{Finc}\nolimits_{T}\left(  p\right)  =p\left(  T\right)  $ for
every $p\in\mathbb{F}_{q}\left[  T\right]  $ (where $p\left(  T\right)  $
means the result of substituting $T$ into the polynomial $p$).

\item We define an $\mathbb{F}_{q}$-algebra homomorphism $\operatorname*{Carl}%
:\mathbb{F}_{q}\left[  T\right]  \rightarrow\mathcal{F}$ by
$\operatorname*{Carl}\left(  T\right)  =F+T$. Thus, $\operatorname*{Carl}%
\left(  p\right)  =p\left(  F+T\right)  $ for every $p\in\mathbb{F}_{q}\left[
T\right]  $ (where $p\left(  F+T\right)  $ means the result of substituting
$F+T$ into the polynomial $p$).
\end{itemize}

Furthermore, recall that $\mathcal{F}$ is the $\mathbb{F}_{q}$-algebra
$\mathbb{F}_{q}\left\langle F,T\ \mid\ FT=T^{q}F\right\rangle $. Thus,
$\mathcal{F}$ has the following universal property: If $u$ and $v$ are two
elements of an $\mathbb{F}_{q}$-algebra $\mathcal{U}$ satisfying $uv=v^{q}u$,
then there exists a unique $\mathbb{F}_{q}$-algebra homomorphism
$\mathcal{F}\rightarrow\mathcal{U}$ sending $F$ and $T$ to $u$ and $v$,
respectively. This allows us to define $\mathbb{F}_{q}$-algebra homomorphisms
out of $\mathcal{F}$, such as the following:

\begin{itemize}
\item We define an $\mathbb{F}_{q}$-algebra homomorphism $\operatorname*{Fpro}%
\nolimits_{F}:\mathcal{F}\rightarrow\mathbb{F}_{q}\left[  F\right]  $ by
$\operatorname*{Fpro}\nolimits_{F}\left(  F\right)  =F$ and
$\operatorname*{Fpro}\nolimits_{F}\left(  T\right)  =0$. It is easy to see
that $\operatorname*{Fpro}\nolimits_{F}\circ\operatorname*{Finc}%
\nolimits_{F}=\operatorname*{id}$. Hence, the $\mathbb{F}_{q}$-algebra
homomorphism $\operatorname*{Finc}\nolimits_{F}$ is injective. Thus, we shall
regard $\operatorname*{Finc}\nolimits_{F}$ as an inclusion, so that
$\mathbb{F}_{q}\left[  F\right]  \subseteq\mathcal{F}$. (Notice that this does
not make $\mathcal{F}$ into an $\mathbb{F}_{q}\left[  F\right]  $-algebra,
since $\mathbb{F}_{q}\left[  F\right]  $ is not contained in the center of
$\mathcal{F}$.)

\item We define an $\mathbb{F}_{q}$-algebra homomorphism $\operatorname*{Fpro}%
\nolimits_{T}:\mathcal{F}\rightarrow\mathbb{F}_{q}\left[  T\right]  $ by
$\operatorname*{Fpro}\nolimits_{T}\left(  F\right)  =0$ and
$\operatorname*{Fpro}\nolimits_{T}\left(  T\right)  =T$. It is easy to see
that $\operatorname*{Fpro}\nolimits_{T}\circ\operatorname*{Finc}%
\nolimits_{T}=\operatorname*{id}$. Hence, the $\mathbb{F}_{q}$-algebra
homomorphism $\operatorname*{Finc}\nolimits_{T}$ is injective. Thus, we shall
regard $\operatorname*{Finc}\nolimits_{T}$ as an inclusion, so that
$\mathbb{F}_{q}\left[  T\right]  \subseteq\mathcal{F}$. (Notice that this does
not make $\mathcal{F}$ into an $\mathbb{F}_{q}\left[  T\right]  $-algebra,
since $\mathbb{F}_{q}\left[  T\right]  $ is not contained in the center of
$\mathcal{F}$.)

\item For every $a\in\mathbb{F}_{q}$ and $b\in\mathbb{F}_{q}$, we define an
$\mathbb{F}_{q}$-algebra homomorphism $\operatorname*{Fscal}\nolimits_{a,b}%
:\mathcal{F}\rightarrow\mathcal{F}$ by $\operatorname*{Fscal}\nolimits_{a,b}%
\left(  F\right)  =aF$ and $\operatorname*{Fscal}\nolimits_{a,b}\left(
T\right)  =bT$. (This is well-defined, since $\left(  aF\right)  \left(
bT\right)  =\left(  bT\right)  ^{q}\left(  aF\right)  $.) If $a$ and $b$ are
nonzero, then $\operatorname*{Fscal}\nolimits_{a,b}$ is invertible (with
inverse $\operatorname*{Fscal}\nolimits_{a^{-1},b^{-1}}$).
\end{itemize}

Now, we shall derive some structural properties of $\mathcal{F}$ straight from
Proposition \ref{prop.F-gen.bases}:

\begin{proposition}
\label{prop.F.bases}The homomorphisms $\operatorname*{Finc}\nolimits_{T}$ and
$\operatorname*{Finc}\nolimits_{F}$ make $\mathcal{F}$ into a left
$\mathbb{F}_{q}\left[  T\right]  $-module, a right $\mathbb{F}_{q}\left[
T\right]  $-module, a left $\mathbb{F}_{q}\left[  F\right]  $-module, and a
right $\mathbb{F}_{q}\left[  F\right]  $-module. Any of these two left module
structures can be combined with any of these two right module structures to
form a bimodule structure on $\mathcal{F}$ (for example, the left
$\mathbb{F}_{q}\left[  T\right]  $-module structure and the right
$\mathbb{F}_{q}\left[  F\right]  $-module structure on $\mathcal{F}$ can be
combined to form an $\mathbb{F}_{q}\left[  T\right]  $-$\mathbb{F}_{q}\left[
F\right]  $-bimodule structure on $\mathcal{F}$).

\textbf{(a)} We have $F^{a}T^{b}=T^{q^{a}b}F^{a}$ in $\mathcal{F}$ for every
$a\in\mathbb{N}$ and $b\in\mathbb{N}$.

\textbf{(b)} The $\mathbb{F}_{q}$-module $\mathcal{F}$ is free with basis
$\left(  T^{j}F^{i}\right)  _{i\geq0,\ j\geq0}$.

\textbf{(c)} As left $\mathbb{F}_{q}\left[  T\right]  $-module, $\mathcal{F}$
is free with basis $\left(  F^{i}\right)  _{i\geq0}$.

\textbf{(d)} As right $\mathbb{F}_{q}\left[  T\right]  $-module, $\mathcal{F}$
is free with basis $\left(  T^{j}F^{i}\right)  _{i\geq0,\ 0\leq j<q^{i}}$.

\textbf{(e)} As right $\mathbb{F}_{q}\left[  F\right]  $-module, $\mathcal{F}$
is free with basis $\left(  T^{j}\right)  _{j\geq0}$.

\textbf{(f)} As left $\mathbb{F}_{q}\left[  F\right]  $-module, $\mathcal{F}$
is free with basis $\left(  T^{j}F^{i}\right)  _{i=0\text{ or }q\nmid j}$.

\textbf{(g)} As $\mathbb{F}_{q}\left[  F\right]  $-$\mathbb{F}_{q}\left[
T\right]  $-bimodule, $\mathcal{F}$ is free with basis $\left(  T^{j}%
F^{i}\right)  _{\left(  i=0\text{ or }q\nmid j\right)  \text{ and }0\leq
j<q^{i}}$ (that is, we have $\mathcal{F}=\bigoplus\limits_{\substack{\left(
i,j\right)  \in\mathbb{N}^{2};\\\left(  i=0\text{ or }q\nmid j\right)  \text{
and }0\leq j<q^{i}}}\mathbb{F}_{q}\left[  F\right]  \cdot\left(  T^{j}%
F^{i}\right)  \cdot\mathbb{F}_{q}\left[  T\right]  $, and each $\mathbb{F}%
_{q}\left[  F\right]  \cdot\left(  T^{j}F^{i}\right)  \cdot\mathbb{F}%
_{q}\left[  T\right]  $ is isomorphic to $\mathbb{F}_{q}\left[  F\right]
\otimes\mathbb{F}_{q}\left[  T\right]  $ as an $\mathbb{F}_{q}\left[
F\right]  $-$\mathbb{F}_{q}\left[  T\right]  $-bimodule, where the tensor
product is taken over $\mathbb{F}_{q}$).
\end{proposition}

\begin{proof}
[Proof of Proposition \ref{prop.F.bases}.]Proposition \ref{prop.F.bases}
follows immediately from Proposition \ref{prop.F-gen.bases} by setting
$\mathbb{K}=\mathbb{F}_{q}$ and $r=q$.
\end{proof}

One simple identity in $\mathcal{F}$ is the following:

\begin{proposition}
\label{prop.F.FP}Let $P\in\mathbb{F}_{q}\left[  T\right]  $. Then, $FP=P^{q}F$
in $\mathcal{F}$.
\end{proposition}

\begin{proof}
[Proof of Proposition \ref{prop.F.FP}.]We are going to prove that $FP=\left(
\operatorname*{Frob}P\right)  F$. Since both sides of this equality are
$\mathbb{F}_{q}$-linear in $P$ (because $\operatorname*{Frob}$ is an
$\mathbb{F}_{q}$-linear map), we can WLOG assume that $P$ belongs to the basis
$\left(  T^{i}\right)  _{i\geq0}$ of the $\mathbb{F}_{q}$-vector space
$\mathbb{F}_{q}\left[  T\right]  $. Assume this. Thus, $P=T^{i}$ for some
$i\in\mathbb{N}$. Consider this $i$. The definition of $\operatorname*{Frob}$
yields $\operatorname*{Frob}P=\left(  \underbrace{P}_{=T^{i}}\right)
^{q}=\left(  T^{i}\right)  ^{q}=T^{qi}$.

Now, $\underbrace{F}_{=F^{1}}\underbrace{P}_{=T^{i}}=F^{1}T^{i}=T^{q^{1}%
i}F^{1}$ (by Proposition \ref{prop.F.bases} \textbf{(a)}), so that
$FP=\underbrace{T^{q^{1}i}}_{=T^{qi}=\operatorname*{Frob}P}\underbrace{F^{1}%
}_{=F}=\left(  \operatorname*{Frob}P\right)  F$.

Thus, $FP=\left(  \operatorname*{Frob}P\right)  F$ is proven. Hence,
$FP=\underbrace{\left(  \operatorname*{Frob}P\right)  }_{=P^{q}}F=P^{q}F$.
This proves Proposition \ref{prop.F.FP}.
\end{proof}

\begin{corollary}
\label{cor.F.FPF}Let $P\in\mathbb{F}_{q}\left[  T\right]  $. Then,
$\mathcal{F}\cdot P\cdot\mathcal{F}\subseteq P\cdot\mathcal{F}$.
\end{corollary}

\begin{proof}
[Proof of Corollary \ref{cor.F.FPF}.]We first claim that%
\begin{equation}
F^{i}P\in P\cdot\mathcal{F}\ \ \ \ \ \ \ \ \ \ \text{for every }i\in
\mathbb{N}. \label{pf.cor.F.FPF.FiP}%
\end{equation}


\textit{Proof of (\ref{pf.cor.F.FPF.FiP}):} We shall prove
(\ref{pf.cor.F.FPF.FiP}) by induction on $i$.

The \textit{induction base} (i.e., the case $i=0$) is trivial.

For the \textit{induction step}, we fix an $n\in\mathbb{N}$, and we assume
that (\ref{pf.cor.F.FPF.FiP}) holds for $i=n$. We then must prove that
(\ref{pf.cor.F.FPF.FiP}) holds for $i=n+1$.

By assumption, (\ref{pf.cor.F.FPF.FiP}) holds for $i=n$. In other words,
$F^{n}P\in P\cdot\mathcal{F}$. Now,%
\begin{align*}
\underbrace{F^{n+1}}_{=FF^{n}}P  &  =F\underbrace{F^{n}P}_{\in P\cdot
\mathcal{F}}\in\underbrace{FP}_{\substack{=P^{q}F\\\text{(by Proposition
\ref{prop.F.FP})}}}\cdot\mathcal{F}=\underbrace{P^{q}}_{=PP^{q-1}}%
F\cdot\mathcal{F}\\
&  =P\underbrace{P^{q-1}F\cdot\mathcal{F}}_{\subseteq\mathcal{F}}\subseteq
P\cdot\mathcal{F}.
\end{align*}
In other words, (\ref{pf.cor.F.FPF.FiP}) holds for $i=n+1$. This completes the
induction step. Thus, (\ref{pf.cor.F.FPF.FiP}) is proven.

Recall that $\left(  T^{j}F^{i}\right)  _{i\geq0,\ j\geq0}$ is a basis of the
$\mathbb{F}_{q}$-module $\mathcal{F}$ (by Proposition \ref{prop.F.bases}
\textbf{(b)}).

Now, we shall prove that%
\begin{equation}
uP\in P\cdot\mathcal{F}\ \ \ \ \ \ \ \ \ \ \text{for every }u\in\mathcal{F}.
\label{pf.cor.F.FPF.main}%
\end{equation}


\textit{Proof of (\ref{pf.cor.F.FPF.main}):} Let $u\in\mathcal{F}$. We must
prove the equality (\ref{pf.cor.F.FPF.main}). Since this equality is
$\mathbb{F}_{q}$-linear in $u$, we can WLOG assume that $u$ belongs to the
basis $\left(  T^{j}F^{i}\right)  _{i\geq0,\ j\geq0}$ of the $\mathbb{F}_{q}%
$-module $\mathcal{F}$. Assume this. Thus, $u=T^{j}F^{i}$ for some $\left(
i,j\right)  \in\mathbb{N}^{2}$. Consider this $\left(  i,j\right)  $. Now,%
\[
\underbrace{u}_{=T^{j}F^{i}}P=T^{j}\underbrace{F^{i}P}_{\substack{\in
P\cdot\mathcal{F}\\\text{(by (\ref{pf.cor.F.FPF.FiP}))}}}\in\underbrace{T^{j}%
P}_{\substack{=PT^{j}\\\text{(since }P\text{ and }T^{j}\text{ both}\\\text{lie
in }\mathbb{F}_{q}\left[  T\right]  \text{)}}}\cdot\mathcal{F}%
=P\underbrace{T^{j}\cdot\mathcal{F}}_{\subseteq\mathcal{F}}\subseteq
P\cdot\mathcal{F}.
\]
This proves (\ref{pf.cor.F.FPF.main}).

Now, (\ref{pf.cor.F.FPF.main}) immediately yields $\mathcal{F}\cdot P\subseteq
P\cdot\mathcal{F}$. Hence, $\underbrace{\mathcal{F}\cdot P}_{\subseteq
P\cdot\mathcal{F}}\cdot\mathcal{F}\subseteq P\cdot\underbrace{\mathcal{F}%
\cdot\mathcal{F}}_{\subseteq\mathcal{F}}\subseteq P\cdot\mathcal{F}$. This
proves Corollary \ref{cor.F.FPF}.
\end{proof}

\subsection{$q$-polynomials}

Next, we shall see an alternative description of the $\mathbb{F}_{q}$-algebra
$\mathcal{F}$. We begin with a general definition:

\begin{definition}
\label{def.q-pol}Let $A$ be a commutative $\mathbb{F}_{q}$-algebra. A
polynomial in $A\left[  X\right]  $ is said to be a $q$\textit{-polynomial} if
it is an $A$-linear combination of the monomials $X^{q^{0}},X^{q^{1}}%
,X^{q^{2}},\ldots$. We let $A\left[  X\right]  _{q-\operatorname*{lin}}$ be
the set of all $q$-polynomials in $A\left[  X\right]  $. Thus, $A\left[
X\right]  _{q-\operatorname*{lin}}$ is an $A$-submodule of $A\left[  X\right]
$; as an $A$-submodule, it has basis $\left(  X^{q^{0}},X^{q^{1}},X^{q^{2}%
},\ldots\right)  $.
\end{definition}

Thus, a polynomial in $A\left[  X\right]  $ belongs to $A\left[  X\right]
_{q-\operatorname*{lin}}$ if and only if the only monomials it contains are
(some of) the monomials $X^{q^{0}},X^{q^{1}},X^{q^{2}},\ldots$.

The $A$-submodule $A\left[  X\right]  _{q-\operatorname*{lin}}$ of $A\left[
X\right]  $ is not a subring of $A\left[  X\right]  $ (unless $A=0$). However,
it is closed under a different operation: namely, composition of polynomials.
Let us see this in more detail:

\begin{definition}
\label{def.q-pol.comp}Let $A$ be a commutative ring. Let $f\in A\left[
X\right]  $ and $g\in A\left[  X\right]  $. Then, $f\circ g$ denotes the
polynomial $f\left(  g\right)  \in A\left[  X\right]  $. (This is the
polynomial obtained from $f$ by substituting $g$ for $X$.) This defines a
binary operation $\circ$ on the set $A\left[  X\right]  $.
\end{definition}

\begin{proposition}
\label{prop.q-pol.comp.basics}Let $A$ be a commutative ring.

\textbf{(a)} The pair $\left(  A\left[  X\right]  ,\circ\right)  $ is a monoid
with neutral element $X$.

\textbf{(b)} Assume that $A$ is a commutative $\mathbb{F}_{q}$-algebra. Then,
$A\left[  X\right]  _{q-\operatorname*{lin}}$ is a submonoid of the monoid
$\left(  A\left[  X\right]  ,\circ\right)  $. Moreover, $\left(  A\left[
X\right]  _{q-\operatorname*{lin}},+,\circ\right)  $ is a (noncommutative)
$\mathbb{F}_{q}$-algebra with unity $X$ (where the $\mathbb{F}_{q}$-module
structure is the one obtained by restricting the $A\left[  X\right]  $-module
structure to $\mathbb{F}_{q}$).
\end{proposition}

\begin{proof}
[Proof of Proposition \ref{prop.q-pol.comp.basics}.]\textbf{(a)} If $B$ is any
commutative $A$-algebra, and if $b\in B$ is any element, then there exists a
unique $A$-algebra homomorphism $\varphi:A\left[  X\right]  \rightarrow B$
satisfying $\varphi\left(  X\right)  =b$.\ \ \ \ \footnote{This is simply the
universal property of the polynomial ring $A\left[  X\right]  $.} We shall
denote this homomorphism $\varphi$ by $\operatorname*{ev}\nolimits_{b}$. It
has the property that%
\begin{equation}
\operatorname*{ev}\nolimits_{b}\left(  f\right)  =f\left(  b\right)
\ \ \ \ \ \ \ \ \ \ \text{for every }f\in A\left[  X\right]  .
\label{pf.prop.q-pol.comp.basics.a.ev}%
\end{equation}


Now, every $f,g\in A\left[  X\right]  $ satisfy%
\begin{align}
\operatorname*{ev}\nolimits_{g}\left(  f\right)   &  =f\left(  g\right)
\ \ \ \ \ \ \ \ \ \ \left(  \text{by (\ref{pf.prop.q-pol.comp.basics.a.ev}),
applied to }B=A\left[  X\right]  \text{ and }b=g\right) \nonumber\\
&  =f\circ g\ \ \ \ \ \ \ \ \ \ \left(  \text{since }f\circ g=f\left(
g\right)  \right)  . \label{pf.prop.q-pol.comp.basics.a.evgf}%
\end{align}


Let $f,g,h\in A\left[  X\right]  $. Then,
(\ref{pf.prop.q-pol.comp.basics.a.evgf}) yields $\operatorname*{ev}%
\nolimits_{g}\left(  f\right)  =f\circ g$. Furthermore,
(\ref{pf.prop.q-pol.comp.basics.a.evgf}) (applied to $f\circ g$ and $h$
instead of $f$ and $g$) yields $\operatorname*{ev}\nolimits_{h}\left(  f\circ
g\right)  =\left(  f\circ g\right)  \circ h$. But
(\ref{pf.prop.q-pol.comp.basics.a.evgf}) (applied to $g$ and $h$ instead of
$f$ and $g$) yields $\operatorname*{ev}\nolimits_{h}\left(  g\right)  =g\circ
h$. Finally, (\ref{pf.prop.q-pol.comp.basics.a.evgf}) (applied to $g\circ h$
instead of $g$) yields $\operatorname*{ev}\nolimits_{g\circ h}\left(
f\right)  =f\circ\left(  g\circ h\right)  $.

The defining property of $\operatorname*{ev}\nolimits_{g\circ h}$ yields
$\operatorname*{ev}\nolimits_{g\circ h}\left(  X\right)  =g\circ h$. But the
defining property of $\operatorname*{ev}\nolimits_{g}$ yields
$\operatorname*{ev}\nolimits_{g}\left(  X\right)  =g$. Now,%
\[
\left(  \operatorname*{ev}\nolimits_{h}\circ\operatorname*{ev}\nolimits_{g}%
\right)  \left(  X\right)  =\operatorname*{ev}\nolimits_{h}\left(
\underbrace{\operatorname*{ev}\nolimits_{g}\left(  X\right)  }_{=g}\right)
=\operatorname*{ev}\nolimits_{h}\left(  g\right)  =g\circ h.
\]
Comparing this with $\operatorname*{ev}\nolimits_{g\circ h}\left(  X\right)
=g\circ h$, we obtain $\left(  \operatorname*{ev}\nolimits_{h}\circ
\operatorname*{ev}\nolimits_{g}\right)  \left(  X\right)  =\operatorname*{ev}%
\nolimits_{g\circ h}\left(  X\right)  $. The two maps $\operatorname*{ev}%
\nolimits_{h}\circ\operatorname*{ev}\nolimits_{g}$ and $\operatorname*{ev}%
\nolimits_{g\circ h}$ thus agree on the generator $X$ of the $A$-algebra
$A\left[  X\right]  $. Since these two maps are $A$-algebra homomorphisms
(because $\operatorname*{ev}\nolimits_{h}$, $\operatorname*{ev}\nolimits_{g}$
and $\operatorname*{ev}\nolimits_{g\circ h}$ are $A$-algebra homomorphisms),
this shows that these two maps are equal. In other words, $\operatorname*{ev}%
\nolimits_{h}\circ\operatorname*{ev}\nolimits_{g}=\operatorname*{ev}%
\nolimits_{g\circ h}$. Hence, $\underbrace{\left(  \operatorname*{ev}%
\nolimits_{h}\circ\operatorname*{ev}\nolimits_{g}\right)  }%
_{=\operatorname*{ev}\nolimits_{g\circ h}}\left(  f\right)
=\operatorname*{ev}\nolimits_{g\circ h}\left(  f\right)  =f\circ\left(  g\circ
h\right)  $. Thus,%
\[
f\circ\left(  g\circ h\right)  =\left(  \operatorname*{ev}\nolimits_{h}%
\circ\operatorname*{ev}\nolimits_{g}\right)  \left(  f\right)
=\operatorname*{ev}\nolimits_{h}\left(  \underbrace{\operatorname*{ev}%
\nolimits_{g}\left(  f\right)  }_{=f\circ g}\right)  =\operatorname*{ev}%
\nolimits_{h}\left(  f\circ g\right)  =\left(  f\circ g\right)  \circ h.
\]


Now, let us forget that we fixed $f,g,h$. We thus have shown that
$f\circ\left(  g\circ h\right)  =\left(  f\circ g\right)  \circ h$ for every
$f,g,h\in A\left[  X\right]  $. Thus, $\left(  A\left[  X\right]
,\circ\right)  $ is a semigroup. Furthermore, $X$ is a neutral element of this
semigroup (since every $f\in A\left[  X\right]  $ satisfies $X\circ f=X\left(
f\right)  =f$ and $f\circ X=f\left(  X\right)  =f$). Therefore, this semigroup
$\left(  A\left[  X\right]  ,\circ\right)  $ is a monoid with neutral element
$X$. This proves Proposition \ref{prop.q-pol.comp.basics} \textbf{(a)}.

\textbf{(b)} \textit{Step 1:} Let $\operatorname*{End}\left(  A\left[
X\right]  \right)  $ denote the $\mathbb{F}_{q}$-algebra of all endomorphisms
of the $\mathbb{F}_{q}$-vector space $A\left[  X\right]  $. It is easy to see
that $\operatorname*{Frob}=\operatorname*{Frob}\nolimits_{A\left[  X\right]
}\in\operatorname*{End}\left(  A\left[  X\right]  \right)  $. Hence,
$\operatorname*{Frob}\nolimits^{n}\in\operatorname*{End}\left(  A\left[
X\right]  \right)  $ for every $n\in\mathbb{N}$. It is straightforward to see
(by induction over $n$) that%
\begin{equation}
\operatorname*{Frob}\nolimits^{n}\left(  f\right)  =f^{q^{n}}%
\ \ \ \ \ \ \ \ \ \ \text{for every }f\in A\left[  X\right]  \text{ and }%
n\in\mathbb{N}. \label{pf.prop.q-pol.comp.basics.b.Frobn}%
\end{equation}
It is easy to see that
\begin{equation}
\operatorname*{Frob}\left(  A\left[  X\right]  _{q-\operatorname*{lin}%
}\right)  \subseteq A\left[  X\right]  _{q-\operatorname*{lin}}
\label{pf.prop.q-pol.comp.basics.b.Frob-contains}%
\end{equation}
\footnote{\textit{Proof of (\ref{pf.prop.q-pol.comp.basics.b.Frob-contains}):}
Let $g\in A\left[  X\right]  _{q-\operatorname*{lin}}$. We shall prove that
$\operatorname*{Frob}g\in A\left[  X\right]  _{q-\operatorname*{lin}}$.
\par
Indeed, $g\in A\left[  X\right]  _{q-\operatorname*{lin}}$. Thus, $g$ is an
$A$-linear combination of $\left(  X^{q^{0}},X^{q^{1}},X^{q^{2}}%
,\ldots\right)  $ (since the $A$-module $A\left[  X\right]
_{q-\operatorname*{lin}}$ has basis $\left(  X^{q^{0}},X^{q^{1}},X^{q^{2}%
},\ldots\right)  $). In other words, there exists a sequence $\left(
a_{0},a_{1},a_{2},\ldots\right)  \in A^{\mathbb{N}}$ of elements of $A$ such
that $g=\sum_{n\in\mathbb{N}}a_{n}X^{q^{n}}$, and such that all but finitely
many $n\in\mathbb{N}$ satisfy $a_{n}=0$. Consider this sequence.
\par
Applying the map $\operatorname*{Frob}$ to the equality $g=\sum_{n\in
\mathbb{N}}a_{n}X^{q^{n}}$, we obtain%
\begin{align*}
\operatorname*{Frob}g  &  =\operatorname*{Frob}\left(  \sum_{n\in\mathbb{N}%
}a_{n}X^{q^{n}}\right)  =\sum_{n\in\mathbb{N}}\underbrace{\operatorname*{Frob}%
\left(  a_{n}X^{q^{n}}\right)  }_{=\left(  a_{n}X^{q^{n}}\right)  ^{q}%
=a_{n}^{q}\left(  X^{q^{n}}\right)  ^{q}}\ \ \ \ \ \ \ \ \ \ \left(
\text{since the map }\operatorname*{Frob}\text{ is }\mathbb{F}_{q}%
\text{-linear}\right) \\
&  =\sum_{n\in\mathbb{N}}a_{n}^{q}\underbrace{\left(  X^{q^{n}}\right)  ^{q}%
}_{=X^{q^{n}q}=X^{q^{n+1}}\in A\left[  X\right]  _{q-\operatorname*{lin}}}%
\in\sum_{n\in\mathbb{N}}a_{n}^{q}A\left[  X\right]  _{q-\operatorname*{lin}%
}\subseteq A\left[  X\right]  _{q-\operatorname*{lin}}%
\end{align*}
(since $A\left[  X\right]  _{q-\operatorname*{lin}}$ is an $A$-module).
\par
Now, let us forget that we fixed $g$. We thus have proven that
$\operatorname*{Frob}g\in A\left[  X\right]  _{q-\operatorname*{lin}}$ for
every $g\in A\left[  X\right]  _{q-\operatorname*{lin}}$. In other words,
$\operatorname*{Frob}\left(  A\left[  X\right]  _{q-\operatorname*{lin}%
}\right)  \subseteq A\left[  X\right]  _{q-\operatorname*{lin}}$. This proves
(\ref{pf.prop.q-pol.comp.basics.b.Frob-contains}).}. Using this fact, it is
straightforward to see (by induction over $n$) that%
\begin{equation}
\operatorname*{Frob}\nolimits^{n}\left(  A\left[  X\right]
_{q-\operatorname*{lin}}\right)  \subseteq A\left[  X\right]
_{q-\operatorname*{lin}}\ \ \ \ \ \ \ \ \ \ \text{for every }n\in\mathbb{N}.
\label{pf.prop.q-pol.comp.basics.b.Frob-contains-n}%
\end{equation}


\textit{Step 2:} Now, let us prove that%
\begin{equation}
f\circ\left(  \lambda_{1}g_{1}+\lambda_{2}g_{2}\right)  =\lambda_{1}\left(
f\circ g_{1}\right)  +\lambda_{2}\left(  f\circ g_{2}\right)
\label{pf.prop.q-pol.comp.basics.b.dist1}%
\end{equation}
for every $f\in A\left[  X\right]  _{q-\operatorname*{lin}}$, $g_{1}\in
A\left[  X\right]  $, $g_{2}\in A\left[  X\right]  $, $\lambda_{1}%
\in\mathbb{F}_{q}$ and $\lambda_{2}\in\mathbb{F}_{q}$.

\textit{Proof of (\ref{pf.prop.q-pol.comp.basics.b.dist1}):} Let $f\in
A\left[  X\right]  _{q-\operatorname*{lin}}$.

We have $f\in A\left[  X\right]  _{q-\operatorname*{lin}}$. Thus, $f$ is an
$A$-linear combination of $\left(  X^{q^{0}},X^{q^{1}},X^{q^{2}}%
,\ldots\right)  $ (since the $A$-module $A\left[  X\right]
_{q-\operatorname*{lin}}$ has basis $\left(  X^{q^{0}},X^{q^{1}},X^{q^{2}%
},\ldots\right)  $). In other words, there exists a sequence $\left(
a_{0},a_{1},a_{2},\ldots\right)  \in A^{\mathbb{N}}$ of elements of $A$ such
that $f=\sum_{n\in\mathbb{N}}a_{n}X^{q^{n}}$, and such that all but finitely
many $n\in\mathbb{N}$ satisfy $a_{n}=0$. Consider this sequence.

Let $\widehat{f}$ denote the element $\sum_{n\in\mathbb{N}}a_{n}%
\operatorname*{Frob}\nolimits^{n}$ of $\operatorname*{End}\left(  A\left[
X\right]  \right)  $. (This is well-defined, since $\operatorname*{Frob}%
\nolimits^{n}\in\operatorname*{End}\left(  A\left[  X\right]  \right)  $ for
every $n\in\mathbb{N}$.) Now, every $h\in A\left[  X\right]  $ satisfies%
\begin{equation}
f\circ h=\widehat{f}\left(  h\right)
\label{pf.prop.q-pol.comp.basics.b.dist1.pf.1}%
\end{equation}
\footnote{\textit{Proof of (\ref{pf.prop.q-pol.comp.basics.b.dist1.pf.1}):}
Let $h\in A\left[  X\right]  $. Then,%
\[
f\circ h=f\left(  h\right)  =\sum_{n\in\mathbb{N}}a_{n}h^{q^{n}}%
\ \ \ \ \ \ \ \ \ \ \left(  \text{since }f=\sum_{n\in\mathbb{N}}a_{n}X^{q^{n}%
}\right)  .
\]
Comparing this with%
\begin{align*}
\widehat{f}\left(  h\right)   &  =\sum_{n\in\mathbb{N}}a_{n}%
\underbrace{\operatorname*{Frob}\nolimits^{n}\left(  h\right)  }%
_{\substack{=h^{q^{n}}\\\text{(by (\ref{pf.prop.q-pol.comp.basics.b.Frobn}),
applied to }h\\\text{instead of }f\text{)}}}\ \ \ \ \ \ \ \ \ \ \left(
\text{since }\widehat{f}=\sum_{n\in\mathbb{N}}a_{n}\operatorname*{Frob}%
\nolimits^{n}\right) \\
&  =\sum_{n\in\mathbb{N}}a_{n}h^{q^{n}},
\end{align*}
this yields $f\circ h=\widehat{f}\left(  h\right)  $, qed.}.

Now, let $g_{1}\in A\left[  X\right]  $, $g_{2}\in A\left[  X\right]  $,
$\lambda_{1}\in\mathbb{F}_{q}$ and $\lambda_{2}\in\mathbb{F}_{q}$. Applying
(\ref{pf.prop.q-pol.comp.basics.b.dist1.pf.1}) to $h=\lambda_{1}g_{1}%
+\lambda_{2}g_{2}$, we obtain%
\[
f\circ\left(  \lambda_{1}g_{1}+\lambda_{2}g_{2}\right)  =\widehat{f}\left(
\lambda_{1}g_{1}+\lambda_{2}g_{2}\right)  =\lambda_{1}\widehat{f}\left(
g_{1}\right)  +\lambda_{2}\widehat{f}\left(  g_{2}\right)
\]
(since $\widehat{f}\in\operatorname*{End}\left(  A\left[  X\right]  \right)
$). Comparing this with%
\[
\lambda_{1}\underbrace{\left(  f\circ g_{1}\right)  }_{\substack{=\widehat{f}%
\left(  g_{1}\right)  \\\text{(by
(\ref{pf.prop.q-pol.comp.basics.b.dist1.pf.1}))}}}+\lambda_{2}%
\underbrace{\left(  f\circ g_{2}\right)  }_{\substack{=\widehat{f}\left(
g_{2}\right)  \\\text{(by (\ref{pf.prop.q-pol.comp.basics.b.dist1.pf.1}))}%
}}=\lambda_{1}\widehat{f}\left(  g_{1}\right)  +\lambda_{2}\widehat{f}\left(
g_{2}\right)  ,
\]
we obtain $f\circ\left(  \lambda_{1}g_{1}+\lambda_{2}g_{2}\right)
=\lambda_{1}\left(  f\circ g_{1}\right)  +\lambda_{2}\left(  f\circ
g_{2}\right)  $. Thus, (\ref{pf.prop.q-pol.comp.basics.b.dist1}) is proven.

\textit{Step 3:} Furthermore, we have%
\begin{equation}
\left(  \lambda_{1}f_{1}+\lambda_{2}f_{2}\right)  \circ g=\lambda_{1}\left(
f_{1}\circ g\right)  +\lambda_{2}\left(  f_{2}\circ g\right)
\label{pf.prop.q-pol.comp.basics.b.dist2}%
\end{equation}
for every $f_{1}\in A\left[  X\right]  $, $f_{2}\in A\left[  X\right]  $,
$g\in A\left[  X\right]  $, $\lambda_{1}\in\mathbb{F}_{q}$ and $\lambda_{2}%
\in\mathbb{F}_{q}$.

\textit{Proof of (\ref{pf.prop.q-pol.comp.basics.b.dist2}):} Let $f_{1}\in
A\left[  X\right]  $, $f_{2}\in A\left[  X\right]  $, $g\in A\left[  X\right]
$, $\lambda_{1}\in\mathbb{F}_{q}$ and $\lambda_{2}\in\mathbb{F}_{q}$. Then,%
\[
\left(  \lambda_{1}f_{1}+\lambda_{2}f_{2}\right)  \circ g=\left(  \lambda
_{1}f_{1}+\lambda_{2}f_{2}\right)  \left(  g\right)  =\lambda_{1}f_{1}\left(
g\right)  +\lambda_{2}f_{2}\left(  g\right)  .
\]
Comparing this with $\lambda_{1}\underbrace{\left(  f_{1}\circ g\right)
}_{=f_{1}\left(  g\right)  }+\lambda_{2}\underbrace{\left(  f_{2}\circ
g\right)  }_{=f_{2}\left(  g\right)  }=\lambda_{1}f_{1}\left(  g\right)
+\lambda_{2}f_{2}\left(  g\right)  $, we obtain $\left(  \lambda_{1}%
f_{1}+\lambda_{2}f_{2}\right)  \circ g=\lambda_{1}\left(  f_{1}\circ g\right)
+\lambda_{2}\left(  f_{2}\circ g\right)  $. This proves
(\ref{pf.prop.q-pol.comp.basics.b.dist2}).

\textit{Step 4:} Now, let us show that%
\begin{equation}
f\circ g\in A\left[  X\right]  _{q-\operatorname*{lin}}%
\ \ \ \ \ \ \ \ \ \ \text{for every }f,g\in A\left[  X\right]
_{q-\operatorname*{lin}}. \label{pf.prop.q-pol.comp.basics.b.closed}%
\end{equation}


\textit{Proof of (\ref{pf.prop.q-pol.comp.basics.b.closed}):} Let $f,g\in
A\left[  X\right]  _{q-\operatorname*{lin}}$. Define the sequence $\left(
a_{0},a_{1},a_{2},\ldots\right)  \in A^{\mathbb{N}}$ and the element
$\widehat{f}\in\operatorname*{End}\left(  A\left[  X\right]  \right)  $ as in
the proof of (\ref{pf.prop.q-pol.comp.basics.b.dist1}). Then,
(\ref{pf.prop.q-pol.comp.basics.b.dist1.pf.1}) holds. Applying
(\ref{pf.prop.q-pol.comp.basics.b.dist1.pf.1}) to $h=g$, we obtain%
\begin{align*}
f\circ g  &  =\widehat{f}\left(  g\right)  =\sum_{n\in\mathbb{N}}%
a_{n}\operatorname*{Frob}\nolimits^{n}\left(  \underbrace{g}_{\in A\left[
X\right]  _{q-\operatorname*{lin}}}\right)  \ \ \ \ \ \ \ \ \ \ \left(
\text{since }\widehat{f}=\sum_{n\in\mathbb{N}}a_{n}\operatorname*{Frob}%
\nolimits^{n}\right) \\
&  \in\sum_{n\in\mathbb{N}}a_{n}\underbrace{\operatorname*{Frob}%
\nolimits^{n}\left(  A\left[  X\right]  _{q-\operatorname*{lin}}\right)
}_{\substack{\subseteq A\left[  X\right]  _{q-\operatorname*{lin}}\\\text{(by
(\ref{pf.prop.q-pol.comp.basics.b.Frob-contains-n}))}}}\subseteq\sum
_{n\in\mathbb{N}}a_{n}A\left[  X\right]  _{q-\operatorname*{lin}}\subseteq
A\left[  X\right]  _{q-\operatorname*{lin}}%
\end{align*}
(since $A\left[  X\right]  _{q-\operatorname*{lin}}$ is an $A$-module). Thus,
we have proven (\ref{pf.prop.q-pol.comp.basics.b.closed}).

\textit{Step 5:} We have $X=X^{1}\in A\left[  X\right]
_{q-\operatorname*{lin}}$. This, combined with
(\ref{pf.prop.q-pol.comp.basics.b.closed}), shows that $A\left[  X\right]
_{q-\operatorname*{lin}}$ is a submonoid of the monoid $\left(  A\left[
X\right]  ,\circ\right)  $. Furthermore, the binary operation $\circ$ on
$A\left[  X\right]  _{q-\operatorname*{lin}}$ is $\mathbb{F}_{q}$-bilinear (by
(\ref{pf.prop.q-pol.comp.basics.b.dist1}) and
(\ref{pf.prop.q-pol.comp.basics.b.dist2})) and associative (since $\left(
A\left[  X\right]  ,\circ\right)  $ is a monoid) and has neutral element $X$
(since $\left(  A\left[  X\right]  ,\circ\right)  $ is a monoid with neutral
element $X$). Thus, $\left(  A\left[  X\right]  _{q-\operatorname*{lin}%
},+,\circ\right)  $ is a (noncommutative) $\mathbb{F}_{q}$-algebra with unity
$X$. This concludes the proof of Proposition \ref{prop.q-pol.comp.basics}
\textbf{(b)}.
\end{proof}

\begin{definition}
\label{def.q-pol.mon.not}Let $A$ be a commutative ring. Whenever $f\in
A\left[  X\right]  $ and $n\in\mathbb{N}$, we shall use the notation $f^{\circ
n}$ for the $n$-th power of $f$ in the monoid $\left(  A\left[  X\right]
,\circ\right)  $.
\end{definition}

\begin{definition}
\label{def.q-pol.ring}Let $A$ be a commutative $\mathbb{F}_{q}$-algebra. The
(noncommutative) $\mathbb{F}_{q}$-algebra $\left(  A\left[  X\right]
_{q-\operatorname*{lin}},+,\circ\right)  $ constructed in Proposition
\ref{prop.q-pol.comp.basics} \textbf{(b)} will be called the \textit{Ore
polynomial ring over }$A$, and simply denoted by $A\left[  X\right]
_{q-\operatorname*{lin}}$ (since there are no other $\mathbb{F}_{q}$-algebra
structures on $A\left[  X\right]  _{q-\operatorname*{lin}}$ that could be
confused with this one).
\end{definition}

The connection between these Ore polynomial rings and our $\mathcal{F}$ is the following:

\begin{theorem}
\label{thm.q-pol.=F}Consider the Ore polynomial ring over $\mathbb{F}%
_{q}\left[  T\right]  \left[  X\right]  _{q-\operatorname*{lin}}$; recall that
this is the $\mathbb{F}_{q}$-algebra $\left(  \mathbb{F}_{q}\left[  T\right]
\left[  X\right]  _{q-\operatorname*{lin}},+,\circ\right)  $. (Notice that
polynomials in $\mathbb{F}_{q}\left[  T\right]  \left[  X\right]
_{q-\operatorname*{lin}}$ can contain arbitrary powers of $T$, but the only
powers of $X$ they can contain are $X^{q^{0}},X^{q^{1}},X^{q^{2}},\ldots$.)
Define an $\mathbb{F}_{q}$-algebra homomorphism $\operatorname*{Fqpol}%
:\mathcal{F}\rightarrow\mathbb{F}_{q}\left[  T\right]  \left[  X\right]
_{q-\operatorname*{lin}}$ by $\operatorname*{Fqpol}\left(  F\right)  =X^{q}$
and $\operatorname*{Fqpol}\left(  T\right)  =TX$.

\textbf{(a)} This homomorphism $\operatorname*{Fqpol}$ is well-defined.

\textbf{(b)} This homomorphism $\operatorname*{Fqpol}$ is an $\mathbb{F}_{q}%
$-algebra isomorphism.

\textbf{(c)} We have $\operatorname*{Fqpol}\left(  T^{j}F^{i}\right)
=T^{j}X^{q^{i}}$ for every $i\in\mathbb{N}$ and $j\in\mathbb{N}$.

\textbf{(d)} We have $\operatorname*{Fqpol}t=t\cdot X$ for every
$t\in\mathbb{F}_{q}\left[  T\right]  $. (Here, we regard $\mathbb{F}%
_{q}\left[  T\right]  $ as an $\mathbb{F}_{q}$-subalgebra of $\mathcal{F}$ as
before. The expression \textquotedblleft$t\cdot X$\textquotedblright\ means
the product of $t\in\mathbb{F}_{q}\left[  T\right]  \subseteq\mathbb{F}%
_{q}\left[  T\right]  \left[  X\right]  $ with $X$ in $\mathbb{F}_{q}\left[
T\right]  \left[  X\right]  $.)
\end{theorem}

\begin{proof}
[Proof of Theorem \ref{thm.q-pol.=F}.]For every $n\in\mathbb{N}$, we have%
\begin{equation}
\left(  TX\right)  ^{\circ n}=T^{n}X\ \ \ \ \ \ \ \ \ \ \text{in }%
\mathbb{F}_{q}\left[  T\right]  \left[  X\right]  _{q-\operatorname*{lin}}.
\label{pf.thm.q-pol.=F.TXon}%
\end{equation}
(This follows by a straightforward induction on $n$.) Furthermore, for every
$n\in\mathbb{N}$, we have%
\begin{equation}
\left(  X^{q}\right)  ^{\circ n}=X^{q^{n}}\ \ \ \ \ \ \ \ \ \ \text{in
}\mathbb{F}_{q}\left[  T\right]  \left[  X\right]  _{q-\operatorname*{lin}}.
\label{pf.thm.q-pol.=F.Xqon}%
\end{equation}
(Again, this is easy to prove by induction.)

\textbf{(a)} In $\mathbb{F}_{q}\left[  T\right]  \left[  X\right]
_{q-\operatorname*{lin}}$, we have $X^{q}\circ\left(  TX\right)  =\left(
TX\right)  ^{\circ q}\circ X^{q}$ (indeed, this follows by comparing
$X^{q}\circ\left(  TX\right)  =X^{q}\left(  TX\right)  =\left(  TX\right)
^{q}=T^{q}X^{q}$ and $\underbrace{\left(  TX\right)  ^{\circ q}}%
_{\substack{=T^{q}X\\\text{(by (\ref{pf.thm.q-pol.=F.TXon}), applied to
}n=q\text{)}}}\circ X^{q}=\left(  T^{q}X\right)  \circ X^{q}=T^{q}X^{q}$).
Now, recall that if $u$ and $v$ are two elements of an $\mathbb{F}_{q}%
$-algebra $\mathcal{U}$ satisfying $uv=v^{q}u$, then there exists a unique
$\mathbb{F}_{q}$-algebra homomorphism $\mathcal{F}\rightarrow\mathcal{U}$
sending $F$ and $T$ to $u$ and $v$, respectively. Applying this to
$\mathcal{U}=\mathbb{F}_{q}\left[  T\right]  \left[  X\right]
_{q-\operatorname*{lin}}$, $u=X^{q}$ and $v=TX$, we thus conclude that there
exists a unique $\mathbb{F}_{q}$-algebra homomorphism $\mathcal{F}%
\rightarrow\mathcal{U}$ sending $F$ and $T$ to $X^{q}$ and $TX$, respectively.
In other words, the homomorphism $\operatorname*{Fqpol}$ is well-defined. This
proves Theorem \ref{thm.q-pol.=F} \textbf{(a)}.

\textbf{(c)} For every $i\in\mathbb{N}$ and $j\in\mathbb{N}$, we have%
\begin{align*}
\operatorname*{Fqpol}\left(  T^{j}F^{i}\right)   &  =\left(
\underbrace{\operatorname*{Fqpol}T}_{=TX}\right)  ^{\circ j}\circ\left(
\underbrace{\operatorname*{Fqpol}F}_{=X^{q}}\right)  ^{\circ i}\\
&  \ \ \ \ \ \ \ \ \ \ \left(  \text{since }\operatorname*{Fqpol}\text{ is an
}\mathbb{F}_{q}\text{-algebra homomorphism}\right) \\
&  =\underbrace{\left(  TX\right)  ^{\circ j}}_{\substack{=T^{j}X\\\text{(by
(\ref{pf.thm.q-pol.=F.TXon}))}}}\circ\underbrace{\left(  X^{q}\right)  ^{\circ
i}}_{\substack{=X^{q^{i}}\\\text{(by (\ref{pf.thm.q-pol.=F.Xqon}))}}}=\left(
T^{j}X\right)  \circ X^{q^{i}}=T^{j}X^{q^{i}}.
\end{align*}
This proves Theorem \ref{thm.q-pol.=F} \textbf{(c)}.

\textbf{(b)} The $\mathbb{F}_{q}\left[  T\right]  $-module $\mathbb{F}%
_{q}\left[  T\right]  \left[  X\right]  _{q-\operatorname*{lin}}$ has basis
$\left(  X^{q^{0}},X^{q^{1}},X^{q^{2}},\ldots\right)  =\left(  X^{q^{i}%
}\right)  _{i\geq0}$. Thus, as an $\mathbb{F}_{q}$-module, it has basis
$\left(  T^{j}X^{q^{i}}\right)  _{i\geq0,\ j\geq0}$.

On the other hand, Proposition \ref{prop.F.bases} \textbf{(b)} says that the
$\mathbb{F}_{q}$-module $\mathcal{F}$ is free with basis $\left(  T^{j}%
F^{i}\right)  _{i\geq0,\ j\geq0}$.

For every $i\in\mathbb{N}$ and $j\in\mathbb{N}$, we have
$\operatorname*{Fqpol}\left(  T^{j}F^{i}\right)  =T^{j}X^{q^{i}}$ (by Theorem
\ref{thm.q-pol.=F} \textbf{(c)}). Hence, the $\mathbb{F}_{q}$-linear map
$\operatorname*{Fqpol}$ sends the basis $\left(  T^{j}F^{i}\right)
_{i\geq0,\ j\geq0}$ of the $\mathbb{F}_{q}$-module $\mathcal{F}$ to the basis
$\left(  T^{j}X^{q^{i}}\right)  _{i\geq0,\ j\geq0}$ of the $\mathbb{F}_{q}%
$-module $\mathbb{F}_{q}\left[  T\right]  \left[  X\right]
_{q-\operatorname*{lin}}$. Consequently, $\operatorname*{Fqpol}$ is an
$\mathbb{F}_{q}$-module isomorphism, thus an $\mathbb{F}_{q}$-algebra
isomorphism. This proves Theorem \ref{thm.q-pol.=F} \textbf{(b)}.

\textbf{(d)} Let $t\in\mathbb{F}_{q}\left[  T\right]  $. We must prove the
equality $\operatorname*{Fqpol}t=t\cdot X$. Since this equality is clearly
$\mathbb{F}_{q}$-linear in $t$, we can WLOG assume that $t$ belongs to the
basis $\left(  T^{j}\right)  _{j\geq0}$ of the $\mathbb{F}_{q}$-module
$\mathbb{F}_{q}\left[  T\right]  $. Assume this. Thus, $t=T^{j}$ for some
$j\in\mathbb{N}$. Consider this $j$. We have $t=T^{j}=T^{j}F^{0}$ in
$\mathcal{F}$. Thus, $\operatorname*{Fqpol}t=\operatorname*{Fqpol}\left(
T^{j}F^{0}\right)  =T^{j}X^{q^{0}}$ (by Theorem \ref{thm.q-pol.=F}
\textbf{(c)}, applied to $i=0$). Hence, $\operatorname*{Fqpol}%
t=\underbrace{T^{j}}_{=t}\underbrace{X^{q^{0}}}_{=X^{1}=X}=t\cdot X$. Thus,
Theorem \ref{thm.q-pol.=F} \textbf{(d)} is proven.
\end{proof}

Theorem \ref{thm.q-pol.=F} \textbf{(b)} shows that the $\mathbb{F}_{q}%
$-algebra $\mathbb{F}_{q}\left[  T\right]  \left[  X\right]
_{q-\operatorname*{lin}}$ is isomorphic to $\mathcal{F}$; this algebra can
thus be regarded as a rather concrete manifestation of $\mathcal{F}$. We shall
make more use of this later.

Let us prove one further simple property of $A\left[  X\right]
_{q-\operatorname*{lin}}$ (for general $A$):

\begin{proposition}
\label{prop.q-pol.A-q-lin}Let $A$ be a commutative $\mathbb{F}_{q}$-algebra.
Let $f\in A\left[  X\right]  _{q-\operatorname*{lin}}$. Let $B$ be a
commutative $A$-algebra. Then, the map $B\rightarrow B,\ b\mapsto f\left(
b\right)  $ is $\mathbb{F}_{q}$-linear. (It might not be $A$-linear.)
\end{proposition}

\begin{proof}
[Proof of Proposition \ref{prop.q-pol.A-q-lin}.]Let $\operatorname*{End}B$
denote the $\mathbb{F}_{q}$-algebra of all endomorphisms of the $\mathbb{F}%
_{q}$-vector space $B$. It is easy to see that $\operatorname*{Frob}%
=\operatorname*{Frob}\nolimits_{B}\in\operatorname*{End}B$. Hence,
$\operatorname*{Frob}\nolimits^{n}\in\operatorname*{End}B$ for every
$n\in\mathbb{N}$. It is straightforward to see (by induction over $n$) that%
\begin{equation}
\operatorname*{Frob}\nolimits^{n}\left(  b\right)  =b^{q^{n}}%
\ \ \ \ \ \ \ \ \ \ \text{for every }b\in B\text{ and }n\in\mathbb{N}.
\label{pf.prop.q-pol.A-q-lin.Frobn}%
\end{equation}


We have $f\in A\left[  X\right]  _{q-\operatorname*{lin}}$. Thus, $f$ is an
$A$-linear combination of $\left(  X^{q^{0}},X^{q^{1}},X^{q^{2}}%
,\ldots\right)  $ (since the $A$-module $A\left[  X\right]
_{q-\operatorname*{lin}}$ has basis $\left(  X^{q^{0}},X^{q^{1}},X^{q^{2}%
},\ldots\right)  $). In other words, there exists a sequence $\left(
a_{0},a_{1},a_{2},\ldots\right)  \in A^{\mathbb{N}}$ of elements of $A$ such
that $f=\sum_{n\in\mathbb{N}}a_{n}X^{q^{n}}$, and such that all but finitely
many $n\in\mathbb{N}$ satisfy $a_{n}=0$. Consider this sequence.

Let $\widehat{f}$ denote the element $\sum_{n\in\mathbb{N}}a_{n}%
\operatorname*{Frob}\nolimits^{n}$ of $\operatorname*{End}B$. (This is
well-defined, since $\operatorname*{Frob}\nolimits^{n}\in\operatorname*{End}B$
for every $n\in\mathbb{N}$.) Now, every $b\in B$ satisfies%
\begin{equation}
f\left(  b\right)  =\widehat{f}\left(  b\right)
\label{pf.prop.q-pol.A-q-lin.fh}%
\end{equation}
\footnote{\textit{Proof of (\ref{pf.prop.q-pol.A-q-lin.fh}):} Let $b\in B$.
From $f=\sum_{n\in\mathbb{N}}a_{n}X^{q^{n}}$, we obtain $f\left(  b\right)
=\sum_{n\in\mathbb{N}}a_{n}b^{q^{n}}$. Comparing this with%
\begin{align*}
\widehat{f}\left(  b\right)   &  =\sum_{n\in\mathbb{N}}a_{n}%
\underbrace{\operatorname*{Frob}\nolimits^{n}\left(  b\right)  }%
_{\substack{=b^{q^{n}}\\\text{(by (\ref{pf.prop.q-pol.A-q-lin.Frobn}))}%
}}\ \ \ \ \ \ \ \ \ \ \left(  \text{since }\widehat{f}=\sum_{n\in\mathbb{N}%
}a_{n}\operatorname*{Frob}\nolimits^{n}\right) \\
&  =\sum_{n\in\mathbb{N}}a_{n}b^{q^{n}},
\end{align*}
this yields $f\left(  b\right)  =\widehat{f}\left(  b\right)  $, qed.}. Hence,
the map $B\rightarrow B,\ b\mapsto f\left(  b\right)  $ equals the map
$B\rightarrow B,\ b\mapsto\widehat{f}\left(  b\right)  $. But the latter map
is simply the map $\widehat{f}\in\operatorname*{End}B$, and thus clearly
$\mathbb{F}_{q}$-linear. Hence, the former map is $\mathbb{F}_{q}$-linear.
Proposition \ref{prop.q-pol.A-q-lin} is thus proven.
\end{proof}

Proposition \ref{prop.q-pol.A-q-lin} also has a partial converse:

\begin{proposition}
\label{prop.q-pol.A-q-lin-converse}Let $A$ be a commutative $\mathbb{F}_{q}%
$-algebra which is an integral domain. Let $f\in A\left[  X\right]  $ be such
that, for every commutative $A$-algebra $B$, the map $B\rightarrow
B,\ b\mapsto f\left(  b\right)  $ is $\mathbb{F}_{q}$-linear. Then, $f\in
A\left[  X\right]  _{q-\operatorname*{lin}}$.
\end{proposition}

The proof of Proposition \ref{prop.q-pol.A-q-lin-converse} can be found in
\cite[Corollary A.3]{kc-carlitz}; we shall not give it here, as we shall not
use Proposition \ref{prop.q-pol.A-q-lin-converse}. Propositions
\ref{prop.q-pol.A-q-lin} and \ref{prop.q-pol.A-q-lin-converse} are the reason
why the $q$-polynomials over $A$ (that is, the elements of $A\left[  X\right]
_{q-\operatorname*{lin}}$) are often called the \textquotedblleft%
$\mathbb{F}_{q}$-linear polynomials over $A$\textquotedblright, but we shall
not use this terminology (as it is mildly misleading: it sounds too much like
degree-$1$ polynomials).

\subsection{$q$-polynomials from subspaces}

We shall now see a classical way to construct $q$-polynomials.

\begin{definition}
Let $A$ be a commutative $\mathbb{F}_{q}$-algebra. For every finite subset $V$
of $A$, let $f_{V}$ be the polynomial $\prod_{v\in V}\left(  X+v\right)  \in
A\left[  X\right]  $.
\end{definition}

The following result is a consequence of \cite[(7.7)]{mac-schurvar} (and also
appears in \cite[Theorem A.1 2)]{kc-carlitz} in the particular case when $A$
is an integral domain):

\begin{theorem}
\label{thm.mac1.subspace}Let $A$ be a commutative $\mathbb{F}_{q}$-algebra.
Let $V$ be a finite $\mathbb{F}_{q}$-vector subspace of $A$. Then, $f_{V}$ is
a $q$-polynomial.
\end{theorem}

We shall prove Theorem \ref{thm.mac1.subspace} following an idea that appears
in \cite[proof of (7.15)]{mac-schurvar}; but first, let us slightly generalize it:

\begin{definition}
Let $A$ be a commutative $\mathbb{F}_{q}$-algebra. For every finite set $V$
and every map $\varphi:V\rightarrow A$, we let $f_{V,\varphi}$ be the
polynomial $\prod_{v\in V}\left(  X+\varphi\left(  v\right)  \right)  \in
A\left[  X\right]  $.
\end{definition}

\begin{theorem}
\label{thm.mac1.map}Let $A$ be a commutative $\mathbb{F}_{q}$-algebra. Let $V$
be a finite $\mathbb{F}_{q}$-vector space, and let $\varphi:V\rightarrow A$ be
an $\mathbb{F}_{q}$-linear map. Then, $f_{V,\varphi}$ is a $q$-polynomial.
\end{theorem}

Theorem \ref{thm.mac1.map} is not significantly more general than Theorem
\ref{thm.mac1.subspace} (it is easily derived from the latter), but this
little generality helps in proving it. The proof will need the following lemmas:

\begin{lemma}
\label{lem.mac1.too-early}Let $A$ be a commutative $\mathbb{F}_{q}$-algebra.
Let $V$ and $W$ be two finite $\mathbb{F}_{q}$-vector spaces. Let
$\varphi:V\rightarrow A$ and $\psi:W\rightarrow A$ be two $\mathbb{F}_{q}%
$-linear maps. Assume that $f_{W,\psi}$ is a $q$-polynomial. Let
$h:A\rightarrow A$ be an $\mathbb{F}_{q}$-linear map such that every $a\in A$
satisfies%
\begin{equation}
h\left(  a\right)  =f_{W,\psi}\left(  a\right)  .
\label{eq.lem.mac1.too-early.ass}%
\end{equation}
Let $\chi:V\oplus W\rightarrow A$ be the $\mathbb{F}_{q}$-linear map which
sends every $\left(  v,w\right)  \in V\oplus W$ to $\varphi\left(  v\right)
+\psi\left(  w\right)  \in A$. Then,%
\[
f_{V\oplus W,\chi}=f_{V,h\circ\varphi}\circ f_{W,\psi}%
\ \ \ \ \ \ \ \ \ \ \text{in }A\left[  X\right]  .
\]

\end{lemma}

\begin{proof}
[Proof of Lemma \ref{lem.mac1.too-early}.]The definition of $f_{W,\psi}$
yields%
\begin{equation}
f_{W,\psi}=\prod_{v\in W}\left(  X+\psi\left(  v\right)  \right)  =\prod_{w\in
W}\left(  X+\psi\left(  w\right)  \right)  \label{pf.lem.mac1.too-early.fW}%
\end{equation}
(here, we renamed the summation index $v$ as $w$).

Fix some $v\in V$. If we substitute $X+\varphi\left(  v\right)  $ for $X$ on
both sides of (\ref{pf.lem.mac1.too-early.fW}), then we obtain%
\begin{equation}
f_{W,\psi}\left(  X+\varphi\left(  v\right)  \right)  =\prod_{w\in W}\left(
X+\varphi\left(  v\right)  +\psi\left(  w\right)  \right)  .
\label{pf.lem.mac1.too-early.1}%
\end{equation}


We have assumed that $f_{W,\psi}$ is a $q$-polynomial. In other words,
$f_{W,\psi}\in A\left[  X\right]  _{q-\operatorname*{lin}}$. Hence,
Proposition \ref{prop.q-pol.A-q-lin} (applied to $B=A\left[  X\right]  $ and
$f=f_{W,\psi}$) shows that the map $A\left[  X\right]  \rightarrow A\left[
X\right]  ,\ b\mapsto f_{W,\psi}\left(  b\right)  $ is $\mathbb{F}_{q}%
$-linear. Hence, $f_{W,\psi}\left(  x_{1}+x_{2}\right)  =f_{W,\psi}\left(
x_{1}\right)  +f_{W,\psi}\left(  x_{2}\right)  $ for every $x_{1},x_{2}\in
A\left[  X\right]  $. Applying this to $x_{1}=X$ and $x_{2}=\varphi\left(
v\right)  $, we obtain
\begin{align*}
f_{W,\psi}\left(  X+\varphi\left(  v\right)  \right)   &
=\underbrace{f_{W,\psi}\left(  X\right)  }_{=f_{W,\psi}}+\underbrace{f_{W,\psi
}\left(  \varphi\left(  v\right)  \right)  }_{\substack{=h\left(
\varphi\left(  v\right)  \right)  \\\text{(because
(\ref{eq.lem.mac1.too-early.ass}) (applied to }a=\varphi\left(  v\right)
\text{)}\\\text{yields }h\left(  \varphi\left(  v\right)  \right)  =f_{W,\psi
}\left(  \varphi\left(  v\right)  \right)  \text{)}}}\\
&  =f_{W,\psi}+\underbrace{h\left(  \varphi\left(  v\right)  \right)
}_{=\left(  h\circ\varphi\right)  \left(  v\right)  }=f_{W,\psi}+\left(
h\circ\varphi\right)  \left(  v\right)  .
\end{align*}
Comparing this with (\ref{pf.lem.mac1.too-early.1}), we obtain%
\begin{equation}
\prod_{w\in W}\left(  X+\varphi\left(  v\right)  +\psi\left(  w\right)
\right)  =f_{W,\psi}+\left(  h\circ\varphi\right)  \left(  v\right)  .
\label{pf.lem.mac1.too-early.4}%
\end{equation}


Let us now forget that we fixed $v$. We thus have shown proven the equality
(\ref{pf.lem.mac1.too-early.4}) for all $v\in V$.

The definition of $f_{V,h\circ\varphi}$ yields%
\[
f_{V,h\circ\varphi}=\prod_{v\in V}\left(  X+\left(  h\circ\varphi\right)
\left(  v\right)  \right)  .
\]
Substituting $f_{W,\psi}$ for $X$ on both sides of this equality, we obtain%
\begin{equation}
f_{V,h\circ\varphi}\left(  f_{W,\psi}\right)  =\prod_{v\in V}\left(
f_{W,\psi}+\left(  h\circ\varphi\right)  \left(  v\right)  \right)  .
\label{pf.lem.mac1.too-early.6}%
\end{equation}


The definition of $f_{V\oplus W,\chi}$ yields%
\begin{align*}
f_{V\oplus W,\chi}  &  =\prod_{v\in V\oplus W}\left(  X+\chi\left(  v\right)
\right)  =\underbrace{\prod_{\left(  v,w\right)  \in V\oplus W}}_{=\prod_{v\in
V}\prod_{w\in W}}\left(  X+\underbrace{\chi\left(  v,w\right)  }%
_{\substack{=\varphi\left(  v\right)  +\psi\left(  w\right)  \\\text{(by the
definition of }\chi\text{)}}}\right) \\
&  \ \ \ \ \ \ \ \ \ \ \left(  \text{here, we renamed the index }v\text{ as
}\left(  v,w\right)  \text{ in the product}\right) \\
&  =\prod_{v\in V}\underbrace{\prod_{w\in W}\left(  X+\varphi\left(  v\right)
+\psi\left(  w\right)  \right)  }_{\substack{=f_{W,\psi}+\left(  h\circ
\varphi\right)  \left(  v\right)  \\\text{(by (\ref{pf.lem.mac1.too-early.4}%
))}}}=\prod_{v\in V}\left(  f_{W,\psi}+\left(  h\circ\varphi\right)  \left(
v\right)  \right) \\
&  =f_{V,h\circ\varphi}\left(  f_{W,\psi}\right)  \ \ \ \ \ \ \ \ \ \ \left(
\text{by (\ref{pf.lem.mac1.too-early.6})}\right) \\
&  =f_{V,h\circ\varphi}\circ f_{W,\psi}.
\end{align*}
This proves Lemma \ref{lem.mac1.too-early}.
\end{proof}

\begin{lemma}
\label{lem.mac1.Xq-X}We have
\begin{equation}
\prod_{\lambda\in\mathbb{F}_{q}}\left(  X-\lambda Y\right)  =X^{q}-XY^{q-1}
\label{eq.lem.mac1.Xq-X.eq}%
\end{equation}
in the polynomial ring $\mathbb{F}_{q}\left[  X,Y\right]  $.
\end{lemma}

\begin{proof}
[Proof of Lemma \ref{lem.mac1.Xq-X}.]It is well-known that
\begin{equation}
\prod_{\lambda\in\mathbb{F}_{q}}\left(  X-\lambda\right)  =X^{q}-X
\label{pf.lem.mac1.Xq-X.X}%
\end{equation}
in the polynomial ring $\mathbb{F}_{q}\left[  X\right]  $\ \ \ \ \footnote{Let
us give a \textit{proof of (\ref{pf.lem.mac1.Xq-X.X})} for the sake of
completeness:
\par
The polynomial $\prod_{\lambda\in\mathbb{F}_{q}}\left(  X-\lambda\right)  $ is
a product of $\left\vert \mathbb{F}_{q}\right\vert =q$ monic polynomials of
degree $1$. Thus, it is a monic polynomial of degree $q$. Hence, both
polynomials $\prod_{\lambda\in\mathbb{F}_{q}}\left(  X-\lambda\right)  $ and
$X^{q}-X$ are monic polynomials of degree $q$. Their difference $\prod
_{\lambda\in\mathbb{F}_{q}}\left(  X-\lambda\right)  -\left(  X^{q}-X\right)
$ therefore is a polynomial of degree $<q$ (since the subtraction causes their
leading terms to cancel).
\par
On the other hand, every $\mu\in\mathbb{F}_{q}$ satisfies%
\[
\underbrace{\prod_{\lambda\in\mathbb{F}_{q}}\left(  \mu-\lambda\right)
}_{\substack{=0\\\text{(since one of the factors of}\\\text{this product is
}\mu-\mu=0\text{)}}}-\left(  \underbrace{\mu^{q}}_{\substack{=\mu
\\\text{(since }\mu\in\mathbb{F}_{q}\text{)}}}-\mu\right)  =0-\left(  \mu
-\mu\right)  =0.
\]
In other words, every $\mu\in\mathbb{F}_{q}$ is a root of the polynomial
$\prod_{\lambda\in\mathbb{F}_{q}}\left(  X-\lambda\right)  -\left(
X^{q}-X\right)  $. Hence, the polynomial $\prod_{\lambda\in\mathbb{F}_{q}%
}\left(  X-\lambda\right)  -\left(  X^{q}-X\right)  $ has at least $q$ roots
(since $\mathbb{F}_{q}$ has at least $q$ elements).
\par
But $\mathbb{F}_{q}$ is a field. Hence, any polynomial in $\mathbb{F}%
_{q}\left[  X\right]  $ whose degree is smaller than its number of roots must
be the zero polynomial. The polynomial $\prod_{\lambda\in\mathbb{F}_{q}%
}\left(  X-\lambda\right)  -\left(  X^{q}-X\right)  $ is such a polynomial
(since its degree is $<q$, but it has at least $q$ roots), and thus must be
the zero polynomial. In other words, $\prod_{\lambda\in\mathbb{F}_{q}}\left(
X-\lambda\right)  =\left(  X^{q}-X\right)  $. This proves
(\ref{pf.lem.mac1.Xq-X.X}).}.

Now, consider the element $X/Y$ in the quotient field $\mathbb{F}_{q}\left(
X,Y\right)  $ of the ring $\mathbb{F}_{q}\left[  X,Y\right]  $. Substituting
this element $X/Y$ for $X$ in (\ref{pf.lem.mac1.Xq-X.X}), we obtain%
\[
\prod_{\lambda\in\mathbb{F}_{q}}\left(  X/Y-\lambda\right)  =\left(
X/Y\right)  ^{q}-X/Y.
\]
Multiplying this equality by $Y^{q}$, we obtain%
\[
Y^{q}\prod_{\lambda\in\mathbb{F}_{q}}\left(  X/Y-\lambda\right)  =Y^{q}\left(
\left(  X/Y\right)  ^{q}-X/Y\right)  =X^{q}-XY^{q-1}.
\]
Hence,
\begin{align*}
X^{q}-XY^{q-1}  &  =Y^{q}\prod_{\lambda\in\mathbb{F}_{q}}\left(
X/Y-\lambda\right)  =\prod_{\lambda\in\mathbb{F}_{q}}\underbrace{\left(
Y\left(  X/Y-\lambda\right)  \right)  }_{=X-\lambda Y}%
\ \ \ \ \ \ \ \ \ \ \left(  \text{since }\left\vert \mathbb{F}_{q}\right\vert
=q\right) \\
&  =\prod_{\lambda\in\mathbb{F}_{q}}\left(  X-\lambda Y\right)  .
\end{align*}
This proves Lemma \ref{lem.mac1.Xq-X}.
\end{proof}

\begin{lemma}
\label{lem.mac1.dim1}Let $A$ be a commutative $\mathbb{F}_{q}$-algebra. Let
$V$ be a one-dimensional $\mathbb{F}_{q}$-vector space. Let $\varphi
:V\rightarrow A$ be an $\mathbb{F}_{q}$-linear map. Let $e$ be a nonzero
element of $V$. Then, $f_{V,\varphi}=X^{q}-\left(  \varphi\left(  e\right)
\right)  ^{q-1}X$.
\end{lemma}

\begin{proof}
[Proof of Lemma \ref{lem.mac1.dim1}.]The element $-e$ of $V$ is nonzero (since
$e$ is nonzero).

The $\mathbb{F}_{q}$-vector space $V$ is one-dimensional, and thus any nonzero
element of $V$ forms a basis of $V$. Thus, $-e$ forms a basis of $V$ (since
$-e$ is a nonzero element of $V$). In other words, the map $\mathbb{F}%
_{q}\rightarrow V,\ \lambda\mapsto\lambda\left(  -e\right)  $ is a bijection.
Now, the definition of $f_{V,\varphi}$ yields%
\begin{align*}
f_{V,\varphi}  &  =\prod_{v\in V}\left(  X+\varphi\left(  v\right)  \right)
=\prod_{\lambda\in\mathbb{F}_{q}}\left(  X+\varphi\left(  \underbrace{\lambda
\left(  -e\right)  }_{=-\lambda e}\right)  \right) \\
&  \ \ \ \ \ \ \ \ \ \ \left(
\begin{array}
[c]{c}%
\text{here, we have substituted }\lambda\left(  -e\right)  \text{ for }v\text{
in the product,}\\
\text{since the map }\mathbb{F}_{q}\rightarrow V,\ \lambda\mapsto
\lambda\left(  -e\right)  \text{ is a bijection}%
\end{array}
\right) \\
&  =\prod_{\lambda\in\mathbb{F}_{q}}\left(  X+\underbrace{\varphi\left(
-\lambda e\right)  }_{\substack{=-\lambda\varphi\left(  e\right)
\\\text{(since }\varphi\text{ is }\mathbb{F}_{q}\text{-linear)}}}\right)
=\prod_{\lambda\in\mathbb{F}_{q}}\left(  X-\lambda\varphi\left(  e\right)
\right) \\
&  =X^{q}-X\left(  \varphi\left(  e\right)  \right)  ^{q-1}%
\ \ \ \ \ \ \ \ \ \ \left(  \text{this follows by substituting }\varphi\left(
e\right)  \text{ for }Y\text{ in (\ref{eq.lem.mac1.Xq-X.eq})}\right) \\
&  =X^{q}-\left(  \varphi\left(  e\right)  \right)  ^{q-1}X.
\end{align*}
This proves Lemma \ref{lem.mac1.dim1}.
\end{proof}

\begin{proof}
[Proof of Theorem \ref{thm.mac1.map}.]We shall prove Theorem
\ref{thm.mac1.map} by induction over $\dim V$:

\textit{Induction base:} Theorem \ref{thm.mac1.map} holds in the case when
$\dim V=0$\ \ \ \ \footnote{\textit{Proof.} Consider the setting of Theorem
\ref{thm.mac1.map}, and assume that $\dim V=0$. From $\dim V=0$, we obtain
$V=0$. The definition of $f_{V,\varphi}$ yields%
\begin{align*}
f_{V,\varphi}  &  =\prod_{v\in V}\left(  X+\varphi\left(  v\right)  \right)
=X+\underbrace{\varphi\left(  0\right)  }_{\substack{=0\\\text{(since }%
\varphi\text{ is }\mathbb{F}_{q}\text{-linear)}}}\ \ \ \ \ \ \ \ \ \ \left(
\text{since }V=0\right) \\
&  =X.
\end{align*}
Thus, $f_{V,\varphi}$ is a $q$-polynomial (since $X$ is a $q$-polynomial).
Thus, Theorem \ref{thm.mac1.map} is proven in the case when $\dim V=0$.}. This
completes the induction base.

\textit{Induction step:} Let $N\in\mathbb{N}$. Assume (as the induction
hypothesis) that Theorem \ref{thm.mac1.map} holds in the case when $\dim V=N$.
We need to show that Theorem \ref{thm.mac1.map} holds in the case when $\dim
V=N+1$.

Consider the setting of Theorem \ref{thm.mac1.map}, and assume that $\dim
V=N+1$. Thus, $\dim V=N+1>0$. Hence, $V$ contains a nonzero element $e$.
Consider this $e$. Let $U$ be the $\mathbb{F}_{q}$-vector subspace
$\mathbb{F}_{q}e$ of $V$; thus, $\dim U=1$ (since $e$ is nonzero). Pick any
complement $W$ to the subspace $U$ of $V$ (such a complement exists by one of
the basic theorems of linear algebra). Then, $W$ is an $\mathbb{F}_{q}$-vector
subspace of $V$ satisfying $U\oplus W=V$. We shall identify $V$ with the
\textbf{external} direct sum of $U$ and $W$ (that is, we shall identify each
element of $v$ with the unique pair $\left(  u,w\right)  \in U\times W$
satisfying $v=u+w$). Thus, the $\mathbb{F}_{q}$-linear map $\varphi
:V\rightarrow A$ can be regarded as an $\mathbb{F}_{q}$-linear map
$\varphi:U\oplus W\rightarrow A$.

Define two $\mathbb{F}_{q}$-linear maps $\gamma:U\rightarrow A$ and
$\psi:W\rightarrow A$ by $\gamma=\varphi\mid_{U}$ and $\psi=\varphi\mid_{W}$.
Then, the $\mathbb{F}_{q}$-linear map $\varphi:U\oplus W\rightarrow A$ sends
every $\left(  v,w\right)  \in U\oplus W$ to $\gamma\left(  v\right)
+\psi\left(  w\right)  $\ \ \ \ \footnote{\textit{Proof.} Let $\left(
v,w\right)  \in U\oplus W$. We must show that $\varphi\left(  v,w\right)
=\gamma\left(  v\right)  +\psi\left(  w\right)  $.
\par
We have $v\in U$, and thus $\gamma\left(  v\right)  =\varphi\left(  v\right)
$ (since $\gamma=\varphi\mid_{U}$). We have $w\in W$, and thus $\psi\left(
w\right)  =\varphi\left(  w\right)  $ (since $\psi=\varphi\mid_{W}$). The map
$\varphi$ is $\mathbb{F}_{q}$-linear, and thus $\varphi\left(  v+w\right)
=\underbrace{\varphi\left(  v\right)  }_{=\gamma\left(  v\right)
}+\underbrace{\varphi\left(  w\right)  }_{=\psi\left(  w\right)  }%
=\gamma\left(  v\right)  +\psi\left(  w\right)  $. But recall that we are
identifying $\left(  v,w\right)  \in U\oplus W$ with $v+w\in V$. Thus,
$\varphi\left(  v,w\right)  =\varphi\left(  v+w\right)  =\gamma\left(
v\right)  +\psi\left(  w\right)  $, qed.}.

From $V=U\oplus W$, we obtain $\dim V=\dim U+\dim W$, so that $\dim
W=\underbrace{\dim V}_{=N+1}-\underbrace{\dim U}_{=1}=N+1-1=N$. Thus,
(according to the induction hypothesis) Theorem \ref{thm.mac1.map} can be
applied to $W$ and $\psi$ instead of $V$ and $\varphi$. As a consequence, we
obtain that $f_{W,\psi}$ is a $q$-polynomial. In other words, $f_{W,\psi}\in
A\left[  X\right]  _{q-\operatorname*{lin}}$. Thus, Proposition
\ref{prop.q-pol.A-q-lin} (applied to $f=f_{W,q}$ and $B=A$) shows that the map
$A\rightarrow A,\ b\mapsto f_{W,q}\left(  b\right)  $ is $\mathbb{F}_{q}%
$-linear. Let us denote this map by $h$. Thus, $h$ is the map $A\rightarrow
A,\ b\mapsto f_{W,q}\left(  b\right)  $, and is $\mathbb{F}_{q}$-linear. Every
$a\in A$ satisfies $h\left(  a\right)  =f_{W,\psi}\left(  a\right)  $ (by the
definition of $h$).

Now, Lemma \ref{lem.mac1.too-early} (applied to $U$, $\gamma$ and $\varphi$
instead of $V$, $\varphi$ and $\chi$) shows that $f_{U\oplus W,\varphi
}=f_{U,h\circ\gamma}\circ f_{W,\psi}$ in $A\left[  X\right]  $.

But the $\mathbb{F}_{q}$-vector space $U$ is one-dimensional (since $\dim
U=1$) and contains the nonzero vector $e$ (since $U=\mathbb{F}_{q}e\supseteq
e$). Thus, Lemma \ref{lem.mac1.dim1} (applied to $U$ and $h\circ\gamma$
instead of $V$ and $\varphi$) shows that $f_{U,h\circ\gamma}=X^{q}-\left(
\left(  h\circ\gamma\right)  \left(  e\right)  \right)  ^{q-1}X$. This is
clearly a $q$-polynomial (since $\left(  \left(  h\circ\gamma\right)  \left(
e\right)  \right)  ^{q-1}$ is just a coefficient in $A$). In other words,
$f_{U,h\circ\gamma}\in A\left[  X\right]  _{q-\operatorname*{lin}}$.

Proposition \ref{prop.q-pol.comp.basics} \textbf{(b)} shows that $A\left[
X\right]  _{q-\operatorname*{lin}}$ is a submonoid of the monoid $\left(
A\left[  X\right]  ,\circ\right)  $. Hence, $A\left[  X\right]
_{q-\operatorname*{lin}}$ is closed under the binary operation $\circ$.
Therefore, $f_{U,h\circ\gamma}\circ f_{W,\psi}\in A\left[  X\right]
_{q-\operatorname*{lin}}$ (since $f_{U,h\circ\gamma}\in A\left[  X\right]
_{q-\operatorname*{lin}}$ and $f_{W,\psi}\in A\left[  X\right]
_{q-\operatorname*{lin}}$). But $V=U\oplus W$, so that $f_{V,\varphi
}=f_{U\oplus W,\varphi}=f_{U,h\circ\gamma}\circ f_{W,\psi}\in A\left[
X\right]  _{q-\operatorname*{lin}}$. In other words, $f_{V,\varphi}$ is a
$q$-polynomial. Thus, Theorem \ref{thm.mac1.map} is proven in the case when
$\dim V=N+1$. This completes the induction step.

The proof of Theorem \ref{thm.mac1.map} is thus complete.
\end{proof}

As a consequence of Theorem \ref{thm.mac1.map}, we can remove one unneeded
assumption from Lemma \ref{lem.mac1.too-early}:

\begin{corollary}
\label{cor.mac1.V+W}Let $A$ be a commutative $\mathbb{F}_{q}$-algebra. Let $V$
and $W$ be two $\mathbb{F}_{q}$-vector spaces. Let $\varphi:V\rightarrow A$
and $\psi:W\rightarrow A$ be two $\mathbb{F}_{q}$-linear maps. Let
$h:A\rightarrow A$ be an $\mathbb{F}_{q}$-linear map such that every $a\in A$
satisfies $h\left(  a\right)  =f_{W,\psi}\left(  a\right)  $. Let
$\chi:V\oplus W\rightarrow A$ be the $\mathbb{F}_{q}$-linear map which sends
every $\left(  v,w\right)  \in V\oplus W$ to $\varphi\left(  v\right)
+\psi\left(  w\right)  \in A$. Then,%
\[
f_{V\oplus W,\chi}=f_{V,h\circ\varphi}\circ f_{W,\psi}%
\ \ \ \ \ \ \ \ \ \ \text{in }A\left[  X\right]  .
\]

\end{corollary}

\begin{proof}
[Proof of Corollary \ref{cor.mac1.V+W}.]Theorem \ref{thm.mac1.map} (applied to
$W$ and $\psi$ instead of $V$ and $\varphi$) shows that $f_{W,\psi}$ is a
$q$-polynomial. Thus, Lemma \ref{lem.mac1.too-early} shows that $f_{V\oplus
W,\chi}=f_{V,h\circ\varphi}\circ f_{W,\psi}$ in $A\left[  X\right]  $. This
proves Corollary \ref{cor.mac1.V+W}.
\end{proof}

Let us finally derive Theorem \ref{thm.mac1.subspace} from Theorem
\ref{thm.mac1.map}:

\begin{proof}
[Proof of Theorem \ref{thm.mac1.subspace}.]Let $\iota$ be the canonical
inclusion map $V\rightarrow A$. Thus, $\iota$ is an $\mathbb{F}_{q}$-linear
map. Hence, Theorem \ref{thm.mac1.map} (applied to $\varphi=\iota$) shows that
$f_{V,\iota}$ is a $q$-polynomial. But the definition of $f_{V,\iota}$ shows
that%
\[
f_{V,\iota}=\prod_{v\in V}\left(  X+\underbrace{\iota\left(  v\right)
}_{\substack{=v\\\text{(since }\iota\text{ is an}\\\text{inclusion map)}%
}}\right)  =\prod_{v\in V}\left(  X+v\right)  =f_{V}%
\]
(since this is how $f_{V}$ is defined). Thus, $f_{V}$ is a $q$-polynomial
(since $f_{V,\iota}$ is a $q$-polynomial). This proves Theorem
\ref{thm.mac1.subspace}.
\end{proof}

\subsection{Further consequences of the $\operatorname*{Fqpol}$ isomorphism}

Let us return to $\mathcal{F}$. We shall now exploit the isomorphism
$\operatorname*{Fqpol}$ to obtain properties of $\mathcal{F}$.

First, let us recall that if $A$ is any commutative $\mathbb{F}_{q}$-algebra,
then $A\left[  X\right]  _{q-\operatorname*{lin}}$ is an $A$-submodule of
$A\left[  X\right]  $. Applying this to $A=\mathbb{F}_{q}\left[  T\right]  $,
we see that
\begin{equation}
\mathbb{F}_{q}\left[  T\right]  \left[  X\right]  _{q-\operatorname*{lin}%
}\text{ is an }\mathbb{F}_{q}\left[  T\right]  \text{-submodule of }%
\mathbb{F}_{q}\left[  T\right]  \left[  X\right]  \text{.}
\label{eq.q-pol.q-lin.leftT}%
\end{equation}
We shall write this $\mathbb{F}_{q}\left[  T\right]  $-module structure on the
left (i.e., we use it to make $\mathbb{F}_{q}\left[  T\right]  \left[
X\right]  _{q-\operatorname*{lin}}$ into a left $\mathbb{F}_{q}\left[
T\right]  $-module). This left $\mathbb{F}_{q}\left[  T\right]  $-module
structure is given by plain multiplication inside $\mathbb{F}_{q}\left[
T\right]  \left[  X\right]  $. It has the following property:

\begin{proposition}
\label{prop.q-pol.Fqlin.leftT}The map $\operatorname*{Fqpol}:\mathcal{F}%
\rightarrow\mathbb{F}_{q}\left[  T\right]  \left[  X\right]
_{q-\operatorname*{lin}}$ is an isomorphism of left $\mathbb{F}_{q}\left[
T\right]  $-modules.
\end{proposition}

\begin{proof}
[Proof of Proposition \ref{prop.q-pol.Fqlin.leftT}.]Proposition
\ref{prop.F.bases} \textbf{(b)} says that the $\mathbb{F}_{q}$-module
$\mathcal{F}$ is free with basis $\left(  T^{j}F^{i}\right)  _{i\geq
0,\ j\geq0}$.

Theorem \ref{thm.q-pol.=F} \textbf{(b)} shows that $\operatorname*{Fqpol}$ is
an $\mathbb{F}_{q}$-algebra isomorphism. Thus, it remains to prove that
$\operatorname*{Fqpol}$ is a homomorphism of left $\mathbb{F}_{q}\left[
T\right]  $-modules. In other words, it remains to prove that
$\operatorname*{Fqpol}\left(  fu\right)  =f\operatorname*{Fqpol}\left(
u\right)  $ for every $f\in\mathbb{F}_{q}\left[  T\right]  $ and
$u\in\mathcal{F}$.

So let $f\in\mathbb{F}_{q}\left[  T\right]  $ and $u\in\mathcal{F}$. We need
to prove the equality $\operatorname*{Fqpol}\left(  fu\right)
=f\operatorname*{Fqpol}\left(  u\right)  $. This equality is $\mathbb{F}_{q}%
$-linear in $u$. Hence, we can WLOG assume that $u$ belongs to the basis
$\left(  T^{j}F^{i}\right)  _{i\geq0,\ j\geq0}$ of the $\mathbb{F}_{q}$-module
$\mathcal{F}$. Assume this. Thus, $u=T^{j}F^{i}$ for some $i\in\mathbb{N}$ and
$j\in\mathbb{N}$. Consider these $i$ and $j$.

We still need to prove the equality $\operatorname*{Fqpol}\left(  fu\right)
=f\operatorname*{Fqpol}\left(  u\right)  $. This equality is $\mathbb{F}_{q}%
$-linear in $f$. Hence, we can WLOG assume that $f$ belongs to the basis
$\left(  T^{k}\right)  _{k\geq0}$ of the $\mathbb{F}_{q}$-module
$\mathbb{F}_{q}\left[  T\right]  $. Assume this. Thus, $f=T^{k}$ for some
$k\in\mathbb{N}$. Consider this $k$.

Multiplying the equalities $f=T^{k}$ and $u=T^{j}F^{i}$, we obtain
$fu=\underbrace{T^{k}T^{j}}_{=T^{k+j}}F^{i}=T^{k+j}F^{i}$. Hence,
$\operatorname*{Fqpol}\left(  fu\right)  =\operatorname*{Fqpol}\left(
T^{k+j}F^{i}\right)  =T^{k+j}X^{q^{i}}$ (by Theorem \ref{thm.q-pol.=F}
\textbf{(c)}, applied to $k+j$ instead of $j$). On the other hand,
$u=T^{j}F^{i}$, so that $\operatorname*{Fqpol}\left(  u\right)
=\operatorname*{Fqpol}\left(  T^{j}F^{i}\right)  =T^{j}X^{q^{i}}$ (by Theorem
\ref{thm.q-pol.=F} \textbf{(c)}). Multiplying the equalities $f=T^{k}$ and
$\operatorname*{Fqpol}\left(  u\right)  =T^{j}X^{q^{i}}$, we obtain
$f\operatorname*{Fqpol}\left(  u\right)  =\underbrace{T^{k}T^{j}}_{=T^{k+j}%
}X^{q^{i}}=T^{k+j}X^{q^{i}}$. Comparing this with $\operatorname*{Fqpol}%
\left(  fu\right)  =T^{k+j}X^{q^{i}}$, we obtain $\operatorname*{Fqpol}\left(
fu\right)  =f\operatorname*{Fqpol}\left(  u\right)  $. As explained, this
completes the proof of Proposition \ref{prop.q-pol.Fqlin.leftT}.
\end{proof}

Notice that we can use Proposition \ref{prop.q-pol.Fqlin.leftT} to recover
Proposition \ref{prop.F.bases} \textbf{(c)}:

\begin{proof}
[Second proof of Proposition \ref{prop.F.bases} \textbf{(c)}.]Proposition
\ref{prop.q-pol.Fqlin.leftT} yields that $\mathcal{F}\cong\mathbb{F}%
_{q}\left[  T\right]  \left[  X\right]  _{q-\operatorname*{lin}}$ as left
$\mathbb{F}_{q}\left[  T\right]  $-modules, via the isomorphism
$\operatorname*{Fqpol}$. Since the left $\mathbb{F}_{q}\left[  T\right]
$-module $\mathbb{F}_{q}\left[  T\right]  \left[  X\right]
_{q-\operatorname*{lin}}$ has basis $\left(  X^{q^{0}},X^{q^{1}},X^{q^{2}%
},\ldots\right)  $, we can therefore conclude that the left $\mathbb{F}%
_{q}\left[  T\right]  $-module $\mathcal{F}$ has basis $\left(
\operatorname*{Fqpol}\nolimits^{-1}\left(  X^{q^{0}}\right)
,\operatorname*{Fqpol}\nolimits^{-1}\left(  X^{q^{1}}\right)
,\operatorname*{Fqpol}\nolimits^{-1}\left(  X^{q^{2}}\right)  ,\ldots\right)
$. Since $\operatorname*{Fqpol}\nolimits^{-1}\left(  X^{q^{i}}\right)  =F^{i}$
for every $i\in\mathbb{N}$\ \ \ \ \footnote{\textit{Proof.} Let $i\in
\mathbb{N}$. Theorem \ref{thm.q-pol.=F} \textbf{(c)} (applied to $j=0$) yields
$\operatorname*{Fqpol}\left(  T^{0}F^{i}\right)  =\underbrace{T^{0}}%
_{=1}X^{q^{i}}=X^{q^{i}}$. Thus, $\operatorname*{Fqpol}\nolimits^{-1}\left(
X^{q^{i}}\right)  =\underbrace{T^{0}}_{=1}F^{i}=F^{i}$, qed.}, this rewrites
as follows: The left $\mathbb{F}_{q}\left[  T\right]  $-module $\mathcal{F}$
has basis\textbf{ }$\left(  F^{i}\right)  _{i\geq0}$. This proves Proposition
\ref{prop.F.bases} \textbf{(c)} again.
\end{proof}

Let us make some more remarks (in less detail, since these will not be used in
the following):

Proposition \ref{prop.q-pol.Fqlin.leftT} can be rewritten as follows: If we
transport the left $\mathbb{F}_{q}\left[  T\right]  $-module structure on
$\mathcal{F}$ to $\mathbb{F}_{q}\left[  T\right]  \left[  X\right]
_{q-\operatorname*{lin}}$ via the isomorphism $\operatorname*{Fqpol}%
:\mathcal{F}\rightarrow\mathbb{F}_{q}\left[  T\right]  \left[  X\right]
_{q-\operatorname*{lin}}$, then we obtain the left $\mathbb{F}_{q}\left[
T\right]  $-module structure on $\mathbb{F}_{q}\left[  T\right]  \left[
X\right]  _{q-\operatorname*{lin}}$ constructed in (\ref{eq.q-pol.q-lin.leftT}%
). Of course, we can also use the isomorphism $\operatorname*{Fqpol}$ to
transport all the other module structures from $\mathcal{F}$ to $\mathbb{F}%
_{q}\left[  T\right]  \left[  X\right]  _{q-\operatorname*{lin}}$ along
$\operatorname*{Fqpol}$. In more detail:

From Proposition \ref{prop.F.bases}, we know that $\mathcal{F}$ is a left
$\mathbb{F}_{q}\left[  T\right]  $-module, a right $\mathbb{F}_{q}\left[
T\right]  $-module, a left $\mathbb{F}_{q}\left[  F\right]  $-module, and a
right $\mathbb{F}_{q}\left[  F\right]  $-module. Thus, we have altogether four
module structures on $\mathcal{F}$. Using the isomorphism
$\operatorname*{Fqpol}:\mathcal{F}\rightarrow\mathbb{F}_{q}\left[  T\right]
\left[  X\right]  _{q-\operatorname*{lin}}$, we can transport them to
$\mathbb{F}_{q}\left[  T\right]  \left[  X\right]  _{q-\operatorname*{lin}}$;
therefore, $\mathbb{F}_{q}\left[  T\right]  \left[  X\right]
_{q-\operatorname*{lin}}$ becomes a left $\mathbb{F}_{q}\left[  T\right]
$-module, a right $\mathbb{F}_{q}\left[  T\right]  $-module, a left
$\mathbb{F}_{q}\left[  F\right]  $-module, and a right $\mathbb{F}_{q}\left[
F\right]  $-module. As we have already said, the first of these four module
structures is precisely the left $\mathbb{F}_{q}\left[  T\right]  $-module
structure on $\mathcal{F}$ constructed in (\ref{eq.q-pol.q-lin.leftT}). The
other three structures are new. Explicitly, two of them are characterized as follows:

\begin{itemize}
\item If $t\in\mathbb{F}_{q}\left[  T\right]  $, then the action of $t$ on the
right $\mathbb{F}_{q}\left[  T\right]  $-module $\mathbb{F}_{q}\left[
T\right]  \left[  X\right]  _{q-\operatorname*{lin}}$ sends every
$m\in\mathbb{F}_{q}\left[  T\right]  \left[  X\right]  _{q-\operatorname*{lin}%
}$ to $m\circ\underbrace{\operatorname*{Fqpol}t}_{\substack{=t\cdot
X\\\text{(by Theorem \ref{thm.q-pol.=F} \textbf{(d)})}}}=m\circ\left(  t\cdot
X\right)  =m\left(  t\cdot X\right)  $ (that is, the result of substituting
$t\cdot X$ for $X$ in $m$).

\item If $f\in\mathbb{F}_{q}\left[  F\right]  $, then the action of $f$ on the
left $\mathbb{F}_{q}\left[  F\right]  $-module $\mathbb{F}_{q}\left[
T\right]  \left[  X\right]  _{q-\operatorname*{lin}}$ sends every
$m\in\mathbb{F}_{q}\left[  T\right]  \left[  X\right]  _{q-\operatorname*{lin}%
}$ to $\operatorname*{Fqpol}f\circ m=f\left(  \operatorname*{Frob}%
\nolimits_{\mathbb{F}_{q}\left[  T\right]  \left[  X\right]  }\right)  \circ
m$.
\end{itemize}

\begin{noncompile}
[This has been rather sketchy. More details would have been in order if I ever
need to use these other module structures.]
\end{noncompile}

\subsection{Frobenius $\mathbb{F}_{q}\left[  T\right]  $-modules}

In the following, \textquotedblleft$\mathcal{F}$-module\textquotedblright%
\ will always mean \textquotedblleft left $\mathcal{F}$%
-module\textquotedblright, unless stated otherwise. The following fact is a
simple consequence of the definition of $\mathcal{F}$ (specifically, of the
fact that $\mathcal{F}$ is generated by $F$ and $T$ as an $\mathbb{F}_{q}$-algebra):

\begin{lemma}
\label{lem.F.modhom}Let $M$ and $N$ be two $\mathcal{F}$-modules. Let
$f:M\rightarrow N$ be an $\mathbb{F}_{q}$-linear map. Assume that%
\[
f\left(  Tu\right)  =Tf\left(  u\right)  \ \ \ \ \ \ \ \ \ \ \text{for every
}u\in M.
\]
Assume also that%
\[
f\left(  Fu\right)  =Ff\left(  u\right)  \ \ \ \ \ \ \ \ \ \ \text{for every
}u\in M.
\]
Then, $f$ is an $\mathcal{F}$-module homomorphism.
\end{lemma}

This lemma shall be used tacitly further below; it is the most reasonable way
to prove that a certain map between two $\mathcal{F}$-modules $M$ and $N$ is
an $\mathcal{F}$-module homomorphism, particularly in the case when the
$\mathcal{F}$-module structure on at least one of $M$ and $N$ is defined not
explicitly but by providing the actions of $F$ and $T$.

Part of the interest in the $\mathbb{F}_{q}$-algebra $\mathcal{F}$ is due to
its category of modules: it can be described as the category of
\textquotedblleft Frobenius $\mathbb{F}_{q}\left[  T\right]  $%
-modules\textquotedblright, by which we mean $\mathbb{F}_{q}\left[  T\right]
$-modules equipped with a \textquotedblleft Frobenius map\textquotedblright%
\ satisfying a certain rule. Let us define this in more detail:

\begin{definition}
\label{def.F.frobmod}\textbf{(a)} A \textit{Frobenius }$\mathbb{F}_{q}\left[
T\right]  $\textit{-module} means a pair $\left(  M,\mathfrak{f}\right)  $,
where $M$ is an $\mathbb{F}_{q}\left[  T\right]  $-module, and where
$\mathfrak{f}:M\rightarrow M$ is an $\mathbb{F}_{q}$-linear map satisfying%
\begin{equation}
\mathfrak{f}\left(  Tm\right)  =T^{q}\mathfrak{f}\left(  m\right)
\ \ \ \ \ \ \ \ \ \ \text{for every }m\in M. \label{eq.def.F.frobmod.axiom}%
\end{equation}
This map $\mathfrak{f}$ is called the \textit{Frobenius map} of the Frobenius
$\mathbb{F}_{q}\left[  T\right]  $-module $\left(  M,\mathfrak{f}\right)  $.
By abuse of notation, we shall often speak of the \textquotedblleft Frobenius
$\mathbb{F}_{q}\left[  T\right]  $-module $M$\textquotedblright\ instead of
the \textquotedblleft Frobenius $\mathbb{F}_{q}\left[  T\right]  $-module
$\left(  M,\mathfrak{f}\right)  $\textquotedblright, leaving the Frobenius map
$\mathfrak{f}$ implicit; in this situation, the Frobenius map $\mathfrak{f}$
will be denoted by $\mathfrak{f}_{M}$.

\textbf{(b)} Let $M$ and $N$ be two Frobenius $\mathbb{F}_{q}\left[  T\right]
$-modules. Then, a map $h:M\rightarrow N$ is said to be a \textit{homomorphism
of Frobenius }$\mathbb{F}_{q}\left[  T\right]  $\textit{-modules} if and only
if it is $\mathbb{F}_{q}\left[  T\right]  $-linear and \textquotedblleft
respects the Frobenius maps\textquotedblright\ (i.e., satisfies $\mathfrak{f}%
_{N}\circ h=h\circ\mathfrak{f}_{M}$).

\textbf{(c)} We let $\operatorname*{FrobMod}\nolimits_{\mathbb{F}_{q}\left[
T\right]  }$ denote the category whose objects are the Frobenius
$\mathbb{F}_{q}\left[  T\right]  $-modules, and whose morphisms are the
homomorphisms of Frobenius $\mathbb{F}_{q}\left[  T\right]  $-modules.
\end{definition}

It turns out that this category $\operatorname*{FrobMod}\nolimits_{\mathbb{F}%
_{q}\left[  T\right]  }$ is isomorphic to the category of $\mathcal{F}$-modules:

\begin{proposition}
\label{prop.F.frobmod.cateq}Let $\operatorname*{Mod}\nolimits_{\mathcal{F}}$
be the category of all (left) $\mathcal{F}$-modules.

Recall that we are regarding the $\mathbb{F}_{q}$-algebra homomorphism
$\operatorname*{Finc}\nolimits_{T}:\mathbb{F}_{q}\left[  T\right]
\rightarrow\mathcal{F}$ as an inclusion. Thus, $\mathbb{F}_{q}\left[
T\right]  $ is an $\mathbb{F}_{q}$-subalgebra of $\mathcal{F}$.

\textbf{(a)} Let $M$ be a Frobenius $\mathbb{F}_{q}\left[  T\right]  $-module.
Then, there exists a unique $\mathcal{F}$-module structure on $M$ which
extends the $\mathbb{F}_{q}\left[  T\right]  $-module structure on $M$ and
satisfies%
\[
F\cdot m=\mathfrak{f}_{M}\left(  m\right)  \ \ \ \ \ \ \ \ \ \ \text{for every
}m\in M.
\]


\textbf{(b)} Let $N$ be an $\mathcal{F}$-module. Then, $N$ becomes an
$\mathbb{F}_{q}\left[  T\right]  $-module (since $\mathbb{F}_{q}\left[
T\right]  \subseteq\mathcal{F}$). Let $\mathfrak{f}$ be the action of
$F\in\mathcal{F}$ on $N$ (that is, the $\mathbb{F}_{q}$-linear map
$N\rightarrow N,\ n\mapsto F\cdot n$). Then, $\left(  N,\mathfrak{f}\right)  $
is a Frobenius $\mathbb{F}_{q}\left[  T\right]  $-module.

\textbf{(c)} Proposition \ref{prop.F.frobmod.cateq} \textbf{(a)} defines a
functor from $\operatorname*{FrobMod}\nolimits_{\mathbb{F}_{q}\left[
T\right]  }$ to $\operatorname*{Mod}\nolimits_{\mathcal{F}}$ (because, to any
Frobenius $\mathbb{F}_{q}\left[  T\right]  $-module $M$, it assigns an
$\mathcal{F}$-module structure on $M$, and this assignment can easily be
extended to morphisms). Proposition \ref{prop.F.frobmod.cateq} \textbf{(b)}
defines a functor from $\operatorname*{Mod}\nolimits_{\mathcal{F}}$ to
$\operatorname*{FrobMod}\nolimits_{\mathbb{F}_{q}\left[  T\right]  }$
(because, to any $\mathcal{F}$-module $N$, it assigns a Frobenius
$\mathbb{F}_{q}\left[  T\right]  $-module $\left(  N,\mathfrak{f}\right)  $,
and this assignment can easily be extended to morphisms). These two functors
are mutually inverse. Thus, the categories $\operatorname*{FrobMod}%
\nolimits_{\mathbb{F}_{q}\left[  T\right]  }$ and $\operatorname*{Mod}%
\nolimits_{\mathcal{F}}$ are isomorphic.
\end{proposition}

\begin{proof}
[Proof of Proposition \ref{prop.F.frobmod.cateq}.]\textbf{(a)} We let
$\operatorname*{End}M$ denote the $\mathbb{F}_{q}$-algebra of all
$\mathbb{F}_{q}$-module endomorphisms of $M$.

It is clear that there exists \textbf{at most one} $\mathcal{F}$-module
structure on $M$ which extends the $\mathbb{F}_{q}\left[  T\right]  $-module
structure on $M$ and satisfies%
\begin{equation}
F\cdot m=\mathfrak{f}_{M}\left(  m\right)  \ \ \ \ \ \ \ \ \ \ \text{for every
}m\in M \label{pf.prop.F.frobmod.cateq.a.want}%
\end{equation}
\footnote{Indeed, the requirement that this structure extends the
$\mathbb{F}_{q}\left[  T\right]  $-module structure on $M$ uniquely determines
how $T$ acts on $M$. Meanwhile, the requirement
(\ref{pf.prop.F.frobmod.cateq.a.want}) uniquely determines how $F$ acts on
$M$. Thus, the actions of both $T$ and $F$ on $M$ are uniquely determined. But
therefore, the action of any element of $\mathcal{F}$ on $M$ is uniquely
determined as well (since the $\mathbb{F}_{q}$-algebra $\mathcal{F}$ is
generated by $T$ and $F$); in other words, the $\mathcal{F}$-module structure
on $M$ is uniquely determined, qed.}. It thus remains to prove that there
exists \textbf{at least one} such structure. So let us construct such a structure.

As usual, we abbreviate $\mathfrak{f}_{M}$ as $\mathfrak{f}$.

Let $\mathfrak{t}$ be the $\mathbb{F}_{q}$-linear map $M\rightarrow
M,\ m\mapsto T\cdot m$. Then, for every $n\in\mathbb{N}$ and $m\in M$, we have%
\begin{equation}
\mathfrak{t}^{n}\left(  m\right)  =T^{n}\cdot m.
\label{pf.prop.F.frobmod.cateq.a.1}%
\end{equation}
(This is easy to prove by induction over $n$.)

For every $m\in M$, we have%
\begin{align*}
\left(  \mathfrak{f}\circ\mathfrak{t}\right)  \left(  m\right)   &
=\mathfrak{f}\left(  \underbrace{\mathfrak{t}\left(  m\right)  }%
_{\substack{=T\cdot m\\\text{(by the definition of }\mathfrak{t}\text{)}%
}}\right)  =\mathfrak{f}\left(  T\cdot m\right)  =\mathfrak{f}\left(
Tm\right)  =T^{q}\mathfrak{f}\left(  m\right)  \ \ \ \ \ \ \ \ \ \ \left(
\text{by (\ref{eq.def.F.frobmod.axiom})}\right) \\
&  =\mathfrak{t}^{q}\left(  \mathfrak{f}\left(  m\right)  \right) \\
&  \ \ \ \ \ \ \ \ \ \ \left(
\begin{array}
[c]{c}%
\text{because (\ref{pf.prop.F.frobmod.cateq.a.1}) (applied to }q\text{ and
}\mathfrak{f}\left(  m\right)  \text{ instead of }n\text{ and }m\text{)}\\
\text{shows that }\mathfrak{t}^{q}\left(  \mathfrak{f}\left(  m\right)
\right)  =T^{q}\cdot\mathfrak{f}\left(  m\right)  =T^{q}\mathfrak{f}\left(
m\right)
\end{array}
\right) \\
&  =\left(  \mathfrak{t}^{q}\circ\mathfrak{f}\right)  \left(  m\right)  .
\end{align*}
Hence, $\mathfrak{f}\circ\mathfrak{t}=\mathfrak{t}^{q}\circ\mathfrak{f}$.

Now, recall the universal property of $\mathcal{F}$: If $u$ and $v$ are two
elements of an $\mathbb{F}_{q}$-algebra $\mathcal{U}$ satisfying $uv=v^{q}u$,
then there exists a unique $\mathbb{F}_{q}$-algebra homomorphism
$\mathcal{F}\rightarrow\mathcal{U}$ sending $F$ and $T$ to $u$ and $v$,
respectively. Applying this to $\mathcal{U}=\operatorname*{End}M$,
$u=\mathfrak{f}$ and $v=\mathfrak{t}$, we conclude that there exists a unique
$\mathbb{F}_{q}$-algebra homomorphism $\mathcal{F}\rightarrow
\operatorname*{End}M$ sending $F$ and $T$ to $\mathfrak{f}$ and $\mathfrak{t}%
$, respectively. Let $\Phi$ be this homomorphism. The definition of $\Phi$
shows that $\Phi\left(  F\right)  =\mathfrak{f}$ and $\Phi\left(  T\right)
=\mathfrak{t}$.

We have
\begin{equation}
\left(  \Phi\left(  f\right)  \right)  \left(  m\right)  =f\cdot
m\ \ \ \ \ \ \ \ \ \ \text{for every }f\in\mathbb{F}_{q}\left[  T\right]
\text{ and }m\in M \label{pf.prop.F.frobmod.cateq.a.3}%
\end{equation}
\footnote{\textit{Proof of (\ref{pf.prop.F.frobmod.cateq.a.1}):} Let
$f\in\mathbb{F}_{q}\left[  T\right]  $ and $m\in M$. We have to prove the
equality $\left(  \Phi\left(  f\right)  \right)  \left(  m\right)  =f\cdot m$.
This equality is $\mathbb{F}_{q}$-linear in $f$; we can therefore WLOG assume
that $f$ belongs to the basis $\left(  T^{n}\right)  _{n\geq0}$ of the
$\mathbb{F}_{q}$-module $\mathbb{F}_{q}\left[  T\right]  $. Assume this.
Hence, $f=T^{n}$ for some $n\in\mathbb{N}$. Consider this $n$. From $f=T^{n}$,
we obtain $\Phi\left(  f\right)  =\Phi\left(  T^{n}\right)  =\left(
\Phi\left(  T\right)  \right)  ^{n}$ (since $\Phi$ is an $\mathbb{F}_{q}%
$-algebra homomorphism). Since $\Phi\left(  T\right)  =\mathfrak{t}$, this
rewrites as $\Phi\left(  f\right)  =\mathfrak{t}^{n}$. Therefore,
$\underbrace{\left(  \Phi\left(  f\right)  \right)  }_{=\mathfrak{t}^{n}%
}\left(  m\right)  =\mathfrak{t}^{n}\left(  m\right)  =T^{n}\cdot m$ (by
(\ref{pf.prop.F.frobmod.cateq.a.1})). Hence, $\left(  \Phi\left(  f\right)
\right)  \left(  m\right)  =\underbrace{T^{n}}_{=f}\cdot m=f\cdot m$. This
proves (\ref{pf.prop.F.frobmod.cateq.a.1}).}. Thus, the $\mathcal{F}$-module
structure on $M$ obtained from the map $\Phi:\mathcal{F}\rightarrow
\operatorname*{End}M$ extends the $\mathbb{F}_{q}\left[  T\right]  $-module
structure on $M$.

Furthermore, $\underbrace{\left(  \Phi\left(  F\right)  \right)
}_{=\mathfrak{f}=\mathfrak{f}_{M}}\left(  m\right)  =\mathfrak{f}_{M}\left(
m\right)  $ for every $m\in M$. Thus, the $\mathcal{F}$-module structure on
$M$ obtained from the map $\Phi:\mathcal{F}\rightarrow\operatorname*{End}M$
satisfies (\ref{pf.prop.F.frobmod.cateq.a.want}).

Hence, there exists at least one $\mathcal{F}$-module structure on $M$ which
extends the $\mathbb{F}_{q}\left[  T\right]  $-module structure on $M$ and
satisfies (\ref{pf.prop.F.frobmod.cateq.a.want}) (namely, the $\mathcal{F}%
$-module structure on $M$ obtained from the map $\Phi:\mathcal{F}%
\rightarrow\operatorname*{End}M$). This completes the proof of Proposition
\ref{prop.F.frobmod.cateq} \textbf{(a)}.

\textbf{(b)} We need to show that $\left(  N,\mathfrak{f}\right)  $ is a
Frobenius $\mathbb{F}_{q}\left[  T\right]  $-module. In other words, we need
to show that $N$ is an $\mathbb{F}_{q}\left[  T\right]  $-module, that
$\mathfrak{f}:N\rightarrow N$ is an $\mathbb{F}_{q}$-linear map, and that this
map $\mathfrak{f}$ satisfies%
\begin{equation}
\mathfrak{f}\left(  Tm\right)  =T^{q}\mathfrak{f}\left(  m\right)
\ \ \ \ \ \ \ \ \ \ \text{for every }m\in N.
\label{pf.prop.F.frobmod.cateq.b.want}%
\end{equation}


The first two of these statements are obvious. It thus remains to prove the
third statement, i.e., to prove that the map $\mathfrak{f}$ satisfies
(\ref{pf.prop.F.frobmod.cateq.b.want}).

So let $m\in N$. The definition of $\mathfrak{f}$ yields $\mathfrak{f}\left(
m\right)  =Fm$ and $\mathfrak{f}\left(  Tm\right)  =F\cdot Tm=\underbrace{FT}%
_{=T^{q}F}m=T^{q}\underbrace{Fm}_{=\mathfrak{f}\left(  m\right)  }%
=T^{q}\mathfrak{f}\left(  m\right)  $. Thus,
(\ref{pf.prop.F.frobmod.cateq.b.want}) is proven. As we have already
explained, this completes the proof of Proposition \ref{prop.F.frobmod.cateq}
\textbf{(b)}.

\textbf{(c)} It is clear that if we apply the functor $\operatorname*{FrobMod}%
\nolimits_{\mathbb{F}_{q}\left[  T\right]  }\rightarrow\operatorname*{Mod}%
\nolimits_{\mathcal{F}}$ first and then the functor $\operatorname*{Mod}%
\nolimits_{\mathcal{F}}\rightarrow\operatorname*{FrobMod}\nolimits_{\mathbb{F}%
_{q}\left[  T\right]  }$, then we get back to where we started. It is somewhat
less obvious, but still easy, to prove that if we apply the functor
$\operatorname*{Mod}\nolimits_{\mathcal{F}}\rightarrow\operatorname*{FrobMod}%
\nolimits_{\mathbb{F}_{q}\left[  T\right]  }$ first and then the functor
$\operatorname*{FrobMod}\nolimits_{\mathbb{F}_{q}\left[  T\right]
}\rightarrow\operatorname*{Mod}\nolimits_{\mathcal{F}}$, then we get back to
where we started\footnote{In order to prove this, it suffices to observe that
an $\mathcal{F}$-module structure on a given $\mathbb{F}_{q}$-vector space is
uniquely determined by the actions of $F$ and $T$ (because the $\mathbb{F}%
_{q}$-algebra $\mathcal{F}$ is generated by $F$ and $T$).}. Thus, the functors
$\operatorname*{FrobMod}\nolimits_{\mathbb{F}_{q}\left[  T\right]
}\rightarrow\operatorname*{Mod}\nolimits_{\mathcal{F}}$ and
$\operatorname*{Mod}\nolimits_{\mathcal{F}}\rightarrow\operatorname*{FrobMod}%
\nolimits_{\mathbb{F}_{q}\left[  T\right]  }$ are mutually inverse. This
proves Proposition \ref{prop.F.frobmod.cateq} \textbf{(c)}.
\end{proof}

An ample supply of Frobenius $\mathbb{F}_{q}\left[  T\right]  $-modules (and
thus, $\mathcal{F}$-module) is given by commutative $\mathbb{F}_{q}\left[
T\right]  $-algebras and their Frobenius homomorphisms:

\begin{proposition}
\label{prop.F.frobmod.alg}\textbf{(a)} If $A$ is a commutative $\mathbb{F}%
_{q}\left[  T\right]  $-algebra, then $\left(  A,\operatorname*{Frob}%
\nolimits_{A}\right)  $ is a Frobenius $\mathbb{F}_{q}\left[  T\right]  $-module.

\textbf{(b)} If $A$ and $B$ are two commutative $\mathbb{F}_{q}\left[
T\right]  $-algebras, and if $f:A\rightarrow B$ is an $\mathbb{F}_{q}\left[
T\right]  $-algebra homomorphism, then $f$ is also a homomorphism of Frobenius
$\mathbb{F}_{q}\left[  T\right]  $-modules from $\left(
A,\operatorname*{Frob}\nolimits_{A}\right)  $ to $\left(
B,\operatorname*{Frob}\nolimits_{B}\right)  $.

\textbf{(c)} Proposition \ref{prop.F.frobmod.alg} \textbf{(a)} assigns a
Frobenius $\mathbb{F}_{q}\left[  T\right]  $-module $\left(
A,\operatorname*{Frob}\nolimits_{A}\right)  $ to each commutative
$\mathbb{F}_{q}\left[  T\right]  $-algebra $A$. This defines a functor from
the category of commutative $\mathbb{F}_{q}\left[  T\right]  $-algebras to the
category $\operatorname*{FrobMod}\nolimits_{\mathbb{F}_{q}\left[  T\right]  }$
of Frobenius $\mathbb{F}_{q}\left[  T\right]  $-modules (the action of this
functor on morphisms just leaves morphisms unchanged), and thus to the
category $\operatorname*{Mod}\nolimits_{\mathcal{F}}$ of $\mathcal{F}$-modules
(because Proposition \ref{prop.F.frobmod.cateq} \textbf{(c)} shows that
$\operatorname*{FrobMod}\nolimits_{\mathbb{F}_{q}\left[  T\right]  }%
\cong\operatorname*{Mod}\nolimits_{\mathcal{F}}$). Explicitly, this shows that
every commutative $\mathbb{F}_{q}\left[  T\right]  $-algebra $A$ canonically
becomes an $\mathcal{F}$-module, and this $\mathcal{F}$-module structure
extends the $\mathbb{F}_{q}\left[  T\right]  $-module structure on $A$ and has
the property that
\[
F\cdot m=\operatorname*{Frob}\nolimits_{A}\left(  m\right)
\ \ \ \ \ \ \ \ \ \ \text{for every }m\in A.
\]

\end{proposition}

\begin{proof}
[Proof of Proposition \ref{prop.F.frobmod.alg}.]\textbf{(a)} Let $A$ be a
commutative $\mathbb{F}_{q}\left[  T\right]  $-algebra. As we know,
$\operatorname*{Frob}\nolimits_{A}:A\rightarrow A$ is an $\mathbb{F}_{q}%
$-algebra homomorphism, and thus an $\mathbb{F}_{q}$-linear map. Furthermore,
it satisfies%
\[
\operatorname*{Frob}\nolimits_{A}\left(  Tm\right)  =T^{q}\operatorname*{Frob}%
\nolimits_{A}\left(  m\right)
\]
for every $m\in A$\ \ \ \ \footnote{\textit{Proof.} Let $m\in A$. Then, the
definition of $\operatorname*{Frob}\nolimits_{A}$ shows that
$\operatorname*{Frob}\nolimits_{A}\left(  m\right)  =m^{q}$ and
$\operatorname*{Frob}\nolimits_{A}\left(  Tm\right)  =\left(  Tm\right)
^{q}=T^{q}\underbrace{m^{q}}_{=\operatorname*{Frob}\nolimits_{A}\left(
m\right)  }=T^{q}\operatorname*{Frob}\nolimits_{A}\left(  m\right)  $, qed.}.
Hence, $\left(  A,\operatorname*{Frob}\nolimits_{A}\right)  $ is a Frobenius
$\mathbb{F}_{q}\left[  T\right]  $-module (by the definition of a
\textquotedblleft Frobenius $\mathbb{F}_{q}\left[  T\right]  $%
-module\textquotedblright). This proves Proposition \ref{prop.F.frobmod.alg}
\textbf{(a)}.

\textbf{(b)} The proof of Proposition \ref{prop.F.frobmod.alg} \textbf{(b)} is straightforward.

\textbf{(c)} Proposition \ref{prop.F.frobmod.alg} \textbf{(c)} follows from
what we have proven above. (Specifically, the statement that the $\mathcal{F}%
$-module structure on $A$ extends the $\mathbb{F}_{q}\left[  T\right]
$-module structure on $A$ and has the property that
\[
F\cdot m=\operatorname*{Frob}\nolimits_{A}\left(  m\right)
\ \ \ \ \ \ \ \ \ \ \text{for every }m\in A
\]
is a consequence of Proposition \ref{prop.F.frobmod.cateq} \textbf{(a)}.)
\end{proof}

Restricted Lie algebras (see, e.g., \cite{jacobson-rl}) can be used as another
source of Frobenius $\mathbb{F}_{q}\left[  T\right]  $-modules, provided they
can be equipped with an appropriate $\mathbb{F}_{q}\left[  T\right]  $-module
structure. We are not currently aware of specific examples of interest, however.

\begin{condition}
\label{conv.F.acts-on-commalg}Let $A$ be a commutative $\mathbb{F}_{q}\left[
T\right]  $-algebra. Then, $\left(  A,\operatorname*{Frob}\nolimits_{A}%
\right)  $ is a Frobenius $\mathbb{F}_{q}\left[  T\right]  $-module (by
Proposition \ref{prop.F.frobmod.alg} \textbf{(a)}), and thus Proposition
\ref{prop.F.frobmod.cateq} \textbf{(a)} (applied to $M=A$) defines an
$\mathcal{F}$-module structure on $A$. In the following, we shall always
regard a commutative $\mathbb{F}_{q}\left[  T\right]  $-algebra $A$ as
equipped with this $\mathcal{F}$-module structure by default. This structure
extends the $\mathbb{F}_{q}\left[  T\right]  $-module structure on $A$, and
satisfies%
\begin{equation}
F\cdot m=\operatorname*{Frob}\nolimits_{A}\left(  m\right)  =m^{q}%
\ \ \ \ \ \ \ \ \ \ \left(  \text{by the definition of }\operatorname*{Frob}%
\nolimits_{A}\right)  \label{eq.conv.F.acts-on-commalg.F}%
\end{equation}
for every $m\in A$.
\end{condition}

\begin{proposition}
\label{prop.F.acts-on-commalg.Fk}Let $A$ be a commutative $\mathbb{F}%
_{q}\left[  T\right]  $-algebra. Then, $A$ is an $\mathcal{F}$-module
(according to Convention \ref{conv.F.acts-on-commalg}). This $\mathcal{F}%
$-module structure has the following property: For every $k\in\mathbb{N}$ and
$m\in A$, we have%
\begin{equation}
F^{k}\cdot m=m^{q^{k}}. \label{eq.prop.F.acts-on-commalg.Fk.eq}%
\end{equation}

\end{proposition}

\begin{proof}
[Proof of Proposition \ref{prop.F.acts-on-commalg.Fk}.]Only
(\ref{eq.prop.F.acts-on-commalg.Fk.eq}) needs to be proven.

From (\ref{eq.conv.F.acts-on-commalg.F}), we know that%
\begin{equation}
F\cdot m=m^{q}\ \ \ \ \ \ \ \ \ \ \text{for every }m\in A.
\label{pf.prop.F.acts-on-commalg.Fk.1}%
\end{equation}
Thus,%
\begin{equation}
F^{k}\cdot m=m^{q^{k}}\ \ \ \ \ \ \ \ \ \ \text{for every }m\in A\text{ and
}k\in\mathbb{N}. \label{pf.prop.F.acts-on-commalg.Fk.2}%
\end{equation}
(Indeed, (\ref{pf.prop.F.acts-on-commalg.Fk.2}) can be proven by a
straightforward induction over $k$; the induction step will rely on
(\ref{pf.prop.F.acts-on-commalg.Fk.1}). The details of this proof are left to
the reader.)

So we know that (\ref{pf.prop.F.acts-on-commalg.Fk.2}) holds. In other words,
(\ref{eq.prop.F.acts-on-commalg.Fk.eq}) holds. This proves Proposition
\ref{prop.F.acts-on-commalg.Fk}.
\end{proof}

\begin{proposition}
\label{prop.F.Fqpol.Fmodhom}The commutative $\mathbb{F}_{q}\left[  T\right]
$-algebra $\mathbb{F}_{q}\left[  T\right]  \left[  X\right]  $ becomes an
$\mathcal{F}$-module (by Convention \ref{conv.F.acts-on-commalg}, applied to
$A=\mathbb{F}_{q}\left[  T\right]  \left[  X\right]  $). Let $\overline
{\operatorname*{Fqpol}}$ denote the map $\operatorname*{Fqpol}:\mathcal{F}%
\rightarrow\mathbb{F}_{q}\left[  T\right]  \left[  X\right]
_{q-\operatorname*{lin}}$, considered as a map $\mathcal{F}\rightarrow
\mathbb{F}_{q}\left[  T\right]  \left[  X\right]  $ (this is well-defined
because $\mathbb{F}_{q}\left[  T\right]  \left[  X\right]
_{q-\operatorname*{lin}}\subseteq\mathbb{F}_{q}\left[  T\right]  \left[
X\right]  $). Then, this map $\overline{\operatorname*{Fqpol}}:\mathcal{F}%
\rightarrow\mathbb{F}_{q}\left[  T\right]  \left[  X\right]  $ is an
$\mathcal{F}$-module homomorphism.
\end{proposition}

\begin{proof}
[Proof of Proposition \ref{prop.F.Fqpol.Fmodhom}.]Proposition
\ref{prop.q-pol.Fqlin.leftT} shows that the map $\operatorname*{Fqpol}%
:\mathcal{F}\rightarrow\mathbb{F}_{q}\left[  T\right]  \left[  X\right]
_{q-\operatorname*{lin}}$ is an isomorphism of left $\mathbb{F}_{q}\left[
T\right]  $-modules. Thus, the map $\overline{\operatorname*{Fqpol}%
}:\mathcal{F}\rightarrow\mathbb{F}_{q}\left[  T\right]  \left[  X\right]  $
(which differs from $\operatorname*{Fqpol}:\mathcal{F}\rightarrow
\mathbb{F}_{q}\left[  T\right]  \left[  X\right]  _{q-\operatorname*{lin}}$
only in its target) is also a homomorphism of left $\mathbb{F}_{q}\left[
T\right]  $-modules. In other words, $\overline{\operatorname*{Fqpol}}\left(
fu\right)  =f\overline{\operatorname*{Fqpol}}\left(  u\right)  $ for every
$f\in\mathbb{F}_{q}\left[  T\right]  $ and $u\in\mathcal{F}$. Applying this to
$f=T$, we obtain%
\begin{equation}
\overline{\operatorname*{Fqpol}}\left(  Tu\right)  =T\overline
{\operatorname*{Fqpol}}\left(  u\right)  \ \ \ \ \ \ \ \ \ \ \text{for every
}u\in\mathcal{F}. \label{pf.prop.F.Fqpol.Fmodhom.1}%
\end{equation}


On the other hand, let $u\in\mathcal{F}$. Then,%
\begin{align*}
&  \overline{\operatorname*{Fqpol}}\left(  Fu\right) \\
&  =\operatorname*{Fqpol}\left(  Fu\right)  \ \ \ \ \ \ \ \ \ \ \left(
\text{by the definition of }\overline{\operatorname*{Fqpol}}\right) \\
&  =\underbrace{\left(  \operatorname*{Fqpol}\left(  F\right)  \right)
}_{=X^{q}}\circ\left(  \operatorname*{Fqpol}\left(  u\right)  \right) \\
&  \ \ \ \ \ \ \ \ \ \ \left(  \text{since }\operatorname*{Fqpol}\text{ is an
}\mathbb{F}_{q}\text{-algebra homomorphism }\mathcal{F}\rightarrow\left(
\mathbb{F}_{q}\left[  T\right]  \left[  X\right]  _{q-\operatorname*{lin}%
},+,\circ\right)  \right) \\
&  =X^{q}\circ\left(  \operatorname*{Fqpol}\left(  u\right)  \right)  =\left(
\operatorname*{Fqpol}\left(  u\right)  \right)  ^{q}.
\end{align*}
Comparing this with%
\begin{align*}
F\overline{\operatorname*{Fqpol}}\left(  u\right)   &  =F\cdot\overline
{\operatorname*{Fqpol}}\left(  u\right)  =\left(  \underbrace{\overline
{\operatorname*{Fqpol}}\left(  u\right)  }_{\substack{=\operatorname*{Fqpol}%
\left(  u\right)  \\\text{(by the definition of }\overline
{\operatorname*{Fqpol}}\text{)}}}\right)  ^{q}\\
&  \ \ \ \ \ \ \ \ \ \ \left(  \text{by (\ref{eq.conv.F.acts-on-commalg.F}),
applied to }A=\mathbb{F}_{q}\left[  T\right]  \left[  X\right]  \text{ and
}m=\overline{\operatorname*{Fqpol}}\left(  u\right)  \right) \\
&  =\left(  \operatorname*{Fqpol}\left(  u\right)  \right)  ^{q},
\end{align*}
we obtain $\overline{\operatorname*{Fqpol}}\left(  Fu\right)  =F\overline
{\operatorname*{Fqpol}}\left(  u\right)  $. Let us now forget that we fixed
$u$. We thus have shown that
\begin{equation}
\overline{\operatorname*{Fqpol}}\left(  Fu\right)  =F\overline
{\operatorname*{Fqpol}}\left(  u\right)  \ \ \ \ \ \ \ \ \ \ \text{for every
}u\in\mathcal{F}. \label{pf.prop.F.Fqpol.Fmodhom.2}%
\end{equation}
Now, Lemma \ref{lem.F.modhom} (applied to $M=\mathcal{F}$, $N=\mathbb{F}%
_{q}\left[  T\right]  \left[  X\right]  $ and $f=\overline
{\operatorname*{Fqpol}}$) shows that $\overline{\operatorname*{Fqpol}}$ is an
$\mathcal{F}$-module homomorphism (because of (\ref{pf.prop.F.Fqpol.Fmodhom.1}%
) and (\ref{pf.prop.F.Fqpol.Fmodhom.2})). This proves Proposition
\ref{prop.F.Fqpol.Fmodhom}.
\end{proof}

\subsection{The Carlitz action}

Now, let us recall the Carlitz polynomials $\left[  M\right]  $ defined in
Definition \ref{def.carlitzpoly}. We can connect these polynomials to
$\mathcal{F}$ in the following way\footnote{Recall that $\operatorname*{Carl}$
is the $\mathbb{F}_{q}$-algebra homomorphism $\mathbb{F}_{q}\left[  T\right]
\rightarrow\mathcal{F}$ sending $T$ to $F+T$.}:

\begin{proposition}
\label{prop.F.carlitz}Let $A$ be a commutative $\mathbb{F}_{q}\left[
T\right]  $-algebra. Thus, $A$ becomes an $\mathcal{F}$-module (by Convention
\ref{conv.F.acts-on-commalg}).

For every $M\in\mathbb{F}_{q}\left[  T\right]  $ and $a\in A$, we have
$\left[  M\right]  \left(  a\right)  =\left(  \operatorname*{Carl}M\right)
\cdot a$. (Here, the $\left[  M\right]  \left(  a\right)  $ on the left hand
side means the result of substituting $a$ for $X$ in the polynomial $\left[
M\right]  \in\mathbb{F}_{q}\left[  T\right]  \left[  X\right]  $, whereas the
$\left(  \operatorname*{Carl}M\right)  \cdot a$ on the right hand side denotes
the action of $\operatorname*{Carl}M\in\mathcal{F}$ on $a\in A$.)
\end{proposition}

\begin{proof}
[Proof of Proposition \ref{prop.F.carlitz}.]We first claim that%
\begin{equation}
\left[  T^{n}\right]  \left(  a\right)  =\left(  F+T\right)  ^{n}%
a\ \ \ \ \ \ \ \ \ \ \text{for every }n\in\mathbb{N}\text{ and }a\in A.
\label{pf.prop.F.carlitz.1}%
\end{equation}


\textit{Proof of (\ref{pf.prop.F.carlitz.1}):} We shall prove
(\ref{pf.prop.F.carlitz.1}) by induction over $n$:

\textit{Induction base:} We have $\left[  T^{0}\right]  =X$, thus $\left[
T^{0}\right]  \left(  a\right)  =X\left(  a\right)  =a$. Comparing this with
$\underbrace{\left(  F+T\right)  ^{0}}_{=1}a=a$, we obtain $\left[
T^{0}\right]  \left(  a\right)  =\left(  F+T\right)  ^{0}a$. In other words,
(\ref{pf.prop.F.carlitz.1}) holds for $n=0$. This completes the induction base.

\textit{Induction step:} Fix a positive integer $N$. Assume that
(\ref{pf.prop.F.carlitz.1}) holds for $n=N-1$. We now need to show that
(\ref{pf.prop.F.carlitz.1}) holds for $n=N$.

We have assumed that (\ref{pf.prop.F.carlitz.1}) holds for $n=N-1$. In other
words, we have%
\begin{equation}
\left[  T^{N-1}\right]  \left(  a\right)  =\left(  F+T\right)  ^{N-1}%
a\ \ \ \ \ \ \ \ \ \ \text{for every }a\in A. \label{pf.prop.F.carlitz.1.pf.1}%
\end{equation}


Now, fix $a\in A$. Applying (\ref{eq.conv.F.acts-on-commalg.F}) to $m=\left[
T^{N-1}\right]  \left(  a\right)  $, we obtain%
\begin{equation}
F\cdot\left[  T^{N-1}\right]  \left(  a\right)  =\left(  \left[
T^{N-1}\right]  \left(  a\right)  \right)  ^{q}.
\label{pf.prop.F.carlitz.1.pf.2}%
\end{equation}


The recursive definition of $\left[  T^{N}\right]  $ yields $\left[
T^{N}\right]  =\left[  T^{N-1}\right]  ^{q}+T\left[  T^{N-1}\right]  $. Hence,%
\begin{align*}
\left[  T^{N}\right]  \left(  a\right)   &  =\left(  \left[  T^{N-1}\right]
^{q}+T\left[  T^{N-1}\right]  \right)  \left(  a\right)  =\underbrace{\left(
\left[  T^{N-1}\right]  \left(  a\right)  \right)  ^{q}}_{=F\cdot\left[
T^{N-1}\right]  \left(  a\right)  }+T\left[  T^{N-1}\right]  \left(  a\right)
\\
&  =F\cdot\left[  T^{N-1}\right]  \left(  a\right)  +T\cdot\left[
T^{N-1}\right]  \left(  a\right)  =\left(  F+T\right)  \underbrace{\left[
T^{N-1}\right]  \left(  a\right)  }_{\substack{=\left(  F+T\right)
^{N-1}a\\\text{(by (\ref{pf.prop.F.carlitz.1.pf.1}))}}}\\
&  =\underbrace{\left(  F+T\right)  \left(  F+T\right)  ^{N-1}}_{=\left(
F+T\right)  ^{N}}a=\left(  F+T\right)  ^{N}a.
\end{align*}
Now, let us forget that we fixed $a$. We thus have shown that $\left[
T^{N}\right]  \left(  a\right)  =\left(  F+T\right)  ^{N}a$ for every $a\in
A$. In other words, (\ref{pf.prop.F.carlitz.1}) holds for $n=N$. This
completes the induction step, and thus (\ref{pf.prop.F.carlitz.1}) is proven.

Now, let $M\in\mathbb{F}_{q}\left[  T\right]  $ and $a\in A$. Write the
polynomial $M$ in the form $M=a_{0}T^{0}+a_{1}T^{1}+\cdots+a_{k}T^{k}$ for
some $k\in\mathbb{N}$ and $a_{0},a_{1},\ldots,a_{k}\in\mathbb{F}_{q}$. Thus,%
\[
M=a_{0}T^{0}+a_{1}T^{1}+\cdots+a_{k}T^{k}=\sum_{n=0}^{k}a_{n}T^{n}.
\]
The definition of $\left[  M\right]  $ now yields
\[
\left[  M\right]  =a_{0}\left[  T^{0}\right]  +a_{1}\left[  T^{1}\right]
+\cdots+a_{k}\left[  T^{k}\right]  =\sum_{n=0}^{k}a_{n}\left[  T^{n}\right]
.
\]


Recall that $\operatorname*{Carl}$ is the $\mathbb{F}_{q}$-algebra
homomorphism $\mathbb{F}_{q}\left[  T\right]  \rightarrow\mathcal{F}$ sending
$T$ to $F+T$. Thus, $\operatorname*{Carl}T=F+T$. The map $\operatorname*{Carl}%
$ commutes with applications of polynomials in $\mathbb{F}_{q}\left[
T\right]  $ (since it is an $\mathbb{F}_{q}$-algebra homomorphism). Thus,%
\[
\operatorname*{Carl}\left(  M\left(  T\right)  \right)  =M\left(
\underbrace{\operatorname*{Carl}T}_{=F+T}\right)  =M\left(  F+T\right)
=\sum_{n=0}^{k}a_{n}\left(  F+T\right)  ^{n}%
\]
(since $M=\sum_{n=0}^{k}a_{n}T^{n}$). Since $M\left(  T\right)  =M$, this
rewrites as%
\[
\operatorname*{Carl}M=\sum_{n=0}^{k}a_{n}\left(  F+T\right)  ^{n}.
\]
Hence,%
\begin{align*}
\left(  \operatorname*{Carl}M\right)  \cdot a  &  =\left(  \sum_{n=0}^{k}%
a_{n}\left(  F+T\right)  ^{n}\right)  \cdot a=\sum_{n=0}^{k}a_{n}%
\underbrace{\left(  F+T\right)  ^{n}a}_{\substack{=\left[  T^{n}\right]
\left(  a\right)  \\\text{(by (\ref{pf.prop.F.carlitz.1}))}}}\\
&  =\sum_{n=0}^{k}a_{n}\left[  T^{n}\right]  \left(  a\right)
=\underbrace{\left(  \sum_{n=0}^{k}a_{n}\left[  T^{n}\right]  \right)
}_{=\left[  M\right]  }\left(  a\right)  =\left[  M\right]  \left(  a\right)
.
\end{align*}
This proves Proposition \ref{prop.F.carlitz}.
\end{proof}

\begin{corollary}
\label{cor.F.carlitz.img}Let $M\in\mathbb{F}_{q}\left[  T\right]  $. Then, the
homomorphism $\operatorname*{Fqpol}:\mathcal{F}\rightarrow\mathbb{F}%
_{q}\left[  T\right]  \left[  X\right]  _{q-\operatorname*{lin}}$ satisfies
$\left[  M\right]  =\operatorname*{Fqpol}\left(  \operatorname*{Carl}M\right)
$.
\end{corollary}

Corollary \ref{cor.F.carlitz.img} yields, in particular, that every
$M\in\mathbb{F}_{q}\left[  T\right]  $ satisfies $\left[  M\right]
=\operatorname*{Fqpol}\left(  \operatorname*{Carl}M\right)  \in
\operatorname*{Fqpol}\mathcal{F}\subseteq\mathbb{F}_{q}\left[  T\right]
\left[  X\right]  _{q-\operatorname*{lin}}$.

\begin{proof}
[Proof of Corollary \ref{cor.F.carlitz.img}.]Let $M\in\mathbb{F}_{q}\left[
T\right]  $.

Consider the map $\overline{\operatorname*{Fqpol}}:\mathcal{F}\rightarrow
\mathbb{F}_{q}\left[  T\right]  \left[  X\right]  $ defined in Proposition
\ref{prop.F.Fqpol.Fmodhom}. This map $\overline{\operatorname*{Fqpol}}$ is an
$\mathcal{F}$-module homomorphism (according to Proposition
\ref{prop.F.Fqpol.Fmodhom}).

The definition of $\overline{\operatorname*{Fqpol}}$ shows that $\overline
{\operatorname*{Fqpol}}\left(  1\right)  =\operatorname*{Fqpol}\left(
1\right)  =X$ (since $\operatorname*{Fqpol}$ is an $\mathbb{F}_{q}$-algebra
homomorphism $\mathcal{F}\rightarrow\left(  \mathbb{F}_{q}\left[  T\right]
\left[  X\right]  _{q-\operatorname*{lin}},+,\circ\right)  $, and since the
unity of the $\mathbb{F}_{q}$-algebra $\left(  \mathbb{F}_{q}\left[  T\right]
\left[  X\right]  _{q-\operatorname*{lin}},+,\circ\right)  $ is $X$).

But the definition of $\overline{\operatorname*{Fqpol}}$ shows that
$\overline{\operatorname*{Fqpol}}\left(  \operatorname*{Carl}M\right)
=\operatorname*{Fqpol}\left(  \operatorname*{Carl}M\right)  $, so that%
\begin{align}
\operatorname*{Fqpol}\left(  \operatorname*{Carl}M\right)   &  =\overline
{\operatorname*{Fqpol}}\left(  \underbrace{\operatorname*{Carl}M}_{=\left(
\operatorname*{Carl}M\right)  \cdot1}\right)  =\overline{\operatorname*{Fqpol}%
}\left(  \left(  \operatorname*{Carl}M\right)  \cdot1\right) \nonumber\\
&  =\left(  \operatorname*{Carl}M\right)  \cdot\underbrace{\overline
{\operatorname*{Fqpol}}\left(  1\right)  }_{=X}\nonumber\\
&  \ \ \ \ \ \ \ \ \ \ \left(  \text{since }\overline{\operatorname*{Fqpol}%
}\text{ is an }\mathcal{F}\text{-module homomorphism}\right) \nonumber\\
&  =\left(  \operatorname*{Carl}M\right)  \cdot X.
\label{pf.cor.F.carlitz.img.1}%
\end{align}
On the other hand, Proposition \ref{prop.F.carlitz} (applied to $A=\mathbb{F}%
_{q}\left[  T\right]  \left[  X\right]  $ and $a=X$) yields $\left[  M\right]
\left(  X\right)  =\left(  \operatorname*{Carl}M\right)  \cdot X$. Comparing
this with (\ref{pf.cor.F.carlitz.img.1}), we obtain $\operatorname*{Fqpol}%
\left(  \operatorname*{Carl}M\right)  =\left[  M\right]  \left(  X\right)
=\left[  M\right]  $. This proves Corollary \ref{cor.F.carlitz.img}.
\end{proof}

\subsection{\textquotedblleft Fermat's Little Theorem\textquotedblright\ for
the Carlitz action}

Let us first state a simple fact:

\begin{lemma}
\label{lem.F.torfree-use}Let $A$ be an $\mathbb{F}_{q}\left[  T\right]
$-algebra which is torsionfree as an $\mathbb{F}_{q}\left[  T\right]
$-module. Let $f$ be a nonzero element of $\mathbb{F}_{q}\left[  T\right]  $.
Let $\mathbf{u}\in A\left[  X\right]  $ be such that $f\mathbf{u}\in A\left[
X\right]  _{q-\operatorname*{lin}}$. Then, $\mathbf{u}\in A\left[  X\right]
_{q-\operatorname*{lin}}$.
\end{lemma}

\begin{proof}
[Proof of Lemma \ref{lem.F.torfree-use}.]We have $f\mathbf{u}\in A\left[
X\right]  _{q-\operatorname*{lin}}$. In other words, the polynomial
$f\mathbf{u}\in A\left[  X\right]  $ is a $q$-polynomial, that is, an
$A$-linear combination of the monomials $X^{q^{0}},X^{q^{1}},X^{q^{2}},\ldots
$. In other words, for every $k\in\mathbb{N}\setminus\left\{  q^{0}%
,q^{1},q^{2},\ldots\right\}  $, we have%
\begin{equation}
\left(  \text{the }X^{k}\text{-coefficient of }f\mathbf{u}\right)  =0.
\label{pf.lem.F.torfree-use.1}%
\end{equation}


Now, for every $k\in\mathbb{N}\setminus\left\{  q^{0},q^{1},q^{2}%
,\ldots\right\}  $, we have%
\[
f\cdot\left(  \text{the }X^{k}\text{-coefficient of }\mathbf{u}\right)
=\left(  \text{the }X^{k}\text{-coefficient of }f\mathbf{u}\right)  =0
\]
(by (\ref{pf.lem.F.torfree-use.1})), and thus $\left(  \text{the }%
X^{k}\text{-coefficient of }\mathbf{u}\right)  =0$ (because $f\neq0$, and
because $A$ is torsionfree as an $\mathbb{F}_{q}\left[  T\right]  $-module).
In other words, the polynomial $\mathbf{u}$ is an $A$-linear combination of
the monomials $X^{q^{0}},X^{q^{1}},X^{q^{2}},\ldots$. In other words,
$\mathbf{u}$ is a $q$-polynomial; that is, $\mathbf{u}\in A\left[  X\right]
_{q-\operatorname*{lin}}$. This proves Lemma \ref{lem.F.torfree-use}.
\end{proof}

We now shall prove a crucial fact:

\begin{proposition}
\label{prop.F.u(pi)}Let $\pi$ be a monic irreducible polynomial in
$\mathbb{F}_{q}\left[  T\right]  $. Then, there exists a unique $u\left(
\pi\right)  \in\mathcal{F}$ such that $\operatorname*{Carl}\pi=F^{\deg\pi}%
+\pi\cdot u\left(  \pi\right)  $. (The notation $u\left(  \pi\right)  $ means
that $u$ depends on $\pi$; it is not meant to imply that $u\left(  \pi\right)
$ is a polynomial in $\pi$.)
\end{proposition}

The first proof of this proposition will reveal it to be a translation of part
of \cite[Theorem 2.11]{kc-carlitz}:

\begin{proof}
[First proof of Proposition \ref{prop.F.u(pi)}.]The left $\mathbb{F}%
_{q}\left[  T\right]  $-module $\mathcal{F}$ is free (by Proposition
\ref{prop.F.bases} \textbf{(c)}), and thus torsionfree.

From \cite[Theorem 2.11]{kc-carlitz}, we know that $\overline{\left[
\pi\right]  }\left(  X\right)  =X^{q^{\deg\pi}}$, where $\overline{\left[
\pi\right]  }\left(  X\right)  $ denotes the projection of $\left[
\pi\right]  \left(  X\right)  =\left[  \pi\right]  \in\mathbb{F}_{q}\left[
T\right]  \left[  X\right]  $ onto $\left(  \mathbb{F}_{q}\left[  T\right]
/\pi\right)  \left[  X\right]  $. In other words, $\left[  \pi\right]  \left(
X\right)  \equiv X^{q^{\deg\pi}}\operatorname{mod}K$, where $K$ is the kernel
of the projection $\mathbb{F}_{q}\left[  T\right]  \left[  X\right]
\rightarrow\left(  \mathbb{F}_{q}\left[  T\right]  /\pi\right)  \left[
X\right]  $. Since this kernel $K$ is simply $\pi\mathbb{F}_{q}\left[
T\right]  \left[  X\right]  $, this rewrites as follows: $\left[  \pi\right]
\left(  X\right)  \equiv X^{q^{\deg\pi}}\operatorname{mod}\pi\mathbb{F}%
_{q}\left[  T\right]  \left[  X\right]  $.

Thus, $\left[  \pi\right]  =\left[  \pi\right]  \left(  X\right)  \equiv
X^{q^{\deg\pi}}\operatorname{mod}\pi\mathbb{F}_{q}\left[  T\right]  \left[
X\right]  $. In other words, $\pi\mid\left[  \pi\right]  -X^{q^{\deg\pi}}$ in
the ring $\mathbb{F}_{q}\left[  T\right]  \left[  X\right]  $. Hence,
$\dfrac{1}{\pi}\left(  \left[  \pi\right]  -X^{q^{\deg\pi}}\right)  $ is a
well-defined polynomial in the ring $\mathbb{F}_{q}\left[  T\right]  \left[
X\right]  $ (since this ring is an integral domain). Let us denote this
polynomial by $\mathbf{u}$.

We have%
\begin{align*}
\left[  \pi\right]   &  =\operatorname*{Fqpol}\left(
\underbrace{\operatorname*{Carl}\pi}_{\in\mathcal{F}}\right)
\ \ \ \ \ \ \ \ \ \ \left(  \text{by Corollary \ref{cor.F.carlitz.img},
applied to }M=\pi\right) \\
&  \in\operatorname*{Carl}\mathcal{F}\subseteq\mathbb{F}_{q}\left[  T\right]
\left[  X\right]  _{q-\operatorname*{lin}}.
\end{align*}


But $\mathbf{u}=\dfrac{1}{\pi}\left(  \left[  \pi\right]  -X^{q^{\deg\pi}%
}\right)  $, so that $\pi\mathbf{u}=\left[  \pi\right]  -X^{q^{\deg\pi}}%
\in\mathbb{F}_{q}\left[  T\right]  \left[  X\right]  _{q-\operatorname*{lin}}$
(since both $\left[  \pi\right]  $ and $X^{q^{\deg\pi}}$ belong to
$\mathbb{F}_{q}\left[  T\right]  \left[  X\right]  _{q-\operatorname*{lin}}$).
Therefore, $\mathbf{u}\in\mathbb{F}_{q}\left[  T\right]  \left[  X\right]
_{q-\operatorname*{lin}}$ (by Lemma \ref{lem.F.torfree-use}, applied to
$A=\mathbb{F}_{q}\left[  T\right]  $ and $f=\pi$).

Theorem \ref{thm.q-pol.=F} \textbf{(c)} (applied to $j=0$ and $i=\deg\pi$)
yields $\operatorname*{Fqpol}\left(  T^{0}F^{\deg\pi}\right)
=\underbrace{T^{0}}_{=1}X^{q^{\deg\pi}}=X^{q^{\deg\pi}}$, so that
$X^{q^{\deg\pi}}=\operatorname*{Fqpol}\left(  \underbrace{T^{0}}_{=1}%
F^{\deg\pi}\right)  =\operatorname*{Fqpol}\left(  F^{\deg\pi}\right)  $.

Theorem \ref{thm.q-pol.=F} \textbf{(b)} shows that the map
$\operatorname*{Fqpol}:\mathcal{F}\rightarrow\mathbb{F}_{q}\left[  T\right]
\left[  X\right]  _{q-\operatorname*{lin}}$ is an $\mathbb{F}_{q}$-algebra
isomorphism. Thus, its inverse map $\operatorname*{Fqpol}\nolimits^{-1}$ is
well-defined. Set $\widetilde{\mathbf{u}}=\operatorname*{Fqpol}\nolimits^{-1}%
\left(  \mathbf{u}\right)  $. Thus, $\widetilde{\mathbf{u}}\in\mathcal{F}$ and
$\operatorname*{Fqpol}\left(  \widetilde{\mathbf{u}}\right)  =\mathbf{u}$.

But $\operatorname*{Fqpol}$ is an isomorphism of left $\mathbb{F}_{q}\left[
T\right]  $-modules (according to Proposition \ref{prop.q-pol.Fqlin.leftT}).
Hence,%
\begin{align*}
\operatorname*{Fqpol}\left(  \pi\widetilde{\mathbf{u}}\right)   &
=\pi\underbrace{\operatorname*{Fqpol}\left(  \widetilde{\mathbf{u}}\right)
}_{=\mathbf{u}}=\pi\mathbf{u}=\underbrace{\left[  \pi\right]  }%
_{\substack{=\operatorname*{Fqpol}\left(  \operatorname*{Carl}\pi\right)
\\\text{(by Corollary \ref{cor.F.carlitz.img},}\\\text{applied to }%
M=\pi\text{)}}}-\underbrace{X^{q^{\deg\pi}}}_{=\operatorname*{Fqpol}\left(
F^{\deg\pi}\right)  }\\
&  =\operatorname*{Fqpol}\left(  \operatorname*{Carl}\pi\right)
-\operatorname*{Fqpol}\left(  F^{\deg\pi}\right)  =\operatorname*{Fqpol}%
\left(  \operatorname*{Carl}\pi-F^{\deg\pi}\right)
\end{align*}
(since the map $\operatorname*{Fqpol}$ is $\mathbb{F}_{q}$-linear). Since
$\operatorname*{Fqpol}$ is injective (because $\operatorname*{Fqpol}$ is an
isomorphism), this yields $\pi\widetilde{\mathbf{u}}=\operatorname*{Carl}%
\pi-F^{\deg\pi}$.

Hence, there exists at least one $u\left(  \pi\right)  \in\mathcal{F}$ such
that $\pi\cdot u\left(  \pi\right)  =\operatorname*{Carl}\pi-F^{\deg\pi}$
(namely, $u\left(  \pi\right)  =\widetilde{\mathbf{u}}$). Moreover, such a
$u\left(  \pi\right)  $ is clearly unique (because any element $u\left(
\pi\right)  \in\mathcal{F}$ is uniquely determined by $\pi\cdot u\left(
\pi\right)  $ (since $\pi\neq0$, and since the left $\mathbb{F}_{q}\left[
T\right]  $-module $\mathcal{F}$ is torsionfree)). Thus, there exists a
\textbf{unique} $u\left(  \pi\right)  \in\mathcal{F}$ such that $\pi\cdot
u\left(  \pi\right)  =\operatorname*{Carl}\pi-F^{\deg\pi}$. In other words,
there exists a \textbf{unique} $u\left(  \pi\right)  \in\mathcal{F}$ such that
$\operatorname*{Carl}\pi=F^{\deg\pi}+\pi\cdot u\left(  \pi\right)  $. This
proves Proposition \ref{prop.F.u(pi)}.
\end{proof}

\subsection{A second proof of Proposition \ref{prop.F.u(pi)}}

Let us next give another proof of Proposition \ref{prop.F.u(pi)}, which does
not rely on Carlitz polynomials. This proof is not directly relevant for the
rest of this report, but illustrates some techniques of working with
$\mathcal{F}$.

\begin{noncompile}
We begin by quoting a well-known fact:

\begin{lemma}
\label{lem.artin-character-lind}Let $G$ be a finite abelian group. Let $F$ be
a field. Let $\chi_{1},\chi_{2},\ldots,\chi_{n}$ be finitely many distinct
group homomorphisms $G\rightarrow F^{\times}$. Then, $\chi_{1},\chi_{2}%
,\ldots,\chi_{n}$ are $F$-linearly independent as elements of the $F$-vector
space $F^{G}$.
\end{lemma}

Lemma \ref{lem.artin-character-lind} is Artin's classical result on the
\textit{linear independency of characters}; it appears, for example, in
\cite[Theorem 2.1]{kc-lind} (where the word \textquotedblleft
character\textquotedblright\ for \textquotedblleft group homomorphism to
$F^{\times}$\textquotedblright\ is used).

\bibitem {kc-lind}Keith Conrad, \textit{Linear independence of characters},
version 10 June 2013.\newline\url{http://www.math.uconn.edu/~kconrad/blurbs/galoistheory/linearchar.pdf}
\end{noncompile}

We first state a classical fact:

\begin{proposition}
\label{prop.F.u(pi).lem1}Let $\pi$ be a monic irreducible polynomial in
$\mathbb{F}_{q}\left[  T\right]  $. Let $d=\deg\pi$.

Let $\mathbb{F}_{\pi}$ denote the field $\mathbb{F}_{q}\left[  T\right]
/\pi\mathbb{F}_{q}\left[  T\right]  $. This is a field extension of
$\mathbb{F}_{q}$. Let $\alpha\in\mathbb{F}_{\pi}$ be the residue class of
$T\in\mathbb{F}_{q}\left[  T\right]  $ modulo the ideal $\pi\mathbb{F}%
_{q}\left[  T\right]  $. Thus, $\mathbb{F}_{\pi}=\mathbb{F}\left[
\alpha\right]  $ and $\pi\left(  \alpha\right)  =0$.

\textbf{(a)} The $\mathbb{F}_{q}$-vector space $\mathbb{F}_{\pi}$ has basis
$\left(  \alpha^{0},\alpha^{1},\ldots,\alpha^{d-1}\right)  $.

\textbf{(b)} The elements $\alpha^{q^{0}},\alpha^{q^{1}},\ldots,\alpha
^{q^{d-1}}$ are pairwise distinct and are precisely the roots of $\pi$.

\textbf{(c)} We have%
\begin{equation}
\pi=\prod_{k=0}^{d-1}\left(  T-\alpha^{q^{k}}\right)
\ \ \ \ \ \ \ \ \ \ \text{in }\mathbb{F}_{\pi}\left[  T\right]  .
\label{eq.prop.F.u(pi).lem1.c}%
\end{equation}

\end{proposition}

\begin{proof}
[Proof of Proposition \ref{prop.F.u(pi).lem1}.]\textbf{(a)} This is well-known
(and holds for any commutative ring instead of $\mathbb{F}_{q}$).

\textbf{(c)} Recall that $\operatorname*{Frob}\nolimits_{A}$ is an
$\mathbb{F}_{q}$-algebra endomorphism of $A$ whenever $A$ is a commutative
$\mathbb{F}_{q}$-algebra. Applying this to $A=\mathbb{F}_{\pi}$, we conclude
that $\operatorname*{Frob}\nolimits_{\mathbb{F}_{\pi}}$ is an $\mathbb{F}_{q}%
$-algebra endomorphism of $\mathbb{F}_{\pi}$. Denote this $\mathbb{F}_{q}%
$-algebra endomorphism by $f$. Thus, $f=\operatorname*{Frob}%
\nolimits_{\mathbb{F}_{\pi}}$.

We have $f=\operatorname*{Frob}\nolimits_{\mathbb{F}_{\pi}}$, and thus
\begin{equation}
f\left(  a\right)  =\operatorname*{Frob}\nolimits_{\mathbb{F}_{\pi}}\left(
a\right)  =a^{q}\ \ \ \ \ \ \ \ \ \ \left(  \text{by the definition of
}\operatorname*{Frob}\nolimits_{\mathbb{F}_{\pi}}\right)
\label{pf.prop.F.u(pi).lem1.f(a)=}%
\end{equation}
for every $a\in\mathbb{F}_{\pi}$. Now,%
\begin{equation}
f^{k}\left(  a\right)  =a^{q^{k}}\ \ \ \ \ \ \ \ \ \ \text{for every }%
k\in\mathbb{N}\text{ and }a\in\mathbb{F}_{\pi}.
\label{pf.prop.F.u(pi).lem1.fk(a)=}%
\end{equation}
(Indeed, this can be proven by a straightforward induction on $k$, using
(\ref{pf.prop.F.u(pi).lem1.f(a)=}).)

But $\mathbb{F}_{\pi}=\mathbb{F}_{q}\left[  T\right]  /\pi\mathbb{F}%
_{q}\left[  T\right]  $ is an $\mathbb{F}_{q}$-vector space of dimension
$\deg\pi=d$. Hence, $\left\vert \mathbb{F}_{\pi}\right\vert =\left\vert
\mathbb{F}_{q}\right\vert ^{d}=q^{d}$ (since $\left\vert \mathbb{F}%
_{q}\right\vert =q$). But it is well-known that if $L$ is a finite field, then
every $a\in L$ satisfies $a^{\left\vert L\right\vert }=a$. Applying this to
$L=\mathbb{F}_{\pi}$, we conclude that every $a\in\mathbb{F}_{\pi}$ satisfies
$a^{\left\vert \mathbb{F}_{\pi}\right\vert }=a$. Hence,%
\begin{equation}
f^{d}=\operatorname*{id} \label{pf.prop.F.u(pi).lem1.fd=}%
\end{equation}
\footnote{\textit{Proof of (\ref{pf.prop.F.u(pi).lem1.fd=}):} We have just
shown that every $a\in\mathbb{F}_{\pi}$ satisfies $a^{\left\vert
\mathbb{F}_{\pi}\right\vert }=a$. Now, every $a\in\mathbb{F}_{\pi}$ satisfies%
\begin{align*}
f^{d}\left(  a\right)   &  =a^{q^{d}}\ \ \ \ \ \ \ \ \ \ \left(  \text{by
(\ref{pf.prop.F.u(pi).lem1.fk(a)=}), applied to }k=d\right) \\
&  =a^{\left\vert \mathbb{F}_{\pi}\right\vert }\ \ \ \ \ \ \ \ \ \ \left(
\text{since }q^{d}=\left\vert \mathbb{F}_{\pi}\right\vert \right) \\
&  =a=\operatorname*{id}\left(  a\right)  .
\end{align*}
In other words, $f^{d}=\operatorname*{id}$. Qed.}. Thus, $\operatorname*{id}%
=f^{d}=f^{d-1}\circ f$. Hence, the map $f$ is left-invertible, and thus injective.

Every nonzero polynomial $g\in\mathbb{F}_{q}\left[  T\right]  $ has at most
$\deg g$ roots (since $\mathbb{F}_{q}$ is a field). Applying this to $g=\pi$,
we conclude that the polynomial $\pi$ has at most $\deg\pi=d$ roots.

Now, we notice that
\begin{equation}
\pi\left(  \alpha^{q^{k}}\right)  =0\text{ for each }k\in\left\{
0,1,\ldots,d-1\right\}  \label{pf.prop.F.u(pi).lem1.b.root1}%
\end{equation}
\footnote{\textit{Proof of (\ref{pf.prop.F.u(pi).lem1.b.root1}):} Let
$k\in\left\{  0,1,\ldots,d-1\right\}  $. Then,
(\ref{pf.prop.F.u(pi).lem1.fk(a)=}) (applied to $a=\alpha$) yields
$f^{k}\left(  \alpha\right)  =\alpha^{q^{k}}$.
\par
Recall that $f$ is an $\mathbb{F}_{q}$-algebra endomorphism of $\mathbb{F}%
_{\pi}$. Thus, $f^{k}$ is an $\mathbb{F}_{q}$-algebra endomorphism of
$\mathbb{F}_{\pi}$ as well. Hence, $f^{k}$ commutes with polynomials in
$\mathbb{F}_{q}\left[  T\right]  $. In other words, $f^{k}\left(  g\left(
\beta\right)  \right)  =g\left(  f^{k}\left(  \beta\right)  \right)  $ for
every $g\in\mathbb{F}_{q}\left[  T\right]  $ and every $\beta\in
\mathbb{F}_{\pi}$. Applying this to $g=\pi$ and $\beta=\alpha$, we obtain
$f^{k}\left(  \pi\left(  \alpha\right)  \right)  =\pi\left(  \underbrace{f^{k}%
\left(  \alpha\right)  }_{=\alpha^{q^{k}}}\right)  =\pi\left(  \alpha^{q^{k}%
}\right)  $. Hence, $\pi\left(  \alpha^{q^{k}}\right)  =f^{k}\left(
\underbrace{\pi\left(  \alpha\right)  }_{=0}\right)  =f^{k}\left(  0\right)
=0$ (since $f^{k}$ is an $\mathbb{F}_{q}$-algebra endomorphism of
$\mathbb{F}_{\pi}$). This proves (\ref{pf.prop.F.u(pi).lem1.b.root1}).}.
Also,
\begin{equation}
\alpha^{q^{k}}\neq\alpha\ \ \ \ \ \ \ \ \ \ \text{for each }k\in\left\{
1,2,\ldots,d-1\right\}  \label{pf.prop.F.u(pi).lem1.b.neq1}%
\end{equation}
\footnote{\textit{Proof of (\ref{pf.prop.F.u(pi).lem1.b.neq1}):} Let
$k\in\left\{  1,2,\ldots,d-1\right\}  $. We shall show that $\alpha^{q^{k}%
}\neq\alpha$.
\par
Indeed, assume the contrary. Thus, $\alpha^{q^{k}}=\alpha$. But
(\ref{pf.prop.F.u(pi).lem1.fk(a)=}) (applied to $a=\alpha$) yields
$f^{k}\left(  \alpha\right)  =\alpha^{q^{k}}=\alpha$.
\par
Let $x\in\mathbb{F}_{\pi}$. We are going to show that $x^{q^{k}}-x=0$.
\par
Indeed, $x\in\mathbb{F}_{\pi}=\mathbb{F}_{q}\left[  \alpha\right]  $. Hence,
$x=h\left(  \alpha\right)  $ for some polynomial $h\in\mathbb{F}_{q}\left[
T\right]  $. Consider this $h$.
\par
Recall that $f$ is an $\mathbb{F}_{q}$-algebra endomorphism of $\mathbb{F}%
_{\pi}$. Thus, $f^{k}$ is an $\mathbb{F}_{q}$-algebra endomorphism of
$\mathbb{F}_{\pi}$ as well. Hence, $f^{k}$ commutes with polynomials in
$\mathbb{F}_{q}\left[  T\right]  $. In other words, $f^{k}\left(  g\left(
\beta\right)  \right)  =g\left(  f^{k}\left(  \beta\right)  \right)  $ for
every $g\in\mathbb{F}_{q}\left[  T\right]  $ and every $\beta\in
\mathbb{F}_{\pi}$. Applying this to $g=h$ and $\beta=\alpha$, we obtain
$f^{k}\left(  h\left(  \alpha\right)  \right)  =h\left(  \underbrace{f^{k}%
\left(  \alpha\right)  }_{=\alpha}\right)  =h\left(  \alpha\right)  $. Since
$x=h\left(  \alpha\right)  $, this rewrites as $f^{k}\left(  x\right)  =x$.
But (\ref{pf.prop.F.u(pi).lem1.fk(a)=}) (applied to $a=x$) yields
$f^{k}\left(  x\right)  =x^{q^{k}}$. Hence, $x^{q^{k}}=f^{k}\left(  x\right)
=x$, so that $x^{q^{k}}-x=0$.
\par
Now, forget that we fixed $x$. We thus have proven that every $x\in
\mathbb{F}_{\pi}$ satisfies $x^{q^{k}}-x=0$. In other words, every
$x\in\mathbb{F}_{\pi}$ is a root of the polynomial $T^{q^{k}}-T\in
\mathbb{F}_{q}\left[  T\right]  $. Hence, the polynomial $T^{q^{k}}-T$ has at
least $\left\vert \mathbb{F}_{\pi}\right\vert $ roots. Since $\left\vert
\mathbb{F}_{\pi}\right\vert =q^{d}>q^{k}$ (since $d>k$ (because $k\in\left\{
1,2,\ldots,d-1\right\}  $)), this shows that the polynomial $T^{q^{k}}-T$ has
$>q^{k}$ roots.
\par
But $k>0$, so that the polynomial $T^{q^{k}}-T$ is a nonzero polynomial of
degree $\deg\left(  T^{q^{k}}-T\right)  =q^{k}$. It is well-known that each
nonzero polynomial $w\in\mathbb{F}_{q}\left[  T\right]  $ has at most $\deg w$
roots (since $\mathbb{F}_{q}$ is a field). Applying this to $w=T^{q^{k}}-T$,
we conclude that the polynomial $T^{q^{k}}-T$ has at most $\deg\left(
T^{q^{k}}-T\right)  =q^{k}$ roots. This contradicts the fact that the
polynomial $T^{q^{k}}-T$ has $>q^{k}$ roots. This contradiction shows that our
assumption was false. Hence, $\alpha^{q^{k}}\neq\alpha$ is proven, qed.}.
Hence,%
\begin{equation}
\text{the elements }\alpha^{q^{0}},\alpha^{q^{1}},\ldots,\alpha^{q^{d-1}%
}\text{ are pairwise distinct} \label{pf.prop.F.u(pi).lem1.b.neq2}%
\end{equation}
\footnote{\textit{Proof of (\ref{pf.prop.F.u(pi).lem1.b.neq2}):} Assume the
contrary. Thus, two of the elements $\alpha^{q^{0}},\alpha^{q^{1}}%
,\ldots,\alpha^{q^{d-1}}$ are equal. In other words, there exist two elements
$i$ and $j$ of $\left\{  0,1,\ldots,d-1\right\}  $ satisfying $i<j$ and
$\alpha^{q^{i}}=\alpha^{q^{j}}$. Consider these $i$ and $j$.
\par
We have $j-i\in\left\{  1,2,\ldots,d-1\right\}  $ (since $i$ and $j$ belong to
$\left\{  0,1,\ldots,d-1\right\}  $ and satisfy $i<j$). Hence,
(\ref{pf.prop.F.u(pi).lem1.b.neq1}) (applied to $k=j-i$) yields $\alpha
^{q^{j-i}}\neq\alpha$. But (\ref{pf.prop.F.u(pi).lem1.fk(a)=}) (applied to
$a=\alpha$ and $k=j-i$) yields $f^{j-i}\left(  \alpha\right)  =\alpha
^{q^{j-i}}\neq\alpha$.
\par
Applying (\ref{pf.prop.F.u(pi).lem1.fk(a)=}) to $a=\alpha$ and $k=i$, we
obtain $f^{i}\left(  \alpha\right)  =\alpha^{q^{i}}$. Applying
(\ref{pf.prop.F.u(pi).lem1.fk(a)=}) to $a=\alpha$ and $k=j$, we obtain
$f^{j}\left(  \alpha\right)  =\alpha^{q^{j}}$. Thus, $\alpha^{q^{j}%
}=\underbrace{f^{j}}_{\substack{=f^{i}\circ f^{j-i}\\\text{(since }%
i<j\text{)}}}\left(  \alpha\right)  =\left(  f^{i}\circ f^{j-i}\right)
\left(  \alpha\right)  =f^{i}\left(  f^{j-i}\left(  \alpha\right)  \right)  $.
\par
Now, $f^{i}\left(  \alpha\right)  =\alpha^{q^{i}}=\alpha^{q^{j}}=f^{i}\left(
f^{j-i}\left(  \alpha\right)  \right)  $. Since the map $f^{i}$ is injective
(because $f$ is injective), this entails $\alpha=f^{j-i}\left(  \alpha\right)
\neq\alpha$. This is clearly absurd. This contradiction proves that our
assumption was false. Hence, (\ref{pf.prop.F.u(pi).lem1.b.neq2}) is proven.}.

Let $\gamma$ be the polynomial%
\[
\pi-\prod_{k=0}^{d-1}\left(  T-\alpha^{q^{k}}\right)  \in\mathbb{F}_{\pi
}\left[  T\right]  .
\]


The polynomial $\pi$ is monic and has degree $\deg\pi=d$. The polynomial
$\prod_{k=0}^{d-1}\left(  T-\alpha^{q^{k}}\right)  $ is also obviously a monic
polynomial of degree $d$ (since it is a product of $d$ monic polynomials of
degree $1$). Thus, $\gamma$ is a difference of two monic polynomials of degree
$d$ (since $\gamma=\pi-\prod_{k=0}^{d-1}\left(  T-\alpha^{q^{k}}\right)  $).
Consequently, $\gamma$ is a polynomial of degree $<d$ (because the difference
of two monic polynomials of degree $d$ must always be a polynomial of degree
$<d$). In other words, $\deg\gamma<d$.

Assume (for the sake of contradiction) that $\gamma\neq0$.

Every nonzero polynomial $g\in\mathbb{F}_{\pi}\left[  T\right]  $ has at most
$\deg g$ roots (since $\mathbb{F}_{\pi}$ is a field). Applying this to
$g=\gamma$, we conclude that $\gamma$ has at most $\deg\gamma$ roots (since
$\gamma\neq0$). Thus, $\gamma$ has $<d$ roots (since $\deg\gamma<d$).

But for every $\ell\in\left\{  0,1,\ldots,d-1\right\}  $, the element
$\alpha^{q^{\ell}}$ of $\mathbb{F}_{\pi}$ is a root of $\gamma$%
\ \ \ \ \footnote{\textit{Proof.} Let $\ell\in\left\{  0,1,\ldots,d-1\right\}
$. From $\gamma=\pi-\prod_{k=0}^{d-1}\left(  T-\alpha^{q^{k}}\right)  $, we
obtain%
\[
\gamma\left(  \alpha^{q^{\ell}}\right)  =\underbrace{\pi\left(  \alpha
^{q^{\ell}}\right)  }_{\substack{=0\\\text{(by
(\ref{pf.prop.F.u(pi).lem1.b.root1}),}\\\text{applied to }k=\ell\text{)}%
}}-\underbrace{\prod_{k=0}^{d-1}\left(  \alpha^{q^{\ell}}-\alpha^{q^{k}%
}\right)  }_{\substack{=0\\\text{(because one of the factors in this product
is }\alpha^{q^{\ell}}-\alpha^{q^{\ell}}\\\text{(namely, the factor for }%
k=\ell\text{), and this factor is clearly }0\text{)}}}=0-0=0.
\]
In other words, the element $\alpha^{q^{\ell}}$ of $\mathbb{F}_{\pi}$ is a
root of $\gamma$. Qed.}. In other words, $\alpha^{q^{0}},\alpha^{q^{1}}%
,\ldots,\alpha^{q^{d-1}}$ are $d$ roots of $\gamma$. These $d$ roots are
pairwise distinct (by (\ref{pf.prop.F.u(pi).lem1.b.neq2})). Thus, the
polynomial $\gamma$ has at least $d$ roots. This contradicts the fact that
$\gamma$ has $<d$ roots. This contradiction proves that our assumption (that
$\gamma\neq0$) was false. Hence, we have $\gamma=0$. Thus, $0=\gamma=\pi
-\prod_{k=0}^{d-1}\left(  T-\alpha^{q^{k}}\right)  $, so that $\pi=\prod
_{k=0}^{d-1}\left(  T-\alpha^{q^{k}}\right)  $. This proves Proposition
\ref{prop.F.u(pi).lem1} \textbf{(c)}.

\textbf{(b)} The elements $\alpha^{q^{0}},\alpha^{q^{1}},\ldots,\alpha
^{q^{d-1}}$ are pairwise distinct (by (\ref{pf.prop.F.u(pi).lem1.b.neq2})) and
are precisely the roots of $\pi$ (because of (\ref{eq.prop.F.u(pi).lem1.c})).
This proves Proposition \ref{prop.F.u(pi).lem1} \textbf{(b)}.
\end{proof}

Here are some more useful lemmas:

\begin{lemma}
\label{lem.F.u(pi).dividiff}Let $\mathbb{K}$ be a commutative ring. Let
$d\in\mathbb{N}$. Let $\pi\in\mathbb{K}\left[  T\right]  $ be a polynomial of
degree $\leq d$. For each $i\in\mathbb{N}$, let $\pi_{i}$ be the coefficient
of $T^{i}$ in $\pi$. For each $k\in\left\{  0,1,\ldots,d\right\}  $, define a
polynomial $p_{k}\in\mathbb{K}\left[  T\right]  $ by $p_{k}=\sum_{i=k+1}%
^{d}\pi_{i}T^{i-1-k}$. Then:

\textbf{(a)} We have $p_{d-1}=\pi_{d}$ (a constant polynomial) and $p_{d}=0$.

\textbf{(b)} We have $\pi\left(  X\right)  -\pi\left(  Y\right)  =\left(
X-Y\right)  \sum_{i=0}^{d-1}p_{i}\left(  X\right)  Y^{i}$ in the ring
$\mathbb{K}\left[  X,Y\right]  $.
\end{lemma}

\begin{proof}
[Proof of Lemma \ref{lem.F.u(pi).dividiff}.]The definition of $p_{d-1}$ yields%
\[
p_{d-1}=\sum_{i=\left(  d-1\right)  +1}^{d}\pi_{i}T^{i-1-\left(  d-1\right)
}=\sum_{i=d}^{d}\pi_{i}T^{i-1-\left(  d-1\right)  }=\pi_{d}%
\underbrace{T^{d-1-\left(  d-1\right)  }}_{=T^{0}=1}=\pi_{d}.
\]
The definition of $p_{d}$ yields%
\[
p_{d}=\sum_{i=d+1}^{d}\pi_{i}T^{i-1-d}=\left(  \text{empty sum}\right)  =0.
\]
This proves Lemma \ref{lem.F.u(pi).dividiff} \textbf{(a)}.

For every $i\in\left\{  0,1,\ldots,d\right\}  $, we have%
\begin{align}
X^{i}-Y^{i}  &  =\left(  X-Y\right)  \underbrace{\sum_{k=0}^{i-1}%
X^{k}Y^{i-1-k}}_{\substack{=\sum_{\ell=0}^{i-1}X^{i-1-\ell}Y^{\ell
}\\\text{(here, we have substituted }\ell\\\text{for }i-1-k\text{ in the
sum)}}}\ \ \ \ \ \ \ \ \ \ \left(  \text{by a known formula}\right)
\nonumber\\
&  =\left(  X-Y\right)  \sum_{\ell=0}^{i-1}X^{i-1-\ell}Y^{\ell}.
\label{pf.lem.F.u(pi).dividiff.geoser}%
\end{align}


We have $\pi=\sum_{i=0}^{d}\pi_{i}T^{i}$ (since $\pi$ is a polynomial of
degree $\leq d$, and since the $\pi_{i}$ are its coefficients). Thus,
$\pi\left(  X\right)  =\sum_{i=0}^{d}\pi_{i}X^{i}$ and $\pi\left(  Y\right)
=\sum_{i=0}^{d}\pi_{i}Y^{i}$. Hence,%
\begin{align*}
\pi\left(  X\right)  -\pi\left(  Y\right)   &  =\sum_{i=0}^{d}\pi_{i}%
X^{i}-\sum_{i=0}^{d}\pi_{i}Y^{i}\\
&  =\sum_{i=0}^{d}\pi_{i}\underbrace{\left(  X^{i}-Y^{i}\right)
}_{\substack{=\left(  X-Y\right)  \sum_{\ell=0}^{i-1}X^{i-1-\ell}Y^{\ell
}\\\text{(by (\ref{pf.lem.F.u(pi).dividiff.geoser}))}}}=\sum_{i=0}^{d}\pi
_{i}\cdot\left(  X-Y\right)  \sum_{\ell=0}^{i-1}X^{i-1-\ell}Y^{\ell}\\
&  =\left(  X-Y\right)  \sum_{i=0}^{d}\pi_{i}\sum_{\ell=0}^{i-1}X^{i-1-\ell
}Y^{\ell}.
\end{align*}
Since%
\begin{align*}
\sum_{i=0}^{d}\pi_{i}\sum_{\ell=0}^{i-1}X^{i-1-\ell}Y^{\ell}  &
=\underbrace{\sum_{i=0}^{d}\sum_{\ell=0}^{i-1}}_{=\sum_{\ell=0}^{d}%
\sum_{i=\ell+1}^{d}}\pi_{i}X^{i-1-\ell}Y^{\ell}=\sum_{\ell=0}^{d}%
\underbrace{\sum_{i=\ell+1}^{d}\pi_{i}X^{i-1-\ell}}_{\substack{=p_{\ell
}\left(  X\right)  \\\text{(since }p_{\ell}=\sum_{i=\ell+1}^{d}\pi
_{i}T^{i-1-\ell}\\\text{(by the definition of }p_{\ell}\text{) and
thus}\\p_{\ell}\left(  X\right)  =\sum_{i=\ell+1}^{d}\pi_{i}X^{i-1-\ell
}\text{)}}}Y^{\ell}\\
&  =\sum_{\ell=0}^{d}p_{\ell}\left(  X\right)  Y^{\ell}=\sum_{\ell=0}%
^{d-1}p_{\ell}\left(  X\right)  Y^{\ell}+\underbrace{p_{d}\left(  X\right)
}_{\substack{=0\\\text{(since }p_{d}=0\text{)}}}Y^{d}\\
&  =\sum_{\ell=0}^{d-1}p_{\ell}\left(  X\right)  Y^{\ell}=\sum_{i=0}%
^{d-1}p_{i}\left(  X\right)  Y^{i}\\
&  \ \ \ \ \ \ \ \ \ \ \left(  \text{here, we have renamed the summation index
}\ell\text{ as }i\right)  ,
\end{align*}
this rewrites as $\pi\left(  X\right)  -\pi\left(  Y\right)  =\left(
X-Y\right)  \sum_{i=0}^{d-1}p_{i}\left(  X\right)  Y^{i}$. This proves Lemma
\ref{lem.F.u(pi).dividiff} \textbf{(b)}.
\end{proof}

\begin{lemma}
\label{lem.F.u(pi).lem2}Let $\pi$ be a monic irreducible polynomial in
$\mathbb{F}_{q}\left[  T\right]  $. Let $d=\deg\pi$.

Let $\mathbb{F}_{\pi}$ denote the field $\mathbb{F}_{q}\left[  T\right]
/\pi\mathbb{F}_{q}\left[  T\right]  $. This is a field extension of
$\mathbb{F}_{q}$. Let $\alpha\in\mathbb{F}_{\pi}$ be the residue class of
$T\in\mathbb{F}_{q}\left[  T\right]  $ modulo the ideal $\pi\mathbb{F}%
_{q}\left[  T\right]  $. Thus, $\mathbb{F}_{\pi}=\mathbb{F}\left[
\alpha\right]  $ and $\pi\left(  \alpha\right)  =0$. Let $\mathcal{F}_{\pi}$
denote the $\mathbb{F}_{\pi}$-algebra $\mathbb{F}_{\pi}\otimes\mathcal{F}$
(where $\mathbb{F}_{\pi}$ acts on the first tensorand).

Let $h\in\mathcal{F}$ be such that $1\otimes h\in\left(  1\otimes
T-\alpha\right)  \mathcal{F}_{\pi}$. (Notice that the $\alpha$ here really
means the element $\alpha1_{\mathcal{F}_{\pi}}=\alpha\otimes1$ of
$\mathcal{F}_{\pi}$.) Then, $h\in\pi\mathcal{F}$.
\end{lemma}

\begin{remark}
\label{rmk.lem.F.u(pi).lem2.exp}Lemma \ref{lem.F.u(pi).lem2} can be viewed as
a noncommutative version of the following known fact: If $h\in\mathbb{F}%
_{q}\left[  T\right]  $ is such that $h\in\left(  T-\alpha\right)
\mathbb{F}_{\pi}\left[  T\right]  $, then $h\in\pi\mathbb{F}_{q}\left[
T\right]  $. (That is, a polynomial in $\mathbb{F}_{q}\left[  T\right]  $ that
vanishes at $\alpha$ must be a multiple of $\pi$.)
\end{remark}

\begin{proof}
[Proof of Lemma \ref{lem.F.u(pi).lem2}.]For each $i\in\mathbb{N}$, let
$\pi_{i}$ be the coefficient of $T^{i}$ in $\pi$. For each $k\in\left\{
0,1,\ldots,d\right\}  $, define a polynomial $p_{k}\in\mathbb{F}_{q}\left[
T\right]  $ by $p_{k}=\sum_{i=k+1}^{d}\pi_{i}T^{i-1-k}$. Then, Lemma
\ref{lem.F.u(pi).dividiff} \textbf{(a)} (applied to $\mathbb{K}=\mathbb{F}%
_{q}$) yields that $p_{d-1}=\pi_{d}$ (a constant polynomial) and $p_{d}=0$.
But $\pi_{d}=1$ (since $\pi$ is a monic polynomial of degree $d$). Thus,
$p_{d-1}=\pi_{d}=1$.

Furthermore, Lemma \ref{lem.F.u(pi).dividiff} \textbf{(b)} (applied to
$\mathbb{K}=\mathbb{F}_{q}$) yields
\[
\pi\left(  X\right)  -\pi\left(  Y\right)  =\left(  X-Y\right)  \sum
_{i=0}^{d-1}p_{i}\left(  X\right)  Y^{i}=\left(  \sum_{i=0}^{d-1}p_{i}\left(
X\right)  Y^{i}\right)  \left(  X-Y\right)
\]
in the ring $\mathbb{K}\left[  X,Y\right]  $. Since the two elements $1\otimes
T$ and $\alpha$ of $\mathcal{F}_{\pi}$ commute with each other, we can
substitute $1\otimes T$ and $\alpha$ for $X$ and $Y$ in this identity. We thus
obtain%
\begin{align*}
\pi\left(  1\otimes T\right)  -\pi\left(  \alpha\right)   &  =\left(
\sum_{i=0}^{d-1}\underbrace{p_{i}\left(  1\otimes T\right)  }%
_{\substack{=1\otimes p_{i}\left(  T\right)  =1\otimes p_{i}\\\text{(since
}p_{i}\left(  T\right)  =p_{i}\text{)}}}\underbrace{\alpha^{i}}%
_{\substack{=\alpha^{i}\otimes1\\\text{(since }\alpha^{i}\in\mathbb{F}_{\pi
}\text{)}}}\right)  \left(  1\otimes T-\alpha\right) \\
&  =\left(  \sum_{i=0}^{d-1}\underbrace{\left(  1\otimes p_{i}\right)  \left(
\alpha^{i}\otimes1\right)  }_{=\alpha^{i}\otimes p_{i}}\right)  \left(
1\otimes T-\alpha\right) \\
&  =\left(  \sum_{i=0}^{d-1}\alpha^{i}\otimes p_{i}\right)  \left(  1\otimes
T-\alpha\right)
\end{align*}
in the ring $\mathcal{F}$. Since $\pi\left(  1\otimes T\right)
-\underbrace{\pi\left(  \alpha\right)  }_{=0}=\pi\left(  1\otimes T\right)
=1\otimes\underbrace{\pi\left(  T\right)  }_{=\pi}=1\otimes\pi$, this rewrites
as
\begin{equation}
1\otimes\pi=\left(  \sum_{i=0}^{d-1}\alpha^{i}\otimes p_{i}\right)  \left(
1\otimes T-\alpha\right)  . \label{pf.lem.F.u(pi).lem2.1}%
\end{equation}


Now,
\begin{align}
&  \sum_{i=0}^{d-1}\underbrace{\alpha^{i}\otimes p_{i}h}_{=\left(  \alpha
^{i}\otimes p_{i}\right)  \left(  1\otimes h\right)  }\nonumber\\
&  =\sum_{i=0}^{d-1}\left(  \alpha^{i}\otimes p_{i}\right)  \left(  1\otimes
h\right)  =\left(  \sum_{i=0}^{d-1}\alpha^{i}\otimes p_{i}\right)
\underbrace{\left(  1\otimes h\right)  }_{\in\left(  1\otimes T-\alpha\right)
\mathcal{F}_{\pi}}\nonumber\\
&  \in\underbrace{\left(  \sum_{i=0}^{d-1}\alpha^{i}\otimes p_{i}\right)
\left(  1\otimes T-\alpha\right)  }_{\substack{=1\otimes\pi\\\text{(by
(\ref{pf.lem.F.u(pi).lem2.1}))}}}\mathcal{F}_{\pi}=\left(  1\otimes\pi\right)
\mathcal{F}_{\pi}. \label{pf.lem.F.u(pi).lem2.3}%
\end{align}


But Proposition \ref{prop.F.u(pi).lem1} \textbf{(a)} shows that the
$\mathbb{F}_{q}$-vector space $\mathbb{F}_{\pi}$ has basis $\left(  \alpha
^{0},\alpha^{1},\ldots,\alpha^{d-1}\right)  $. Hence, we can define an
$\mathbb{F}_{q}$-linear map $\lambda:\mathbb{F}_{\pi}\rightarrow\mathbb{F}%
_{q}$ by%
\begin{equation}
\left(  \lambda\left(  \alpha^{i}\right)  =\delta_{i,d-1}%
\ \ \ \ \ \ \ \ \ \ \text{for each }i\in\left\{  0,1,\ldots,d-1\right\}
\right)  . \label{pf.lem.F.u(pi).lem2.lambda}%
\end{equation}
Consider this $\lambda$. The $\mathbb{F}_{q}$-linear map $\lambda
:\mathbb{F}_{\pi}\rightarrow\mathbb{F}_{q}$ induces an $\mathbb{F}_{q}$-linear
map $\lambda\otimes\operatorname*{id}\nolimits_{\mathcal{F}}:\mathbb{F}_{\pi
}\otimes\mathcal{F}\rightarrow\mathbb{F}_{q}\otimes\mathcal{F}$. In view of
$\mathbb{F}_{\pi}\otimes\mathcal{F}=\mathcal{F}_{\pi}$ and $\mathbb{F}%
_{q}\otimes\mathcal{F}=\mathcal{F}$, this latter map is thus an $\mathbb{F}%
_{q}$-linear map $\lambda\otimes\operatorname*{id}\nolimits_{\mathcal{F}%
}:\mathcal{F}_{\pi}\rightarrow\mathcal{F}$. This map satisfies%
\begin{equation}
\left(  \lambda\otimes\operatorname*{id}\nolimits_{\mathcal{F}}\right)
\left(  \left(  1\otimes\pi\right)  \mathcal{F}_{\pi}\right)  \subseteq
\pi\mathcal{F} \label{pf.lem.F.u(pi).lem2.goesto}%
\end{equation}
\footnote{\textit{Proof of (\ref{pf.lem.F.u(pi).lem2.goesto}):} We have%
\begin{align*}
&  \left(  \lambda\otimes\operatorname*{id}\nolimits_{\mathcal{F}}\right)
\left(  \left(  1\otimes\pi\right)  \underbrace{\mathcal{F}_{\pi}%
}_{=\mathbb{F}_{\pi}\otimes\mathcal{F}}\right) \\
&  =\left(  \lambda\otimes\operatorname*{id}\nolimits_{\mathcal{F}}\right)
\underbrace{\left(  \left(  1\otimes\pi\right)  \left(  \mathbb{F}_{\pi
}\otimes\mathcal{F}\right)  \right)  }_{\substack{=\mathbb{F}_{\pi}\otimes
\pi\mathcal{F}\\\text{(seen as a subspace of }\mathbb{F}_{\pi}\otimes
\mathcal{F}\text{)}}}=\left(  \lambda\otimes\operatorname*{id}%
\nolimits_{\mathcal{F}}\right)  \left(  \mathbb{F}_{\pi}\otimes\pi
\mathcal{F}\right) \\
&  =\underbrace{\lambda\left(  \mathbb{F}_{\pi}\right)  }_{\subseteq
\mathbb{F}_{q}}\otimes\underbrace{\operatorname*{id}\nolimits_{\mathcal{F}%
}\left(  \pi\mathcal{F}\right)  }_{=\pi\mathcal{F}}\ \ \ \ \ \ \ \ \ \ \left(
\text{seen as a subspace of }\mathbb{F}_{q}\otimes\mathcal{F}\right) \\
&  \subseteq\mathbb{F}_{q}\otimes\pi\mathcal{F}=\pi\mathcal{F}%
\ \ \ \ \ \ \ \ \ \ \left(  \text{using our identification of }\mathbb{F}%
_{q}\otimes\mathcal{F}\text{ with }\mathcal{F}\right)  ,
\end{align*}
qed.}. Now, applying the map $\lambda\otimes\operatorname*{id}%
\nolimits_{\mathcal{F}}$ to both sides of the equality
(\ref{pf.lem.F.u(pi).lem2.3}), we obtain%
\[
\left(  \lambda\otimes\operatorname*{id}\nolimits_{\mathcal{F}}\right)
\left(  \sum_{i=0}^{d-1}\alpha^{i}\otimes p_{i}h\right)  \in\left(
\lambda\otimes\operatorname*{id}\nolimits_{\mathcal{F}}\right)  \left(
\left(  1\otimes\pi\right)  \mathcal{F}_{\pi}\right)  \subseteq\pi\mathcal{F}%
\]
(by (\ref{pf.lem.F.u(pi).lem2.goesto})). Since%
\begin{align*}
&  \left(  \lambda\otimes\operatorname*{id}\nolimits_{\mathcal{F}}\right)
\left(  \sum_{i=0}^{d-1}\alpha^{i}\otimes p_{i}h\right) \\
&  =\sum_{i=0}^{d-1}\underbrace{\lambda\left(  \alpha^{i}\right)
}_{\substack{=\delta_{i,d-1}\\\text{(by (\ref{pf.lem.F.u(pi).lem2.lambda}))}%
}}\otimes\underbrace{\operatorname*{id}\nolimits_{\mathcal{F}}\left(
p_{i}h\right)  }_{=p_{i}h}=\sum_{i=0}^{d-1}\delta_{i,d-1}\otimes p_{i}h\\
&  =\sum_{i=0}^{d-1}\delta_{i,d-1}p_{i}h\ \ \ \ \ \ \ \ \ \ \left(
\text{using our identification of }\mathbb{F}_{q}\otimes\mathcal{F}\text{ with
}\mathcal{F}\right) \\
&  =\sum_{i=0}^{d-2}\underbrace{\delta_{i,d-1}}_{\substack{=0\\\text{(since
}i\neq d-1\\\text{(since }i\leq d-2\text{))}}}p_{i}h+\underbrace{\delta
_{d-1,d-1}}_{=1}\underbrace{p_{d-1}}_{=1}h=\underbrace{\sum_{i=0}^{d-2}%
0p_{i}h}_{=0}+h=h,
\end{align*}
this rewrites as $h\in\pi\mathcal{F}$. This proves Lemma
\ref{lem.F.u(pi).lem2}.
\end{proof}

\begin{lemma}
\label{lem.F.u(pi).prod-deform}Let $R$ be a ring (not necessarily
commutative). If $b_{0},b_{1},\ldots,b_{d-1}$ are some elements of $R$ (for
some $d\in\mathbb{N}$), then the product $\prod_{k=0}^{d-1}b_{k}$ shall be
defined as $b_{0}b_{1}\cdots b_{d-1}$. (Thus, we have defined this product
even if the elements $b_{0},b_{1},\ldots,b_{d-1}$ do not commute.)

Let $r\in\mathbb{N}$. Let $f$, $t$ and $a$ be three elements of $R$ satisfying
$ft=t^{r}f$, $fa=af$ and $ta=at$. Let $d\in\mathbb{N}$. Then, every
$d\in\mathbb{N}$ satisfies%
\begin{equation}
\prod_{k=0}^{d-1}\left(  f+t-a^{r^{k}}\right)  \equiv f^{d}\operatorname{mod}%
\left(  t-a\right)  R. \label{eq.lem.F.u(pi).prod-deform.1}%
\end{equation}
(Note that $\left(  t-a\right)  R$ is only a right ideal of $R$, not
necessarily an ideal of $R$.)
\end{lemma}

\begin{proof}
[Proof of Lemma \ref{lem.F.u(pi).prod-deform}.]We have%
\begin{equation}
f^{i}t=t^{r^{i}}f^{i}\ \ \ \ \ \ \ \ \ \ \text{for every }i\in\mathbb{N}.
\label{pf.lem.F.u(pi).prod-deform.1}%
\end{equation}
(This can be proven by a straightforward induction on $i$, using the relation
$ft=t^{r}f$.) Also, the relation $fa=af$ shows that the $\mathbb{Z}%
$-subalgebra of $R$ generated by $a$ and $f$ is commutative. Thus, every
$i\in\mathbb{N}$ and $j\in\mathbb{N}$ satisfy%
\begin{equation}
f^{i}a^{j}=a^{j}f^{i} \label{pf.lem.F.u(pi).prod-deform.2}%
\end{equation}
(since both $f^{i}$ and $a^{j}$ belong to this commutative $\mathbb{Z}$-subalgebra).

Moreover, every $i\in\mathbb{N}$ satisfies%
\begin{equation}
t^{i}-a^{i}\equiv0\operatorname{mod}\left(  t-a\right)  R
\label{pf.lem.F.u(pi).prod-deform.3}%
\end{equation}
\footnote{\textit{Proof of (\ref{pf.lem.F.u(pi).prod-deform.3}):} Let
$i\in\mathbb{N}$. Then, a known formula shows that $X^{i}-Y^{i}=\left(
X-Y\right)  \sum_{k=0}^{i-1}X^{k}Y^{i-1-k}$ in the polynomial ring
$\mathbb{Z}\left[  X,Y\right]  $. Since the elements $t$ and $a$ of $R$
commute (because $ta=at$), we can substitute $t$ and $a$ for $X$ and $Y$ in
this formula. We thus obtain%
\[
t^{i}-a^{i}=\left(  t-a\right)  \underbrace{\sum_{k=0}^{i-1}t^{k}a^{i-1-k}%
}_{\in R}\in\left(  t-a\right)  R.
\]
In other words, $t^{i}-a^{i}\equiv0\operatorname{mod}\left(  t-a\right)  R$.
This proves (\ref{pf.lem.F.u(pi).prod-deform.3}).}.

We shall prove (\ref{eq.lem.F.u(pi).prod-deform.1}) by induction over $d$:

\textit{Induction base:} For $d=0$, the congruence
(\ref{eq.lem.F.u(pi).prod-deform.1}) is obviously true (because both sides of
this congruence equal $1$). This completes the induction base.

\textit{Induction step:} Let $D\in\mathbb{N}$. Assume that
(\ref{eq.lem.F.u(pi).prod-deform.1}) holds for $d=D$. We must prove that
(\ref{eq.lem.F.u(pi).prod-deform.1}) holds for $d=D+1$.

We have assumed that (\ref{eq.lem.F.u(pi).prod-deform.1}) holds for $d=D$. In
other words,%
\begin{equation}
\prod_{k=0}^{D-1}\left(  f+t-a^{r^{k}}\right)  \equiv f^{D}\operatorname{mod}%
\left(  t-a\right)  R. \label{pf.lem.F.u(pi).prod-deform.indhyp}%
\end{equation}


Now,%
\begin{align*}
\prod_{k=0}^{D}\left(  f+t-a^{r^{k}}\right)   &  =\underbrace{\left(
\prod_{k=0}^{D-1}\left(  f+t-a^{r^{k}}\right)  \right)  }_{\substack{\equiv
f^{D}\operatorname{mod}\left(  t-a\right)  R\\\text{(by
(\ref{pf.lem.F.u(pi).prod-deform.indhyp}))}}}\left(  f+t-a^{r^{D}}\right) \\
&  \equiv f^{D}\left(  f+t-a^{r^{D}}\right)  =\underbrace{f^{D}f}_{=f^{D+1}%
}+\underbrace{f^{D}t}_{\substack{=t^{r^{D}}f^{D}\\\text{(by
(\ref{pf.lem.F.u(pi).prod-deform.1}), applied to }i=D\text{)}}%
}-\underbrace{f^{D}a^{r^{D}}}_{\substack{=a^{r^{D}}f^{D}\\\text{(by
(\ref{pf.lem.F.u(pi).prod-deform.2}), applied to}\\i=D\text{ and }%
j=r^{D}\text{)}}}\\
&  =f^{D+1}+\underbrace{t^{r^{D}}f^{D}-a^{r^{D}}f^{D}}_{=\left(  t^{r^{D}%
}-a^{r^{D}}\right)  f^{D}}=f^{D+1}+\underbrace{\left(  t^{r^{D}}-a^{r^{D}%
}\right)  }_{\substack{\equiv0\operatorname{mod}\left(  t-a\right)
R\\\text{(by (\ref{pf.lem.F.u(pi).prod-deform.3}), applied to }i=r^{D}%
\text{)}}}f^{D}\\
&  \equiv f^{D+1}\operatorname{mod}\left(  t-a\right)  R.
\end{align*}
In other words, (\ref{eq.lem.F.u(pi).prod-deform.1}) holds for $d=D+1$. This
completes the induction step. Hence, (\ref{eq.lem.F.u(pi).prod-deform.1}) is
proven by induction. In other words, Lemma \ref{lem.F.u(pi).prod-deform} is proven.
\end{proof}

Now we can prove Proposition \ref{prop.F.u(pi)} again:

\begin{proof}
[Second proof of Proposition \ref{prop.F.u(pi)}.]The left $\mathbb{F}%
_{q}\left[  T\right]  $-module $\mathcal{F}$ is free (by Proposition
\ref{prop.F.bases} \textbf{(c)}), and thus torsionfree.

Define $d$, $\mathbb{F}_{\pi}$, $\alpha$ and $\mathcal{F}_{\pi}$ as in Lemma
\ref{lem.F.u(pi).lem2}. Define $h\in\mathcal{F}$ by $h=\operatorname*{Carl}%
\pi-F^{\deg\pi}$. We shall show that $h\in\pi\mathcal{F}$.

Recall that $\operatorname*{Carl}$ is the $\mathbb{F}_{q}$-algebra
homomorphism $\mathbb{F}_{q}\left[  T\right]  \rightarrow\mathcal{F}$ sending
$T$ to $F+T$. This homomorphism sends every polynomial $g\in\mathbb{F}%
_{q}\left[  T\right]  $ to $g\left(  F+T\right)  $ (where $g\left(
F+T\right)  $ denotes the result of substituting $F+T$ for $T$ in $g$, not the
product of $g$ with $F+T$). In other words, $\operatorname*{Carl}g=g\left(
F+T\right)  $ for every $g\in\mathbb{F}_{q}\left[  T\right]  $. Applying this
to $g=\pi$, we obtain $\operatorname*{Carl}\pi=\pi\left(  F+T\right)  $.

Now, we can substitute $1\otimes F+1\otimes T\in\mathcal{F}_{\pi}$ for $T$ in
the equality (\ref{eq.prop.F.u(pi).lem1.c}) (since $1\otimes F+1\otimes T$ is
an element of the $\mathbb{F}_{\pi}$-algebra $\mathcal{F}_{\pi}$). As a
result, we obtain%
\begin{equation}
\pi\left(  1\otimes F+1\otimes T\right)  =\prod_{k=0}^{d-1}\left(  1\otimes
F+1\otimes T-\alpha^{q^{k}}\right)  . \label{pf.prop.F.u(pi).2nd.1}%
\end{equation}


But the elements $1\otimes F$, $1\otimes T$ and $\alpha$ of $\mathcal{F}_{\pi
}$ satisfy%
\begin{align*}
\left(  1\otimes F\right)  \left(  1\otimes T\right)   &  =1\otimes
\underbrace{FT}_{=T^{q}F}=1\otimes T^{q}F=\left(  1\otimes T\right)
^{q}\left(  1\otimes F\right)  ,\\
\left(  1\otimes F\right)  \alpha &  =\alpha\left(  1\otimes F\right)
\ \ \ \ \ \ \ \ \ \ \left(  \text{since }\alpha\text{ really means }%
\alpha\otimes1\in\mathcal{F}_{\pi}\right)  ,\\
\left(  1\otimes T\right)  \alpha &  =\alpha\left(  1\otimes T\right)
\ \ \ \ \ \ \ \ \ \ \left(  \text{since }\alpha\text{ really means }%
\alpha\otimes1\in\mathcal{F}_{\pi}\right)  .
\end{align*}
Hence, Lemma \ref{lem.F.u(pi).prod-deform} (applied to $R=\mathcal{F}_{\pi}$,
$r=q$, $f=1\otimes F$, $t=1\otimes T$ and $a=\alpha$) yields%
\[
\prod_{k=0}^{d-1}\left(  1\otimes F+1\otimes T-\alpha^{q^{k}}\right)
\equiv\left(  1\otimes F\right)  ^{d}=1\otimes F^{d}\operatorname{mod}\left(
1\otimes T-\alpha\right)  \mathcal{F}_{\pi}.
\]
Hence, (\ref{pf.prop.F.u(pi).2nd.1}) becomes%
\begin{align*}
\pi\left(  1\otimes F+1\otimes T\right)   &  =\prod_{k=0}^{d-1}\left(
1\otimes F+1\otimes T-\alpha^{q^{k}}\right) \\
&  \equiv1\otimes F^{d}\operatorname{mod}\left(  1\otimes T-\alpha\right)
\mathcal{F}_{\pi}.
\end{align*}
Since
\[
\pi\left(  \underbrace{1\otimes F+1\otimes T}_{=1\otimes\left(  F+T\right)
}\right)  =\pi\left(  1\otimes\left(  F+T\right)  \right)  =1\otimes\pi\left(
F+T\right)  ,
\]
this rewrites as%
\begin{equation}
1\otimes\pi\left(  F+T\right)  \equiv1\otimes F^{d}\operatorname{mod}\left(
1\otimes T-\alpha\right)  \mathcal{F}_{\pi}. \label{pf.prop.F.u(pi).2nd.5}%
\end{equation}
Now, $h=\underbrace{\operatorname*{Carl}\pi}_{=\pi\left(  F+T\right)
}-\underbrace{F^{\deg\pi}}_{\substack{=F^{d}\\\text{(since }\deg\pi=d\text{)}%
}}=\pi\left(  F+T\right)  -F^{d}$, so that%
\[
1\otimes h=1\otimes\left(  \pi\left(  F+T\right)  -F^{d}\right)  =1\otimes
\pi\left(  F+T\right)  -1\otimes F^{d}\in\left(  1\otimes T-\alpha\right)
\mathcal{F}_{\pi}%
\]
(by (\ref{pf.prop.F.u(pi).2nd.5})). Hence, Lemma \ref{lem.F.u(pi).lem2} shows
that $h\in\pi\mathcal{F}$. Hence, there exists at least one $u\left(
\pi\right)  \in\mathcal{F}$ such that $\pi\cdot u\left(  \pi\right)  =h$.
Moreover, such a $u\left(  \pi\right)  $ is clearly unique (because any
element $u\left(  \pi\right)  \in\mathcal{F}$ is uniquely determined by
$\pi\cdot u\left(  \pi\right)  $ (since $\pi\neq0$, and since the left
$\mathbb{F}_{q}\left[  T\right]  $-module $\mathcal{F}$ is torsionfree)).
Thus, there exists a \textbf{unique} $u\left(  \pi\right)  \in\mathcal{F}$
such that $\pi\cdot u\left(  \pi\right)  =h$. In other words, there exists a
\textbf{unique} $u\left(  \pi\right)  \in\mathcal{F}$ such that
$\operatorname*{Carl}\pi=F^{\deg\pi}+\pi\cdot u\left(  \pi\right)  $ (because
we have the logical equivalence%
\begin{align*}
\left(  \pi\cdot u\left(  \pi\right)  =\underbrace{h}_{=\operatorname*{Carl}%
\pi-F^{\deg\pi}}\right)  \  &  \Longleftrightarrow\ \left(  \pi\cdot u\left(
\pi\right)  =\operatorname*{Carl}\pi-F^{\deg\pi}\right) \\
&  \Longleftrightarrow\ \left(  \operatorname*{Carl}\pi=F^{\deg\pi}+\pi\cdot
u\left(  \pi\right)  \right)
\end{align*}
). This proves Proposition \ref{prop.F.u(pi)} again.
\end{proof}

\begin{remark}
Now that we have a proof of Proposition \ref{prop.F.u(pi)} that is independent
of \cite[Theorem 2.11]{kc-carlitz}, we can turn the cart around and give a new
proof of \cite[Theorem 2.11, last equality]{kc-carlitz} (though this proof, of
course, will be rather roundabout):

Let $\pi$ be a monic irreducible polynomial in $\mathbb{F}_{q}\left[
T\right]  $. Our goal is to show that $\overline{\left[  \pi\right]  }\left(
X\right)  =X^{q^{\deg\pi}}$, where $\overline{\left[  \pi\right]  }\left(
X\right)  $ denotes the projection of $\left[  \pi\right]  \left(  X\right)
=\left[  \pi\right]  \in\mathbb{F}_{q}\left[  T\right]  \left[  X\right]  $
onto $\left(  \mathbb{F}_{q}\left[  T\right]  /\pi\right)  \left[  X\right]  $.

We have $X^{q^{\deg\pi}}=\operatorname*{Fqpol}\left(  F^{\deg\pi}\right)  $.
(This can be proven as in our first proof of Proposition \ref{prop.F.u(pi)}.)
Also, $\operatorname*{Fqpol}$ is an isomorphism of left $\mathbb{F}_{q}\left[
T\right]  $-modules (according to Proposition \ref{prop.q-pol.Fqlin.leftT}).

Proposition \ref{prop.F.u(pi)} shows that there exists a unique $u\left(
\pi\right)  \in\mathcal{F}$ such that $\operatorname*{Carl}\pi=F^{\deg\pi}%
+\pi\cdot u\left(  \pi\right)  $. Consider this $u\left(  \pi\right)  $.
Corollary \ref{cor.F.carlitz.img}, (applied to $M=\pi$) yields
\begin{align*}
\left[  \pi\right]   &  =\operatorname*{Fqpol}\left(
\underbrace{\operatorname*{Carl}\pi}_{=F^{\deg\pi}+\pi\cdot u\left(
\pi\right)  }\right)  =\operatorname*{Fqpol}\left(  F^{\deg\pi}+\pi\cdot
u\left(  \pi\right)  \right) \\
&  =\underbrace{\left(  \operatorname*{Fqpol}\left(  F^{\deg\pi}\right)
\right)  }_{=X^{q^{\deg\pi}}}+\pi\underbrace{\operatorname*{Fqpol}\left(
u\left(  \pi\right)  \right)  }_{\in\mathbb{F}_{q}\left[  T\right]  \left[
X\right]  }\\
&  \ \ \ \ \ \ \ \ \ \ \left(  \text{since }\operatorname*{Fqpol}\text{ is a
homomorphism of left }\mathbb{F}_{q}\left[  T\right]  \text{-modules}\right)
\\
&  \in X^{q^{\deg\pi}}+\pi\mathbb{F}_{q}\left[  T\right]  \left[  X\right]  .
\end{align*}
In other words, $\left[  \pi\right]  \equiv X^{q^{\deg\pi}}\operatorname{mod}%
\pi\mathbb{F}_{q}\left[  T\right]  \left[  X\right]  $. Projecting both sides
of this congruence down to $\mathbb{F}_{q}\left[  T\right]  \left[  X\right]
/\left(  \pi\mathbb{F}_{q}\left[  T\right]  \left[  X\right]  \right)
=\left(  \mathbb{F}_{q}\left[  T\right]  /\pi\right)  \left[  X\right]  $, we
obtain $\overline{\left[  \pi\right]  }=X^{q^{\deg\pi}}$. In other words,
$\overline{\left[  \pi\right]  }\left(  X\right)  =X^{q^{\deg\pi}}$, qed.
\end{remark}

\subsection{Corollary: Carlitz action vs. Frobenius power}

\begin{corollary}
\label{cor.F.u(pi).mod}Let $\pi$ be a monic irreducible polynomial in
$\mathbb{F}_{q}\left[  T\right]  $. Let $A$ be an $\mathcal{F}$-module. Then,
$\left(  \operatorname*{Carl}\pi\right)  a\equiv F^{\deg\pi}%
a\operatorname{mod}\pi A$ for every $a\in A$.
\end{corollary}

\begin{proof}
[Proof of Corollary \ref{cor.F.u(pi).mod}.]Let $a\in A$. Proposition
\ref{prop.F.u(pi)} shows that there exists a unique $u\left(  \pi\right)
\in\mathcal{F}$ such that $\operatorname*{Carl}\pi=F^{\deg\pi}+\pi\cdot
u\left(  \pi\right)  $. Consider this $u\left(  \pi\right)  $.

Now,%
\[
\underbrace{\left(  \operatorname*{Carl}\pi\right)  }_{=F^{\deg\pi}+\pi\cdot
u\left(  \pi\right)  }a=\left(  F^{\deg\pi}+\pi\cdot u\left(  \pi\right)
\right)  a=F^{\deg\pi}a+\underbrace{\pi\cdot u\left(  \pi\right)  a}%
_{\equiv0\operatorname{mod}\pi A}\equiv F^{\deg\pi}a\operatorname{mod}\pi A.
\]
This proves Corollary \ref{cor.F.u(pi).mod}.
\end{proof}

\subsection{Exponent lifting for $\mathcal{F}$-modules}

Next, we shall show a series of simple propositions which will culminate (if
this can be called a culmination) in a Carlitz analogue of the classical
\textquotedblleft lifting the exponent\textquotedblright\ theorem (see, e.g.,
\cite[version with solution (ancillary file), (12.349)]{reiner-hopf} for it).

\begin{proposition}
\label{prop.F.lift.CarlP-P}\textbf{(a)} The $\mathbb{F}_{q}$-vector subspace
$\mathcal{F}F$ of $\mathcal{F}$ is a two-sided ideal of $\mathcal{F}$.

\textbf{(b)} Let $P\in\mathbb{F}_{q}\left[  T\right]  $. Then,
$\operatorname*{Carl}P\equiv P\operatorname{mod}\mathcal{F}F$.
\end{proposition}

\begin{proof}
[Proof of Proposition \ref{prop.F.lift.CarlP-P}.]\textbf{(a)} First, we claim
that%
\begin{equation}
Fu\in\mathcal{F}F\ \ \ \ \ \ \ \ \ \ \text{for every }u\in\mathcal{F}.
\label{pf.prop.F.lift.CarlP-P.a.1}%
\end{equation}


\textit{Proof of (\ref{pf.prop.F.lift.CarlP-P.a.1}):} Proposition
\ref{prop.F.bases} \textbf{(b)} shows that the $\mathbb{F}_{q}$-module
$\mathcal{F}$ is free with basis $\left(  T^{j}F^{i}\right)  _{i\geq
0,\ j\geq0}$.

Let $u\in\mathcal{F}$. We must prove the relation
(\ref{pf.prop.F.lift.CarlP-P.a.1}). Since this relation is $\mathbb{F}_{q}%
$-linear in $u$ (because $\mathcal{F}F$ is an $\mathbb{F}_{q}$-vector subspace
of $\mathcal{F}$), we can WLOG assume that $u$ belongs to the basis $\left(
T^{j}F^{i}\right)  _{i\geq0,\ j\geq0}$ of the $\mathbb{F}_{q}$-module
$\mathcal{F}$. Assume this. Thus, $u=T^{j}F^{i}$ for some $i\in\mathbb{N}$ and
$j\in\mathbb{N}$. Consider these $i$ and $j$. Now,%
\[
F\underbrace{u}_{=T^{j}F^{i}}=\underbrace{FT^{j}}_{\substack{=\left(
T^{j}\right)  ^{q}F\\\text{(by Proposition \ref{prop.F.FP},}\\\text{applied to
}P=T^{j}\text{)}}}F^{i}=\left(  T^{j}\right)  ^{q}\underbrace{FF^{i}%
}_{=F^{i+1}=F^{i}F}=\underbrace{\left(  T^{j}\right)  ^{q}F^{i}}%
_{\in\mathcal{F}}F\in\mathcal{F}F.
\]
This proves (\ref{pf.prop.F.lift.CarlP-P.a.1}).

Now,
\[
F\mathcal{F}=\left\{  Fu\ \mid\ u\in\mathcal{F}\right\}  \subseteq
\mathcal{F}F\ \ \ \ \ \ \ \ \ \ \left(  \text{by
(\ref{pf.prop.F.lift.CarlP-P.a.1})}\right)  .
\]


But it is clear that $\mathcal{F}F$ is a left ideal of $\mathcal{F}$. Since we
furthermore have $\mathcal{F}\underbrace{F\cdot\mathcal{F}}_{=F\mathcal{F}%
\subseteq\mathcal{F}F}\subseteq\underbrace{\mathcal{FF}}_{\subseteq
\mathcal{F}}F\subseteq\mathcal{F}F$, we thus conclude that $\mathcal{F}F$ is a
two-sided ideal of $\mathcal{F}$. This proves Proposition
\ref{prop.F.lift.CarlP-P} \textbf{(a)}.

\textbf{(b)} Proposition \ref{prop.F.lift.CarlP-P} \textbf{(a)} shows that
$\mathcal{F}F$ is a two-sided ideal of $\mathcal{F}$. Hence, $\mathcal{F}%
/\left(  \mathcal{F}F\right)  $ is a quotient ring of $\mathcal{F}$, hence a
quotient $\mathbb{F}_{q}$-algebra of $\mathcal{F}$. Let $\pi$ denote the
canonical projection map $\mathcal{F}\rightarrow\mathcal{F}/\left(
\mathcal{F}F\right)  $. Then, $\pi$ is an $\mathbb{F}_{q}$-algebra
homomorphism (since $\mathcal{F}/\left(  \mathcal{F}F\right)  $ is a quotient
$\mathbb{F}_{q}$-algebra of $\mathcal{F}$).

But $\operatorname*{Carl}\left(  T\right)  =F+T\equiv T\operatorname{mod}%
\mathcal{F}F$ (since $F=\underbrace{1}_{\in\mathcal{F}}F\in\mathcal{F}F$). In
other words, $\pi\left(  \operatorname*{Carl}\left(  T\right)  \right)
=\pi\left(  T\right)  $ (since $\pi$ is the canonical projection map
$\mathcal{F}\rightarrow\mathcal{F}/\left(  \mathcal{F}F\right)  $). Thus,
\begin{align}
\left(  \pi\circ\operatorname*{Carl}\right)  \left(  T\right)   &  =\pi\left(
\operatorname*{Carl}\left(  T\right)  \right)  =\pi\left(  \underbrace{T}%
_{=\operatorname*{Finc}\nolimits_{T}\left(  T\right)  }\right)  =\pi\left(
\operatorname*{Finc}\nolimits_{T}\left(  T\right)  \right) \nonumber\\
&  =\left(  \pi\circ\operatorname*{Finc}\nolimits_{T}\right)  \left(
T\right)  . \label{pf.prop.F.lift.CarlP-P.b.1}%
\end{align}


But the three maps $\pi$, $\operatorname*{Carl}$ and $\operatorname*{Finc}%
\nolimits_{T}$ are $\mathbb{F}_{q}$-algebra homomorphisms; hence, $\pi
\circ\operatorname*{Carl}$ and $\pi\circ\operatorname*{Finc}\nolimits_{T}$ are
$\mathbb{F}_{q}$-algebra homomorphisms as well. The two $\mathbb{F}_{q}%
$-algebra homomorphisms $\pi\circ\operatorname*{Carl}:\mathbb{F}_{q}\left[
T\right]  \rightarrow\mathcal{F}/\left(  \mathcal{F}F\right)  $ and $\pi
\circ\operatorname*{Finc}\nolimits_{T}:\mathbb{F}_{q}\left[  T\right]
\rightarrow\mathcal{F}/\left(  \mathcal{F}F\right)  $ are equal to each other
on the generator $T$ of the $\mathbb{F}_{q}$-algebra $\mathbb{F}_{q}\left[
T\right]  $ (because of (\ref{pf.prop.F.lift.CarlP-P.b.1})). Therefore, these
two homomorphisms must be identical. In other words, $\pi\circ
\operatorname*{Carl}=\pi\circ\operatorname*{Finc}\nolimits_{T}$.

Now,
\[
\pi\left(  \operatorname*{Carl}P\right)  =\underbrace{\left(  \pi
\circ\operatorname*{Carl}\right)  }_{=\pi\circ\operatorname*{Finc}%
\nolimits_{T}}\left(  P\right)  =\left(  \pi\circ\operatorname*{Finc}%
\nolimits_{T}\right)  \left(  P\right)  =\pi\left(
\underbrace{\operatorname*{Finc}\nolimits_{T}\left(  P\right)  }%
_{\substack{=P\\\text{(since we are regarding the}\\\text{map }%
\operatorname*{Finc}\nolimits_{T}\text{ as an inclusion)}}}\right)
=\pi\left(  P\right)  .
\]
In other words, $\operatorname*{Carl}P\equiv P\operatorname{mod}\mathcal{F}F$
(since $\pi$ is the canonical projection map $\mathcal{F}\rightarrow
\mathcal{F}/\left(  \mathcal{F}F\right)  $). This proves Proposition
\ref{prop.F.lift.CarlP-P} \textbf{(b)}.
\end{proof}

\begin{proposition}
\label{prop.F.lift.FPA}Let $A$ be an $\mathcal{F}$-module. Let $P\in
\mathbb{F}_{q}\left[  T\right]  $.

\textbf{(a)} We have $FPA\subseteq P^{q}A$.

\textbf{(b)} The $\mathbb{F}_{q}$-vector subspace $PA$ of $A$ is a left
$\mathcal{F}$-submodule of $A$.

\textbf{(c)} Let $k$ be a positive integer. Then, $FP^{k}A\subseteq P^{k+1}A$.

\textbf{(d)} Let $k$ be a positive integer. Then, $\left(
\operatorname*{Carl}P\right)  P^{k}A\subseteq P^{k+1}A$.
\end{proposition}

\begin{proof}
[Proof of Proposition \ref{prop.F.lift.FPA}.]\textbf{(a)} Proposition
\ref{prop.F.FP} yields $FP=P^{q}F$ in $\mathcal{F}$. Hence, $\underbrace{FP}%
_{=P^{q}F}A=P^{q}\underbrace{FA}_{\subseteq A}\subseteq P^{q}A$. Thus,
Proposition \ref{prop.F.lift.FPA} \textbf{(a)} is proven.

\textbf{(b)} Proposition \ref{prop.F.lift.FPA} \textbf{(a)} yields
$FPA\subseteq\underbrace{P^{q}}_{\substack{=PP^{q-1}\\\text{(since }%
q\geq1\text{)}}}A=P\underbrace{P^{q-1}A}_{\subseteq A}\subseteq PA$. Also,
$\underbrace{TP}_{=PT}A=P\underbrace{TA}_{\subseteq A}\subseteq PA$.

Now, recall that the $\mathbb{F}_{q}$-algebra $\mathcal{F}$ is generated by
$F$ and $T$. From this, it is easy to derive the following fact: If
$\mathcal{V}$ is an $\mathbb{F}_{q}$-vector subspace of some left
$\mathcal{F}$-module $\mathcal{U}$ satisfying $F\mathcal{V}\subseteq
\mathcal{V}$ and $T\mathcal{V}\subseteq\mathcal{V}$, then $\mathcal{V}$ is a
left $\mathcal{F}$-submodule of $\mathcal{U}$. Applying this to $\mathcal{U}%
=A$ and $\mathcal{V}=PA$, we conclude that $PA$ is a left $\mathcal{F}%
$-submodule of $A$ (since $FPA\subseteq PA$ and $TPA\subseteq PA$).
Proposition \ref{prop.F.lift.FPA} \textbf{(b)} is thus shown.

\textbf{(c)} Proposition \ref{prop.F.lift.FPA} \textbf{(a)} (applied to
$P^{k}$ instead of $P$) yields%
\begin{align*}
FP^{k}A  &  \subseteq\underbrace{\left(  P^{k}\right)  ^{q}}%
_{\substack{=\left(  P^{k}\right)  ^{2}\left(  P^{k}\right)  ^{q-2}%
\\\text{(since }q\geq2\text{)}}}A=\left(  P^{k}\right)  ^{2}%
\underbrace{\left(  P^{k}\right)  ^{q-2}A}_{\subseteq A}\subseteq\left(
P^{k}\right)  ^{2}A=P^{k}\underbrace{P^{k}}_{\substack{=PP^{k-1}\\\text{(since
}k\text{ is a positive}\\\text{integer)}}}A\\
&  =\underbrace{P^{k}P}_{=P^{k+1}}\underbrace{P^{k-1}A}_{\subseteq A}\subseteq
P^{k+1}A.
\end{align*}
This establishes Proposition \ref{prop.F.lift.FPA} \textbf{(c)}.

\textbf{(d)} Proposition \ref{prop.F.lift.CarlP-P} \textbf{(b)} yields
$\operatorname*{Carl}P\equiv P\operatorname{mod}\mathcal{F}F$. In other words,
$\operatorname*{Carl}P-P\in\mathcal{F}F$. In other words, there exists some
$u\in\mathcal{F}$ such that $\operatorname*{Carl}P-P=uF$. Consider this $u$.

Proposition \ref{prop.F.lift.FPA} \textbf{(b)} (applied to $P^{k+1}$ instead
of $P$) shows that the $\mathbb{F}_{q}$-vector subspace $P^{k+1}A$ of $A$ is a
left $\mathcal{F}$-submodule of $A$. Hence, $uP^{k+1}A\subseteq P^{k+1}A$
(since $u\in\mathcal{F}$).

But $\operatorname*{Carl}P-P=uF$ shows that $\operatorname*{Carl}P=P+uF$.
Hence,%
\begin{align*}
\underbrace{\left(  \operatorname*{Carl}P\right)  }_{=P+uF}P^{k}A  &  =\left(
P+uF\right)  P^{k}A\subseteq\underbrace{PP^{k}}_{=P^{k+1}}%
A+u\underbrace{FP^{k}A}_{\substack{\subseteq P^{k+1}A\\\text{(by Proposition
\ref{prop.F.lift.FPA} \textbf{(c)})}}}\subseteq P^{k+1}A+\underbrace{uP^{k+1}%
A}_{\subseteq P^{k+1}A}\\
&  \subseteq P^{k+1}A+P^{k+1}A\subseteq P^{k+1}A.
\end{align*}
This proves Proposition \ref{prop.F.lift.FPA} \textbf{(d)}.
\end{proof}

\begin{proposition}
\label{prop.F.lift.lift-P}Let $A$ be an $\mathcal{F}$-module. Let
$P\in\mathbb{F}_{q}\left[  T\right]  $. Let $k$ be a positive integer.

Let $a$ and $b$ be two elements of $A$ such that $a\equiv b\operatorname{mod}%
P^{k}A$.

\textbf{(a)} We have $F^{\deg P}a\equiv F^{\deg P}b\operatorname{mod}P^{k+1}A$.

\textbf{(b)} We have $\left(  \operatorname*{Carl}P\right)  a\equiv\left(
\operatorname*{Carl}P\right)  b\operatorname{mod}P^{k+1}A$.
\end{proposition}

\begin{proof}
[Proof of Proposition \ref{prop.F.lift.lift-P}.]From $a\equiv
b\operatorname{mod}P^{k}A$, we obtain $a-b\in P^{k}A$.

\textbf{(a)} If $P=0$, then the claim of Proposition \ref{prop.F.lift.lift-P}
\textbf{(a)} is true\footnote{\textit{Proof.} Assume that $P=0$. Thus,
$P^{k}=0^{k}=0$ (since $k$ is positive), so that $P^{k}A=0A=0$. Hence,
$a\equiv b\operatorname{mod}P^{k}A$ rewrites as $a\equiv b\operatorname{mod}%
0$. In other words, $a=b$. Hence, $F^{\deg P}a=F^{\deg P}b$, so that $F^{\deg
P}a\equiv F^{\deg P}b\operatorname{mod}P^{k+1}A$. In other words, the claim of
Proposition \ref{prop.F.lift.lift-P} \textbf{(a)} is true; qed.}. Hence, we
WLOG assume that $P\neq0$.

If $\deg P=0$, then the claim of Proposition \ref{prop.F.lift.lift-P}
\textbf{(a)} is true\footnote{\textit{Proof.} Assume that $\deg P=0$. Thus,
the polynomial $P$ is constant. Since $P\neq0$, this shows that the polynomial
$P$ is invertible in $\mathbb{F}_{q}\left[  T\right]  $. Hence, $P$ is
invertible in $\mathcal{F}$. Therefore, $P^{k+1}$ is also invertible in
$\mathcal{F}$. Hence, $P^{k+1}A=A$. But $F^{\deg P}a\equiv F^{\deg
P}b\operatorname{mod}A$ is obviously true. Since $P^{k+1}A=A$, this rewrites
as $F^{\deg P}a\equiv F^{\deg P}b\operatorname{mod}P^{k+1}A$. In other words,
the claim of Proposition \ref{prop.F.lift.lift-P} \textbf{(a)} is true; qed.}.
Hence, we WLOG assume that $\deg P\neq0$. Thus, $\deg P\geq1$.

Let $d=\deg P$. Then, $d\geq1$, so that $F^{d}=FF^{d-1}$.

But Proposition \ref{prop.F.lift.FPA} \textbf{(b)} (applied to $P^{k}$ instead
of $P$) shows that the $\mathbb{F}_{q}$-vector subspace $P^{k}A$ of $A$ is a
left $\mathcal{F}$-submodule of $A$. Hence, $\mathcal{F}\cdot P^{k}A\subseteq
P^{k}A$.

Now, $\deg P=d$, so that%
\begin{align*}
F^{\deg P}a-F^{\deg P}b  &  =F^{d}a-F^{d}b=\underbrace{F^{d}}_{=FF^{d-1}%
}\underbrace{\left(  a-b\right)  }_{\in P^{k}A}\in F\underbrace{F^{d-1}}%
_{\in\mathcal{F}}P^{k}A\subseteq F\underbrace{\mathcal{F}\cdot P^{k}%
A}_{\subseteq P^{k}A}\\
&  \subseteq FP^{k}A\subseteq P^{k+1}A\ \ \ \ \ \ \ \ \ \ \left(  \text{by
Proposition \ref{prop.F.lift.FPA} \textbf{(c)}}\right)  .
\end{align*}
In other words, $F^{\deg P}a\equiv F^{\deg P}b\operatorname{mod}P^{k+1}A$.
This proves Proposition \ref{prop.F.lift.lift-P} \textbf{(a)}.

\textbf{(b)} We have%
\[
\left(  \operatorname*{Carl}P\right)  a-\left(  \operatorname*{Carl}P\right)
b=\left(  \operatorname*{Carl}P\right)  \underbrace{\left(  a-b\right)  }_{\in
P^{k}A}\in\left(  \operatorname*{Carl}P\right)  P^{k}A\subseteq P^{k+1}A
\]
(by Proposition \ref{prop.F.lift.FPA} \textbf{(d)}). In other words, $\left(
\operatorname*{Carl}P\right)  a\equiv\left(  \operatorname*{Carl}P\right)
b\operatorname{mod}P^{k+1}A$. This proves Proposition \ref{prop.F.lift.lift-P}
\textbf{(b)}.
\end{proof}

\begin{corollary}
\label{cor.F.lift.lift-Pl}Let $A$ be an $\mathcal{F}$-module. Let
$P\in\mathbb{F}_{q}\left[  T\right]  $. Let $k$ be a positive integer.

Let $a$ and $b$ be two elements of $A$ such that $a\equiv b\operatorname{mod}%
P^{k}A$.

\textbf{(a)} We have $F^{\deg\left(  P^{\ell}\right)  }a\equiv F^{\deg\left(
P^{\ell}\right)  }b\operatorname{mod}P^{k+\ell}A$ for every $\ell\in
\mathbb{N}$.

\textbf{(b)} We have $\left(  \operatorname*{Carl}\left(  P^{\ell}\right)
\right)  a\equiv\left(  \operatorname*{Carl}\left(  P^{\ell}\right)  \right)
b\operatorname{mod}P^{k+\ell}A$ for every $\ell\in\mathbb{N}$.
\end{corollary}

\begin{proof}
[Proof of Corollary \ref{cor.F.lift.lift-Pl}.]\textbf{(a)} We can prove
Corollary \ref{cor.F.lift.lift-Pl} \textbf{(a)} by induction over $\ell$:

\textit{Induction base:} We have $\deg\underbrace{\left(  P^{0}\right)  }%
_{=1}=\deg1=0$ and thus $F^{\deg\left(  P^{0}\right)  }=F^{0}=1$. Hence,
$F^{\deg\left(  P^{0}\right)  }a=1a=a$ and similarly $F^{\deg\left(
P^{0}\right)  }b=b$. But $a\equiv b\operatorname{mod}P^{k}A$. Since
$k+\underbrace{0}_{=0}=k$, this rewrites as $a\equiv b\operatorname{mod}%
P^{k+0}A$. Now, $F^{\deg\left(  P^{0}\right)  }a=a\equiv b=F^{\deg\left(
P^{0}\right)  }b\operatorname{mod}P^{k+0}A$. In other words, Corollary
\ref{cor.F.lift.lift-Pl} \textbf{(a)} holds for $\ell=0$. This completes the
induction base.

\textit{Induction step:} Let $L\in\mathbb{N}$. Assume that Corollary
\ref{cor.F.lift.lift-Pl} \textbf{(a)} holds for $\ell=L$. We must now prove
that Corollary \ref{cor.F.lift.lift-Pl} \textbf{(a)} holds for $\ell=L+1$.

We have assumed that Corollary \ref{cor.F.lift.lift-Pl} \textbf{(a)} holds for
$\ell=L$. In other words, we have $F^{\deg\left(  P^{L}\right)  }a\equiv
F^{\deg\left(  P^{L}\right)  }b\operatorname{mod}P^{k+L}A$.

But $k$ is a positive integer, and hence $k+L$ is a positive integer. Hence,
Proposition \ref{prop.F.lift.lift-P} \textbf{(a)} (applied to $k+L$,
$F^{\deg\left(  P^{L}\right)  }a$ and $F^{\deg\left(  P^{L}\right)  }b$
instead of $k$, $a$ and $b$) yields
\begin{equation}
F^{\deg P}F^{\deg\left(  P^{L}\right)  }a\equiv F^{\deg P}F^{\deg\left(
P^{L}\right)  }b\operatorname{mod}P^{k+L+1}A.
\label{pf.cor.F.lift.lift-Pl.a.1}%
\end{equation}


Now, $\deg\left(  \underbrace{P^{L+1}}_{=PP^{L}}\right)  =\deg\left(
PP^{L}\right)  =\deg P+\deg\left(  P^{L}\right)  $. Hence, $F^{\deg\left(
P^{L+1}\right)  }=F^{\deg P+\deg\left(  P^{L}\right)  }=F^{\deg P}%
F^{\deg\left(  P^{L}\right)  }$. Therefore, (\ref{pf.cor.F.lift.lift-Pl.a.1})
rewrites as follows:%
\[
F^{\deg\left(  P^{L+1}\right)  }a\equiv F^{\deg\left(  P^{L+1}\right)
}b\operatorname{mod}P^{k+L+1}A.
\]
In other words, Corollary \ref{cor.F.lift.lift-Pl} \textbf{(a)} holds for
$\ell=L+1$. This completes the induction step. The induction proof of
Corollary \ref{cor.F.lift.lift-Pl} \textbf{(a)} is thus finished.

\textbf{(b)} We can prove Corollary \ref{cor.F.lift.lift-Pl} \textbf{(b)} by
induction over $\ell$:

\textit{Induction base:} We have $\operatorname*{Carl}\underbrace{\left(
P^{0}\right)  }_{=1}=\operatorname*{Carl}1=1$ (since $\operatorname*{Carl}$ is
an $\mathbb{F}_{q}$-algebra homomorphism). Hence, $\left(
\operatorname*{Carl}\left(  P^{0}\right)  \right)  a=1a=a$ and similarly
$\left(  \operatorname*{Carl}\left(  P^{0}\right)  \right)  b=b$. But $a\equiv
b\operatorname{mod}P^{k}A$. Since $k+\underbrace{0}_{=0}=k$, this rewrites as
$a\equiv b\operatorname{mod}P^{k+0}A$. Now, $\left(  \operatorname*{Carl}%
\left(  P^{0}\right)  \right)  a=a\equiv b=\left(  \operatorname*{Carl}\left(
P^{0}\right)  \right)  b\operatorname{mod}P^{k+0}A$. In other words, Corollary
\ref{cor.F.lift.lift-Pl} \textbf{(b)} holds for $\ell=0$. This completes the
induction base.

\textit{Induction step:} Let $L\in\mathbb{N}$. Assume that Corollary
\ref{cor.F.lift.lift-Pl} \textbf{(b)} holds for $\ell=L$. We must now prove
that Corollary \ref{cor.F.lift.lift-Pl} \textbf{(b)} holds for $\ell=L+1$.

We have assumed that Corollary \ref{cor.F.lift.lift-Pl} \textbf{(b)} holds for
$\ell=L$. In other words, we have $\left(  \operatorname*{Carl}\left(
P^{L}\right)  \right)  a\equiv\left(  \operatorname*{Carl}\left(
P^{L}\right)  \right)  b\operatorname{mod}P^{k+L}A$.

But $k$ is a positive integer, and hence $k+L$ is a positive integer. Hence,
Proposition \ref{prop.F.lift.lift-P} \textbf{(b)} (applied to $k+L$, $\left(
\operatorname*{Carl}\left(  P^{L}\right)  \right)  a$ and $\left(
\operatorname*{Carl}\left(  P^{L}\right)  \right)  b$ instead of $k$, $a$ and
$b$) yields
\begin{equation}
\left(  \operatorname*{Carl}P\right)  \left(  \operatorname*{Carl}\left(
P^{L}\right)  \right)  a\equiv\left(  \operatorname*{Carl}P\right)  \left(
\operatorname*{Carl}\left(  P^{L}\right)  \right)  b\operatorname{mod}%
P^{k+L+1}A. \label{pf.cor.F.lift.lift-Pl.b.1}%
\end{equation}


Now, $\operatorname*{Carl}\left(  \underbrace{P^{L+1}}_{=PP^{L}}\right)
=\operatorname*{Carl}\left(  PP^{L}\right)  =\left(  \operatorname*{Carl}%
P\right)  \left(  \operatorname*{Carl}\left(  P^{L}\right)  \right)  $ (since
$\operatorname*{Carl}$ is an $\mathbb{F}_{q}$-algebra homomorphism). Thus,
(\ref{pf.cor.F.lift.lift-Pl.b.1}) rewrites as follows:%
\[
\left(  \operatorname*{Carl}\left(  P^{L+1}\right)  \right)  a\equiv\left(
\operatorname*{Carl}\left(  P^{L+1}\right)  \right)  b\operatorname{mod}%
P^{k+L+1}A.
\]
In other words, Corollary \ref{cor.F.lift.lift-Pl} \textbf{(b)} holds for
$\ell=L+1$. This completes the induction step. The induction proof of
Corollary \ref{cor.F.lift.lift-Pl} \textbf{(b)} is thus finished.
\end{proof}

In order to state the last corollary in this section, we need a definition:

\begin{definition}
\label{def.vpi}Let $\mathbb{K}$ be a field. Let $\pi$ be a monic irreducible
polynomial in $\mathbb{K}\left[  T\right]  $. Let $f$ be any polynomial in
$\mathbb{K}\left[  T\right]  $. Then, $v_{\pi}\left(  f\right)  $ means the
largest nonnegative integer $m$ satisfying $\pi^{m}\mid f$; this is set to be
$+\infty$ if $f=0$. Thus, $v_{\pi}\left(  f\right)  \in\mathbb{N}\cup\left\{
+\infty\right\}  $ for each $f$.

We set $P^{+\infty}=0$ for each $P\in\mathbb{K}\left[  T\right]  $. Thus,
$\pi^{v_{\pi}\left(  f\right)  }\mid f$ holds for each $f\in\mathbb{K}\left[
T\right]  $ (including the case when $f=0$).
\end{definition}

\begin{corollary}
\label{cor.F.lift.lift-all}Let $A$ be an $\mathcal{F}$-module. Let
$N\in\mathbb{F}_{q}\left[  T\right]  $. Let $\pi$ be a monic irreducible
polynomial in $\mathbb{F}_{q}\left[  T\right]  $.

Let $a$ and $b$ be two elements of $A$ such that $a\equiv b\operatorname{mod}%
\pi A$.

\textbf{(a)} We have $F^{\deg N}a\equiv F^{\deg N}b\operatorname{mod}%
\pi^{v_{\pi}\left(  N\right)  +1}A$. (Here, $F^{\deg N}$ is understood to mean
$0$ when $N=0$.)

\textbf{(b)} We have $\left(  \operatorname*{Carl}N\right)  a\equiv\left(
\operatorname*{Carl}N\right)  b\operatorname{mod}\pi^{v_{\pi}\left(  N\right)
+1}A$.
\end{corollary}

\begin{proof}
[Proof of Corollary \ref{cor.F.lift.lift-all}.]We have $a\equiv
b\operatorname{mod}\pi A$. In other words, $a\equiv b\operatorname{mod}\pi
^{1}A$ (since $\pi=\pi^{1}$).

If $N=0$, then Corollary \ref{cor.F.lift.lift-all} is easily seen to hold
(since $F^{\deg N}=0$ and $\operatorname*{Carl}\underbrace{N}_{=0}%
=\operatorname*{Carl}0=0$ in this case). Hence, we WLOG assume that $N\neq0$.
Thus, $v_{\pi}\left(  N\right)  \in\mathbb{N}$. Set $\ell=v_{\pi}\left(
N\right)  $. Then, $\pi^{\ell}\mid N$. In other words, there exists some
polynomial $M\in\mathbb{F}_{q}\left[  T\right]  $ such that $N=M\pi^{\ell}$.
Consider this $N$.

Proposition \ref{prop.F.lift.FPA} \textbf{(b)} (applied to $P=\pi^{1+\ell}$)
shows that the $\mathbb{F}_{q}$-vector subspace $\pi^{1+\ell}A$ of $A$ is a
left $\mathcal{F}$-submodule of $A$. Hence, $\mathcal{F}\cdot\pi^{1+\ell
}A\subseteq\pi^{1+\ell}A$.

\textbf{(a)} From $N=M\pi^{\ell}$, we obtain $\deg N=\deg\left(  M\pi^{\ell
}\right)  =\deg M+\deg\left(  \pi^{\ell}\right)  $, so that $F^{\deg
N}=F^{\deg M+\deg\left(  \pi^{\ell}\right)  }=F^{\deg M}F^{\deg\left(
\pi^{\ell}\right)  }$.

Corollary \ref{cor.F.lift.lift-Pl} \textbf{(a)} (applied to $P=\pi$ and $k=1$)
yields \newline$F^{\deg\left(  \pi^{\ell}\right)  }a\equiv F^{\deg\left(
\pi^{\ell}\right)  }b\operatorname{mod}\pi^{1+\ell}A$ (since $a\equiv
b\operatorname{mod}\pi^{1}A$). In other words, $F^{\deg\left(  \pi^{\ell
}\right)  }a-F^{\deg\left(  \pi^{\ell}\right)  }b\in\pi^{1+\ell}A$. But
\begin{align*}
&  F^{\deg N}a-F^{\deg N}b\\
&  =\underbrace{F^{\deg N}}_{=F^{\deg M}F^{\deg\left(  \pi^{\ell}\right)  }%
}\left(  a-b\right)  =\underbrace{F^{\deg M}}_{\in\mathcal{F}}%
\underbrace{F^{\deg\left(  \pi^{\ell}\right)  }\left(  a-b\right)  }%
_{=F^{\deg\left(  \pi^{\ell}\right)  }a-F^{\deg\left(  \pi^{\ell}\right)
}b\in\pi^{1+\ell}A}\\
&  \in\mathcal{F}\cdot\pi^{1+\ell}A\subseteq\pi^{1+\ell}A.
\end{align*}
In other words, $F^{\deg N}a\equiv F^{\deg N}b\operatorname{mod}\pi^{1+\ell}%
A$. Since $1+\underbrace{\ell}_{=v_{\pi}\left(  N\right)  }=1+v_{\pi}\left(
N\right)  =v_{\pi}\left(  N\right)  +1$, this rewrites as $F^{\deg N}a\equiv
F^{\deg N}b\operatorname{mod}\pi^{v_{\pi}\left(  N\right)  +1}A$. This proves
Corollary \ref{cor.F.lift.lift-all} \textbf{(a)}.

\textbf{(b)} From $N=M\pi^{\ell}$, we obtain $\operatorname*{Carl}%
N=\operatorname*{Carl}\left(  M\pi^{\ell}\right)  =\left(
\operatorname*{Carl}M\right)  \left(  \operatorname*{Carl}\left(  \pi^{\ell
}\right)  \right)  $ (since $\operatorname*{Carl}$ is an $\mathbb{F}_{q}%
$-algebra homomorphism).

Corollary \ref{cor.F.lift.lift-Pl} \textbf{(b)} (applied to $P=\pi$ and $k=1$)
yields $\left(  \operatorname*{Carl}\left(  \pi^{\ell}\right)  \right)
a\equiv\left(  \operatorname*{Carl}\left(  \pi^{\ell}\right)  \right)
b\operatorname{mod}\pi^{1+\ell}A$ (since $a\equiv b\operatorname{mod}\pi^{1}%
A$). In other words, $\left(  \operatorname*{Carl}\left(  \pi^{\ell}\right)
\right)  a-\left(  \operatorname*{Carl}\left(  \pi^{\ell}\right)  \right)
b\in\pi^{1+\ell}A$. But%
\begin{align*}
&  \left(  \operatorname*{Carl}N\right)  a-\left(  \operatorname*{Carl}%
N\right)  b\\
&  =\underbrace{\left(  \operatorname*{Carl}N\right)  }_{=\left(
\operatorname*{Carl}M\right)  \left(  \operatorname*{Carl}\left(  \pi^{\ell
}\right)  \right)  }\left(  a-b\right)  =\underbrace{\left(
\operatorname*{Carl}M\right)  }_{\in\mathcal{F}}\underbrace{\left(
\operatorname*{Carl}\left(  \pi^{\ell}\right)  \right)  \left(  a-b\right)
}_{=\left(  \operatorname*{Carl}\left(  \pi^{\ell}\right)  \right)  a-\left(
\operatorname*{Carl}\left(  \pi^{\ell}\right)  \right)  b\in\pi^{1+\ell}A}\\
&  \in\mathcal{F}\cdot\pi^{1+\ell}A\subseteq\pi^{1+\ell}A.
\end{align*}
In other words, $\left(  \operatorname*{Carl}N\right)  a\equiv\left(
\operatorname*{Carl}N\right)  b\operatorname{mod}\pi^{1+\ell}A$. Since
$1+\underbrace{\ell}_{=v_{\pi}\left(  N\right)  }=1+v_{\pi}\left(  N\right)
=v_{\pi}\left(  N\right)  +1$, this rewrites as $\left(  \operatorname*{Carl}%
N\right)  a\equiv\left(  \operatorname*{Carl}N\right)  b\operatorname{mod}%
\pi^{v_{\pi}\left(  N\right)  +1}A$. This proves Corollary
\ref{cor.F.lift.lift-all} \textbf{(b)}.
\end{proof}

Each of the two parts of Corollary \ref{cor.F.lift.lift-all} can be viewed as
an analogue of the classical \textquotedblleft exponent lifting
lemma\textquotedblright\ \cite[version with solution (ancillary file),
(12.349)]{reiner-hopf}.

\subsection{The Chinese Remainder Theorem}

Next, we recall one of the many versions of the Chinese Remainder Theorem:

\begin{theorem}
\label{thm.CRT.dg-witt5c}Let $A$ be a commutative ring. Let $M$ be an
$A$-module. Let $N\in\mathbb{N}$. Let $I_{1},I_{2},\ldots,I_{N}$ be $N$ ideals
of $A$. Assume that $I_{i}+I_{j}=A$ for any two elements $i$ and $j$ of
$\left\{  1,2,\ldots,N\right\}  $ satisfying $i<j$.

\textbf{(a)} We have $I_{1}I_{2}\cdots I_{N}\cdot M=I_{1}M\cap I_{2}%
M\cap\cdots\cap I_{N}M$.

\textbf{(b)} The canonical $A$-module homomorphism
\begin{align*}
M/\left(  I_{1}I_{2}\cdots I_{N}\cdot M\right)   &  \rightarrow\prod
\limits_{k=1}^{N}\left(  M/I_{k}M\right)  ,\\
m+I_{1}I_{2}\cdots I_{N}\cdot M  &  \mapsto\left(  m+I_{1}M,m+I_{2}%
M,\ldots,m+I_{N}M\right)
\end{align*}
is well-defined and an $A$-module isomorphism.
\end{theorem}

Theorem \ref{thm.CRT.dg-witt5c} is precisely \cite[Theorem 1 \textbf{(a)} and
\textbf{(b)}]{dg-witt5c}; thus, we are not giving a proof of it here.

For us, the following restatement of Theorem \ref{thm.CRT.dg-witt5c} will be
more useful:

\begin{theorem}
\label{thm.CRT}Let $A$ be a commutative ring. Let $M$ be an $A$-module. Let
$\mathbf{S}$ be a finite set. For every $s\in\mathbf{S}$, let $I_{s}$ be an
ideal of $A$. Assume that the ideals $I_{s}$ of $A$ are \textit{comaximal};
this means that every two distinct elements $s$ and $t$ of $\mathbf{S}$
satisfy $I_{s}+I_{t}=A$. Then:

\textbf{(a)} We have
\[
\left(  \prod\limits_{s\in\mathbf{S}}I_{s}\right)  \cdot M=\bigcap
_{s\in\mathbf{S}}\left(  I_{s}M\right)  .
\]


\textbf{(b)} The canonical $A$-module homomorphism
\begin{align*}
M/\left(  \left(  \prod\limits_{s\in\mathbf{S}}I_{s}\right)  \cdot M\right)
&  \rightarrow\prod\limits_{s\in\mathbf{S}}\left(  M/I_{s}M\right)  ,\\
m+\left(  \prod\limits_{s\in\mathbf{S}}I_{s}\right)  \cdot M  &
\mapsto\left(  m+I_{s}M\right)  _{s\in\mathbf{S}}%
\end{align*}
is well-defined and an $A$-module isomorphism.
\end{theorem}

\begin{proof}
[Proof of Theorem \ref{thm.CRT}.]We can freely relabel the elements of
$\mathbf{S}$. Thus, we can WLOG assume that $\mathbf{S}=\left\{
1,2,\ldots,N\right\}  $ for some $N\in\mathbb{N}$. Assume this, and consider
this $N$. Then, the claim of Theorem \ref{thm.CRT} becomes identical with the
claim of Theorem \ref{thm.CRT.dg-witt5c}. But since we already know that
Theorem \ref{thm.CRT.dg-witt5c} holds, we thus conclude that Theorem
\ref{thm.CRT} holds as well.
\end{proof}

We shall only use part \textbf{(a)} of Theorem \ref{thm.CRT}.

As a consequence of Theorem \ref{thm.CRT} \textbf{(a)}, we have the following:

\begin{corollary}
\label{cor.CRT.FqT}Let $A$ be an $\mathbb{F}_{q}\left[  T\right]  $-module.
Let $P$ be a monic polynomial in $\mathbb{F}_{q}\left[  T\right]  $. Then,
\[
\bigcap_{\pi\in\operatorname*{PF}P}\pi^{v_{\pi}\left(  P\right)  }A=PA.
\]

\end{corollary}

Before we can prove Corollary \ref{cor.CRT.FqT}, we need a simple lemma:

\begin{lemma}
\label{lem.CRT.FqT.Is+It}Let $\mathbb{F}$ be a field. Let $s$ and $t$ be two
distinct monic irreducible polynomials in $\mathbb{F}\left[  T\right]  $. Let
$n\in\mathbb{N}$ and $m\in\mathbb{N}$. Let $R$ be the ring $\mathbb{F}\left[
T\right]  $. Then, $s^{n}R+t^{m}R=R$.
\end{lemma}

\begin{proof}
[Proof of Lemma \ref{lem.CRT.FqT.Is+It}.]The polynomials $s$ and $t$ are two
distinct monic irreducible polynomials in $\mathbb{F}\left[  T\right]  $.
Hence, $s$ and $t$ are coprime. Consequently, $s^{n}$ and $t^{m}$ are coprime
as well (since $\mathbb{F}\left[  T\right]  $ is a principal ideal domain). By
Bezout's theorem, we thus conclude that there exist polynomials $a$ and $b$ in
$\mathbb{F}\left[  T\right]  $ satisfying $as^{n}+bt^{m}=1$. Consider these
$a$ and $b$.

The unity $1$ of the ring $R=\mathbb{F}\left[  T\right]  $ satisfies
\[
1=as^{n}+bt^{m}=s^{n}\underbrace{a}_{\in\mathbb{F}\left[  T\right]  =R}%
+t^{m}\underbrace{b}_{\in\mathbb{F}\left[  T\right]  =R}\in s^{n}R+t^{m}R.
\]
But $s^{n}R+t^{m}R$ is an ideal of $R$ (since $s^{n}R$ and $t^{m}R$ are ideals
of $R$). This ideal $s^{n}R+t^{m}R$ contains $1$ (since $1\in s^{n}R+t^{m}R$),
and thus must equal the whole ring $R$ (because if an ideal of some ring
contains $1$, then this ideal must equal the whole ring). In other words,
$s^{n}R+t^{m}R=R$. This proves Lemma \ref{lem.CRT.FqT.Is+It}.
\end{proof}

\begin{proof}
[Proof of Corollary \ref{cor.CRT.FqT}.]For each $s\in\operatorname*{PF}P$,
define an ideal $I_{s}$ of $\mathbb{F}_{q}\left[  T\right]  $ by
$I_{s}=s^{v_{s}\left(  P\right)  }\mathbb{F}_{q}\left[  T\right]  $. Notice
that $\mathbb{F}_{q}\left[  T\right]  $ is a principal ideal domain.

For each $s\in\operatorname*{PF}P$, we have%
\begin{equation}
I_{s}A=s^{v_{s}\left(  P\right)  }A \label{pf.cor.CRT.FqT.1}%
\end{equation}
\footnote{\textit{Proof of (\ref{pf.cor.CRT.FqT.1}):} Let $s\in
\operatorname*{PF}P$. Then, the definition of $I_{s}$ yields $I_{s}%
=s^{v_{s}\left(  P\right)  }\mathbb{F}_{q}\left[  T\right]  $. Now,%
\[
\underbrace{I_{s}}_{=s^{v_{s}\left(  P\right)  }\mathbb{F}_{q}\left[
T\right]  }A=s^{v_{s}\left(  P\right)  }\underbrace{\mathbb{F}_{q}\left[
T\right]  \cdot A}_{=A}=s^{v_{s}\left(  P\right)  }A.
\]
This proves (\ref{pf.cor.CRT.FqT.1}).}.

On the other hand, $P$ is a monic polynomial in $\mathbb{F}_{q}\left[
T\right]  $. Hence, the prime factorization of $P$ in the principal ideal
domain $\mathbb{F}_{q}\left[  T\right]  $ is $P=\prod_{s\in\operatorname*{PF}%
P}s^{v_{s}\left(  P\right)  }$ (indeed, for each $s\in\operatorname*{PF}P$,
the multiplicity of $s$ in the prime factorization of $P$ is $v_{s}\left(
P\right)  $). Now,%
\begin{align}
\prod\limits_{s\in\operatorname*{PF}P}\underbrace{I_{s}}_{\substack{=s^{v_{s}%
\left(  P\right)  }\mathbb{F}_{q}\left[  T\right]  \\\text{(by the}%
\\\text{definition of }I_{s}\text{)}}}  &  =\prod\limits_{s\in
\operatorname*{PF}P}\left(  s^{v_{s}\left(  P\right)  }\mathbb{F}_{q}\left[
T\right]  \right)  =\underbrace{\left(  \prod\limits_{s\in\operatorname*{PF}%
P}s^{v_{s}\left(  P\right)  }\right)  }_{=P}\mathbb{F}_{q}\left[  T\right]
\nonumber\\
&  =P\cdot\mathbb{F}_{q}\left[  T\right]  . \label{pf.cor.CRT.FqT.2}%
\end{align}


If $s$ and $t$ are two distinct elements of $\operatorname*{PF}P$, then
$I_{s}+I_{t}=\mathbb{F}_{q}\left[  T\right]  $%
\ \ \ \ \footnote{\textit{Proof.} Let $s$ and $t$ be two distinct elements of
$\operatorname*{PF}P$. Thus, $s$ and $t$ are two distinct monic irreducible
polynomials in $\mathbb{F}_{q}\left[  T\right]  $. Hence, Lemma
\ref{lem.CRT.FqT.Is+It} (applied to $\mathbb{F}=\mathbb{F}_{q}$,
$n=v_{s}\left(  P\right)  $, $m=v_{t}\left(  P\right)  $ and $R=\mathbb{F}%
_{q}\left[  T\right]  $) yields $s^{v_{s}\left(  P\right)  }\mathbb{F}%
_{q}\left[  T\right]  +t^{v_{t}\left(  P\right)  }\mathbb{F}_{q}\left[
T\right]  =\mathbb{F}_{q}\left[  T\right]  $.
\par
The definition of $I_{s}$ yields $I_{s}=s^{v_{s}\left(  P\right)  }%
\mathbb{F}_{q}\left[  T\right]  $. The definition of $I_{t}$ shows that
$I_{t}=t^{v_{t}\left(  P\right)  }\mathbb{F}_{q}\left[  T\right]  $. Hence,%
\[
\underbrace{I_{s}}_{=s^{v_{s}\left(  P\right)  }\mathbb{F}_{q}\left[
T\right]  }+\underbrace{I_{t}}_{=t^{v_{t}\left(  P\right)  }\mathbb{F}%
_{q}\left[  T\right]  }=s^{v_{s}\left(  P\right)  }\mathbb{F}_{q}\left[
T\right]  +t^{v_{t}\left(  P\right)  }\mathbb{F}_{q}\left[  T\right]
=\mathbb{F}_{q}\left[  T\right]  .
\]
Qed.}. Hence, Theorem \ref{thm.CRT} \textbf{(a)} (applied to $\mathbb{F}%
_{q}\left[  T\right]  $, $A$ and $\operatorname*{PF}P$ instead of $A$, $M$ and
$\mathbf{S}$) shows that%
\[
\left(  \prod\limits_{s\in\operatorname*{PF}P}I_{s}\right)  \cdot
A=\bigcap_{s\in\operatorname*{PF}P}\underbrace{\left(  I_{s}A\right)
}_{\substack{=s^{v_{s}\left(  P\right)  }A\\\text{(by (\ref{pf.cor.CRT.FqT.1}%
))}}}=\bigcap_{s\in\operatorname*{PF}P}s^{v_{s}\left(  P\right)  }%
A=\bigcap_{\pi\in\operatorname*{PF}P}\pi^{v_{\pi}\left(  P\right)  }A
\]
(here, we have renamed the index $s$ as $\pi$ in the intersection). Thus,%
\[
\bigcap_{\pi\in\operatorname*{PF}P}\pi^{v_{\pi}\left(  P\right)
}A=\underbrace{\left(  \prod\limits_{s\in\operatorname*{PF}P}I_{s}\right)
}_{\substack{=P\cdot\mathbb{F}_{q}\left[  T\right]  \\\text{(by
(\ref{pf.cor.CRT.FqT.2}))}}}\cdot A=P\cdot\underbrace{\mathbb{F}_{q}\left[
T\right]  \cdot A}_{=A}=PA.
\]
This proves Corollary \ref{cor.CRT.FqT}.
\end{proof}

Let me also state the \textquotedblleft ring version\textquotedblright\ of the
Chinese Remainder theorem:

\begin{theorem}
\label{thm.CRT.ring}Let $A$ be a commutative ring. Let $\mathbf{S}$ be a
finite set. For every $s\in\mathbf{S}$, let $I_{s}$ be an ideal of $A$. Assume
that the ideals $I_{s}$ of $A$ are \textit{comaximal}; this means that every
two distinct elements $s$ and $t$ of $\mathbf{S}$ satisfy $I_{s}+I_{t}=A$. Then:

\textbf{(a)} We have
\[
\prod\limits_{s\in\mathbf{S}}I_{s}=\bigcap_{s\in\mathbf{S}}I_{s}.
\]


\textbf{(b)} The canonical $A$-algebra homomorphism
\[
A/\left(  \prod\limits_{s\in\mathbf{S}}I_{s}\right)  \rightarrow
\prod\limits_{s\in\mathbf{S}}\left(  A/I_{s}\right)
,\ \ \ \ \ \ \ \ \ \ a+\prod\limits_{s\in\mathbf{S}}I_{s}\mapsto\left(
a+I_{s}\right)  _{s\in\mathbf{S}}%
\]
is well-defined and an $A$-algebra isomorphism.
\end{theorem}

Theorem \ref{thm.CRT.ring} can easily be derived by applying Theorem
\ref{thm.CRT} to $M=A$. (The extra claim that the homomorphism in Theorem
\ref{thm.CRT.ring} \textbf{(b)} is an $A$-algebra homomorphism is
straightforward to check.) But Theorem \ref{thm.CRT.ring} is also a classical
fact that appears in many textbooks on algebra (it is probably easier to find
than Theorem \ref{thm.CRT}).

Let me continue with another simple lemma about divisibility of polynomials:

\begin{lemma}
\label{lem.FqT.exact-divisor}Let $P$ be a polynomial in $\mathbb{F}_{q}\left[
T\right]  $. Let $\pi$ be a monic irreducible divisor of $P$. Let $D$ be a
divisor of $P$ satisfying $D\nmid P/\pi$. Then, $\pi^{v_{\pi}\left(  P\right)
}\mid D$.
\end{lemma}

\begin{proof}
[Proof of Lemma \ref{lem.FqT.exact-divisor}.]From $D\nmid P/\pi$, we obtain
$P/\pi\neq0$, hence $P\neq0$.

We have $D\nmid P/\pi$. In other words, $\dfrac{P/\pi}{D}\notin\mathbb{F}%
_{q}\left[  T\right]  $. This rewrites as $\dfrac{P/D}{\pi}\notin%
\mathbb{F}_{q}\left[  T\right]  $ (since $\dfrac{P/\pi}{D}=\dfrac{P/D}{\pi}$).
Equivalently, $\pi\nmid P/D$ (since $P/D\in\mathbb{F}_{q}\left[  T\right]  $
(because $D$ is a divisor of $P$)). In other words, $v_{\pi}\left(
P/D\right)  =0$. Hence, $0=v_{\pi}\left(  P/D\right)  =v_{\pi}\left(
P\right)  -v_{\pi}\left(  D\right)  $, so that $v_{\pi}\left(  P\right)
=v_{\pi}\left(  D\right)  $.

But $\pi^{v_{\pi}\left(  D\right)  }\mid D$ (obviously). Since $v_{\pi}\left(
P\right)  =v_{\pi}\left(  D\right)  $, we now have $\pi^{v_{\pi}\left(
P\right)  }=\pi^{v_{\pi}\left(  D\right)  }\mid D$. This proves Lemma
\ref{lem.FqT.exact-divisor}.
\end{proof}

Here is a well-known fact about quotients of polynomial rings over fields:

\begin{proposition}
\label{prop.FT.modsn}Let $\mathbb{F}$ be a field. Let $s\in\mathbb{F}\left[
T\right]  $ be a monic irreducible polynomial. Let $n$ be a positive integer.
Let $B$ be the ring $\mathbb{F}\left[  T\right]  /s^{n}\mathbb{F}\left[
T\right]  $. Then:

\textbf{(a)} We have $B^{\times}=B\setminus sB$. (Here, $B^{\times}$ denotes
the group of units of the ring $B$.)

\textbf{(b)} We have $sB\cong\mathbb{F}\left[  T\right]  /s^{n-1}%
\mathbb{F}\left[  T\right]  $ as $\mathbb{F}$-vector spaces.
\end{proposition}

\begin{proof}
[Proof of Proposition \ref{prop.FT.modsn}.]For every $a\in\mathbb{F}\left[
T\right]  $, we let $\overline{a}$ denote the canonical projection of $a$ on
$\mathbb{F}\left[  T\right]  /s^{n}\mathbb{F}\left[  T\right]  =B$.

\textbf{(a)} We shall prove the inclusions $B^{\times}\subseteq B\setminus sB$
and $B\setminus sB\subseteq B^{\times}$ separately:

\textit{Proof of }$B^{\times}\subseteq B\setminus sB$\textit{:} Let $b\in
B^{\times}$.

We have $b\in B^{\times}$. In other words, the element $b$ of $B$ is
invertible. In other words, there exists some $d\in B$ such that $bd=1$.
Consider this $d$.

We have $d\in B$. Thus, $d=\overline{c}$ for some $c\in\mathbb{F}\left[
T\right]  $. Consider this $c$.

Now, assume (for the sake of contradiction) that $b\in sB$. In other words,
$b=sf$ for some $f\in B$. Consider this $f$.

We have $f\in B$. Thus, $f=\overline{e}$ for some $e\in\mathbb{F}\left[
T\right]  $. Consider this $e$. Multiplying the equalities $f=\overline{e}$
and $d=\overline{c}$, we obtain $fd=\overline{e}\cdot\overline{c}%
=\overline{ec}=\overline{ce}$.

Now, $bd=1$, so that $1=\underbrace{b}_{=sf}d=s\underbrace{fd}_{=\overline
{ce}}=s\overline{ce}=\overline{sce}$. In other words, $1\equiv
sce\operatorname{mod}s^{n}\mathbb{F}\left[  T\right]  $. In other words,
$s^{n}\mid1-sce$. But since $n$ is positive, we have $s\mid s^{n}\mid1-sce$.
Thus, the polynomial $1-sce$ is divisible by $s$. Also, the polynomial $sce$
is divisible by $s$ (clearly). Hence, the sum of these two polynomials $1-sce$
and $sce$ must also divisible by $s$. In other words, $\left(  1-sce\right)
+sce$ is divisible by $s$. In other words, $1$ is divisible by $s$ (since
$\left(  1-sce\right)  +sce=1$). This is clearly absurd (since $s$ is
irreducible). Thus, we have found a contradiction. This shows that our
assumption (that $b\in sB$) was false.

Hence, $b\notin sB$. Combining this with $b\in B$, we obtain $b\in B\setminus
sB$.

Now, forget that we fixed $b$. We thus have proven that $b\in B\setminus sB$
for each $b\in B^{\times}$. In other words, $B^{\times}\subseteq B\setminus
sB$.

\textit{Proof of }$B\setminus sB\subseteq B^{\times}$\textit{:} Let $b\in
B\setminus sB$. Then, $b\in B\setminus sB\subseteq B$. Hence, $b=\overline{a}$
for some $a\in\mathbb{F}\left[  T\right]  $. Consider this $a$.

We have $s\nmid a$\ \ \ \ \footnote{\textit{Proof.} Assume the contrary. Thus,
$s\mid a$. In other words, $a=cs$ for some $c\in\mathbb{F}\left[  T\right]  $.
Consider this $c$. From $a=cs=sc$, we obtain $\overline{a}=\overline
{cs}=\overline{sc}=s\underbrace{\overline{c}}_{\in B}\in sB$. But
$\overline{a}=b\in B\setminus sB$ and thus $\overline{a}\notin sB$. This
contradicts $\overline{a}\in sB$. This contradiction shows that our assumption
was wrong; qed.}. Hence, the polynomials $a$ and $s$ are coprime (since $s$ is
irreducible, and since $\mathbb{F}\left[  T\right]  $ is a principal ideal
domain). Therefore, the polynomials $a$ and $s^{n}$ are coprime (since
$\mathbb{F}\left[  T\right]  $ is a principal ideal domain). By Bezout's
theorem, we thus conclude that there exist polynomials $\alpha$ and $\beta$ in
$\mathbb{F}\left[  T\right]  $ satisfying $\alpha a+\beta s^{n}=1$. Consider
these $\alpha$ and $\beta$.

The unity $1$ of the ring $\mathbb{F}\left[  T\right]  $ satisfies $1=\alpha
a+\underbrace{\beta s^{n}}_{\substack{\equiv0\operatorname{mod}s^{n}%
\mathbb{F}\left[  T\right]  \\\text{(since }s^{n}\mid\beta s^{n}\text{)}%
}}\equiv\alpha a\operatorname{mod}s^{n}\mathbb{F}\left[  T\right]  $. In other
words, $\overline{1}=\overline{\alpha a}$. Comparing this with $\overline
{\alpha}\underbrace{b}_{=\overline{a}}=\overline{\alpha}\cdot\overline
{a}=\overline{\alpha a}$, we obtain $\overline{\alpha}b=\overline{1}=1$.
Hence, the element $b$ of $B$ is invertible. In other words, $b\in B^{\times}$.

Now, forget that we fixed $b$. We thus have proven that $b\in B^{\times}$ for
each $b\in B\setminus sB$. In other words, $B\setminus sB\subseteq B^{\times}$.

Combining the two relations $B^{\times}\subseteq B\setminus sB$ and
$B\setminus sB\subseteq B^{\times}$, we obtain $B^{\times}=B\setminus sB$.
Thus, Proposition \ref{prop.FT.modsn} \textbf{(a)} is proven.

\textbf{(b)} Let $\rho$ be the map $\mathbb{F}\left[  T\right]  \rightarrow
sB,\ f\mapsto s\overline{f}$. It is straightforward to see that this map
$\rho$ is well-defined and $\mathbb{F}$-linear. Moreover, $\operatorname*{Ker}%
\rho\subseteq s^{n-1}\mathbb{F}\left[  T\right]  $%
\ \ \ \ \footnote{\textit{Proof.} Let $a\in\operatorname*{Ker}\rho$. Thus,
$a\in\mathbb{F}\left[  T\right]  $ and $\rho\left(  a\right)  =0$. Now, the
definition of $\rho$ yields $\rho\left(  a\right)  =s\overline{a}%
=\overline{sa}$. Hence, $\overline{sa}=\rho\left(  a\right)  =0$. In other
words, $sa\in s^{n}\mathbb{F}\left[  T\right]  $. In other words, $s^{n}\mid
sa$ in $\mathbb{F}\left[  T\right]  $. In other words, there exists some
$g\in\mathbb{F}\left[  T\right]  $ satisfying $sa=s^{n}g$. Consider this $g$.
\par
The polynomial $s$ is irreducible and thus nonzero. Hence, we can cancel $s$
from the equation $sa=\underbrace{s^{n}}_{=ss^{n-1}}g=ss^{n-1}g$ (since
$\mathbb{F}\left[  T\right]  $ is an integral domain). We thus obtain
$a=s^{n-1}\underbrace{g}_{\in\mathbb{F}\left[  T\right]  }\in s^{n-1}%
\mathbb{F}\left[  T\right]  $.
\par
Now, forget that we fixed $a$. We thus have shown that $a\in s^{n-1}%
\mathbb{F}\left[  T\right]  $ for each $a\in\operatorname*{Ker}\rho$. In other
words, $\operatorname*{Ker}\rho\subseteq s^{n-1}\mathbb{F}\left[  T\right]  $.
Qed.} and $s^{n-1}\mathbb{F}\left[  T\right]  \subseteq\operatorname*{Ker}%
\rho$\ \ \ \ \footnote{\textit{Proof.} Let $f\in s^{n-1}\mathbb{F}\left[
T\right]  $. Thus, there exists some $g\in\mathbb{F}\left[  T\right]  $
satisfying $f=s^{n-1}g$. Consider this $g$. Now, the definition of $\rho$
yields
\begin{align*}
\rho\left(  f\right)   &  =s\overline{f}=s\overline{s^{n-1}g}%
\ \ \ \ \ \ \ \ \ \ \left(  \text{since }f=s^{n-1}g\right) \\
&  =\overline{ss^{n-1}g}=0\ \ \ \ \ \ \ \ \ \ \left(  \text{since
}\underbrace{ss^{n-1}}_{=s^{n}}g=s^{n}\underbrace{g}_{\in\mathbb{F}\left[
T\right]  }\in s^{n}\mathbb{F}\left[  T\right]  \right)  .
\end{align*}
In other words, $f\in\operatorname*{Ker}\rho$.
\par
Now, forget that we fixed $f$. We thus have proven that $f\in
\operatorname*{Ker}\rho$ for each $f\in s^{n-1}\mathbb{F}\left[  T\right]  $.
In other words, $s^{n-1}\mathbb{F}\left[  T\right]  \subseteq
\operatorname*{Ker}\rho$. Qed.}. Combining these two inclusions, we obtain
$\operatorname*{Ker}\rho=s^{n-1}\mathbb{F}\left[  T\right]  $. Moreover, the
map $\rho$ is surjective\footnote{\textit{Proof.} Let $a\in sB$. Thus, there
exists some $b\in B$ such that $a=sb$. Consider this $b$. Now, we have $b\in
B$. Hence, $b=\overline{f}$ for some $f\in\mathbb{F}\left[  T\right]  $.
Consider this $f$. The definition of $\rho$ yields $\rho\left(  f\right)
=s\overline{f}=sb$ (since $\overline{f}=b$). Compared with $a=sb$, this yields
$a=\rho\left(  \underbrace{f}_{\in\mathbb{F}\left[  T\right]  }\right)
\in\rho\left(  \mathbb{F}\left[  T\right]  \right)  $.
\par
Now, forget that we fixed $a$. We thus have proven that $a\in\rho\left(
\mathbb{F}\left[  T\right]  \right)  $ for each $a\in sB$. In other words,
$sB\subseteq\rho\left(  \mathbb{F}\left[  T\right]  \right)  $. In other
words, the map $\rho$ is surjective. Qed.}. Hence, $\rho\left(  \mathbb{F}%
\left[  T\right]  \right)  =sB$.

Now, the first isomorphism theorem (applied to the $\mathbb{F}$-linear map
$\rho:\mathbb{F}\left[  T\right]  \rightarrow sB$) yields $\rho\left(
\mathbb{F}\left[  T\right]  \right)  \cong\mathbb{F}\left[  T\right]
/\underbrace{\operatorname*{Ker}\rho}_{=s^{n-1}\mathbb{F}\left[  T\right]
}=\mathbb{F}\left[  T\right]  /s^{n-1}\mathbb{F}\left[  T\right]  $ as
$\mathbb{F}$-vector spaces. In light of $\rho\left(  \mathbb{F}\left[
T\right]  \right)  =sB$, this rewrites as $sB\cong\mathbb{F}\left[  T\right]
/s^{n-1}\mathbb{F}\left[  T\right]  $. Thus, Proposition \ref{prop.FT.modsn}
\textbf{(b)} is proven.
\end{proof}

\subsection{Ghost-Witt integrality: a general equivalence}

Recall the notion of a \textquotedblleft$q$-nest\textquotedblright\ defined in
Definition \ref{def.q-nest}. Recall also Definition \ref{def.PF(q)}.
Furthermore, recall the following convention:

\begin{definition}
Let $P$ be a monic polynomial in $\mathbb{F}_{q}\left[  T\right]  $. Then, the
summation sign $\sum_{D\mid P}$ means a sum over all \textbf{monic}
polynomials $D$ dividing $P$.
\end{definition}

We shall now prove a very general fact that encompasses some of the claims of
Theorem \ref{thm.carlitz.gW}:

\begin{theorem}
\label{thm.F.gW-general}Let $N$ be a $q$-nest. Let $A$ be an $\mathcal{F}%
$-module. For every $P\in N$, let $\varphi_{P}$ and $\psi_{P}$ be two
endomorphisms of the $\mathbb{F}_{q}$-vector space $A$. Let us make the
following five assumptions:

\textit{Assumption 1:} For every $P\in N$, the map $\varphi_{P}$ is an
endomorphism of the $\mathcal{F}$-module $A$.

\textit{Assumption 2:} We have $\varphi_{\pi}\left(  a\right)  \equiv\left(
\operatorname*{Carl}\pi\right)  a\operatorname{mod}\pi A$ for every $a\in A$
and every monic irreducible $\pi\in N$.

\textit{Assumption 3:} We have $\varphi_{1}=\operatorname*{id}$. Furthermore,
$\varphi_{P}\circ\varphi_{Q}=\varphi_{PQ}$ for every $P\in N$ and every $Q\in
N$ satisfying $PQ\in N$.

\textit{Assumption 4:} We have $\psi_{P}\left(  a\right)  \equiv\varphi_{\pi
}\left(  \psi_{P/\pi}\left(  a\right)  \right)  \operatorname{mod}\pi^{v_{\pi
}\left(  P\right)  }A$ for every $a\in A$, every $P\in N$ and every $\pi
\in\operatorname*{PF}P$.

\textit{Assumption 5:} We have $\psi_{1}=\operatorname*{id}$.

Let $\left(  b_{P}\right)  _{P\in N}\in A^{N}$ be a family of elements of $A$.
Then, the following assertions $\mathcal{C}_{1}$ and $\mathcal{E}_{\psi}$ are equivalent:

\textit{Assertion }$\mathcal{C}_{1}$\textit{:} Every $P\in N$ and every
$\pi\in\operatorname{PF}P$ satisfy%
\[
\varphi_{\pi}\left(  b_{P / \pi}\right)  \equiv b_{P}\operatorname{mod}%
\pi^{v_{\pi}\left(  P\right)  }A.
\]


\textit{Assertion }$\mathcal{E}_{\psi}$\textit{:} There exists a family
$\left(  z_{P}\right)  _{P\in N}\in A^{N}$ of elements of $A$ such that%
\[
\left(  b_{P}=\sum_{D\mid P}D\psi_{P / D}\left(  z_{D}\right)  \text{ for
every }P\in N\right)  .
\]

\end{theorem}

Before we prove this theorem, let us make a few comments.

\begin{remark}
\label{rmk.F.gW-general.ass2}Let $N$ be a $q$-nest. Let $A$ be an
$\mathcal{F}$-module. For every $P\in N$, let $\varphi_{P}$ be an endomorphism
of the $\mathbb{F}_{q}$-vector space $A$. Then, Assumption 2 in Theorem
\ref{thm.F.gW-general} is equivalent to the following statement: We have
$\varphi_{\pi}\left(  a\right)  \equiv F^{\deg\pi}a\operatorname{mod}\pi A$
for every $a\in A$ and every monic irreducible $\pi\in N$.
\end{remark}

\begin{proof}
[Proof of Remark \ref{rmk.F.gW-general.ass2}.]It is clearly enough to show
that $\left(  \operatorname*{Carl}\pi\right)  a\equiv F^{\deg\pi
}a\operatorname{mod}\pi A$ for every $a\in A$ and every monic irreducible
$\pi\in N$. But this follows from Corollary \ref{cor.F.u(pi).mod}. Thus,
Remark \ref{rmk.F.gW-general.ass2} is proven.
\end{proof}

Next, let us show examples of endomorphisms $\psi_{P}$ satisfying the
Assumption 4 of Theorem \ref{thm.F.gW-general}:

\begin{proposition}
\label{prop.F.gW-general.ex1}Let $N$ be a $q$-nest. Let $A$ be an
$\mathcal{F}$-module. For every $P\in N$, let $\varphi_{P}$ be an endomorphism
of the $\mathbb{F}_{q}$-vector space $A$. Assume that the Assumptions 1 and 2
of Theorem \ref{thm.F.gW-general} are satisfied.

For every $P\in N$, define an endomorphism $\psi_{P}$ of the $\mathbb{F}_{q}%
$-vector space $A$ by%
\[
\left(  \psi_{P}\left(  a\right)  =\left(  \operatorname*{Carl}P\right)
a\ \ \ \ \ \ \ \ \ \ \text{for every }a\in A\right)  .
\]
Then, Assumptions 4 and 5 of Theorem \ref{thm.F.gW-general} are satisfied.
\end{proposition}

\begin{proposition}
\label{prop.F.gW-general.ex2}Let $N$ be a $q$-nest. Let $A$ be an
$\mathcal{F}$-module. For every $P\in N$, let $\varphi_{P}$ be an endomorphism
of the $\mathbb{F}_{q}$-vector space $A$. Assume that the Assumption 1 and 2
of Theorem \ref{thm.F.gW-general} are satisfied.

For every $P\in N$, define an endomorphism $\psi_{P}$ of the $\mathbb{F}_{q}%
$-vector space $A$ by%
\[
\left(  \psi_{P}\left(  a\right)  =F^{\deg P}a\ \ \ \ \ \ \ \ \ \ \text{for
every }a\in A\right)  .
\]
Then, Assumptions 4 and 5 of Theorem \ref{thm.F.gW-general} are satisfied.
\end{proposition}

\begin{proposition}
\label{prop.F.gW-general.ex3}Let $N$ be a $q$-nest. Let $A$ be an
$\mathcal{F}$-module. For every $P\in N$, let $\varphi_{P}$ be an endomorphism
of the $\mathbb{F}_{q}$-vector space $A$. Assume that the Assumption 3 of
Theorem \ref{thm.F.gW-general} is satisfied.

For every $P\in N$, define an endomorphism $\psi_{P}$ of the $\mathbb{F}_{q}%
$-vector space $A$ by%
\[
\psi_{P}=\varphi_{P}.
\]
Then, Assumptions 4 and 5 of Theorem \ref{thm.F.gW-general} are satisfied.
\end{proposition}

\begin{proof}
[Proof of Proposition \ref{prop.F.gW-general.ex1}.]Assumption 5 of Theorem
\ref{thm.F.gW-general} is satisfied\footnote{\textit{Proof.} We have
$\operatorname*{Carl}1=1$ (since $\operatorname*{Carl}$ is an $\mathbb{F}_{q}%
$-algebra homomorphism). Now, every $a\in A$ satisfies
\begin{align*}
\psi_{1}\left(  a\right)   &  =\underbrace{\left(  \operatorname*{Carl}%
1\right)  }_{=1}a\ \ \ \ \ \ \ \ \ \ \left(  \text{by the definition of }%
\psi_{1}\right) \\
&  =1a=a=\operatorname*{id}\left(  a\right)  .
\end{align*}
In other words, $\psi_{1}=1$. In other words, Assumption 5 of Theorem
\ref{thm.F.gW-general} is satisfied, qed.}. Hence, it remains to show that
Assumption 4 of Theorem \ref{thm.F.gW-general} is satisfied. In other words,
we must prove that we have $\psi_{P}\left(  a\right)  \equiv\varphi_{\pi
}\left(  \psi_{P/\pi}\left(  a\right)  \right)  \operatorname{mod}\pi^{v_{\pi
}\left(  P\right)  }A$ for every $a\in A$, every $P\in N$ and every $\pi
\in\operatorname*{PF}P$.

So let us fix $a\in A$, $P\in N$ and $\pi\in\operatorname*{PF}P$. Clearly,
$\pi\mid P$ (since $\pi\in\operatorname*{PF}P$), and $\pi$ is a monic
irreducible polynomial in $\mathbb{F}_{q}\left[  T\right]  $ (since $\pi
\in\operatorname*{PF}P$). From these two facts, we obtain $\pi\in N$ (since
$N$ is a $q$-nest). Thus, Assumption 2 of Theorem \ref{thm.F.gW-general}
yields $\varphi_{\pi}\left(  a\right)  \equiv\left(  \operatorname*{Carl}%
\pi\right)  \left(  a\right)  \operatorname{mod}\pi A$.

Also, $P/\pi\in\mathbb{F}_{q}\left[  T\right]  $ (since $\pi\mid P$). Hence,
$\psi_{P/\pi}\left(  a\right)  =\left(  \operatorname*{Carl}\left(
P/\pi\right)  \right)  \left(  a\right)  $ (by the definition of $\psi_{P/\pi
}$).

Corollary \ref{cor.F.lift.lift-all} \textbf{(b)} (applied to $P/\pi$,
$\varphi_{\pi}\left(  a\right)  $ and $\left(  \operatorname*{Carl}\pi\right)
a$ instead of $N$, $a$ and $b$) shows that%
\[
\left(  \operatorname*{Carl}\left(  P/\pi\right)  \right)  \left(
\varphi_{\pi}\left(  a\right)  \right)  \equiv\left(  \operatorname*{Carl}%
\left(  P/\pi\right)  \right)  \left(  \left(  \operatorname*{Carl}\pi\right)
a\right)  \operatorname{mod}\pi^{v_{\pi}\left(  P/\pi\right)  +1}A.
\]
In view of
\[
v_{\pi}\left(  P/\pi\right)  +\underbrace{1}_{=v_{\pi}\left(  \pi\right)
}=v_{\pi}\left(  P/\pi\right)  +v_{\pi}\left(  \pi\right)  =v_{\pi}\left(
\underbrace{\left(  P/\pi\right)  \pi}_{=P}\right)  =v_{\pi}\left(  P\right)
,
\]
this rewrites as
\begin{equation}
\left(  \operatorname*{Carl}\left(  P/\pi\right)  \right)  \left(
\varphi_{\pi}\left(  a\right)  \right)  \equiv\left(  \operatorname*{Carl}%
\left(  P/\pi\right)  \right)  \left(  \left(  \operatorname*{Carl}\pi\right)
a\right)  \operatorname{mod}\pi^{v_{\pi}\left(  P\right)  }A.
\label{pf.prop.F.gW-general.ex1.a1}%
\end{equation}


But $\varphi_{\pi}$ is an endomorphism of the $\mathcal{F}$-module $A$ (by
Assumption 1 of Theorem \ref{thm.F.gW-general}, applied to $\pi$ instead of
$P$). Hence,%
\[
\left(  \operatorname*{Carl}\left(  P/\pi\right)  \right)  \left(
\varphi_{\pi}\left(  a\right)  \right)  =\varphi_{\pi}\left(
\underbrace{\left(  \operatorname*{Carl}\left(  P/\pi\right)  \right)  \left(
a\right)  }_{=\psi_{P/\pi}\left(  a\right)  }\right)  =\varphi_{\pi}\left(
\psi_{P/\pi}\left(  a\right)  \right)  .
\]
Thus,%
\begin{align*}
\varphi_{\pi}\left(  \psi_{P/\pi}\left(  a\right)  \right)   &  =\left(
\operatorname*{Carl}\left(  P/\pi\right)  \right)  \left(  \varphi_{\pi
}\left(  a\right)  \right)  \equiv\left(  \operatorname*{Carl}\left(
P/\pi\right)  \right)  \left(  \left(  \operatorname*{Carl}\pi\right)
a\right)  \ \ \ \ \ \ \ \ \ \ \left(  \text{by
(\ref{pf.prop.F.gW-general.ex1.a1})}\right) \\
&  =\underbrace{\left(  \operatorname*{Carl}\left(  P/\pi\right)
\cdot\operatorname*{Carl}\pi\right)  }_{\substack{=\operatorname*{Carl}\left(
\left(  P/\pi\right)  \pi\right)  \\\text{(since }\operatorname*{Carl}\text{
is an }\mathbb{F}_{q}\text{-algebra}\\\text{homomorphism)}}}a\\
&  =\left(  \operatorname*{Carl}\underbrace{\left(  \left(  P/\pi\right)
\pi\right)  }_{=P}\right)  a=\left(  \operatorname*{Carl}P\right)  a\\
&  =\psi_{P}\left(  a\right)  \operatorname{mod}\pi^{v_{\pi}\left(  P\right)
}A
\end{align*}
(since $\psi_{P}\left(  a\right)  =\left(  \operatorname*{Carl}P\right)  a$
(by the definition of $\psi_{P}$)). In other words, $\psi_{P}\left(  a\right)
\equiv\varphi_{\pi}\left(  \psi_{P/\pi}\left(  a\right)  \right)
\operatorname{mod}\pi^{v_{\pi}\left(  P\right)  }A$. Thus, Assumption 4 of
Theorem \ref{thm.F.gW-general} is satisfied. This proves Proposition
\ref{prop.F.gW-general.ex1}.
\end{proof}

\begin{proof}
[Proof of Proposition \ref{prop.F.gW-general.ex2}.]Assumption 5 of Theorem
\ref{thm.F.gW-general} is satisfied\footnote{\textit{Proof.} Every $a\in A$
satisfies
\begin{align*}
\psi_{1}\left(  a\right)   &  =F^{\deg1}a\ \ \ \ \ \ \ \ \ \ \left(  \text{by
the definition of }\psi_{1}\right) \\
&  =1a\ \ \ \ \ \ \ \ \ \ \left(  \text{since }\deg1=0\text{ and thus }%
F^{\deg1}=F^{0}=1\right) \\
&  =a=\operatorname*{id}\left(  a\right)  .
\end{align*}
In other words, $\psi_{1}=1$. In other words, Assumption 5 of Theorem
\ref{thm.F.gW-general} is satisfied, qed.}. Hence, it remains to show that
Assumption 4 of Theorem \ref{thm.F.gW-general} is satisfied. In other words,
we must prove that we have $\psi_{P}\left(  a\right)  \equiv\varphi_{\pi
}\left(  \psi_{P/\pi}\left(  a\right)  \right)  \operatorname{mod}\pi^{v_{\pi
}\left(  P\right)  }A$ for every $a\in A$, every $P\in N$ and every $\pi
\in\operatorname*{PF}P$.

So let us fix $a\in A$, $P\in N$ and $\pi\in\operatorname*{PF}P$. Clearly,
$\pi\mid P$ (since $\pi\in\operatorname*{PF}P$), and $\pi$ is a monic
irreducible polynomial in $\mathbb{F}_{q}\left[  T\right]  $ (since $\pi
\in\operatorname*{PF}P$). From these two facts, we obtain $\pi\in N$ (since
$N$ is a $q$-nest). Thus, Assumption 2 of Theorem \ref{thm.F.gW-general}
yields $\varphi_{\pi}\left(  a\right)  \equiv\left(  \operatorname*{Carl}%
\pi\right)  \left(  a\right)  \operatorname{mod}\pi A$. Thus,%
\begin{equation}
\varphi_{\pi}\left(  a\right)  \equiv\left(  \operatorname*{Carl}\pi\right)
\left(  a\right)  \equiv F^{\deg\pi}a\operatorname{mod}\pi A
\label{pf.prop.F.gW-general.ex2.0}%
\end{equation}
(by Corollary \ref{cor.F.u(pi).mod}).

Also, $P/\pi\in\mathbb{F}_{q}\left[  T\right]  $ (since $\pi\mid P$). Hence,
$\psi_{P/\pi}\left(  a\right)  =F^{\deg\left(  P/\pi\right)  }\left(
a\right)  $ (by the definition of $\psi_{P/\pi}$).

Corollary \ref{cor.F.lift.lift-all} \textbf{(a)} (applied to $P/\pi$,
$\varphi_{\pi}\left(  a\right)  $ and $F^{\deg\pi}a$ instead of $N$, $a$ and
$b$) shows that%
\[
F^{\deg\left(  P/\pi\right)  }\left(  \varphi_{\pi}\left(  a\right)  \right)
\equiv F^{\deg\left(  P/\pi\right)  }\left(  F^{\deg\pi}a\right)
\operatorname{mod}\pi^{v_{\pi}\left(  P/\pi\right)  +1}A.
\]
In view of
\[
v_{\pi}\left(  P/\pi\right)  +\underbrace{1}_{=v_{\pi}\left(  \pi\right)
}=v_{\pi}\left(  P/\pi\right)  +v_{\pi}\left(  \pi\right)  =v_{\pi}\left(
\underbrace{\left(  P/\pi\right)  \pi}_{=P}\right)  =v_{\pi}\left(  P\right)
,
\]
this rewrites as
\begin{equation}
F^{\deg\left(  P/\pi\right)  }\left(  \varphi_{\pi}\left(  a\right)  \right)
\equiv F^{\deg\left(  P/\pi\right)  }\left(  F^{\deg\pi}a\right)
\operatorname{mod}\pi^{v_{\pi}\left(  P\right)  }A.
\label{pf.prop.F.gW-general.ex2.a1}%
\end{equation}


But $\varphi_{\pi}$ is an endomorphism of the $\mathcal{F}$-module $A$ (by
Assumption 1 of Theorem \ref{thm.F.gW-general}, applied to $\pi$ instead of
$P$). Hence,%
\[
F^{\deg\left(  P/\pi\right)  }\left(  \varphi_{\pi}\left(  a\right)  \right)
=\varphi_{\pi}\left(  \underbrace{F^{\deg\left(  P/\pi\right)  }\left(
a\right)  }_{=\psi_{P/\pi}\left(  a\right)  }\right)  =\varphi_{\pi}\left(
\psi_{P/\pi}\left(  a\right)  \right)  .
\]
Thus,%
\begin{align*}
\varphi_{\pi}\left(  \psi_{P/\pi}\left(  a\right)  \right)   &  =F^{\deg
\left(  P/\pi\right)  }\left(  \varphi_{\pi}\left(  a\right)  \right)  \equiv
F^{\deg\left(  P/\pi\right)  }\left(  F^{\deg\pi}a\right)
\ \ \ \ \ \ \ \ \ \ \left(  \text{by (\ref{pf.prop.F.gW-general.ex2.a1}%
)}\right) \\
&  =\underbrace{\left(  F^{\deg\left(  P/\pi\right)  }F^{\deg\pi}\right)
}_{\substack{=F^{\deg\left(  P/\pi\right)  +\deg\pi}=F^{\deg P}\\\text{(since
}\deg\left(  P/\pi\right)  +\deg\pi=\deg P\\\text{(since }\deg\left(
P/\pi\right)  =\deg P-\deg\pi\text{))}}}a=F^{\deg P}a\\
&  =\psi_{P}\left(  a\right)  \operatorname{mod}\pi^{v_{\pi}\left(  P\right)
}A
\end{align*}
(since $\psi_{P}\left(  a\right)  =F^{\deg P}a$ (by the definition of
$\psi_{P}$)). In other words, $\psi_{P}\left(  a\right)  \equiv\varphi_{\pi
}\left(  \psi_{P/\pi}\left(  a\right)  \right)  \operatorname{mod}\pi^{v_{\pi
}\left(  P\right)  }A$. Thus, Assumption 4 of Theorem \ref{thm.F.gW-general}
is satisfied. This proves Proposition \ref{prop.F.gW-general.ex2}.
\end{proof}

\begin{proof}
[Proof of Proposition \ref{prop.F.gW-general.ex3}.]Assumption 5 of Theorem
\ref{thm.F.gW-general} is satisfied\footnote{\textit{Proof.} Assumption 3 of
Theorem \ref{thm.F.gW-general} shows that $\varphi_{1}=1$. Now, the definition
of $\psi_{1}$ yields $\psi_{1}=\varphi_{1}=1$. In other words, Assumption 5 of
Theorem \ref{thm.F.gW-general} is satisfied, qed.}. Hence, it remains to show
that Assumption 4 of Theorem \ref{thm.F.gW-general} is satisfied. In other
words, we must prove that we have $\psi_{P}\left(  a\right)  \equiv
\varphi_{\pi}\left(  \psi_{P/\pi}\left(  a\right)  \right)  \operatorname{mod}%
\pi^{v_{\pi}\left(  P\right)  }A$ for every $a\in A$, every $P\in N$ and every
$\pi\in\operatorname*{PF}P$.

So let us fix $a\in A$, $P\in N$ and $\pi\in\operatorname*{PF}P$. Clearly,
$\pi\mid P$ (since $\pi\in\operatorname*{PF}P$), and $\pi$ is a monic
irreducible polynomial in $\mathbb{F}_{q}\left[  T\right]  $ (since $\pi
\in\operatorname*{PF}P$). From these two facts, we obtain $\pi\in N$ (since
$N$ is a $q$-nest). Also, $P/\pi$ is a monic polynomial in $\mathbb{F}%
_{q}\left[  T\right]  $ (since $P$ and $\pi$ are monic and since $\pi\mid P$),
and divides $P$. Therefore, $P/\pi\in N$ (since $P\in N$). Now, the second
sentence of Assumption 3 of Theorem \ref{thm.F.gW-general} (applied to $\pi$
and $P/\pi$ instead of $P$ and $Q$) shows that $\varphi_{\pi}\circ
\varphi_{P/\pi}=\varphi_{\pi\cdot\left(  P/\pi\right)  }$ (since $\pi
\cdot\left(  P/\pi\right)  =P\in N$). Since $\pi\cdot\left(  P/\pi\right)
=P$, this rewrites as $\varphi_{\pi}\circ\varphi_{P/\pi}=\varphi_{P}$. But the
definition of $\psi_{P}$ yields $\psi_{P}=\varphi_{P}$. Hence, $\psi
_{P}=\varphi_{P}=\varphi_{\pi}\circ\varphi_{P/\pi}$, so that%
\begin{equation}
\underbrace{\psi_{P}}_{=\varphi_{\pi}\circ\varphi_{P/\pi}}\left(  a\right)
=\left(  \varphi_{\pi}\circ\varphi_{P/\pi}\right)  \left(  a\right)
=\varphi_{\pi}\left(  \psi_{P/\pi}\left(  a\right)  \right)  .
\label{pf.prop.F.gW-general.ex3.1}%
\end{equation}
On the other hand, the definition of $\psi_{P/\pi}$ yields $\psi_{P/\pi
}=\varphi_{P/\pi}$. Thus, (\ref{pf.prop.F.gW-general.ex3.1}) rewrites as
$\psi_{P}\left(  a\right)  =\varphi_{\pi}\left(  \psi_{P/\pi}\left(  a\right)
\right)  $. Therefore, $\psi_{P}\left(  a\right)  \equiv\varphi_{\pi}\left(
\psi_{P/\pi}\left(  a\right)  \right)  \operatorname{mod}\pi^{v_{\pi}\left(
P\right)  }A$. Thus, Assumption 4 of Theorem \ref{thm.F.gW-general} is
satisfied. This proves Proposition \ref{prop.F.gW-general.ex3}.
\end{proof}

Let us now turn to the proof of Theorem \ref{thm.F.gW-general}\footnote{Our
proof imitates \cite[solution to Exercise 2.83]{reiner-hopf}.}:

\begin{proof}
[Proof of Theorem \ref{thm.F.gW-general}.]We shall prove the two implications
$\mathcal{C}_{1}\Longrightarrow\mathcal{E}_{\psi}$ and $\mathcal{E}_{\psi
}\Longrightarrow\mathcal{C}_{1}$ separately:

\textit{Proof of the implication }$\mathcal{E}_{\psi}\Longrightarrow
\mathcal{C}_{1}$\textit{:} Assume that Assertion $\mathcal{E}_{\psi}$ holds.
That is, there exists a family $\left(  z_{P}\right)  _{P\in N}\in A^{N}$ of
elements of $A$ such that%
\begin{equation}
\left(  b_{P}=\sum_{D\mid P}D\psi_{P / D}\left(  z_{D}\right)  \text{ for
every }P\in N\right)  . \label{pf.thm.F.gW-general.EC.ass}%
\end{equation}
Consider this family $\left(  z_{P}\right)  _{P\in N}$.

We need to prove that Assertion $\mathcal{C}_{1}$ holds, i.e., that every
$P\in N$ and every $\pi\in\operatorname{PF}P$ satisfy%
\begin{equation}
\varphi_{\pi}\left(  b_{P / \pi}\right)  \equiv b_{P}\operatorname{mod}%
\pi^{v_{\pi}\left(  P\right)  }A. \label{pf.thm.F.gW-general.EC.goal}%
\end{equation}
So let us fix a $P\in N$ and a $\pi\in\operatorname*{PF}P$. We need to prove
(\ref{pf.thm.F.gW-general.EC.goal}).

The polynomial $P$ is monic (since $P\in N$). We have $\pi\in
\operatorname*{PF}P$. Thus, $\pi$ is a monic irreducible divisor of $P$.
Hence, $P/\pi$ is a monic polynomial in $\mathbb{F}_{q}\left[  T\right]  $
(since $P$ and $\pi$ are monic). Since $N$ is a $q$-nest, we obtain $P/\pi\in
N$ (since $P\in N$, and since $P/\pi$ is a monic divisor of $N$). Since $N$ is
a $q$-nest, we also obtain $\pi\in N$ (since $P\in N$, and since $\pi$ is a
monic divisor of $N$).

Assumption 1 (applied to $\pi$ instead of $P$) shows that $\varphi_{\pi}$ is
an endomorphism of the $\mathcal{F}$-module $A$.

Applying (\ref{pf.thm.F.gW-general.EC.ass}) to $P/\pi$ instead of $P$, we
obtain $b_{P/\pi}=\sum_{D\mid P/\pi}D\psi_{\left(  P/\pi\right)  /D}\left(
z_{D}\right)  $. Applying the map $\varphi_{\pi}$ to both sides of this
equality, we obtain%
\begin{equation}
\varphi_{\pi}\left(  b_{P/\pi}\right)  =\varphi_{\pi}\left(  \sum_{D\mid
P/\pi}D\psi_{\left(  P/\pi\right)  /D}\left(  z_{D}\right)  \right)
=\sum_{D\mid P/\pi}D\varphi_{\pi}\left(  \psi_{\left(  P/\pi\right)
/D}\left(  z_{D}\right)  \right)  \label{pf.thm.F.gW-general.EC.0}%
\end{equation}
(since $\varphi_{\pi}$ is an endomorphism of the $\mathcal{F}$-module $A$). On
the other hand, every monic divisor $D$ of $P/\pi$ satisfies%
\begin{equation}
D\psi_{P/D}\left(  z_{D}\right)  \equiv D\varphi_{\pi}\left(  \psi_{\left(
P/\pi\right)  /D}\left(  z_{D}\right)  \right)  \operatorname{mod}\pi^{v_{\pi
}\left(  P\right)  }A \label{pf.thm.F.gW-general.EC.2}%
\end{equation}
\footnote{\textit{Proof of (\ref{pf.thm.F.gW-general.EC.2}):} Let $D$ be a
monic divisor of $P/\pi$. Thus, $D\mid P/\pi$, so that $D\mid P/\pi\mid P$ and
therefore $P/D\in\mathbb{F}_{q}\left[  T\right]  $.
\par
Also, $\dfrac{P/D}{\pi}=\dfrac{P/\pi}{D}\in\mathbb{F}_{q}\left[  T\right]  $
(since $D\mid P/\pi$). In other words, $\pi\mid P/D$ (since $P/D\in
\mathbb{F}_{q}\left[  T\right]  $). Hence, $\pi\in\operatorname*{PF}\left(
P/D\right)  $ (since $\pi$ is monic irreducible). Also, $P/D$ is a monic
divisor of $P$ (since $P$ and $D$ are monic, and since $D\mid P$); thus,
$P/D\in N$ (since $P\in N$ and since $N$ is a q-nest). Hence, Assumption 4
(applied to $z_{D}$ and $P/D$ instead of $a$ and $P$) yields%
\[
\psi_{P/D}\left(  z_{D}\right)  \equiv\varphi_{\pi}\left(  \psi_{\left(
P/D\right)  /\pi}\left(  z_{D}\right)  \right)  \operatorname{mod}\pi^{v_{\pi
}\left(  P/D\right)  }A.
\]
In other words, $\psi_{P/D}\left(  z_{D}\right)  -\varphi_{\pi}\left(
\psi_{\left(  P/D\right)  /\pi}\left(  z_{D}\right)  \right)  \in\pi^{v_{\pi
}\left(  P/D\right)  }A$. Since $\left(  P/D\right)  /\pi=\left(
P/\pi\right)  /D$, this rewrites as $\psi_{P/D}\left(  z_{D}\right)
-\varphi_{\pi}\left(  \psi_{\left(  P/\pi\right)  /D}\left(  z_{D}\right)
\right)  \in\pi^{v_{\pi}\left(  P/D\right)  }A$.
\par
Now,%
\begin{align*}
&  D\psi_{P/D}\left(  z_{D}\right)  -D\varphi_{\pi}\left(  \psi_{\left(
P/\pi\right)  /D}\left(  z_{D}\right)  \right) \\
&  =D\underbrace{\left(  \psi_{P/D}\left(  z_{D}\right)  -\varphi_{\pi}\left(
\psi_{\left(  P/\pi\right)  /D}\left(  z_{D}\right)  \right)  \right)  }%
_{\in\pi^{v_{\pi}\left(  P/D\right)  }A}\\
&  \in D\pi^{v_{\pi}\left(  P/D\right)  }A=\pi^{v_{\pi}\left(  P/D\right)
}\underbrace{DA}_{\substack{\subseteq\pi^{v_{\pi}\left(  D\right)
}A\\\text{(since }\pi^{v_{\pi}\left(  D\right)  }\mid D\text{)}}%
}\subseteq\underbrace{\pi^{v_{\pi}\left(  P/D\right)  }\pi^{v_{\pi}\left(
D\right)  }}_{=\pi^{v_{\pi}\left(  P/D\right)  +v_{\pi}\left(  D\right)  }}A\\
&  =\pi^{v_{\pi}\left(  P/D\right)  +v_{\pi}\left(  D\right)  }A=\pi^{v_{\pi
}\left(  P\right)  }A
\end{align*}
(since $v_{\pi}\left(  P/D\right)  +v_{\pi}\left(  D\right)  =v_{\pi}\left(
\underbrace{\left(  P/D\right)  D}_{=P}\right)  =v_{\pi}\left(  P\right)  $).
In other words, $D\psi_{P/D}\left(  z_{D}\right)  \equiv D\varphi_{\pi}\left(
\psi_{\left(  P/\pi\right)  /D}\left(  z_{D}\right)  \right)
\operatorname{mod}\pi^{v_{\pi}\left(  P\right)  }A$. This proves
(\ref{pf.thm.F.gW-general.EC.2}).}. Now,
\begin{align}
&  \sum_{D\mid P}D\psi_{P/D}\left(  z_{D}\right) \nonumber\\
&  =\underbrace{\sum_{\substack{D\mid P;\\D\mid P/\pi}}}_{=\sum_{D\mid P/\pi}%
}D\psi_{P/D}\left(  z_{D}\right)  +\sum_{\substack{D\mid P;\\D\nmid P/\pi
}}\underbrace{D\psi_{P/D}\left(  z_{D}\right)  }_{\substack{\equiv
0\operatorname{mod}\pi^{v_{\pi}\left(  P\right)  }A\\\text{(since Lemma
\ref{lem.FqT.exact-divisor} shows that}\\\pi^{v_{\pi}\left(  P\right)  }\mid
D\text{)}}}\nonumber\\
&  \equiv\sum_{D\mid P/\pi}D\psi_{P/D}\left(  z_{D}\right)  +\underbrace{\sum
_{\substack{D\mid P;\\D\nmid P/\pi}}0}_{=0}=\sum_{D\mid P/\pi}%
\underbrace{D\psi_{P/D}\left(  z_{D}\right)  }_{\substack{\equiv D\varphi
_{\pi}\left(  \psi_{\left(  P/\pi\right)  /D}\left(  z_{D}\right)  \right)
\operatorname{mod}\pi^{v_{\pi}\left(  P\right)  }A\\\text{(by
(\ref{pf.thm.F.gW-general.EC.2}))}}}\label{pf.thm.F.gW-general.EC.1}\\
&  \equiv\sum_{D\mid P/\pi}D\varphi_{\pi}\left(  \psi_{\left(  P/\pi\right)
/D}\left(  z_{D}\right)  \right) \label{pf.thm.F.gW-general.EC.1b}\\
&  =\varphi_{\pi}\left(  b_{P/\pi}\right)  \operatorname{mod}\pi^{v_{\pi
}\left(  P\right)  }A \label{pf.thm.F.gW-general.EC.1c}%
\end{align}
(by (\ref{pf.thm.F.gW-general.EC.0})). But (\ref{pf.thm.F.gW-general.EC.ass})
yields%
\[
b_{P}=\sum_{D\mid P}D\psi_{P / D}\left(  z_{D}\right)  \equiv\varphi_{\pi
}\left(  b_{P/\pi}\right)  \operatorname{mod}\pi^{v_{\pi}\left(  P\right)  }A
\]
(by (\ref{pf.thm.F.gW-general.EC.1c})). Thus,
(\ref{pf.thm.F.gW-general.EC.goal}) is proven. In other words, Assertion
$\mathcal{C}_{1}$ holds. This completes the proof of the implication
$\mathcal{E}_{\psi}\Longrightarrow\mathcal{C}_{1}$.

\textit{Proof of the implication }$\mathcal{C}_{1}\Longrightarrow
\mathcal{E}_{\psi}$\textit{:} Assume that Assertion $\mathcal{C}_{1}$ holds.
In other words, every $P\in N$ and every $\pi\in\operatorname{PF}P$ satisfy%
\begin{equation}
\varphi_{\pi}\left(  b_{P / \pi}\right)  \equiv b_{P}\operatorname{mod}%
\pi^{v_{\pi}\left(  P\right)  }A. \label{pf.thm.F.gW-general.CE.C}%
\end{equation}


We now need to prove that Assertion $\mathcal{E}_{\psi}$ holds as well. In
other words, we need to show that there exists a family $\left(  z_{P}\right)
_{P\in N}\in A^{N}$ of elements of $A$ such that%
\[
\left(  b_{P}=\sum_{D\mid P}D\psi_{P / D}\left(  z_{D}\right)  \text{ for
every }P\in N\right)  .
\]
In other words (renaming $P$ as $Q$), we need to show that there exists a
family $\left(  z_{Q}\right)  _{Q\in N}\in A^{N}$ of elements of $A$ such that%
\[
\left(  b_{Q}=\sum_{D\mid Q}D\psi_{Q / D}\left(  z_{D}\right)  \text{ for
every }Q\in N\right)  .
\]


We construct this family $\left(  z_{Q}\right)  _{Q\in N}$ recursively, by
induction over $\deg Q$. So we fix some $P\in N$, and assume that an element
$z_{Q}$ of $A$ is already constructed for every $Q\in N$ satisfying $\deg
Q<\deg P$; we furthermore assume that these $z_{Q}$ satisfy%
\begin{equation}
b_{Q}=\sum_{D\mid Q}D\psi_{Q/D}\left(  z_{D}\right)
\label{pf.thm.F.gW-general.CE.C.indass}%
\end{equation}
for every $Q\in N$ satisfying $\deg Q<\deg P$. We now need to construct a
$z_{P}\in A$ such that (\ref{pf.thm.F.gW-general.CE.C.indass}) is satisfied
for $Q=P$. In other words, we need to construct a $z_{P}\in A$ satisfying
$b_{P}=\sum_{D\mid P}D\psi_{P/D}\left(  z_{D}\right)  $.

Let us first choose $z_{P}$ \textbf{arbitrarily} (with the intention to tweak
it later). Let $\pi\in\operatorname*{PF}P$ be arbitrary. Thus, $\pi$ is a
monic irreducible divisor of $P$. Then, the polynomial $P/\pi$ is monic (since
$P$ and $\pi$ are monic), and is a divisor of $P$; hence, $P/\pi\in N$ (since
$P\in N$, and since $N$ is a $q$-nest). Moreover, it satisfies $\deg\left(
P/\pi\right)  =\deg P-\underbrace{\deg\pi}_{>0}<\deg P$. Hence,
(\ref{pf.thm.F.gW-general.CE.C.indass}) (applied to $Q=P/\pi$) shows that%
\[
b_{P/\pi}=\sum_{D\mid P/\pi}D\psi_{\left(  P/\pi\right)  /D}\left(
z_{D}\right)  .
\]
Thus, (\ref{pf.thm.F.gW-general.EC.1c}) holds (indeed, this can be proven
precisely as in our proof of the implication $\mathcal{E}_{\psi}%
\Longrightarrow\mathcal{C}_{1}$ above). Hence,%
\[
\sum_{D\mid P}D\psi_{P/D}\left(  z_{D}\right)  \equiv\varphi_{\pi}\left(
b_{P/\pi}\right)  \equiv b_{P}\operatorname{mod}\pi^{v_{\pi}\left(  P\right)
}A
\]
(by (\ref{pf.thm.F.gW-general.CE.C})). In other words, $b_{P}\equiv\sum_{D\mid
P}D\psi_{P/D}\left(  z_{D}\right)  \operatorname{mod}\pi^{v_{\pi}\left(
P\right)  }A$. In other words, $b_{P}-\sum_{D\mid P}D\psi_{P/D}\left(
z_{D}\right)  \in\pi^{v_{\pi}\left(  P\right)  }A$.

Now, let us forget that we fixed $\pi$. We thus have shown (for our
arbitrarily chosen $z_{P}$) that%
\[
b_{P}-\sum_{D\mid P}D\psi_{P/D}\left(  z_{D}\right)  \in\pi^{v_{\pi}\left(
P\right)  }A\ \ \ \ \ \ \ \ \ \ \text{for each }\pi\in\operatorname*{PF}P.
\]
As a consequence,%
\[
b_{P}-\sum_{D\mid P}D\psi_{P/D}\left(  z_{D}\right)  \in\bigcap_{\pi
\in\operatorname*{PF}P}\pi^{v_{\pi}\left(  P\right)  }A=PA
\]
(by Corollary \ref{cor.CRT.FqT}). In other words, there exists a $\gamma\in A$
such that
\[
b_{P}-\sum_{D\mid P}D\psi_{P/D}\left(  z_{D}\right)  =P\gamma.
\]
Consider this $\gamma$.

We have assumed that Assumption 5 of Theorem \ref{thm.F.gW-general} is
satisfied. In other words, $\psi_{1}=\operatorname*{id}$. Hence,%
\begin{align*}
&  P\psi_{P/P}\left(  z_{P}+\gamma\right)  -P\psi_{P/P}\left(  z_{P}\right) \\
&  =P\operatorname*{id}\left(  z_{P}+\gamma\right)  -P\operatorname*{id}%
\left(  z_{P}\right)  \ \ \ \ \ \ \ \ \ \ \left(  \text{since }\psi_{P/P}%
=\psi_{1}=\operatorname*{id}\right) \\
&  =P\cdot\left(  z_{P}+\gamma\right)  -P\cdot z_{P}=P\gamma\\
&  =b_{P}-\sum_{D\mid P}D\psi_{P/D}\left(  z_{D}\right)  .
\end{align*}
In other words,%
\begin{align}
&  \sum_{D\mid P}D\psi_{P/D}\left(  z_{D}\right)  +\left(  P\psi_{P/P}\left(
z_{P}+\gamma\right)  -P\psi_{P/P}\left(  z_{P}\right)  \right) \nonumber\\
&  =b_{P}. \label{pf.thm.F.gW-general.CE.C.sum-upd}%
\end{align}


Now, if we replace $z_{P}$ by $z_{P}+\gamma$, then the sum $\sum_{D\mid
P}D\psi_{P/D}\left(  z_{D}\right)  $ increases by $P\psi_{P/P}\left(
z_{P}+\gamma\right)  -P\psi_{P/P}\left(  z_{P}\right)  $ (because the only
addend of the sum that changes is the addend for $D=P$), and thus the new
value of this sum is $b_{P}$ (by (\ref{pf.thm.F.gW-general.CE.C.sum-upd})).
Hence, by replacing $z_{P}$ by $z_{P}+\gamma$, we achieve that $b_{P}%
=\sum_{D\mid P}D\psi_{P/D}\left(  z_{D}\right)  $ holds. Thus, we have found
the $z_{P}$ we were searching for, and the recursive construction of the
family $\left(  z_{Q}\right)  _{Q\in N}$ has proceeded by one more step. The
proof of the implication $\mathcal{C}_{1}\Longrightarrow\mathcal{E}_{\psi}$ is
thus complete.

We have now proven both implications $\mathcal{C}_{1}\Longrightarrow
\mathcal{E}_{\psi}$ and $\mathcal{E}_{\psi}\Longrightarrow\mathcal{C}_{1}$.
Combining them, we obtain the equivalence $\mathcal{C}_{1}\Longleftrightarrow
\mathcal{E}_{\psi}$. Thus, Theorem \ref{thm.F.gW-general} is proven.
\end{proof}

\subsection{\label{subsect.proofs.numthefuns}$\mathbb{F}_{q}\left[  T\right]
_{+}$-analogues of the M\"{o}bius and Euler totient functions}

Next, we shall discuss the functions $\mu$, $\varphi$ and $\varphi_{C}$
introduced in Section \ref{sect.nots}. Let me first repeat their definitions:

\begin{definition}
\label{def.moebius-q}Define a function $\mu:\mathbb{F}_{q}\left[  T\right]
_{+}\rightarrow\left\{  -1,0,1\right\}  $ by%
\[
\mu\left(  M\right)  =%
\begin{cases}
\left(  -1\right)  ^{\left\vert \operatorname*{PF}M\right\vert }, & \text{if
}M\text{ is squarefree;}\\
0, & \text{if }M\text{ is not squarefree}%
\end{cases}
\ \ \ \ \ \ \ \ \ \ \text{for all }M\in\mathbb{F}_{q}\left[  T\right]  _{+}.
\]
(Recall that a monic polynomial $M\in\mathbb{F}_{q}\left[  T\right]  _{+}$ is
said to be \textit{squarefree} if it satisfies the following three equivalent conditions:

\begin{itemize}
\item No nonconstant polynomial $P\in\mathbb{F}_{q}\left[  T\right]  $
satisfies $P^{2}\mid M$.

\item Every monic irreducible polynomial $\pi\in\mathbb{F}_{q}\left[
T\right]  $ satisfies $v_{\pi}\left(  M\right)  \leq1$.

\item The polynomial $M$ is a product of pairwise distinct monic irreducible polynomials.
\end{itemize}

) The function $\mu$ is called the \textit{M\"{o}bius function on }%
$\mathbb{F}_{q}\left[  T\right]  _{+}$.
\end{definition}

\begin{definition}
\label{def.phiC-q}Define a function $\varphi_{C}:\mathbb{F}_{q}\left[
T\right]  _{+}\rightarrow\mathbb{F}_{q}\left[  T\right]  $ by%
\[
\varphi_{C}\left(  M\right)  =\sum\limits_{D\mid M}\mu\left(  D\right)
\dfrac{M}{D}\ \ \ \ \ \ \ \ \ \ \text{for all }M\in\mathbb{F}_{q}\left[
T\right]  _{+}.
\]

\end{definition}

\begin{definition}
\label{def.phi-q}Define a function $\varphi:\mathbb{F}_{q}\left[  T\right]
_{+}\rightarrow\mathbb{Z}$ by%
\[
\varphi\left(  M\right)  =\sum\limits_{D\mid M}\mu\left(  D\right)
q^{\deg\left(  M/D\right)  }\ \ \ \ \ \ \ \ \ \ \text{for all }M\in
\mathbb{F}_{q}\left[  T\right]  _{+}.
\]

\end{definition}

The function $\mu$ is an analogue of the number-theoretical M\"{o}bius
function, whereas the functions $\varphi_{C}$ and $\varphi$ are two distinct
analogues of the Euler totient function. These functions have a number of
properties (some well-known) that often imitate analogous properties of the
number-theoretical M\"{o}bius function and the Euler totient function. See
\cite[Theorem 4.5]{kc-carlitz} for some properties of $\varphi_{C}$, and see
\cite[Section 6]{kc-carlitz} for the function $\varphi$. We shall prove a
number of their properties, many of which will be used below. We begin by
citing a well-known combinatorial fact:

\begin{lemma}
\label{lem.moebius-Q.Z}Let $Z$ be a finite set.

\textbf{(a)} We have%
\[
\sum_{I\subseteq Z}\left(  -1\right)  ^{\left\vert I\right\vert }=\left[
Z=\varnothing\right]  .
\]


\textbf{(b)} Let $R$ be a commutative ring. Let $r_{i}$ be an element of $R$
for each $i\in Z$. Then,%
\[
\sum_{I\subseteq Z}\prod_{i\in I}r_{i}=\prod_{i\in Z}\left(  1+r_{i}\right)
.
\]

\end{lemma}

\begin{proof}
[Proof of Lemma \ref{lem.moebius-Q.Z}.]Lemma \ref{lem.moebius-Q.Z}
\textbf{(b)} can be proven by induction over $\left\vert Z\right\vert $ (or,
less rigorously, just by expanding the product $\prod_{i\in Z}\left(
1+r_{i}\right)  $). Lemma \ref{lem.moebius-Q.Z} \textbf{(a)} can be proven in
many ways (e.g., it can be obtained by setting $R=\mathbb{Z}$ and $r_{i}=-1$
in Lemma \ref{lem.moebius-Q.Z} \textbf{(b)}).
\end{proof}

\begin{proposition}
\label{prop.moebius-Q.sum}Let $M\in\mathbb{F}_{q}\left[  T\right]  _{+}$.
Then, $\sum_{D\mid M}\mu\left(  D\right)  =\left[  M=1\right]  $. Here, we are
using the \textit{Iverson bracket notation}: If $\mathcal{A}$ is any logical
statement, then $\left[  \mathcal{A}\right]  $ stands for the integer $%
\begin{cases}
1, & \text{if }\mathcal{A}\text{ is true};\\
0, & \text{if }\mathcal{A}\text{ is false}%
\end{cases}
$.
\end{proposition}

\begin{proof}
[Proof of Proposition \ref{prop.moebius-Q.sum}.](This proof is a carbon copy
of \cite[proof of (12.344)]{reiner-hopf}, with minor changes.)

Let $M=P_{1}^{a_{1}}P_{2}^{a_{2}}\cdots P_{k}^{a_{k}}$ be the factorization of
$M$ into monic irreducible polynomials, with all of $a_{1},a_{2},\ldots,a_{k}$
being positive integers (and with $P_{1},P_{2},\ldots,P_{k}$ being
distinct).\footnote{This is well-defined, since $M$ is monic and since
$\mathbb{F}_{q}\left[  T\right]  $ is a principal ideal domain. Of course, $k$
can be $0$ (when $M=1$).} Then, the \textbf{squarefree} monic divisors $D$ of
$M$ all have the form $\prod_{i\in I}P_{i}$ for some subset $I$ of $\left\{
1,2,\ldots,k\right\}  $. More precisely, there exists a bijection%
\begin{align}
\left\{  I\subseteq\left\{  1,2,\ldots,k\right\}  \right\}   &  \rightarrow
\left(  \text{the set of all squarefree monic divisors of }M\right)
,\nonumber\\
I  &  \mapsto\prod_{i\in I}P_{i}. \label{pf.prop.moebius-Q.sum.bij}%
\end{align}
Moreover, every subset $I$ of $\left\{  1,2,\ldots,k\right\}  $ satisfies
$\operatorname*{PF}\left(  \prod_{i\in I}P_{i}\right)  =\left\{  P_{i}%
\ \mid\ i\in I\right\}  $ and thus%
\begin{equation}
\left\vert \operatorname*{PF}\left(  \prod_{i\in I}P_{i}\right)  \right\vert
=\left\vert \left\{  P_{i}\ \mid\ i\in I\right\}  \right\vert =\left\vert
I\right\vert \label{pf.prop.moebius-Q.sum.1}%
\end{equation}
(since $P_{1},P_{2},\ldots,P_{k}$ are distinct) and therefore
\begin{align}
\mu\left(  \prod_{i\in I}P_{i}\right)   &  =\left(  -1\right)  ^{\left\vert
\operatorname*{PF}\left(  \prod_{i\in I}P_{i}\right)  \right\vert
}\ \ \ \ \ \ \ \ \ \ \left(  \text{since }\prod_{i\in I}P_{i}\text{ is
squarefree}\right) \nonumber\\
&  =\left(  -1\right)  ^{\left\vert I\right\vert }\ \ \ \ \ \ \ \ \ \ \left(
\text{by (\ref{pf.prop.moebius-Q.sum.1})}\right)  .
\label{pf.prop.moebius-Q.sum.2}%
\end{align}
Now,%
\begin{align*}
\sum_{D\mid M}\mu\left(  D\right)   &  =\sum_{\substack{D\mid M;\\D\text{ is
squarefree}}}\mu\left(  D\right)  +\sum_{\substack{D\mid M;\\D\text{ is not
squarefree}}}\underbrace{\mu\left(  D\right)  }_{\substack{=0\\\text{(by the
definition}\\\text{of }\mu\text{, since }D\\\text{is not squarefree)}}}\\
&  =\sum_{\substack{D\mid M;\\D\text{ is squarefree}}}\mu\left(  D\right)
+\underbrace{\sum_{\substack{D\mid M;\\D\text{ is not squarefree}}}0}%
_{=0}=\sum_{\substack{D\mid M;\\D\text{ is squarefree}}}\mu\left(  D\right) \\
&  =\sum_{I\subseteq\left\{  1,2,\ldots,k\right\}  }\underbrace{\mu\left(
\prod_{i\in I}P_{i}\right)  }_{\substack{=\left(  -1\right)  ^{\left\vert
I\right\vert }\\\text{(by (\ref{pf.prop.moebius-Q.sum.2}))}}%
}\ \ \ \ \ \ \ \ \ \ \left(
\begin{array}
[c]{c}%
\text{here, we have substituted }\prod_{i\in I}P_{i}\text{ for }D\\
\text{due to the bijection (\ref{pf.prop.moebius-Q.sum.bij})}%
\end{array}
\right) \\
&  =\sum_{I\subseteq\left\{  1,2,\ldots,k\right\}  }\left(  -1\right)
^{\left\vert I\right\vert }=\left[  \underbrace{\left\{  1,2,\ldots,k\right\}
=\varnothing}_{\text{This is equivalent to }k=0}\right] \\
&  \ \ \ \ \ \ \ \ \ \ \left(  \text{by Lemma \ref{lem.moebius-Q.Z}
\textbf{(a)}, applied to }Z=\left\{  1,2,\ldots,k\right\}  \right) \\
&  =\left[  k=0\right]  =\left[  M\text{ is constant}\right] \\
&  \ \ \ \ \ \ \ \ \ \ \left(
\begin{array}
[c]{c}%
\text{since }k\text{ is the number of monic irreducible divisors of }%
M\text{,}\\
\text{and thus we have }k=0\text{ if and only if }M\text{ is constant}%
\end{array}
\right) \\
&  =\left[  M=1\right]  \ \ \ \ \ \ \ \ \ \ \left(  \text{since }M\text{ is
monic}\right)  .
\end{align*}
This proves Proposition \ref{prop.moebius-Q.sum}.
\end{proof}

Let us explicitly state a simple consequence of Proposition
\ref{prop.moebius-Q.sum} for the sake of convenience:

\begin{corollary}
\label{cor.moebius-Q.sum-rel}Let $M\in\mathbb{F}_{q}\left[  T\right]  _{+}$.
Let $E$ be a monic divisor of $M$. Then,%
\[
\sum_{\substack{B\mid M;\\BE\mid M}}\mu\left(  B\right)  =\left[  E=M\right]
.
\]

\end{corollary}

\begin{proof}
[Proof of Corollary \ref{cor.moebius-Q.sum-rel}.]We have $\dfrac{M}{E}%
\in\mathbb{F}_{q}\left[  T\right]  $ (since $E$ is a divisor of $M$).
Moreover, the polynomial $\dfrac{M}{E}$ is monic (since $M$ and $E$ are
monic). Hence,$\dfrac{M}{E}\in\mathbb{F}_{q}\left[  T\right]  _{+}$.
Proposition \ref{prop.moebius-Q.sum} (applied to $\dfrac{M}{E}$ instead of
$M$) thus shows that $\sum_{B\mid\dfrac{M}{E}}\mu\left(  B\right)  =\left[
\underbrace{\dfrac{M}{E}=1}_{\substack{\text{This is equivalent to}%
\\E=M}}\right]  =\left[  E=M\right]  $.

But $E\mid M$. Hence, the monic divisors $B$ of $M$ satisfying $BE\mid M$ are
exactly the monic divisors $B$ of $\dfrac{M}{E}$. Therefore, $\sum
_{\substack{B\mid M;\\BE\mid M}}=\sum_{B\mid\dfrac{M}{E}}$. Thus,%
\[
\underbrace{\sum_{\substack{B\mid M;\\BE\mid M}}}_{=\sum_{B\mid\dfrac{M}{E}}%
}\mu\left(  B\right)  =\sum_{B\mid\dfrac{M}{E}}\mu\left(  B\right)  =\left[
E=M\right]  .
\]
Corollary \ref{cor.moebius-Q.sum-rel} is therefore proven.
\end{proof}

Next come some simple properties of $\varphi_{C}$:

\begin{proposition}
\label{prop.phiC-Q.formula}Let $M\in\mathbb{F}_{q}\left[  T\right]  _{+}$.

\textbf{(a)} We have $\varphi_{C}\left(  M\right)  \in\mathbb{F}_{q}\left[
T\right]  _{+}$.

\textbf{(b)} We have $\varphi_{C}\left(  M\right)  =M\prod\limits_{\pi
\in\operatorname*{PF}M}\left(  1-\dfrac{1}{\pi}\right)  $.

\textbf{(c)} We have $\varphi_{C}\left(  M\right)  =\sum_{D\mid M}D\mu\left(
\dfrac{M}{D}\right)  $.
\end{proposition}

\begin{proof}
[Proof of Proposition \ref{prop.phiC-Q.formula}.]\textbf{(a)} Let $d=\deg M$.
Then, the polynomial $M$ is monic of degree $d$.

Now, let $V_{d}$ be the $\mathbb{F}_{q}$-vector subspace of $\mathbb{F}%
_{q}\left[  T\right]  $ consisting of all polynomials of degree $\leq d-1$.
(This subspace is spanned by $T^{0},T^{1},\ldots,T^{d-1}$.) Then, the monic
polynomials in $\mathbb{F}_{q}\left[  T\right]  $ of degree $d$ are precisely
the polynomials in $\mathbb{F}_{q}\left[  T\right]  $ that are congruent to
$T^{d}$ modulo $V_{d}$. Thus, the polynomial $M$ is congruent to $T^{d}$
modulo $V_{d}$ (since $M$ is monic of degree $d$). In other words, $M\equiv
T^{d}\operatorname{mod}V_{d}.$

If $D$ is a monic divisor of $M$ satisfying $D\neq1$, then
\begin{equation}
\mu\left(  D\right)  \dfrac{M}{D}\equiv0\operatorname{mod}V_{d}
\label{pf.prop.phiC-Q.formula.a.1}%
\end{equation}
\footnote{\textit{Proof of (\ref{pf.prop.phiC-Q.formula.a.1}):} Let $D$ be a
monic divisor of $M$ satisfying $D\neq1$.
\par
We have $\dfrac{M}{D}\in\mathbb{F}_{q}\left[  T\right]  $ (since $D$ is a
divisor of $M$). If we had $\deg D=0$, then we would have $D=1$ (because $D$
is monic), which would contradict $D\neq1$. Thus, we cannot have $\deg D=0$.
Hence, we must have $\deg D\geq1$ (since $D\in\mathbb{F}_{q}\left[  T\right]
$). Thus, the polynomial $\dfrac{M}{D}\in\mathbb{F}_{q}\left[  T\right]  $
satisfies $\deg\dfrac{M}{D}=\underbrace{\deg M}_{=d}-\underbrace{\deg D}%
_{\geq1}\leq d-1$. Hence, $\dfrac{M}{D}$ is a polynomial of degree $\leq d-1$.
In other words, $\dfrac{M}{D}\in V_{d}$ (since $V_{d}$ is the $\mathbb{F}_{q}%
$-vector subspace of $\mathbb{F}_{q}\left[  T\right]  $ consisting of all
polynomials of degree $\leq d-1$). In other words, $\dfrac{M}{D}%
\equiv0\operatorname{mod}V_{d}$. Hence, $\mu\left(  D\right)  \dfrac{M}%
{D}\equiv0\operatorname{mod}V_{d}$ as well (since $\mu\left(  D\right)
\in\left\{  -1,0,1\right\}  \subseteq\mathbb{Z}$). This proves
(\ref{pf.prop.phiC-Q.formula.a.1}).}. Now, the definition of $\varphi_{C}$
yields%
\begin{align*}
\varphi_{C}\left(  M\right)   &  =\sum\limits_{D\mid M}\mu\left(  D\right)
\dfrac{M}{D}\\
&  =\underbrace{\mu\left(  1\right)  }_{=1}\underbrace{\dfrac{M}{1}}_{=M\equiv
T^{d}\operatorname{mod}V_{d}}+\sum\limits_{\substack{D\mid M;\\D\neq
1}}\underbrace{\mu\left(  D\right)  \dfrac{M}{D}}_{\substack{\equiv
0\operatorname{mod}V_{d}\\\text{(by (\ref{pf.prop.phiC-Q.formula.a.1}))}}}\\
&  \ \ \ \ \ \ \ \ \ \ \left(  \text{here, we have split off the addend for
}D=1\text{ from the sum}\right) \\
&  \equiv T^{d}+\underbrace{\sum\limits_{\substack{D\mid M;\\D\neq1}}0}%
_{=0}=T^{d}\operatorname{mod}V_{d}.
\end{align*}
In other words, the polynomial $\varphi_{C}\left(  M\right)  $ is congruent to
$T^{d}$ modulo $V_{d}$. In other words, the polynomial $\varphi_{C}\left(
M\right)  $ is monic of degree $d$ (since the monic polynomials in
$\mathbb{F}_{q}\left[  T\right]  $ of degree $d$ are precisely the polynomials
in $\mathbb{F}_{q}\left[  T\right]  $ that are congruent to $T^{d}$ modulo
$V_{d}$). Hence, $\varphi_{C}\left(  M\right)  \in\mathbb{F}_{q}\left[
T\right]  _{+}$. This proves Proposition \ref{prop.phiC-Q.formula}
\textbf{(a)}.

\textbf{(b)} Let $M=P_{1}^{a_{1}}P_{2}^{a_{2}}\cdots P_{k}^{a_{k}}$ be the
factorization of $M$ into monic irreducible polynomials, with all of
$a_{1},a_{2},\ldots,a_{k}$ being positive integers (and with $P_{1}%
,P_{2},\ldots,P_{k}$ being distinct).\footnote{This is well-defined, since $M$
is monic and since $\mathbb{F}_{q}\left[  T\right]  $ is a principal ideal
domain. Of course, $k$ can be $0$ (when $M=1$).} Then, the \textbf{squarefree}
monic divisors $D$ of $M$ all have the form $\prod_{i\in I}P_{i}$ for some
subset $I$ of $\left\{  1,2,\ldots,k\right\}  $. More precisely, there exists
a bijection%
\begin{align}
\left\{  I\subseteq\left\{  1,2,\ldots,k\right\}  \right\}   &  \rightarrow
\left(  \text{the set of all squarefree monic divisors of }M\right)
,\nonumber\\
I  &  \mapsto\prod_{i\in I}P_{i}. \label{pf.prop.phiC-Q.formula.b.bij}%
\end{align}


Moreover, every subset $I$ of $\left\{  1,2,\ldots,k\right\}  $ satisfies
(\ref{pf.prop.moebius-Q.sum.2}). (This is proven as in our proof of
Proposition \ref{prop.moebius-Q.sum}.)

The definition of $P_{1},P_{2},\ldots,P_{k}$ shows that $\left(  P_{1}%
,P_{2},\ldots,P_{k}\right)  $ is a list of all prime factors of $M$, with no
repetitions. Thus, the map $\left\{  1,2,\ldots,k\right\}  \rightarrow
\operatorname*{PF}M,\ i\mapsto P_{i}$ is a bijection.

The definition of $\varphi_{C}$ yields%
\begin{align*}
\varphi_{C}\left(  M\right)   &  =\sum_{D\mid M}\mu\left(  D\right)  \dfrac
{M}{D}=\sum_{\substack{D\mid M;\\D\text{ is squarefree}}}\mu\left(  D\right)
\dfrac{M}{D}+\sum_{\substack{D\mid M;\\D\text{ is not squarefree}%
}}\underbrace{\mu\left(  D\right)  }_{\substack{=0\\\text{(by the
definition}\\\text{of }\mu\text{, since }D\\\text{is not squarefree)}}%
}\dfrac{M}{D}\\
&  =\sum_{\substack{D\mid M;\\D\text{ is squarefree}}}\mu\left(  D\right)
\dfrac{M}{D}+\underbrace{\sum_{\substack{D\mid M;\\D\text{ is not squarefree}%
}}0\dfrac{M}{D}}_{=0}=\sum_{\substack{D\mid M;\\D\text{ is squarefree}}%
}\mu\left(  D\right)  \dfrac{M}{D}\\
&  =\sum_{I\subseteq\left\{  1,2,\ldots,k\right\}  }\underbrace{\mu\left(
\prod_{i\in I}P_{i}\right)  }_{\substack{=\left(  -1\right)  ^{\left\vert
I\right\vert }\\\text{(by (\ref{pf.prop.moebius-Q.sum.2}))}}}\dfrac{M}%
{\prod_{i\in I}P_{i}}\\
&  \ \ \ \ \ \ \ \ \ \ \left(
\begin{array}
[c]{c}%
\text{here, we have substituted }\prod_{i\in I}P_{i}\text{ for }D\\
\text{due to the bijection (\ref{pf.prop.moebius-Q.sum.bij})}%
\end{array}
\right) \\
&  =\sum_{I\subseteq\left\{  1,2,\ldots,k\right\}  }\underbrace{\left(
-1\right)  ^{\left\vert I\right\vert }}_{=\prod_{i\in I}\left(  -1\right)
}\dfrac{M}{\prod_{i\in I}P_{i}}=\sum_{I\subseteq\left\{  1,2,\ldots,k\right\}
}\left(  \prod_{i\in I}\left(  -1\right)  \right)  \dfrac{M}{\prod_{i\in
I}P_{i}}\\
&  =M\sum_{I\subseteq\left\{  1,2,\ldots,k\right\}  }\underbrace{\dfrac
{\prod_{i\in I}\left(  -1\right)  }{\prod_{i\in I}P_{i}}}_{=\prod_{i\in
I}\dfrac{-1}{P_{i}}}=M\underbrace{\sum_{I\subseteq\left\{  1,2,\ldots
,k\right\}  }\prod_{i\in I}\dfrac{-1}{P_{i}}}_{\substack{=\prod_{i\in\left\{
1,2,\ldots,k\right\}  }\left(  1+\dfrac{-1}{P_{i}}\right)  \\\text{(by Lemma
\ref{lem.moebius-Q.Z} \textbf{(b)}, applied to }R=\mathbb{F}_{q}\left[
T\right]  \text{,}\\Z=\left\{  1,2,\ldots,k\right\}  \text{ and }r_{i}%
=\dfrac{-1}{P_{i}}\text{)}}}\\
&  =M\prod_{i\in\left\{  1,2,\ldots,k\right\}  }\left(  1+\dfrac{-1}{P_{i}%
}\right)  =M\prod_{\pi\in\operatorname*{PF}M}\underbrace{\left(  1+\dfrac
{-1}{\pi}\right)  }_{=1-\dfrac{1}{\pi}}\\
&  \ \ \ \ \ \ \ \ \ \ \left(
\begin{array}
[c]{c}%
\text{here, we have substituted }\pi\text{ for }P_{i}\text{ in the product,}\\
\text{since the map }\left\{  1,2,\ldots,k\right\}  \rightarrow
\operatorname*{PF}M,\ i\mapsto P_{i}\text{ is a bijection}%
\end{array}
\right) \\
&  =M\prod\limits_{\pi\in\operatorname*{PF}M}\left(  1-\dfrac{1}{\pi}\right)
.
\end{align*}
This proves Proposition \ref{prop.phiC-Q.formula} \textbf{(b)}.

\textbf{(c)} Let $\mathfrak{A}$ be the set of all monic divisors of $M$. Thus,
$\sum_{D\in\mathfrak{A}}=\sum_{D\mid M}$.

But $M$ itself is monic. Hence, the map $\mathfrak{A}\rightarrow
\mathfrak{A},\ D\mapsto\dfrac{M}{D}$ is well-defined and a bijection. Thus, we
can substitute $\dfrac{M}{D}$ for $D$ in the sum $\sum_{D\in\mathfrak{A}}%
\mu\left(  D\right)  \dfrac{M}{D}$. As a result, we obtain%
\[
\sum_{D\in\mathfrak{A}}\mu\left(  D\right)  \dfrac{M}{D}=\underbrace{\sum
_{D\in\mathfrak{A}}}_{=\sum_{D\mid M}}\mu\left(  \dfrac{M}{D}\right)
\underbrace{\dfrac{M}{\left(  \dfrac{M}{D}\right)  }}_{=D}=\sum_{D\mid M}%
\mu\left(  \dfrac{M}{D}\right)  D=\sum_{D\mid M}D\mu\left(  \dfrac{M}%
{D}\right)  .
\]
Comparing this with%
\[
\underbrace{\sum_{D\in\mathfrak{A}}}_{=\sum_{D\mid M}}\mu\left(  D\right)
\dfrac{M}{D}=\sum_{D\mid M}\mu\left(  D\right)  \dfrac{M}{D}=\varphi
_{C}\left(  M\right)  \ \ \ \ \ \ \ \ \ \ \left(
\begin{array}
[c]{c}%
\text{since }\varphi_{C}\left(  M\right)  \text{ is defined}\\
\text{to be }\sum_{D\mid M}\mu\left(  D\right)  \dfrac{M}{D}%
\end{array}
\right)  ,
\]
we obtain $\varphi_{C}\left(  M\right)  =\sum_{D\mid M}D\mu\left(  \dfrac
{M}{D}\right)  $. This proves Proposition \ref{prop.phiC-Q.formula}
\textbf{(c)}.
\end{proof}

\begin{proposition}
\label{prop.phiC-Q.sum}Let $M\in\mathbb{F}_{q}\left[  T\right]  _{+}$. Then,
$M=\sum\limits_{D\mid M}\varphi_{C}\left(  D\right)  $.
\end{proposition}

Proposition \ref{prop.phiC-Q.sum} is \cite[Theorem 4.5 (2)]{kc-carlitz}, but
let me nevertheless give an independent proof of it:

\begin{proof}
[Proof of Proposition \ref{prop.phiC-Q.sum}.]We shall use the notation of
Proposition \ref{prop.moebius-Q.sum}.

Every $E\in\mathbb{F}_{q}\left[  T\right]  _{+}$ satisfies%
\begin{align}
\varphi_{C}\left(  E\right)   &  =\sum\limits_{D\mid E}\mu\left(  D\right)
\dfrac{E}{D}\ \ \ \ \ \ \ \ \ \ \left(  \text{by the definition of }%
\varphi_{C}\right) \nonumber\\
&  =\sum_{B\mid E}\mu\left(  B\right)  \dfrac{E}{B}
\label{pf.prop.phiC-Q.sum.phiCE=}%
\end{align}
(here, we have renamed the summation index $D$ as $B$).

For any monic divisor $B$ of $M$, we have%
\begin{equation}
\sum_{\substack{D\mid M;\\B\mid D}}\dfrac{D}{B}=\sum_{E\mid\dfrac{M}{B}}E
\label{pf.prop.phiC-Q.sum.down}%
\end{equation}
\footnote{\textit{Proof of (\ref{pf.prop.phiC-Q.sum.down}):} Let $B$ be a
monic divisor of $M$. Then, the map%
\begin{align*}
\left\{  D\text{ is a monic divisor of }M\text{ such that }B\mid D\right\}
&  \rightarrow\left\{  E\text{ is a monic divisor of }\dfrac{M}{B}\right\}
,\\
D  &  \mapsto\dfrac{D}{B}%
\end{align*}
(where the symbol \textquotedblleft$\mid$\textquotedblright\ means
\textquotedblleft divides\textquotedblright, not \textquotedblleft such
that\textquotedblright) is well-defined and a bijection. Hence, we can
substitute $E$ for $\dfrac{D}{B}$ in the sum $\sum_{\substack{D\mid M;\\B\mid
D}}\dfrac{D}{B}$. We thus obtain $\sum_{\substack{D\mid M;\\B\mid D}}\dfrac
{D}{B}=\sum_{E\mid\dfrac{M}{B}}E$. This proves (\ref{pf.prop.phiC-Q.sum.down}%
).}.

Now,%
\begin{align*}
&  \sum\limits_{D\mid M}\underbrace{\varphi_{C}\left(  D\right)
}_{\substack{=\sum_{B\mid D}\mu\left(  B\right)  \dfrac{D}{B}\\\text{(by
(\ref{pf.prop.phiC-Q.sum.phiCE=}), applied to }E=D\text{)}}}\\
&  =\sum\limits_{D\mid M}\underbrace{\sum_{B\mid D}}_{\substack{=\sum
_{\substack{B\mid M;\\B\mid D}}\\\text{(since }D\mid M\text{)}}}\mu\left(
B\right)  \dfrac{D}{B}=\underbrace{\sum\limits_{D\mid M}\sum_{\substack{B\mid
M;\\B\mid D}}}_{=\sum_{B\mid M}\sum_{\substack{D\mid M;\\B\mid D}}}\mu\left(
B\right)  \dfrac{D}{B}=\sum_{B\mid M}\sum_{\substack{D\mid M;\\B\mid D}%
}\mu\left(  B\right)  \dfrac{D}{B}\\
&  =\sum_{B\mid M}\mu\left(  B\right)  \underbrace{\sum_{\substack{D\mid
M;\\B\mid D}}\dfrac{D}{B}}_{\substack{=\sum_{E\mid\dfrac{M}{B}}E\\\text{(by
(\ref{pf.prop.phiC-Q.sum.down}))}}}=\sum_{B\mid M}\mu\left(  B\right)
\underbrace{\sum_{E\mid\dfrac{M}{B}}}_{\substack{=\sum_{\substack{E\mid
M;\\BE\mid M}}\\\text{(since the monic divisors }E\text{ of }\dfrac{M}%
{B}\text{ are precisely}\\\text{the monic divisors }E\text{ of }M\text{
satisfying }BE\mid M\text{)}}}E\\
&  =\sum_{B\mid M}\mu\left(  B\right)  \sum_{\substack{E\mid M;\\BE\mid
M}}E=\underbrace{\sum_{B\mid M}\sum_{\substack{E\mid M;\\BE\mid M}}}%
_{=\sum_{E\mid M}\sum_{\substack{B\mid M;\\BE\mid M}}}\mu\left(  B\right)
E=\sum_{E\mid M}\underbrace{\sum_{\substack{B\mid M;\\BE\mid M}}\mu\left(
B\right)  }_{\substack{=\left[  E=M\right]  \\\text{(by Corollary
\ref{cor.moebius-Q.sum-rel})}}}E\\
&  =\sum_{E\mid M}\left[  E=M\right]  E=\underbrace{\left[  M=M\right]  }%
_{=1}M+\sum_{\substack{E\mid M;\\E\neq M}}\underbrace{\left[  E=M\right]
}_{\substack{=0\\\text{(since }E\neq M\text{)}}}E\\
&  \ \ \ \ \ \ \ \ \ \ \left(  \text{here, we have split off the addend for
}E=M\text{ from the sum}\right) \\
&  =M+\underbrace{\sum_{\substack{E\mid M;\\E\neq M}}0E}_{=0}=M.
\end{align*}
This proves Proposition \ref{prop.phiC-Q.sum}.
\end{proof}

Next, let us study the function $\varphi$:

\begin{proposition}
\label{prop.phi-Q.formula}Let $M\in\mathbb{F}_{q}\left[  T\right]  _{+}$.

\textbf{(a)} We have $\varphi\left(  M\right)  \in\mathbb{N}_{+}$.

\textbf{(b)} We have $\varphi\left(  M\right)  =q^{\deg M}\prod\limits_{\pi
\in\operatorname*{PF}M}\left(  1-\dfrac{1}{q^{\deg\pi}}\right)  $.

\textbf{(c)} We have $\varphi\left(  M\right)  \equiv\mu\left(  M\right)
\operatorname{mod}p$.

\textbf{(d)} We have $\varphi\left(  M\right)  =\mu\left(  M\right)  $ in
$\mathbb{F}_{q}$.

\textbf{(e)} Let $A$ be the ring $\mathbb{F}_{q}\left[  T\right]  $. For any
ring $B$, we let $B^{\times}$ denote the group of units of $B$. Then,
$\varphi\left(  M\right)  =\left\vert \left(  A/MA\right)  ^{\times
}\right\vert $.
\end{proposition}

Proposition \ref{prop.phi-Q.formula} \textbf{(e)} is used as a definition of
$\varphi\left(  M\right)  $ in \cite[\S 6]{kc-carlitz}.

\begin{proof}
[Proof of Proposition \ref{prop.phi-Q.formula}.]\textbf{(b)} Let
$M=P_{1}^{a_{1}}P_{2}^{a_{2}}\cdots P_{k}^{a_{k}}$ be the factorization of $M$
into monic irreducible polynomials, with all of $a_{1},a_{2},\ldots,a_{k}$
being positive integers (and with $P_{1},P_{2},\ldots,P_{k}$ being
distinct).\footnote{This is well-defined, since $M$ is monic and since
$\mathbb{F}_{q}\left[  T\right]  $ is a principal ideal domain. Of course, $k$
can be $0$ (when $M=1$).} Then, the \textbf{squarefree} monic divisors $D$ of
$M$ all have the form $\prod_{i\in I}P_{i}$ for some subset $I$ of $\left\{
1,2,\ldots,k\right\}  $. More precisely, there exists a bijection%
\begin{align}
\left\{  I\subseteq\left\{  1,2,\ldots,k\right\}  \right\}   &  \rightarrow
\left(  \text{the set of all squarefree monic divisors of }M\right)
,\nonumber\\
I  &  \mapsto\prod_{i\in I}P_{i}. \label{pf.prop.phi-Q.formula.b.bij}%
\end{align}


Moreover, every subset $I$ of $\left\{  1,2,\ldots,k\right\}  $ satisfies
(\ref{pf.prop.moebius-Q.sum.2}). (This is proven as in our proof of
Proposition \ref{prop.moebius-Q.sum}.)

Furthermore, every subset $I$ of $\left\{  1,2,\ldots,k\right\}  $ satisfies%
\begin{align}
q^{\deg\left(  M/\prod_{i\in I}P_{i}\right)  }  &  =q^{\deg M-\sum_{i\in
I}\deg\left(  P_{i}\right)  }\ \ \ \ \ \ \ \ \ \ \left(  \text{since }%
\deg\left(  M/\prod_{i\in I}P_{i}\right)  =\deg M-\sum_{i\in I}\deg\left(
P_{i}\right)  \right) \nonumber\\
&  =\dfrac{q^{\deg M}}{\prod_{i\in I}q^{\deg\left(  P_{i}\right)  }}.
\label{pf.prop.phi-Q.formula.b.qdeg}%
\end{align}


The definition of $P_{1},P_{2},\ldots,P_{k}$ shows that $\left(  P_{1}%
,P_{2},\ldots,P_{k}\right)  $ is a list of all prime factors of $M$, with no
repetitions. Thus, the map $\left\{  1,2,\ldots,k\right\}  \rightarrow
\operatorname*{PF}M,\ i\mapsto P_{i}$ is a bijection.

The definition of $\varphi$ yields%
\begin{align*}
\varphi\left(  M\right)   &  =\sum\limits_{D\mid M}\mu\left(  D\right)
q^{\deg\left(  M/D\right)  }\\
&  =\sum_{\substack{D\mid M;\\D\text{ is squarefree}}}\mu\left(  D\right)
q^{\deg\left(  M/D\right)  }+\sum_{\substack{D\mid M;\\D\text{ is not
squarefree}}}\underbrace{\mu\left(  D\right)  }_{\substack{=0\\\text{(by the
definition}\\\text{of }\mu\text{, since }D\\\text{is not squarefree)}}%
}q^{\deg\left(  M/D\right)  }\\
&  =\sum_{\substack{D\mid M;\\D\text{ is squarefree}}}\mu\left(  D\right)
q^{\deg\left(  M/D\right)  }+\underbrace{\sum_{\substack{D\mid M;\\D\text{ is
not squarefree}}}0q^{\deg\left(  M/D\right)  }}_{=0}\\
&  =\sum_{\substack{D\mid M;\\D\text{ is squarefree}}}\mu\left(  D\right)
q^{\deg\left(  M/D\right)  }=\sum_{I\subseteq\left\{  1,2,\ldots,k\right\}
}\underbrace{\mu\left(  \prod_{i\in I}P_{i}\right)  }_{\substack{=\left(
-1\right)  ^{\left\vert I\right\vert }\\\text{(by
(\ref{pf.prop.moebius-Q.sum.2}))}}}\underbrace{q^{\deg\left(  M/\prod_{i\in
I}P_{i}\right)  }}_{\substack{=\dfrac{q^{\deg M}}{\prod_{i\in I}q^{\deg\left(
P_{i}\right)  }}\\\text{(by (\ref{pf.prop.phi-Q.formula.b.qdeg}))}}}\\
&  \ \ \ \ \ \ \ \ \ \ \left(
\begin{array}
[c]{c}%
\text{here, we have substituted }\prod_{i\in I}P_{i}\text{ for }D\\
\text{due to the bijection (\ref{pf.prop.moebius-Q.sum.bij})}%
\end{array}
\right) \\
&  =\sum_{I\subseteq\left\{  1,2,\ldots,k\right\}  }\underbrace{\left(
-1\right)  ^{\left\vert I\right\vert }}_{=\prod_{i\in I}\left(  -1\right)
}\dfrac{q^{\deg M}}{\prod_{i\in I}q^{\deg\left(  P_{i}\right)  }}%
=\sum_{I\subseteq\left\{  1,2,\ldots,k\right\}  }\left(  \prod_{i\in I}\left(
-1\right)  \right)  \dfrac{q^{\deg M}}{\prod_{i\in I}q^{\deg\left(
P_{i}\right)  }}\\
&  =q^{\deg M}\sum_{I\subseteq\left\{  1,2,\ldots,k\right\}  }%
\underbrace{\dfrac{\prod_{i\in I}\left(  -1\right)  }{\prod_{i\in I}%
q^{\deg\left(  P_{i}\right)  }}}_{=\prod_{i\in I}\dfrac{-1}{q^{\deg\left(
P_{i}\right)  }}}=q^{\deg M}\underbrace{\sum_{I\subseteq\left\{
1,2,\ldots,k\right\}  }\prod_{i\in I}\dfrac{-1}{q^{\deg\left(  P_{i}\right)
}}}_{\substack{=\prod_{i\in\left\{  1,2,\ldots,k\right\}  }\left(
1+\dfrac{-1}{q^{\deg\left(  P_{i}\right)  }}\right)  \\\text{(by Lemma
\ref{lem.moebius-Q.Z} \textbf{(b)}, applied to }R=\mathbb{Q}\text{,}%
\\Z=\left\{  1,2,\ldots,k\right\}  \text{ and }r_{i}=\dfrac{-1}{q^{\deg\left(
P_{i}\right)  }}\text{)}}}\\
&  =q^{\deg M}\prod_{i\in\left\{  1,2,\ldots,k\right\}  }\left(  1+\dfrac
{-1}{q^{\deg\left(  P_{i}\right)  }}\right)  =q^{\deg M}\prod_{\pi
\in\operatorname*{PF}M}\underbrace{\left(  1+\dfrac{-1}{q^{\deg\pi}}\right)
}_{=1-\dfrac{1}{q^{\deg\pi}}}\\
&  \ \ \ \ \ \ \ \ \ \ \left(
\begin{array}
[c]{c}%
\text{here, we have substituted }\pi\text{ for }P_{i}\text{ in the product,}\\
\text{since the map }\left\{  1,2,\ldots,k\right\}  \rightarrow
\operatorname*{PF}M,\ i\mapsto P_{i}\text{ is a bijection}%
\end{array}
\right) \\
&  =q^{\deg M}\prod_{\pi\in\operatorname*{PF}M}\left(  1-\dfrac{1}{q^{\deg\pi
}}\right)  .
\end{align*}
This proves Proposition \ref{prop.phi-Q.formula} \textbf{(b)}.

\textbf{(a)} The definition of $\varphi$ yields $\varphi\left(  M\right)
=\sum\limits_{D\mid M}\mu\left(  D\right)  q^{\deg\left(  M/D\right)  }%
\in\mathbb{Z}$ (since $\mu\left(  D\right)  $ and $q^{\deg\left(  M/D\right)
}$ are integers for all $D\mid M$). But every $\pi\in\operatorname*{PF}M$
satisfies $\deg\pi>0$ (since $\pi$ is irreducible) and thus $q^{\deg\pi}>1$
(since $q>1$) and therefore
\begin{equation}
1>\dfrac{1}{q^{\deg\pi}}. \label{pf.prop.phi-Q.formula.a.1}%
\end{equation}


Proposition \ref{prop.phi-Q.formula} \textbf{(b)} yields%
\[
\varphi\left(  M\right)  =\underbrace{q^{\deg M}}_{>0}\prod\limits_{\pi
\in\operatorname*{PF}M}\underbrace{\left(  1-\dfrac{1}{q^{\deg\pi}}\right)
}_{\substack{>0\\\text{(by (\ref{pf.prop.phi-Q.formula.a.1}))}}}>0.
\]
Combining this with $\varphi\left(  M\right)  \in\mathbb{Z}$, we find that
$\varphi\left(  M\right)  \in\mathbb{N}_{+}$. This proves Proposition
\ref{prop.phi-Q.formula} \textbf{(a)}.

\textbf{(c)} If $D$ is a monic divisor of $M$ satisfying $D\neq M$, then
\begin{equation}
\mu\left(  D\right)  q^{\deg\left(  M/D\right)  }\equiv0\operatorname{mod}p
\label{pf.prop.phi-Q.formula.c.1}%
\end{equation}
\footnote{\textit{Proof of (\ref{pf.prop.phi-Q.formula.c.1}):} Let $D$ be a
monic divisor of $M$ satisfying $D\neq M$. From $M\neq D$, we obtain
$M/D\neq1$.
\par
We have $M/D\in\mathbb{F}_{q}\left[  T\right]  $ (since $D$ is a divisor of
$M$). Also, the polynomial $M/D$ is monic (since $M$ and $D$ are monic). If we
had $\deg\left(  M/D\right)  =0$, then we would have $M/D=1$ (because $M/D$ is
monic), which would contradict $M/D\neq1$. Thus, we cannot have $\deg\left(
M/D\right)  =0$. Hence, we must have $\deg\left(  M/D\right)  \geq1$ (since
$M/D\in\mathbb{F}_{q}\left[  T\right]  $). Hence, $q^{\deg\left(  M/D\right)
}$ is divisible by $q$, and thus also divisible by $p$ (since $p\mid q$). In
other words, $q^{\deg\left(  M/D\right)  }\equiv0\operatorname{mod}p$. Hence,
$\mu\left(  D\right)  \underbrace{q^{\deg\left(  M/D\right)  }}_{\equiv
0\operatorname{mod}p}\equiv0\operatorname{mod}p$ (since $\mu\left(  D\right)
\in\left\{  -1,0,1\right\}  \subseteq\mathbb{Z}$). This proves
(\ref{pf.prop.phi-Q.formula.c.1}).}. Now, the definition of $\varphi$ yields%
\begin{align*}
\varphi\left(  M\right)   &  =\sum\limits_{D\mid M}\mu\left(  D\right)
q^{\deg\left(  M/D\right)  }=\mu\left(  M\right)  \underbrace{q^{\deg\left(
M/M\right)  }}_{\substack{=q^{0}\\\text{(since }\deg\left(  M/M\right)
=\deg1=0\text{)}}}+\sum\limits_{\substack{D\mid M;\\D\neq M}}\underbrace{\mu
\left(  D\right)  q^{\deg\left(  M/D\right)  }}_{\substack{\equiv
0\operatorname{mod}p\\\text{(by (\ref{pf.prop.phi-Q.formula.c.1}))}}}\\
&  \ \ \ \ \ \ \ \ \ \ \left(  \text{here, we have split off the addend for
}D=M\text{ from the sum}\right) \\
&  \equiv\mu\left(  M\right)  \underbrace{q^{0}}_{=1}+\underbrace{\sum
\limits_{\substack{D\mid M;\\D\neq M}}0}_{=0}=\mu\left(  M\right)
\operatorname{mod}p.
\end{align*}
This proves Proposition \ref{prop.phi-Q.formula} \textbf{(c)}.

\textbf{(d)} Proposition \ref{prop.phi-Q.formula} \textbf{(c)} shows that
$\varphi\left(  M\right)  \equiv\mu\left(  M\right)  \operatorname{mod}p$.
Hence, $\varphi\left(  M\right)  =\mu\left(  M\right)  $ holds in any field of
characteristic $p$. In particular, $\varphi\left(  M\right)  =\mu\left(
M\right)  $ holds in $\mathbb{F}_{q}$ (since $\mathbb{F}_{q}$ is a field of
characteristic $p$).

\textbf{(e)} Let us first observe two general facts:

\begin{itemize}
\item If $s\in\mathbb{F}_{q}\left[  T\right]  $ is a nonzero polynomial, then%
\begin{equation}
\left\vert A/sA\right\vert =q^{\deg s} \label{pf.prop.phi-Q.formula.e.qdegs}%
\end{equation}
\footnote{\textit{Proof of (\ref{pf.prop.phi-Q.formula.e.qdegs}):} Let
$s\in\mathbb{F}_{q}\left[  T\right]  $ be a nonzero polynomial. Then, it is
well-known that $A/sA$ is an $\deg s$-dimensional $\mathbb{F}_{q}$-vector
space (since $A=\mathbb{F}_{q}\left[  T\right]  $). Hence, $\left\vert
A/sA\right\vert =\left\vert \mathbb{F}_{q}\right\vert ^{\deg s}$. Since
$\left\vert \mathbb{F}_{q}\right\vert =q$, this rewrites as $\left\vert
A/sA\right\vert =q^{\deg s}$. This proves (\ref{pf.prop.phi-Q.formula.e.qdegs}%
).}.

\item If $s\in\mathbb{F}_{q}\left[  T\right]  $ is a monic irreducible
polynomial, and if $n$ is a positive integer, then%
\begin{equation}
\left\vert \left(  A/s^{n}A\right)  ^{\times}\right\vert =q^{n\deg
s}-q^{\left(  n-1\right)  \deg s} \label{pf.prop.phi-Q.formula.e.locality}%
\end{equation}
\footnote{\textit{Proof of (\ref{pf.prop.phi-Q.formula.e.locality}):} Let
$s\in\mathbb{F}_{q}\left[  T\right]  $ be a monic irreducible polynomial, and
let $n$ be a positive integer.
\par
Applying (\ref{pf.prop.phi-Q.formula.e.qdegs}) to $s^{n-1}$ instead of $s$, we
obtain $\left\vert A/s^{n-1}A\right\vert =q^{\deg\left(  s^{n-1}\right)
}=q^{\left(  n-1\right)  \deg s}$ (since $\deg\left(  s^{n-1}\right)  =\left(
n-1\right)  \deg s$).
\par
Applying (\ref{pf.prop.phi-Q.formula.e.qdegs}) to $s^{n}$ instead of $s$, we
obtain $\left\vert A/s^{n}A\right\vert =q^{\deg\left(  s^{n}\right)
}=q^{n\deg s}$ (since $\deg\left(  s^{n}\right)  =n\deg s$).
\par
Let $B$ be the ring $A/s^{n}A$. Then, $B=\underbrace{A}_{=\mathbb{F}%
_{q}\left[  T\right]  }/s^{n}\underbrace{A}_{=\mathbb{F}_{q}\left[  T\right]
}=\mathbb{F}_{q}\left[  T\right]  /s^{n}\mathbb{F}_{q}\left[  T\right]  $.
Hence, Proposition \ref{prop.FT.modsn} \textbf{(b)} (applied to $\mathbb{F}%
=\mathbb{F}_{q}$) shows that $sB\cong\mathbb{F}_{q}\left[  T\right]
/s^{n-1}\mathbb{F}_{q}\left[  T\right]  $ as $\mathbb{F}_{q}$-vector spaces.
Thus, $sB\cong\underbrace{\mathbb{F}_{q}\left[  T\right]  }_{=A}%
/s^{n-1}\underbrace{\mathbb{F}_{q}\left[  T\right]  }_{=A}=A/s^{n-1}A$ as
$\mathbb{F}_{q}$-vector spaces. Hence, $\left\vert sB\right\vert =\left\vert
A/s^{n-1}A\right\vert =q^{\left(  n-1\right)  \deg s}$. Also, from
$B=A/s^{n}A$, we obtain $\left\vert B\right\vert =\left\vert A/s^{n}%
A\right\vert =q^{n\deg s}$.
\par
But Proposition \ref{prop.FT.modsn} \textbf{(a)} (applied to $\mathbb{F}%
=\mathbb{F}_{q}$) yields $B^{\times}=B\setminus sB$. Hence,%
\begin{align*}
\left\vert B^{\times}\right\vert  &  =\left\vert B\setminus sB\right\vert
=\underbrace{\left\vert B\right\vert }_{=q^{n\deg s}}-\underbrace{\left\vert
sB\right\vert }_{=q^{\left(  n-1\right)  \deg s}}\ \ \ \ \ \ \ \ \ \ \left(
\text{since }sB\subseteq B\right) \\
&  =q^{n\deg s}-q^{\left(  n-1\right)  \deg s}.
\end{align*}
Since $B=A/s^{n}A$, this rewrites as $\left\vert \left(  A/s^{n}A\right)
^{\times}\right\vert =q^{n\deg s}-q^{\left(  n-1\right)  \deg s}$. Hence,
(\ref{pf.prop.phi-Q.formula.e.locality}) is proven.}.
\end{itemize}

The polynomial $M$ is monic. Hence, the factorization of $M$ into monic
irreducible polynomials is $M=\prod_{s\in\operatorname*{PF}M}s^{v_{s}\left(
M\right)  }$. Notice that $v_{s}\left(  M\right)  $ is a positive integer for
each $s\in\operatorname*{PF}M$.

From $M=\prod_{s\in\operatorname*{PF}M}s^{v_{s}\left(  M\right)  }$, we
conclude that%
\[
\deg M=\deg\prod_{s\in\operatorname*{PF}M}s^{v_{s}\left(  M\right)  }%
=\sum_{s\in\operatorname*{PF}M}\deg\left(  s^{v_{s}\left(  M\right)  }\right)
,
\]
and thus%
\begin{equation}
q^{\deg M}=q^{\sum_{s\in\operatorname*{PF}M}\deg\left(  s^{v_{s}\left(
M\right)  }\right)  }=\prod\limits_{s\in\operatorname*{PF}M}q^{\deg\left(
s^{v_{s}\left(  M\right)  }\right)  }. \label{pf.prop.phi-Q.formula.e.qdegM}%
\end{equation}


For each $s\in\operatorname*{PF}M$, define an ideal $I_{s}$ of $A$ by
$I_{s}=s^{v_{s}\left(  M\right)  }A$. Notice that $A$ is a principal ideal
domain (since $A=\mathbb{F}_{q}\left[  T\right]  $). We have%
\begin{equation}
\left\vert \left(  A/I_{s}\right)  ^{\times}\right\vert =q^{\deg\left(
s^{v_{s}\left(  M\right)  }\right)  }\left(  1-\dfrac{1}{q^{\deg s}}\right)
\label{pf.prop.phi-Q.formula.e.locality-specific}%
\end{equation}
for each $s\in\operatorname*{PF}M$\ \ \ \ \footnote{\textit{Proof of
(\ref{pf.prop.phi-Q.formula.e.locality-specific}):} Let $s\in
\operatorname*{PF}M$. Thus, $s$ is a monic irreducible polynomial dividing
$M$.
\par
Let $n=v_{s}\left(  M\right)  $. Then, $n=v_{s}\left(  M\right)  $ is a
positive integer (since $s$ divides $n$). Hence,
(\ref{pf.prop.phi-Q.formula.e.locality}) yields
\begin{align*}
\left\vert \left(  A/s^{n}A\right)  ^{\times}\right\vert  &  =q^{n\deg
s}-\underbrace{q^{\left(  n-1\right)  \deg s}}_{\substack{=q^{n\deg s-\deg
s}\\\text{(since }\left(  n-1\right)  \deg s=n\deg s-\deg s\text{)}}}=q^{n\deg
s}-\underbrace{q^{n\deg s-\deg s}}_{=\dfrac{q^{n\deg s}}{q^{\deg s}}}\\
&  =q^{n\deg s}-\dfrac{q^{n\deg s}}{q^{\deg s}}=\underbrace{q^{n\deg s}%
}_{\substack{=q^{\deg\left(  s^{n}\right)  }\\\text{(since }n\deg
s=\deg\left(  s^{n}\right)  \text{)}}}\left(  1-\dfrac{1}{q^{\deg s}}\right)
=q^{\deg\left(  s^{n}\right)  }\left(  1-\dfrac{1}{q^{\deg s}}\right) \\
&  =q^{\deg\left(  s^{v_{s}\left(  M\right)  }\right)  }\left(  1-\dfrac
{1}{q^{\deg s}}\right)  \ \ \ \ \ \ \ \ \ \ \left(  \text{since }%
n=v_{s}\left(  M\right)  \right)  .
\end{align*}
\par
Also, $I_{s}=s^{v_{s}\left(  M\right)  }A=s^{n}A$ (since $v_{s}\left(
M\right)  =n$). Hence, $\left\vert \left(  A/I_{s}\right)  ^{\times
}\right\vert =\left\vert \left(  A/s^{n}A\right)  ^{\times}\right\vert
=q^{\deg\left(  s^{v_{s}\left(  M\right)  }\right)  }\left(  1-\dfrac
{1}{q^{\deg s}}\right)  $. This proves
(\ref{pf.prop.phi-Q.formula.e.locality-specific}).}.

Every two distinct elements $s$ and $t$ of $\operatorname*{PF}M$ satisfy
$I_{s}+I_{t}=A$\ \ \ \ \footnote{\textit{Proof.} Let $s$ and $t$ be two
distinct elements of $\operatorname*{PF}M$. Thus, $s$ and $t$ are two distinct
monic irreducible polynomials in $\mathbb{F}_{q}\left[  T\right]  $. Hence,
Lemma \ref{lem.CRT.FqT.Is+It} (applied to $\mathbb{F=F}_{q}$, $n=v_{s}\left(
M\right)  $, $m=v_{t}\left(  M\right)  $ and $R=A$) yields $s^{v_{s}\left(
M\right)  }A+t^{v_{t}\left(  M\right)  }A=A$.
\par
On the other hand, $I_{s}=s^{v_{s}\left(  M\right)  }A$ (by the definition of
$I_{s}$) and $I_{t}=t^{v_{t}\left(  M\right)  }A$ (by the definition of
$I_{t}$). Adding these two equalities, we obtain $I_{s}+I_{t}=s^{v_{s}\left(
M\right)  }A+t^{v_{t}\left(  M\right)  }A=A$. Qed.}. Hence, Theorem
\ref{thm.CRT.ring} \textbf{(b)} (applied to $\mathbf{S}=\operatorname*{PF}M$)
shows that the canonical $A$-algebra homomorphism
\[
A/\left(  \prod\limits_{s\in\operatorname*{PF}M}I_{s}\right)  \rightarrow
\prod\limits_{s\in\operatorname*{PF}M}\left(  A/I_{s}\right)
,\ \ \ \ \ \ \ \ \ \ a+\prod\limits_{s\in\operatorname*{PF}M}I_{s}%
\mapsto\left(  a+I_{s}\right)  _{s\in\operatorname*{PF}M}%
\]
is well-defined and an $A$-algebra isomorphism. Hence, $A/\left(
\prod\limits_{s\in\operatorname*{PF}M}I_{s}\right)  \cong\prod\limits_{s\in
\operatorname*{PF}M}\left(  A/I_{s}\right)  $ as $A$-algebras.

But
\[
\prod\limits_{s\in\operatorname*{PF}M}\underbrace{I_{s}}_{=s^{v_{s}\left(
M\right)  }A}=\prod\limits_{s\in\operatorname*{PF}M}\left(  s^{v_{s}\left(
M\right)  }A\right)  =\underbrace{\left(  \prod_{s\in\operatorname*{PF}%
M}s^{v_{s}\left(  M\right)  }\right)  }_{=M}A=MA.
\]
Thus, $A/\underbrace{\left(  \prod\limits_{s\in\operatorname*{PF}M}%
I_{s}\right)  }_{=MA}=A/MA$. Hence, $A/MA=A/\left(  \prod\limits_{s\in
\operatorname*{PF}M}I_{s}\right)  \cong\prod\limits_{s\in\operatorname*{PF}%
M}\left(  A/I_{s}\right)  $ as $A$-algebras. Therefore,%
\[
\left(  A/MA\right)  ^{\times}\cong\left(  \prod\limits_{s\in
\operatorname*{PF}M}\left(  A/I_{s}\right)  \right)  ^{\times}\cong%
\prod\limits_{s\in\operatorname*{PF}M}\left(  A/I_{s}\right)  ^{\times}%
\]
as groups. Hence,%
\begin{align*}
\left\vert \left(  A/MA\right)  ^{\times}\right\vert  &  =\left\vert
\prod\limits_{s\in\operatorname*{PF}M}\left(  A/I_{s}\right)  ^{\times
}\right\vert =\prod\limits_{s\in\operatorname*{PF}M}\underbrace{\left\vert
\left(  A/I_{s}\right)  ^{\times}\right\vert }_{\substack{=q^{\deg\left(
s^{v_{s}\left(  M\right)  }\right)  }\left(  1-\dfrac{1}{q^{\deg s}}\right)
\\\text{(by (\ref{pf.prop.phi-Q.formula.e.locality-specific}))}}}\\
&  =\prod\limits_{s\in\operatorname*{PF}M}\left(  q^{\deg\left(
s^{v_{s}\left(  M\right)  }\right)  }\left(  1-\dfrac{1}{q^{\deg s}}\right)
\right) \\
&  =\underbrace{\left(  \prod\limits_{s\in\operatorname*{PF}M}q^{\deg\left(
s^{v_{s}\left(  M\right)  }\right)  }\right)  }_{\substack{=q^{\deg
M}\\\text{(by (\ref{pf.prop.phi-Q.formula.e.qdegM}))}}}\underbrace{\prod
\limits_{s\in\operatorname*{PF}M}\left(  1-\dfrac{1}{q^{\deg s}}\right)
}_{\substack{=\prod\limits_{\pi\in\operatorname*{PF}M}\left(  1-\dfrac
{1}{q^{\deg\pi}}\right)  \\\text{(here, we have renamed the}\\\text{index
}s\text{ as }\pi\text{ in the product)}}}\\
&  =q^{\deg M}\prod\limits_{\pi\in\operatorname*{PF}M}\left(  1-\dfrac
{1}{q^{\deg\pi}}\right)  =\varphi\left(  M\right)
\end{align*}
(by Proposition \ref{prop.phi-Q.formula} \textbf{(b)}). This proves
Proposition \ref{prop.phi-Q.formula} \textbf{(e)}.
\end{proof}

Finally, here is an identity that connects the functions $\mu$ and
$\varphi_{C}$ (an analogue of \cite[(12.347)]{reiner-hopf}):

\begin{proposition}
\label{prop.phiC-Q.andmu}Let $M\in\mathbb{F}_{q}\left[  T\right]  _{+}$. Then,%
\[
\sum_{D\mid M}D\mu\left(  D\right)  \varphi_{C}\left(  \dfrac{M}{D}\right)
=\mu\left(  M\right)  \ \ \ \ \ \ \ \ \ \ \text{in }\mathbb{F}_{q}\left[
T\right]  .
\]

\end{proposition}

\begin{proof}
[Proof of Proposition \ref{prop.phiC-Q.andmu}.]We shall use the notation of
Proposition \ref{prop.moebius-Q.sum}.

Every $E\in\mathbb{F}_{q}\left[  T\right]  _{+}$ satisfies
(\ref{pf.prop.phiC-Q.sum.phiCE=}). (This can be proven as in our proof of
Proposition \ref{prop.phiC-Q.sum} above.) Now, every monic divisor $D$ of $M$
satisfies%
\begin{equation}
\varphi_{C}\left(  \dfrac{M}{D}\right)  =\sum_{\substack{B\mid M;\\BD\mid
M}}\mu\left(  B\right)  \dfrac{M}{BD} \label{pf.prop.phiC-Q.andmu.1}%
\end{equation}
\footnote{\textit{Proof of (\ref{pf.prop.phiC-Q.andmu.1}):} Let $D$ be a monic
divisor of $M$. Thus, $M/D\in\mathbb{F}_{q}\left[  T\right]  $. Also, the
polynomial $M/D$ is monic (since $M$ and $D$ are monic). Hence, $M/D\in
\mathbb{F}_{q}\left[  T\right]  _{+}$. Thus, (\ref{pf.prop.phiC-Q.sum.phiCE=})
(applied to $E=M/D$) yields%
\[
\varphi_{C}\left(  M/D\right)  =\underbrace{\sum_{B\mid M/D}}_{\substack{=\sum
_{\substack{B\mid M;\\BD\mid M}}\\\text{(since the monic divisors }B\text{ of
}M/D\\\text{are exactly the monic divisors }B\text{ of }M\\\text{that satisfy
}BD\mid M\text{)}}}\mu\left(  B\right)  \underbrace{\dfrac{M/D}{B}}%
_{=\dfrac{M}{BD}}=\sum_{\substack{B\mid M;\\BD\mid M}}\mu\left(  B\right)
\dfrac{M}{BD}.
\]
Thus, $\varphi_{C}\left(  \dfrac{M}{D}\right)  =\varphi_{C}\left(  M/D\right)
=\sum_{\substack{B\mid M;\\BD\mid M}}\mu\left(  B\right)  \dfrac{M}{BD}$. This
proves (\ref{pf.prop.phiC-Q.andmu.1}).}. Also, every monic divisor $B$ of $M$
satisfies%
\begin{equation}
\sum_{\substack{D\mid M;\\BD\mid M}}\mu\left(  D\right)  =\left[  B=M\right]
\label{pf.prop.phiC-Q.andmu.2}%
\end{equation}
\footnote{\textit{Proof of (\ref{pf.prop.phiC-Q.andmu.2}):} We can rename the
variables $E$ and $B$ as $B$ and $D$ in Corollary \ref{cor.moebius-Q.sum-rel}.
As a result, we conclude that $\sum_{\substack{D\mid M;\\DB\mid M}}\mu\left(
D\right)  =\left[  B=M\right]  $. Hence, $\left[  B=M\right]  =\sum
_{\substack{D\mid M;\\DB\mid M}}\mu\left(  D\right)  =\sum_{\substack{D\mid
M;\\BD\mid M}}\mu\left(  D\right)  $. This proves
(\ref{pf.prop.phiC-Q.andmu.2}).}.

Now,%
\begin{align*}
&  \sum_{D\mid M}D\mu\left(  D\right)  \underbrace{\varphi_{C}\left(
\dfrac{M}{D}\right)  }_{\substack{=\sum_{\substack{B\mid M;\\BD\mid M}%
}\mu\left(  B\right)  \dfrac{M}{BD}\\\text{(by (\ref{pf.prop.phiC-Q.andmu.1}%
))}}}\\
&  =\sum_{D\mid M}D\mu\left(  D\right)  \sum_{\substack{B\mid M;\\BD\mid
M}}\mu\left(  B\right)  \dfrac{M}{BD}=\underbrace{\sum_{D\mid M}%
\sum_{\substack{B\mid M;\\BD\mid M}}}_{=\sum_{B\mid M}\sum_{\substack{D\mid
M;\\BD\mid M}}}\underbrace{D\mu\left(  D\right)  \mu\left(  B\right)
\dfrac{M}{BD}}_{=\dfrac{M}{B}\mu\left(  B\right)  \mu\left(  D\right)  }\\
&  =\sum_{B\mid M}\sum_{\substack{D\mid M;\\BD\mid M}}\dfrac{M}{B}\mu\left(
B\right)  \mu\left(  D\right)  =\sum_{B\mid M}\dfrac{M}{B}\mu\left(  B\right)
\underbrace{\sum_{\substack{D\mid M;\\BD\mid M}}\mu\left(  D\right)
}_{\substack{=\left[  B=M\right]  \\\text{(by (\ref{pf.prop.phiC-Q.andmu.2}%
))}}}\\
&  =\sum_{B\mid M}\dfrac{M}{B}\mu\left(  B\right)  \left[  B=M\right]
=\underbrace{\dfrac{M}{M}}_{=1}\mu\left(  M\right)  \underbrace{\left[
M=M\right]  }_{=1}+\sum_{\substack{B\mid M;\\B\neq M}}\dfrac{M}{B}\mu\left(
B\right)  \underbrace{\left[  B=M\right]  }_{\substack{=0\\\text{(since }B\neq
M\text{)}}}\\
&  \ \ \ \ \ \ \ \ \ \ \left(  \text{here, we have split off the addend for
}B=M\text{ from the sum}\right) \\
&  =\mu\left(  M\right)  +\underbrace{\sum_{\substack{B\mid M;\\B\neq
M}}\dfrac{M}{B}\mu\left(  B\right)  0}_{=0}=\mu\left(  M\right)
\end{align*}
in $\mathbb{F}_{q}\left[  T\right]  $. This proves Proposition
\ref{prop.phiC-Q.andmu}.
\end{proof}

\subsection{The Carlitz ghost-Witt equivalence}

We are now ready to prove a generalization of Theorem \ref{thm.carlitz.gW}:

\begin{theorem}
\label{thm.F.gW}Let $N$ be a $q$-nest. Let $A$ be an $\mathcal{F}$-module. For
every $P\in N$, let $\varphi_{P}$ be an endomorphism of the $\mathbb{F}_{q}%
$-vector space $A$. (The notation $\varphi_{P}$ for these endomorphisms should
not be confused with the notation $\varphi_{C}$ defined in Definition
\ref{def.phiC-q}; we shall ensure this by never using the notation $C$ for a
polynomial in this context.) Let us make the following three assumptions:

\textit{Assumption 1:} For every $P\in N$, the map $\varphi_{P}$ is an
endomorphism of the $\mathcal{F}$-module $A$.

\textit{Assumption 2:} We have $\varphi_{\pi}\left(  a\right)  \equiv\left(
\operatorname*{Carl}\pi\right)  a\operatorname{mod}\pi A$ for every $a\in A$
and every monic irreducible $\pi\in N$.

\textit{Assumption 3:} We have $\varphi_{1}=\operatorname*{id}$. Furthermore,
$\varphi_{P}\circ\varphi_{Q}=\varphi_{PQ}$ for every $P\in N$ and every $Q\in
N$ satisfying $PQ\in N$.

Let $\left(  b_{P}\right)  _{P\in N}\in A^{N}$ be a family of elements of $A$.
Then, the following assertions $\mathcal{C}_{1}$, $\mathcal{D}_{1}$,
$\mathcal{D}_{2}$, $\mathcal{E}_{1}$, $\mathcal{F}_{1}$, $\mathcal{G}_{1}$,
and $\mathcal{G}_{2}$ are equivalent:

\textit{Assertion }$\mathcal{C}_{1}$\textit{:} Every $P\in N$ and every
$\pi\in\operatorname{PF}P$ satisfy%
\[
\varphi_{\pi}\left(  b_{P / \pi}\right)  \equiv b_{P}\operatorname{mod}%
\pi^{v_{\pi}\left(  P\right)  }A.
\]


\textit{Assertion }$\mathcal{D}_{1}$\textit{:} There exists a family $\left(
x_{P}\right)  _{P\in N}\in A^{N}$ of elements of $A$ such that%
\[
\left(  b_{P}=\sum_{D\mid P}D\cdot\left(  \operatorname*{Carl}\dfrac{P}%
{D}\right)  x_{D}\text{ for every }P\in N\right)  .
\]


\textit{Assertion }$\mathcal{D}_{2}$\textit{:} There exists a family $\left(
\widetilde{x}_{P}\right)  _{P\in N}\in A^{N}$ of elements of $A$ such that%
\[
\left(  b_{P}=\sum_{D\mid P}DF^{\deg\left(  P/D\right)  }\widetilde{x}%
_{D}\text{ for every }P\in N\right)  .
\]


\textit{Assertion }$\mathcal{E}_{1}$\textit{:} There exists a family $\left(
y_{P}\right)  _{P\in N}\in A^{N}$ of elements of $A$ such that%
\[
\left(  b_{P}=\sum_{D\mid P}D\varphi_{P / D}\left(  y_{D}\right)  \text{ for
every }P\in N\right)  .
\]


\textit{Assertion }$\mathcal{F}_{1}$\textit{:} Every $P\in N$ satisfies%
\[
\sum_{D\mid P}\mu\left(  D\right)  \varphi_{D}\left(  b_{P / D}\right)  \in
PA.
\]


\textit{Assertion }$\mathcal{G}_{1}$\textit{:} Every $P\in N$ satisfies%
\[
\sum_{D\mid P}\varphi_{C}\left(  D\right)  \varphi_{D}\left(  b_{P /
D}\right)  \in PA.
\]


\textit{Assertion }$\mathcal{G}_{2}$\textit{:} Every $P\in N$ satisfies%
\[
\sum_{D\mid P}\varphi\left(  D\right)  \varphi_{D}\left(  b_{P/D}\right)  \in
PA.
\]

\end{theorem}

Theorem \ref{thm.F.gW} is a generalization of Theorem \ref{thm.carlitz.gW} --
namely, it is precisely the generalization outlined in Remark
\ref{rmk.carlitz.gW.1'}. In order to see this, the reader should recall
Proposition \ref{prop.F.frobmod.cateq}, which says that (roughly speaking)
$\mathcal{F}$-modules are the same as Frobenius $\mathbb{F}_{q}\left[
T\right]  $-modules (which are precisely $\mathbb{F}_{q}\left[  T\right]
$-modules $A$ with an $\mathbb{F}_{q}$-linear Frobenius map $F:A\rightarrow A$
which satisfies (\ref{eq.frobcond})\footnote{This is slightly nontrivial,
because the equalities (\ref{eq.frobcond}) and (\ref{eq.def.F.frobmod.axiom})
are not obviously equivalent. Nevertheless, the equivalence of the equalities
(\ref{eq.frobcond}) and (\ref{eq.def.F.frobmod.axiom}) is easy to show.}).

Before we prove Theorem \ref{thm.F.gW}, let us show two more general facts:

\begin{lemma}
\label{lem.F.gW.F-G-gen0}Let $N$ be a $q$-nest. Let $A$ be an $\mathbb{F}%
_{q}\left[  T\right]  $-module. For every $P\in N$ and every monic divisor $D$
of $P$, let $g_{P,D}$ be an element of $A$. Let $\alpha$, $\beta$ and $\gamma$
are three maps from $N$ to $\mathbb{F}_{q}\left[  T\right]  $.

Assume that
\begin{equation}
\beta\left(  P\right)  =\sum_{D\mid P}D\gamma\left(  D\right)  \alpha\left(
\dfrac{P}{D}\right)  \ \ \ \ \ \ \ \ \ \ \text{for every }P\in N.
\label{eq.lem.F.gW.F-gen0.beta-through-alpha}%
\end{equation}


Furthermore, assume that every $P\in N$ and every monic divisor $E$ of $P$
satisfy%
\begin{equation}
E\sum_{\substack{D\mid P;\\DE\mid P}}\alpha\left(  D\right)  g_{P,DE}\in PA.
\label{eq.lem.F.gW.F-G-gen0.ass}%
\end{equation}
Then, every $P\in N$ and every monic divisor $E$ of $P$ satisfy%
\begin{equation}
E\sum_{\substack{D\mid P;\\DE\mid P}}\beta\left(  D\right)  g_{P,DE}\in PA.
\label{eq.lem.F.gW.F-G-gen0.claim}%
\end{equation}

\end{lemma}

\begin{proof}
[Proof of Lemma \ref{lem.F.gW.F-G-gen0}.]Let $P\in N$. Let $E$ be a monic
divisor of $P$. Then, every monic divisor $F$ of $P$ satisfies%
\begin{equation}
F\sum_{\substack{M\mid P;\\MF\mid P}}\alpha\left(  M\right)  g_{P,MF}\in PA
\label{pf.lem.F.gW.F-G-gen0.1}%
\end{equation}
\footnote{\textit{Proof of (\ref{pf.lem.F.gW.F-G-gen0.1}):} Let $F$ be a monic
divisor of $P$. Then,%
\begin{align*}
F\sum_{\substack{M\mid P;\\MF\mid P}}\alpha\left(  M\right)  g_{P,MF}  &
=F\sum_{\substack{D\mid P;\\DF\mid P}}\alpha\left(  D\right)  g_{P,DF}%
\ \ \ \ \ \ \ \ \ \ \left(  \text{here, we have renamed the summation index
}M\text{ as }D\right) \\
&  \in PA
\end{align*}
(by (\ref{eq.lem.F.gW.F-G-gen0.ass}) (applied to $E=F$)). This proves
(\ref{pf.lem.F.gW.F-G-gen0.1}).}. Furthermore, every monic divisor $D$ of $P$
satisfies%
\begin{equation}
\sum_{\substack{M\mid P;\\ME\mid P;\\D\mid M}}\alpha\left(  \dfrac{M}%
{D}\right)  g_{P,ME}=\sum_{\substack{M\mid P;\\MDE\mid P}}\alpha\left(
M\right)  g_{P,MDE} \label{pf.lem.F.gW.F-G-gen0.3}%
\end{equation}
\footnote{\textit{Proof of (\ref{pf.lem.F.gW.F-G-gen0.3}):} Let $D$ be a monic
divisor of $P$.
\par
Let $\mathfrak{A}$ be the set of all monic divisors $M$ of $P$ satisfying
$ME\mid P$ and $D\mid M$. Thus, $\sum_{M\in\mathfrak{A}}=\sum_{\substack{M\mid
P;\\ME\mid P;\\D\mid M}}$.
\par
Let $\mathfrak{B}$ be the set of all monic divisors $M$ of $P$ satisfying
$MDE\mid P$. Thus, $\sum_{M\in\mathfrak{B}}=\sum_{\substack{M\mid P;\\MDE\mid
P}}$.
\par
We have
\begin{equation}
M/D\in\mathfrak{B}\ \ \ \ \ \ \ \ \ \ \text{for each }M\in\mathfrak{A}\text{.}
\label{pf.lem.F.gW.F-G-gen0.3.pf.1}%
\end{equation}
\par
[\textit{Proof of (\ref{pf.lem.F.gW.F-G-gen0.3.pf.1}):} Let $M\in\mathfrak{A}%
$. In other words, $M$ is a monic divisor of $P$ satisfying $ME\mid P$ and
$D\mid M$ (by the definition of $\mathfrak{A}$). Now, $D\mid M$, so that
$M/D\in\mathbb{F}_{q}\left[  T\right]  _{+}$. The polynomial $M/D$ is monic
(since $M$ and $D$ are monic), and is a divisor of $P$ (since $M/D\mid M\mid
P$). It furthermore satisfies $\left(  M/D\right)  DE=ME\mid P$. Thus, $M/D$
is a monic divisor of $P$ satisfying $\left(  M/D\right)  DE\mid P$. In other
words, $M/D\in\mathfrak{B}$ (by the definition of $\mathfrak{B}$). This proves
(\ref{pf.lem.F.gW.F-G-gen0.3.pf.1}).]
\par
Furthermore, we have%
\begin{equation}
MD\in\mathfrak{A}\ \ \ \ \ \ \ \ \ \ \text{for each }M\in\mathfrak{B}\text{.}
\label{pf.lem.F.gW.F-G-gen0.3.pf.2}%
\end{equation}
\par
[\textit{Proof of (\ref{pf.lem.F.gW.F-G-gen0.3.pf.2}):} Let $M\in\mathfrak{B}%
$. In other words, $M$ is a monic divisor of $P$ satisfying $MDE\mid P$ (by
the definition of $\mathfrak{B}$). Now, the polynomial $MD$ is monic (since
$M$ and $D$ are monic), and is a divisor of $P$ (since $MD\mid MDE\mid P$).
Furthermore, it satisfies $\left(  MD\right)  E=MDE\mid P$ and $D\mid MD$.
Thus, $MD$ is a monic divisor of $P$ satisfying $\left(  MD\right)  E\mid P$
and $D\mid MD$. In other words, $MD\in\mathfrak{A}$ (by the definition of
$\mathfrak{A}$). This proves (\ref{pf.lem.F.gW.F-G-gen0.3.pf.2}).]
\par
Now, the map
\[
\mathfrak{A}\rightarrow\mathfrak{B},\ \ \ \ \ \ \ \ \ \ M\mapsto M/D
\]
is well-defined (according to (\ref{pf.lem.F.gW.F-G-gen0.3.pf.1})).
Furthermore, the map%
\[
\mathfrak{B}\rightarrow\mathfrak{A},\ \ \ \ \ \ \ \ \ \ M\mapsto MD
\]
is well-defined (according to (\ref{pf.lem.F.gW.F-G-gen0.3.pf.2})). These two
maps are mutually inverse (because one of them divides input by $D$, whereas
the other multiplies its input by $D$). Hence, they are both invertible. In
particular, the map
\[
\mathfrak{A}\rightarrow\mathfrak{B},\ \ \ \ \ \ \ \ \ \ M\mapsto M/D
\]
is invertible, i.e., is a bijection. Thus, we can substitute $M/D$ for $M$ in
the sum $\sum_{M\in\mathfrak{B}}\alpha\left(  M\right)  g_{P,MDE}$. We thus
obtain%
\[
\sum_{M\in\mathfrak{B}}\alpha\left(  M\right)  g_{P,MDE}=\underbrace{\sum
_{M\in\mathfrak{A}}}_{=\sum_{\substack{M\mid P;\\ME\mid P;\\D\mid M}}}%
\alpha\left(  \underbrace{M/D}_{=\dfrac{M}{D}}\right)
\underbrace{g_{P,\left(  M/D\right)  DE}}_{=g_{P,ME}}=\sum_{\substack{M\mid
P;\\ME\mid P;\\D\mid M}}\alpha\left(  \dfrac{M}{D}\right)  g_{P,ME}.
\]
Thus,%
\[
\sum_{\substack{M\mid P;\\ME\mid P;\\D\mid M}}\alpha\left(  \dfrac{M}%
{D}\right)  g_{P,ME}=\underbrace{\sum_{M\in\mathfrak{B}}}_{=\sum
_{\substack{M\mid P;\\MDE\mid P}}}\alpha\left(  M\right)  g_{P,MDE}%
=\sum_{\substack{M\mid P;\\MDE\mid P}}\alpha\left(  M\right)  g_{P,MDE}.
\]
This proves (\ref{pf.lem.F.gW.F-G-gen0.3}).}. Finally, every monic divisor $D$
of $P$ satisfies
\begin{equation}
DE\sum_{\substack{M\mid P;\\MDE\mid P}}\alpha\left(  M\right)  g_{P,MDE}\in PA
\label{pf.lem.F.gW.F-G-gen0.5}%
\end{equation}
\footnote{\textit{Proof of (\ref{pf.lem.F.gW.F-G-gen0.5}):} Let $D$ be a monic
divisor of $P$. We must prove (\ref{pf.lem.F.gW.F-G-gen0.5}).
\par
We are in one of the following two cases:
\par
\textit{Case 1:} We have $DE\mid P$.
\par
\textit{Case 2:} We have $DE\nmid P$.
\par
Let us consider Case 1 first. In this case, we have $DE\mid P$. Also, the
polynomial $DE$ is monic (since $D$ and $E$ are monic). Hence, $DE$ is a monic
divisor of $P$. Thus, (\ref{pf.lem.F.gW.F-G-gen0.1}) (applied to $F=DE$)
yields $DE\sum_{\substack{M\mid P;\\MDE\mid P}}\alpha\left(  M\right)
g_{P,MDE}\in PA$. Thus, (\ref{pf.lem.F.gW.F-G-gen0.5}) is proven in Case 1.
\par
Let us now consider Case 2. In this case, we have $DE\nmid P$. Thus, there
exists no $M\mid P$ satisfying $MDE\mid P$ (because if such an $M$ would
exist, then it would satisfy $DE\mid MDE\mid P$, which would contradict
$DE\nmid P$). Hence, the sum $\sum_{\substack{M\mid P;\\MDE\mid P}%
}\alpha\left(  M\right)  g_{P,MDE}$ is empty, and thus equals $0$. In other
words, $\sum_{\substack{M\mid P;\\MDE\mid P}}\alpha\left(  M\right)
g_{P,MDE}=0$. Now, $DE\underbrace{\sum_{\substack{M\mid P;\\MDE\mid P}%
}\alpha\left(  M\right)  g_{P,MDE}}_{=0}=0\in PA$. Thus,
(\ref{pf.lem.F.gW.F-G-gen0.5}) is proven in Case 2.
\par
We have now proven (\ref{pf.lem.F.gW.F-G-gen0.5}) in both Cases 1 and 2. Thus,
(\ref{pf.lem.F.gW.F-G-gen0.5}) always holds.}.

Now,%
\begin{align*}
&  \sum_{\substack{D\mid P;\\DE\mid P}}\beta\left(  D\right)  g_{P,DE}\\
&  =\sum_{\substack{M\mid P;\\ME\mid P}}\underbrace{\beta\left(  M\right)
}_{\substack{=\sum_{D\mid M}D\gamma\left(  D\right)  \alpha\left(  \dfrac
{M}{D}\right)  \\\text{(by (\ref{eq.lem.F.gW.F-gen0.beta-through-alpha})
(applied}\\\text{to }M\text{ instead of }P\text{))}}}g_{P,ME}%
\ \ \ \ \ \ \ \ \ \ \left(  \text{here, we have renamed the summation index
}D\text{ as }M\right) \\
&  =\sum_{\substack{M\mid P;\\ME\mid P}}\underbrace{\sum_{D\mid M}%
}_{\substack{=\sum_{\substack{D\mid P;\\D\mid M}}\\\text{(since every monic
divisor }D\text{ of }M\\\text{is also a monic divisor of }P\text{ (since
}M\mid P\text{))}}}D\gamma\left(  D\right)  \alpha\left(  \dfrac{M}{D}\right)
g_{P,ME}\\
&  =\underbrace{\sum_{\substack{M\mid P;\\ME\mid P}}\sum_{\substack{D\mid
P;\\D\mid M}}}_{=\sum_{D\mid P}\sum_{\substack{M\mid P;\\ME\mid P;\\D\mid M}%
}}D\gamma\left(  D\right)  \alpha\left(  \dfrac{M}{D}\right)  g_{P,ME}%
=\sum_{D\mid P}\sum_{\substack{M\mid P;\\ME\mid P;\\D\mid M}}D\gamma\left(
D\right)  \alpha\left(  \dfrac{M}{D}\right)  g_{P,ME}\\
&  =\sum_{D\mid P}D\gamma\left(  D\right)  \underbrace{\sum_{\substack{M\mid
P;\\ME\mid P;\\D\mid M}}\alpha\left(  \dfrac{M}{D}\right)  g_{P,ME}%
}_{\substack{=\sum_{\substack{M\mid P;\\MDE\mid P}}\alpha\left(  M\right)
g_{P,MDE}\\\text{(by (\ref{pf.lem.F.gW.F-G-gen0.3}))}}}=\sum_{D\mid P}%
D\gamma\left(  D\right)  \sum_{\substack{M\mid P;\\MDE\mid P}}\alpha\left(
M\right)  g_{P,MDE}.
\end{align*}
Multiplying both sides of this equality by $E$, we find%
\begin{align*}
&  E\sum_{\substack{D\mid P;\\DE\mid P}}\beta\left(  D\right)  g_{P,DE}\\
&  =E\sum_{D\mid P}D\gamma\left(  D\right)  \sum_{\substack{M\mid P;\\MDE\mid
P}}\alpha\left(  M\right)  g_{P,MDE}=\sum_{D\mid P}DE\gamma\left(  D\right)
\sum_{\substack{M\mid P;\\MDE\mid P}}\alpha\left(  M\right)  g_{P,MDE}\\
&  =\sum_{D\mid P}\gamma\left(  D\right)  \underbrace{DE\sum_{\substack{M\mid
P;\\MDE\mid P}}\alpha\left(  M\right)  g_{P,MDE}}_{\substack{\in PA\\\text{(by
(\ref{pf.lem.F.gW.F-G-gen0.5}))}}}\in\sum_{D\mid P}\gamma\left(  D\right)
PA\subseteq PA.
\end{align*}
This proves Lemma \ref{lem.F.gW.F-G-gen0}.
\end{proof}

\begin{lemma}
\label{lem.F.gW.F-G-gen}Let $N$ be a $q$-nest. Let $A$ be an $\mathbb{F}%
_{q}\left[  T\right]  $-module. For every $P\in N$ and every monic divisor $D$
of $P$, let $g_{P,D}$ be an element of $A$. Then, the following two assertions
are equivalent:

\textit{Assertion }$\mathcal{L}$\textit{:} Every $P\in N$ and every monic
divisor $E$ of $P$ satisfy%
\[
E\sum_{\substack{D\mid P;\\DE\mid P}}\mu\left(  D\right)  g_{P,DE}\in PA.
\]


\textit{Assertion }$\mathcal{M}$\textit{:} Every $P\in N$ and every monic
divisor $E$ of $P$ satisfy%
\[
E\sum_{\substack{D\mid P;\\DE\mid P}}\varphi_{C}\left(  D\right)  g_{P,DE}\in
PA.
\]

\end{lemma}

\begin{proof}
[Proof of Lemma \ref{lem.F.gW.F-G-gen}.]We shall consider $\varphi
_{C}:\mathbb{F}_{q}\left[  T\right]  _{+}\rightarrow\mathbb{F}_{q}\left[
T\right]  $ as a map $N\rightarrow\mathbb{F}_{q}\left[  T\right]  $ (by
restricting it to the subset $N$ of $\mathbb{F}_{q}\left[  T\right]  _{+}$).
We shall also consider $\mu:\mathbb{F}_{q}\left[  T\right]  _{+}%
\rightarrow\left\{  -1,0,1\right\}  $ as a map $N\rightarrow\mathbb{F}%
_{q}\left[  T\right]  $ (by restricting it to the subset $N$ of $\mathbb{F}%
_{q}\left[  T\right]  _{+}$, and by composing it with the canonical map
$\left\{  -1,0,1\right\}  \rightarrow\mathbb{Z}\rightarrow\mathbb{F}%
_{q}\left[  T\right]  $).

We shall prove the implications $\mathcal{L}\Longrightarrow\mathcal{M}$ and
$\mathcal{M}\Longrightarrow\mathcal{L}$ separately:

\textit{Proof of the implication }$\mathcal{L}\Longrightarrow\mathcal{M}%
$\textit{:} Assume that Assertion $\mathcal{L}$ holds. We must show that
Assertion $\mathcal{M}$ holds.

Define a map $\gamma:N\rightarrow\mathbb{F}_{q}\left[  T\right]  $ by $\left(
\gamma\left(  P\right)  =1\text{ for every }P\in N\right)  $.

For every $P\in N$, we have%
\begin{align*}
\varphi_{C}\left(  P\right)   &  =\sum_{D\mid P}\underbrace{D}_{=D1}\mu\left(
\dfrac{P}{D}\right)  \ \ \ \ \ \ \ \ \ \ \left(  \text{by Proposition
\ref{prop.phiC-Q.formula} \textbf{(c)}, applied to }M=P\right) \\
&  =\sum_{D\mid P}D\underbrace{1}_{\substack{=\gamma\left(  D\right)
\\\text{(since }\gamma\left(  D\right)  =1\\\text{(by the definition of
}\gamma\text{))}}}\mu\left(  \dfrac{P}{D}\right)  =\sum_{D\mid P}%
D\gamma\left(  D\right)  \mu\left(  \dfrac{P}{D}\right)  .
\end{align*}
Furthermore, every $P\in N$ and every monic divisor $E$ of $P$ satisfy%
\[
E\sum_{\substack{D\mid P;\\DE\mid P}}\mu\left(  D\right)  g_{P,DE}\in PA
\]
(because Assertion $\mathcal{L}$ holds). Thus, Lemma \ref{lem.F.gW.F-G-gen0}
(applied to $\alpha=\mu$ and $\beta=\varphi_{C}$) shows that every $P\in N$
and every monic divisor $E$ of $P$ satisfy%
\[
E\sum_{\substack{D\mid P;\\DE\mid P}}\varphi_{C}\left(  D\right)  g_{P,DE}\in
PA.
\]
In other words, Assertion $\mathcal{M}$ holds. Thus, we have proven the
implication $\mathcal{L}\Longrightarrow\mathcal{M}$.

\textit{Proof of the implication }$\mathcal{M}\Longrightarrow\mathcal{L}%
$\textit{:} Assume that Assertion $\mathcal{M}$ holds. We must show that
Assertion $\mathcal{L}$ holds.

For every $P\in N$, we have%
\[
\sum_{D\mid P}D\mu\left(  D\right)  \varphi_{C}\left(  \dfrac{P}{D}\right)
=\mu\left(  P\right)
\]
(by Proposition \ref{prop.phiC-Q.andmu}, applied to $M=P$) and thus%
\[
\mu\left(  P\right)  =\sum_{D\mid P}D\mu\left(  D\right)  \varphi_{C}\left(
\dfrac{P}{D}\right)  .
\]
Furthermore, every $P\in N$ and every monic divisor $E$ of $P$ satisfy%
\[
E\sum_{\substack{D\mid P;\\DE\mid P}}\varphi_{C}\left(  D\right)  g_{P,DE}\in
PA
\]
(because Assertion $\mathcal{M}$ holds). Thus, Lemma \ref{lem.F.gW.F-G-gen0}
(applied to $\alpha=\varphi_{C}$, $\beta=\mu$ and $\gamma=\mu$) shows that
every $P\in N$ and every monic divisor $E$ of $P$ satisfy%
\[
E\sum_{\substack{D\mid P;\\DE\mid P}}\mu\left(  D\right)  g_{P,DE}\in PA.
\]
In other words, Assertion $\mathcal{L}$ holds. Thus, we have proven the
implication $\mathcal{M}\Longrightarrow\mathcal{L}$.

We have now proven the two implications $\mathcal{L}\Longrightarrow
\mathcal{M}$ and $\mathcal{M}\Longrightarrow\mathcal{L}$. Combining them, we
obtain the equivalence $\mathcal{L}\Longleftrightarrow\mathcal{M}$. Thus,
Lemma \ref{lem.F.gW.F-G-gen} is proven.
\end{proof}

\begin{proof}
[Proof of Theorem \ref{thm.F.gW}.]Let us observe a few simple facts:

\begin{itemize}
\item If $D$ and $E$ are two monic polynomials in $\mathbb{F}_{q}\left[
T\right]  $ satisfying $DE\in N$, then%
\begin{equation}
\varphi_{D}\circ\varphi_{E}=\varphi_{DE} \label{pf.thm.F.gW.compositionDE}%
\end{equation}
\footnote{\textit{Proof of (\ref{pf.thm.F.gW.compositionDE}):} Let $D$ and $E$
be two monic polynomials in $\mathbb{F}_{q}\left[  T\right]  $ satisfying
$DE\in N$.
\par
The polynomial $D$ is a monic divisor of $DE$ (since $D$ is monic and $D\mid
DE$). Since $DE\in N$, this entails $D\in N$ (because $N$ is a $q$-nest).
Similarly, $E\in N$.
\par
But Assumption 3 shows that $\varphi_{P}\circ\varphi_{Q}=\varphi_{PQ}$ for
every $P\in N$ and every $Q\in N$ satisfying $PQ\in N$. Applying this to $P=D$
and $Q=E$, we obtain $\varphi_{D}\circ\varphi_{E}=\varphi_{DE}$. This proves
(\ref{pf.thm.F.gW.compositionDE}).}.

\item Every $P\in N$ and every monic divisor $D$ of $P$ satisfy
\begin{equation}
\varphi_{D}\circ\varphi_{P/D}=\varphi_{P} \label{pf.thm.F.gW.composition}%
\end{equation}
\footnote{\textit{Proof of (\ref{pf.thm.F.gW.composition}):} Let $P\in N$, and
let $D$ be a monic divisor of $P$. Then, $P/D\in\mathbb{F}_{q}\left[
T\right]  $ (since $D$ is a divisor of $P$). The polynomial $P/D$ is monic
(since $P$ and $D$ are monic). Also, $D\cdot\left(  P/D\right)  =P\in N$.
Hence, (\ref{pf.thm.F.gW.compositionDE}) (applied to $E=P/D$) yields
$\varphi_{D}\circ\varphi_{P/D}=\varphi_{D\cdot\left(  P/D\right)  }%
=\varphi_{P}$. This proves (\ref{pf.thm.F.gW.composition}).}.

\item Assumption 3 furthermore shows that $\varphi_{1}=\operatorname*{id}$.
\end{itemize}

Assumption 1 shows that, for every $P\in N$, the map $\varphi_{P}$ is an
endomorphism of the $\mathcal{F}$-module $A$. In other words, for every $P\in
N$,%
\begin{equation}
\text{the map }\varphi_{P}\text{ is }\mathcal{F}\text{-linear.}
\label{pf.thm.F.gW.Flin}%
\end{equation}


Notice that Assertion $\mathcal{C}_{1}$ of Theorem \ref{thm.F.gW} is identical
with Assertion $\mathcal{C}_{1}$ of Theorem \ref{thm.F.gW-general}.

Let us now prove the equivalences $\mathcal{C}_{1}\Longleftrightarrow
\mathcal{D}_{1}$, $\mathcal{C}_{1}\Longleftrightarrow\mathcal{D}_{2}$ and
$\mathcal{C}_{1}\Longleftrightarrow\mathcal{E}_{1}$. These three equivalences
will be derived from Theorem \ref{thm.F.gW-general}.

\textit{Proof of the equivalence }$\mathcal{C}_{1}\Longleftrightarrow
\mathcal{D}_{1}$\textit{:} For every $P\in N$, define an endomorphism
$\psi_{P}$ of the $\mathbb{F}_{q}$-vector space $A$ by%
\[
\left(  \psi_{P}\left(  a\right)  =\left(  \operatorname*{Carl}P\right)
a\ \ \ \ \ \ \ \ \ \ \text{for every }a\in A\right)  .
\]
The Assumptions 1, 2 and 3 of Theorem \ref{thm.F.gW-general} are satisfied
(because they are precisely the Assumptions 1, 2 and 3 of Theorem
\ref{thm.F.gW}). Hence, Proposition \ref{prop.F.gW-general.ex1} shows that
Assumptions 4 and 5 of Theorem \ref{thm.F.gW-general} are satisfied. Hence,
Theorem \ref{thm.F.gW-general} shows that the assertions $\mathcal{C}_{1}$ and
$\mathcal{E}_{\psi}$ of Theorem \ref{thm.F.gW-general} are equivalent. In
other words, $\mathcal{C}_{1}\Longleftrightarrow\mathcal{E}_{\psi}$.

But Assertion $\mathcal{D}_{1}$ can be rewritten as follows:

\begin{statement}
\textit{Assertion }$\mathcal{D}_{1}^{\prime}$\textit{:} There exists a family
$\left(  z_{P}\right)  _{P\in N}\in A^{N}$ of elements of $A$ such that%
\[
\left(  b_{P}=\sum_{D\mid P}D\cdot\left(  \operatorname*{Carl}\dfrac{P}%
{D}\right)  z_{D}\text{ for every }P\in N\right)  .
\]

\end{statement}

Assertion $\mathcal{D}_{1}^{\prime}$ is obtained from Assertion $\mathcal{D}%
_{1}$ by renaming the family $\left(  x_{P}\right)  _{P\in N}$ as $\left(
z_{P}\right)  _{P\in N}$. Hence, we have the equivalence $\mathcal{D}%
_{1}\Longleftrightarrow\mathcal{D}_{1}^{\prime}$.

But every $P\in N$ and every monic divisor $D$ of $P$ satisfy%
\begin{align*}
\psi_{P/D}\left(  z_{D}\right)   &  =\left(  \operatorname*{Carl}\left(
P/D\right)  \right)  z_{D}\ \ \ \ \ \ \ \ \ \ \left(  \text{by the definition
of }\psi_{P/D}\right) \\
&  =\left(  \operatorname*{Carl}\dfrac{P}{D}\right)  z_{D}.
\end{align*}
Thus, Assertion $\mathcal{E}_{\psi}$ of Theorem \ref{thm.F.gW-general} is
equivalent to our Assertion $\mathcal{D}_{1}^{\prime}$. In other words, we
have the equivalence $\mathcal{E}_{\psi}\Longleftrightarrow\mathcal{D}%
_{1}^{\prime}$. Thus, we have the chain of equivalences $\mathcal{D}%
_{1}\Longleftrightarrow\mathcal{D}_{1}^{\prime}\Longleftrightarrow
\mathcal{E}_{\psi}\Longleftrightarrow\mathcal{C}_{1}$. This proves the
equivalence $\mathcal{C}_{1}\Longleftrightarrow\mathcal{D}_{1}$.

\textit{Proof of the equivalence }$\mathcal{C}_{1}\Longleftrightarrow
\mathcal{D}_{2}$\textit{:} For every $P\in N$, define an endomorphism
$\psi_{P}$ of the $\mathbb{F}_{q}$-vector space $A$ by%
\[
\left(  \psi_{P}\left(  a\right)  =F^{\deg P}a\ \ \ \ \ \ \ \ \ \ \text{for
every }a\in A\right)  .
\]
The Assumptions 1, 2 and 3 of Theorem \ref{thm.F.gW-general} are satisfied
(because they are precisely the Assumptions 1, 2 and 3 of Theorem
\ref{thm.F.gW}). Hence, Proposition \ref{prop.F.gW-general.ex2} shows that
Assumptions 4 and 5 of Theorem \ref{thm.F.gW-general} are satisfied. Hence,
Theorem \ref{thm.F.gW-general} shows that the assertions $\mathcal{C}_{1}$ and
$\mathcal{E}_{\psi}$ of Theorem \ref{thm.F.gW-general} are equivalent. In
other words, $\mathcal{C}_{1}\Longleftrightarrow\mathcal{E}_{\psi}$.

But Assertion $\mathcal{D}_{2}$ can be rewritten as follows:

\begin{statement}
\textit{Assertion }$\mathcal{D}_{2}^{\prime}$\textit{:} There exists a family
$\left(  z_{P}\right)  _{P\in N}\in A^{N}$ of elements of $A$ such that%
\[
\left(  b_{P}=\sum_{D\mid P}DF^{\deg\left(  P/D\right)  }z_{D}\text{ for every
}P\in N\right)  .
\]

\end{statement}

Assertion $\mathcal{D}_{2}^{\prime}$ is obtained from Assertion $\mathcal{D}%
_{2}$ by renaming the family $\left(  x_{P}\right)  _{P\in N}$ as $\left(
z_{P}\right)  _{P\in N}$. Hence, we have the equivalence $\mathcal{D}%
_{2}\Longleftrightarrow\mathcal{D}_{2}^{\prime}$.

But every $P\in N$ and every monic divisor $D$ of $P$ satisfy%
\[
\psi_{P/D}\left(  z_{D}\right)  =F^{\deg\left(  P/D\right)  }z_{D}%
\ \ \ \ \ \ \ \ \ \ \left(  \text{by the definition of }\psi_{P/D}\right)  .
\]
Thus, Assertion $\mathcal{E}_{\psi}$ of Theorem \ref{thm.F.gW-general} is
equivalent to our Assertion $\mathcal{D}_{2}^{\prime}$. In other words, we
have the equivalence $\mathcal{E}_{\psi}\Longleftrightarrow\mathcal{D}%
_{2}^{\prime}$. Thus, we have the chain of equivalences $\mathcal{D}%
_{2}\Longleftrightarrow\mathcal{D}_{2}^{\prime}\Longleftrightarrow
\mathcal{E}_{\psi}\Longleftrightarrow\mathcal{C}_{1}$. This proves the
equivalence $\mathcal{C}_{1}\Longleftrightarrow\mathcal{D}_{2}$.

\textit{Proof of the equivalence }$\mathcal{C}_{1}\Longleftrightarrow
\mathcal{E}_{1}$\textit{:} For every $P\in N$, define an endomorphism
$\psi_{P}$ of the $\mathbb{F}_{q}$-vector space $A$ by $\psi_{P}=\varphi_{P}$.
The Assumptions 1, 2 and 3 of Theorem \ref{thm.F.gW-general} are satisfied
(because they are precisely the Assumptions 1, 2 and 3 of Theorem
\ref{thm.F.gW}). Hence, Proposition \ref{prop.F.gW-general.ex3} shows that
Assumptions 4 and 5 of Theorem \ref{thm.F.gW-general} are satisfied. Hence,
Theorem \ref{thm.F.gW-general} shows that the assertions $\mathcal{C}_{1}$ and
$\mathcal{E}_{\psi}$ of Theorem \ref{thm.F.gW-general} are equivalent. In
other words, $\mathcal{C}_{1}\Longleftrightarrow\mathcal{E}_{\psi}$.

But Assertion $\mathcal{E}_{1}$ can be rewritten as follows:

\begin{statement}
\textit{Assertion }$\mathcal{E}_{1}^{\prime}$\textit{:} There exists a family
$\left(  z_{P}\right)  _{P\in N}\in A^{N}$ of elements of $A$ such that%
\[
\left(  b_{P}=\sum_{D\mid P}D\varphi_{P/D}\left(  z_{D}\right)  \text{ for
every }P\in N\right)  .
\]

\end{statement}

Assertion $\mathcal{E}_{1}^{\prime}$ is obtained from Assertion $\mathcal{E}%
_{1}$ by renaming the family $\left(  y_{P}\right)  _{P\in N}$ as $\left(
z_{P}\right)  _{P\in N}$. Hence, we have the equivalence $\mathcal{E}%
_{1}\Longleftrightarrow\mathcal{E}_{1}^{\prime}$.

But every $P\in N$ and every monic divisor $D$ of $P$ satisfy $\psi
_{P/D}=\varphi_{P/D}$ (by the definition of $\psi_{P/D}$). Thus, Assertion
$\mathcal{E}_{\psi}$ of Theorem \ref{thm.F.gW-general} is equivalent to our
Assertion $\mathcal{E}_{1}^{\prime}$. In other words, we have the equivalence
$\mathcal{E}_{\psi}\Longleftrightarrow\mathcal{E}_{1}^{\prime}$. Thus, we have
the chain of equivalences $\mathcal{E}_{1}\Longleftrightarrow\mathcal{E}%
_{1}^{\prime}\Longleftrightarrow\mathcal{E}_{\psi}\Longleftrightarrow
\mathcal{C}_{1}$. This proves the equivalence $\mathcal{C}_{1}%
\Longleftrightarrow\mathcal{E}_{1}$.

Combining the equivalences $\mathcal{C}_{1}\Longleftrightarrow\mathcal{D}_{1}%
$, $\mathcal{C}_{1}\Longleftrightarrow\mathcal{D}_{2}$ and $\mathcal{C}%
_{1}\Longleftrightarrow\mathcal{E}_{1}$, we obtain the chain of equivalences
$\mathcal{C}_{1}\Longleftrightarrow\mathcal{D}_{1}\Longleftrightarrow
\mathcal{D}_{2}\Longleftrightarrow\mathcal{E}_{1}$. Let us now show some
further logical implications. We shall use the notations of Proposition
\ref{prop.moebius-Q.sum}.

\textit{Proof of the implication }$\mathcal{E}_{1}\Longrightarrow
\mathcal{F}_{1}$\textit{:} Assume that Assertion $\mathcal{E}_{1}$ holds. That
is, there exists a family $\left(  y_{P}\right)  _{P\in N}\in A^{N}$ of
elements of $A$ such that%
\begin{equation}
\left(  b_{P}=\sum_{D\mid P}D\varphi_{P/D}\left(  y_{D}\right)  \text{ for
every }P\in N\right)  . \label{pf.thm.F.gW.EtoF.ass}%
\end{equation}
Consider this family $\left(  y_{P}\right)  _{P\in N}$. We need to prove that
Assertion $\mathcal{F}_{1}$ holds, i.e., that every $P\in N$ satisfies%
\[
\sum_{D\mid P}\mu\left(  D\right)  \varphi_{D}\left(  b_{P/D}\right)  \in PA.
\]


Fix $P\in N$. Then, every monic divisor $D$ of $P$ satisfies%
\begin{equation}
b_{P/D}=\sum_{\substack{E\mid P;\\DE\mid P}}E\varphi_{\left(  P/E\right)
/D}\left(  y_{E}\right)  \label{pf.thm.F.gW.EtoF.1}%
\end{equation}
\footnote{\textit{Proof of (\ref{pf.thm.F.gW.EtoF.1}):} Let $B$ be a monic
divisor of $P$. Thus, $P/B\in\mathbb{F}_{q}\left[  T\right]  _{+}$. Moreover,
the polynomial $P/B$ is monic (since $P$ and $B$ are monic), and is a divisor
of $P$. Hence, $P/B\in N$ (since $N$ is a $q$-nest, and since $P\in N$). Thus,
(\ref{pf.thm.F.gW.EtoF.ass}) (applied to $P/B$ instead of $P$) yields%
\begin{align*}
b_{P/B}  &  =\underbrace{\sum_{D\mid P/B}}_{\substack{=\sum_{\substack{D\mid
P;\\BD\mid P}}\\\text{(since the monic divisors }D\text{ of }P/B\\\text{are
precisely the monic divisors }D\text{ of }P\\\text{satisfying }BD\mid
P\text{)}}}D\underbrace{\varphi_{\left(  P/B\right)  /D}}_{\substack{=\varphi
_{\left(  P/D\right)  /B}\\\text{(since }\left(  P/B\right)  /D=\left(
P/D\right)  /B\text{)}}}\left(  y_{D}\right)  =\sum_{\substack{D\mid
P;\\BD\mid P}}D\varphi_{\left(  P/D\right)  /B}\left(  y_{D}\right) \\
&  =\sum_{\substack{E\mid P;\\BE\mid P}}E\varphi_{\left(  P/E\right)
/B}\left(  y_{E}\right)  \ \ \ \ \ \ \ \ \ \ \left(
\begin{array}
[c]{c}%
\text{here, we have renamed the}\\
\text{summation index }D\text{ as }E
\end{array}
\right)  .
\end{align*}
\par
Now, forget that we fixed $B$. We thus have shown that every monic divisor $B$
of $P$ satisfies $b_{P/B}=\sum_{\substack{E\mid P;\\BE\mid P}}E\varphi
_{\left(  P/E\right)  /B}\left(  y_{E}\right)  $. Renaming $B$ as $D$ in this
result, we obtain the following: Every monic divisor $D$ of $P$ satisfies
$b_{P/D}=\sum_{\substack{E\mid P;\\DE\mid P}}E\varphi_{\left(  P/E\right)
/D}\left(  y_{E}\right)  $. This proves (\ref{pf.thm.F.gW.EtoF.1}).}.
Moreover, if $D$ and $E$ are two monic divisors of $P$ satisfying $DE\mid P$,
then%
\begin{equation}
\varphi_{D}\left(  \varphi_{\left(  P/E\right)  /D}\left(  y_{E}\right)
\right)  =\varphi_{P/E}\left(  y_{E}\right)  \label{pf.thm.F.gW.EtoF.3}%
\end{equation}
\footnote{\textit{Proof of (\ref{pf.thm.F.gW.EtoF.3}):} Let $D$ and $E$ be two
monic divisors of $P$ satisfying $DE\mid P$. We have $E\mid DE\mid P$. Thus,
$P/E\in\mathbb{F}_{q}\left[  T\right]  $. Moreover, the polynomial $P/E$ is
monic (since $P$ and $E$ are monic). Hence, $P/E$ is a monic divisor of $P\in
N$. Thus, $P/E\in N$ (since $N$ is a $q$-nest). Moreover, $D\mid P/E$ (since
$\dfrac{P/E}{D}=\dfrac{P}{DE}\in\mathbb{F}_{q}\left[  T\right]  $ (since
$DE\mid P$)). Hence, $D$ is a monic divisor of $P/E$. Thus,
(\ref{pf.thm.F.gW.composition}) (applied to $P/E$ instead of $P$) yields
$\varphi_{D}\circ\varphi_{\left(  P/E\right)  /D}=\varphi_{P/E}$.
\par
Now, $\varphi_{D}\left(  \varphi_{\left(  P/E\right)  /D}\left(  y_{E}\right)
\right)  =\underbrace{\left(  \varphi_{D}\circ\varphi_{\left(  P/E\right)
/D}\right)  }_{=\varphi_{P/E}}\left(  y_{E}\right)  =\varphi_{P/E}\left(
y_{E}\right)  $. This proves (\ref{pf.thm.F.gW.EtoF.3}).}.

Hence, every monic divisor $D$ of $P$ satisfies%
\begin{align}
\varphi_{D}\left(  \underbrace{b_{P/D}}_{\substack{=\sum_{\substack{E\mid
P;\\DE\mid P}}E\varphi_{\left(  P/E\right)  /D}\left(  y_{E}\right)
\\\text{(by (\ref{pf.thm.F.gW.EtoF.1}))}}}\right)   &  =\varphi_{D}\left(
\sum_{\substack{E\mid P;\\DE\mid P}}E\varphi_{\left(  P/E\right)  /D}\left(
y_{E}\right)  \right)  =\sum_{\substack{E\mid P;\\DE\mid P}%
}E\underbrace{\varphi_{D}\left(  \varphi_{\left(  P/E\right)  /D}\left(
y_{E}\right)  \right)  }_{\substack{=\varphi_{P/E}\left(  y_{E}\right)
\\\text{(by (\ref{pf.thm.F.gW.EtoF.3}))}}}\nonumber\\
&  \ \ \ \ \ \ \ \ \ \ \left(
\begin{array}
[c]{c}%
\text{since the map }\varphi_{D}\text{ is }\mathcal{F}\text{-linear}\\
\text{(by (\ref{pf.thm.F.gW.Flin}), applied to }D\text{ instead of }P\text{)}%
\end{array}
\right) \nonumber\\
&  =\sum_{\substack{E\mid P;\\DE\mid P}}E\varphi_{P/E}\left(  y_{E}\right)  .
\label{pf.thm.F.gW.EtoF.6}%
\end{align}
Hence,%
\begin{align*}
&  \sum_{D\mid P}\mu\left(  D\right)  \underbrace{\varphi_{D}\left(
b_{P/D}\right)  }_{\substack{=\sum_{\substack{E\mid P;\\DE\mid P}%
}E\varphi_{P/E}\left(  y_{E}\right)  \\\text{(by (\ref{pf.thm.F.gW.EtoF.6}))}%
}}\\
&  =\sum_{D\mid P}\mu\left(  D\right)  \sum_{\substack{E\mid P;\\DE\mid
P}}E\varphi_{P/E}\left(  y_{E}\right)  =\sum_{B\mid P}\mu\left(  B\right)
\sum_{\substack{E\mid P;\\BE\mid P}}E\varphi_{P/E}\left(  y_{E}\right) \\
&  \ \ \ \ \ \ \ \ \ \ \left(
\begin{array}
[c]{c}%
\text{here, we have renamed the summation}\\
\text{index }D\text{ as }B\text{ in the outer sum}%
\end{array}
\right) \\
&  =\underbrace{\sum_{B\mid P}\sum_{\substack{E\mid P;\\BE\mid P}}}%
_{=\sum_{E\mid P}\sum_{\substack{B\mid P;\\BE\mid P}}}\mu\left(  B\right)
E\varphi_{P/E}\left(  y_{E}\right)  =\sum_{E\mid P}\underbrace{\sum
_{\substack{B\mid P;\\BE\mid P}}\mu\left(  B\right)  }_{\substack{=\left[
E=P\right]  \\\text{(by Corollary \ref{cor.moebius-Q.sum-rel},}\\\text{applied
to }M=P\text{)}}}E\varphi_{P/E}\left(  y_{E}\right) \\
&  =\sum_{E\mid P}\left[  E=P\right]  E\varphi_{P/E}\left(  y_{E}\right) \\
&  =\underbrace{\left[  P=P\right]  }_{=1}P\underbrace{\varphi_{P/P}%
}_{\substack{=\operatorname*{id}\\\text{(by Assumption 1)}}}\left(
y_{P}\right)  +\sum_{\substack{E\mid P;\\E\neq P}}\underbrace{\left[
E=P\right]  }_{\substack{=0\\\text{(since }E\neq P\text{)}}}E\varphi
_{P/E}\left(  y_{E}\right) \\
&  \ \ \ \ \ \ \ \ \ \ \left(  \text{here, we have split off the addend for
}E=P\text{ from the sum}\right) \\
&  =P\underbrace{\operatorname*{id}\left(  y_{P}\right)  }_{=y_{P}%
}+\underbrace{\sum_{\substack{E\mid P;\\E\neq P}}0E\varphi_{P/E}\left(
y_{E}\right)  }_{=0}=P\underbrace{y_{P}}_{\in A}\in PA.
\end{align*}
Thus, Assertion $\mathcal{F}_{1}$ holds. We have thus proven the implication
$\mathcal{E}_{1}\Longrightarrow\mathcal{F}_{1}$.

\textit{Proof of the implication }$\mathcal{F}_{1}\Longrightarrow
\mathcal{E}_{1}$\textit{:} Assume that Assertion $\mathcal{F}_{1}$ holds. That
is, every $P\in N$ satisfies%
\begin{equation}
\sum_{D\mid P}\mu\left(  D\right)  \varphi_{D}\left(  b_{P/D}\right)  \in PA.
\label{pf.thm.F.gW.FtoE.ass}%
\end{equation}


Now we need to prove that Assertion $\mathcal{E}_{1}$ holds, i.e., that there
exists a family $\left(  y_{P}\right)  _{P\in N}\in A^{N}$ of elements of $A$
such that every $P\in N$ satisfies%
\begin{equation}
\left(  b_{P}=\sum_{D\mid P}D\varphi_{P/D}\left(  y_{D}\right)  \text{ for
every }P\in N\right)  . \label{pf.thm.F.gW.FtoE.goal}%
\end{equation}
We shall construct such a family $\left(  y_{P}\right)  _{P\in N}$
recursively, by induction over $\deg P$. That is, we fix some $Q\in N$, and we
assume that we already have constructed a $y_{P}$ for every $P\in N$
satisfying $\deg P<\deg Q$; we furthermore assume that these $y_{Q}$ satisfy%
\begin{equation}
b_{P}=\sum_{D\mid P}D\varphi_{P/D}\left(  y_{D}\right)
\label{pf.thm.F.gW.FtoE.indass}%
\end{equation}
for every $P\in N$ satisfying $\deg P<\deg Q$. We now need to construct a
$y_{Q}\in A$ such that (\ref{pf.thm.F.gW.FtoE.indass}) is satisfied for $P=Q$.
In other words, we need to construct a $y_{Q}\in A$ satisfying $b_{Q}%
=\sum_{D\mid Q}D\varphi_{Q/D}\left(  y_{D}\right)  $.

From (\ref{pf.thm.F.gW.FtoE.ass}) (applied to $P=Q$), we obtain $\sum_{D\mid
Q}\mu\left(  D\right)  \varphi_{D}\left(  b_{Q/D}\right)  \in QA$. Thus, there
exists a $t\in A$ such that $\sum_{D\mid Q}\mu\left(  D\right)  \varphi
_{D}\left(  b_{Q/D}\right)  =Qt$. Consider this $t$. Set $y_{Q}=t$.

For every monic divisor $E$ of $Q$ satisfying $E\neq1$, we have%
\begin{equation}
b_{Q/E}=\sum_{\substack{D\mid Q;\\DE\mid Q}}D\varphi_{\left(  Q/D\right)
/E}\left(  y_{D}\right)  \label{pf.thm.F.gW.FtoE.1}%
\end{equation}
\footnote{\textit{Proof of (\ref{pf.thm.F.gW.FtoE.1}):} Let $E$ be a monic
divisor of $Q$ satisfying $E\neq1$. We have $E\mid Q$ and thus $Q/E\in
\mathbb{F}_{q}\left[  T\right]  $. The polynomial $Q/E$ is monic (since $Q$
and $E$ are monic) and thus is a monic divisor of $Q\in N$. Hence, $Q/E\in N$
(since $N$ is a $Q$-nest). Also, $E$ is a monic polynomial satisfying $E\neq
1$; therefore, $\deg E>0$. Hence, $\deg\left(  Q/E\right)  =\deg
Q-\underbrace{\deg E}_{>0}<\deg Q$. Thus, we can apply
(\ref{pf.thm.F.gW.FtoE.indass}) to $P=Q/E$ (since we have assumed that
(\ref{pf.thm.F.gW.FtoE.indass}) holds for every $P\in N$ satisfying $\deg
P<\deg Q$). As a result, we obtain%
\[
b_{Q/E}=\underbrace{\sum_{D\mid Q/E}}_{\substack{=\sum_{\substack{D\mid
Q;\\DE\mid Q}}\\\text{(since the monic divisors }D\text{ of }Q/E\\\text{are
precisely the monic divisors }D\text{ of }Q\\\text{satisfying }DE\mid Q\text{
(since }E\mid Q\text{))}}}D\underbrace{\varphi_{\left(  Q/E\right)  /D}%
}_{\substack{=\varphi_{\left(  Q/D\right)  /E}\\\text{(since }\left(
Q/E\right)  /D=\left(  Q/D\right)  /E\text{)}}}\left(  y_{D}\right)
=\sum_{\substack{D\mid Q;\\DE\mid Q}}D\varphi_{\left(  Q/D\right)  /E}\left(
y_{D}\right)  .
\]
This proves (\ref{pf.thm.F.gW.FtoE.1}).}. If $D$ and $E$ are two monic
divisors of $Q$ satisfying $DE\mid Q$, then%
\begin{equation}
\varphi_{E}\left(  \varphi_{\left(  Q/D\right)  /E}\left(  y_{D}\right)
\right)  =\varphi_{Q/D}\left(  y_{D}\right)  \label{pf.thm.F.gW.FtoE.3}%
\end{equation}
\footnote{\textit{Proof of (\ref{pf.thm.F.gW.FtoE.3}):} Let $D$ and $E$ be two
monic divisors of $Q$ satisfying $DE\mid Q$. We have $D\mid Q$ and thus
$Q/D\in\mathbb{F}_{q}\left[  T\right]  $. The polynomial $Q/D$ is monic (since
$Q$ and $D$ are monic), and thus is a monic divisor of $Q\in N$. Hence,
$Q/D\in N$ (since $N$ is a $q$-nest). Moreover, $DE\mid Q$, and thus
$\dfrac{Q}{DE}\in\mathbb{F}_{q}\left[  T\right]  $. Hence, $\dfrac{Q/D}%
{E}=\dfrac{Q}{DE}\in\mathbb{F}_{q}\left[  T\right]  $. Thus, $E$ is a divisor
of $Q/D$ (since $Q/D\in\mathbb{F}_{q}\left[  T\right]  $). Hence,
(\ref{pf.thm.F.gW.composition}) (applied to $Q/D$ and $E$ instead of $P$ and
$D$) shows that
\[
\varphi_{E}\circ\varphi_{\left(  Q/D\right)  /E}=\varphi_{Q/D}.
\]
Now, $\varphi_{E}\left(  \varphi_{\left(  Q/D\right)  /E}\left(  y_{D}\right)
\right)  =\underbrace{\left(  \varphi_{E}\circ\varphi_{\left(  Q/D\right)
/E}\right)  }_{=\varphi_{Q/D}}\left(  y_{D}\right)  =\varphi_{Q/D}\left(
y_{D}\right)  $. This proves (\ref{pf.thm.F.gW.FtoE.3}).}. If $D$ is a monic
divisor of $Q$, then%
\begin{equation}
\sum_{\substack{E\mid Q;\\DE\mid Q;\\E\neq1}}\mu\left(  E\right)  =\left[
D=Q\right]  -1 \label{pf.thm.F.gW.FtoE.5}%
\end{equation}
\footnote{\textit{Proof of (\ref{pf.thm.F.gW.FtoE.5}):} Let $D$ be a monic
divisor of $Q$. We must prove (\ref{pf.thm.F.gW.FtoE.5}).
\par
The polynomial $1$ is a monic divisor of $Q$ satisfying $D\cdot1\mid Q$ (since
$D\cdot1=D\mid Q$). Hence, we can split off the addend for $E=1$ from the sum
$\sum_{\substack{E\mid Q;\\DE\mid Q}}\mu\left(  E\right)  $. As a result, we
obtain%
\[
\sum_{\substack{E\mid Q;\\DE\mid Q}}\mu\left(  E\right)  =\sum
_{\substack{E\mid Q;\\DE\mid Q;\\E\neq1}}\mu\left(  E\right)  +\underbrace{\mu
\left(  1\right)  }_{=1}=\sum_{\substack{E\mid Q;\\DE\mid Q;\\E\neq1}%
}\mu\left(  E\right)  +1.
\]
Comparing this with%
\begin{align*}
\sum_{\substack{E\mid Q;\\DE\mid Q}}\mu\left(  E\right)   &  =\underbrace{\sum
_{\substack{B\mid Q;\\DB\mid Q}}}_{=\sum_{\substack{B\mid Q;\\BD\mid
Q\\\text{(since }DB=BD\\\text{for every }B\mid Q\text{)}}}}\mu\left(
B\right)  \ \ \ \ \ \ \ \ \ \ \left(  \text{here, we have renamed the
summation index }E\text{ as }B\right) \\
&  =\sum_{\substack{B\mid Q;\\BD\mid Q}}\mu\left(  B\right)  =\left[
D=Q\right]  \ \ \ \ \ \ \ \ \ \ \left(
\begin{array}
[c]{c}%
\text{by Corollary \ref{cor.moebius-Q.sum-rel}, applied to }Q\text{ and }D\\
\text{instead of }M\text{ and }E
\end{array}
\right)  ,
\end{align*}
we obtain $\sum_{\substack{E\mid Q;\\DE\mid Q;\\E\neq1}}\mu\left(  E\right)
+1=\left[  D=Q\right]  $. In other words, $\sum_{\substack{E\mid Q;\\DE\mid
Q;\\E\neq1}}\mu\left(  E\right)  =\left[  D=Q\right]  -1$. This proves
(\ref{pf.thm.F.gW.FtoE.5}).}.

Now,%
\begin{align*}
Qt  &  =\sum_{D\mid Q}\mu\left(  D\right)  \varphi_{D}\left(  b_{Q/D}\right)
=\sum_{E\mid Q}\mu\left(  E\right)  \varphi_{E}\left(  b_{Q/E}\right) \\
&  \ \ \ \ \ \ \ \ \ \ \left(  \text{here, we have renamed the summation index
}D\text{ as }E\right) \\
&  =\underbrace{\mu\left(  1\right)  }_{=1}\underbrace{\varphi_{1}%
}_{=\operatorname*{id}}\left(  \underbrace{b_{Q/1}}_{=b_{Q}}\right)
+\sum_{\substack{E\mid Q;\\E\neq1}}\mu\left(  E\right)  \varphi_{E}\left(
\underbrace{b_{Q/E}}_{\substack{=\sum_{\substack{D\mid Q;\\DE\mid Q}%
}D\varphi_{\left(  Q/D\right)  /E}\left(  y_{D}\right)  \\\text{(by
(\ref{pf.thm.F.gW.FtoE.1}))}}}\right) \\
&  \ \ \ \ \ \ \ \ \ \ \left(  \text{here, we have split off the addend for
}E=1\text{ from the sum}\right) \\
&  =\underbrace{\operatorname*{id}\left(  b_{Q}\right)  }_{=b_{Q}}%
+\sum_{\substack{E\mid Q;\\E\neq1}}\mu\left(  E\right)  \underbrace{\varphi
_{E}\left(  \sum_{\substack{D\mid Q;\\DE\mid Q}}D\varphi_{\left(  Q/D\right)
/E}\left(  y_{D}\right)  \right)  }_{\substack{=\sum_{\substack{D\mid
Q;\\DE\mid Q}}D\varphi_{E}\left(  \varphi_{\left(  Q/D\right)  /E}\left(
y_{D}\right)  \right)  \\\text{(since the map }\varphi_{E}\text{ is
}\mathcal{F}\text{-linear}\\\text{(by (\ref{pf.thm.F.gW.Flin}), applied to
}E\text{ instead of }P\text{)}}}\\
&  =b_{Q}+\sum_{\substack{E\mid Q;\\E\neq1}}\mu\left(  E\right)
\sum_{\substack{D\mid Q;\\DE\mid Q}}D\underbrace{\varphi_{E}\left(
\varphi_{\left(  Q/D\right)  /E}\left(  y_{D}\right)  \right)  }%
_{\substack{=\varphi_{Q/D}\left(  y_{D}\right)  \\\text{(by
(\ref{pf.thm.F.gW.FtoE.3}))}}}=b_{Q}+\sum_{\substack{E\mid Q;\\E\neq1}%
}\mu\left(  E\right)  \sum_{\substack{D\mid Q;\\DE\mid Q}}D\varphi
_{Q/D}\left(  y_{D}\right)  .
\end{align*}
Subtracting $b_{Q}$ from both sides of this equality, we obtain%
\begin{align*}
Qt-b_{Q}  &  =\sum_{\substack{E\mid Q;\\E\neq1}}\mu\left(  E\right)
\sum_{\substack{D\mid Q;\\DE\mid Q}}D\varphi_{Q/D}\left(  y_{D}\right)
=\underbrace{\sum_{\substack{E\mid Q;\\E\neq1}}\sum_{\substack{D\mid
Q;\\DE\mid Q}}}_{=\sum_{D\mid Q}\sum_{\substack{E\mid Q;\\DE\mid Q;\\E\neq1}%
}}\mu\left(  E\right)  D\varphi_{Q/D}\left(  y_{D}\right) \\
&  =\sum_{D\mid Q}\underbrace{\sum_{\substack{E\mid Q;\\DE\mid Q;\\E\neq1}%
}\mu\left(  E\right)  }_{\substack{=\left[  D=Q\right]  -1\\\text{(by
(\ref{pf.thm.F.gW.FtoE.5}))}}}D\varphi_{Q/D}\left(  y_{D}\right)  =\sum_{D\mid
Q}\left(  \left[  D=Q\right]  -1\right)  D\varphi_{Q/D}\left(  y_{D}\right) \\
&  =\underbrace{\sum_{D\mid Q}\left[  D=Q\right]  D\varphi_{Q/D}\left(
y_{D}\right)  }_{\substack{=\left[  Q=Q\right]  Q\varphi_{Q/Q}\left(
y_{Q}\right)  +\sum_{\substack{D\mid Q;\\D\neq Q}}\left[  D=Q\right]
D\varphi_{Q/D}\left(  y_{D}\right)  \\\text{(here, we have split off the
addend for }D=Q\text{ from the sum)}}}-\sum_{D\mid Q}\underbrace{1D}%
_{=D}\varphi_{Q/D}\left(  y_{D}\right) \\
&  =\underbrace{\left[  Q=Q\right]  }_{=1}Q\underbrace{\varphi_{Q/Q}%
}_{=\varphi_{1}=\operatorname*{id}}\left(  y_{Q}\right)  +\sum
_{\substack{D\mid Q;\\D\neq Q}}\underbrace{\left[  D=Q\right]  }%
_{\substack{=0\\\text{(since }D\neq Q\text{)}}}D\varphi_{Q/D}\left(
y_{D}\right)  -\sum_{D\mid Q}D\varphi_{Q/D}\left(  y_{D}\right) \\
&  =Q\underbrace{\operatorname*{id}\left(  y_{Q}\right)  }_{=y_{Q}%
=t}+\underbrace{\sum_{\substack{D\mid Q;\\D\neq Q}}0D\varphi_{Q/D}\left(
y_{D}\right)  }_{=0}-\sum_{D\mid Q}D\varphi_{Q/D}\left(  y_{D}\right) \\
&  =Qt-\sum_{D\mid Q}D\varphi_{Q/D}\left(  y_{D}\right)  .
\end{align*}
Subtracting $Qt$ from both sides of this equality, we obtain%
\[
-b_{Q}=-\sum_{D\mid Q}D\varphi_{Q/D}\left(  y_{D}\right)  .
\]
In other words, $b_{Q}=\sum_{D\mid Q}D\varphi_{Q/D}\left(  y_{D}\right)  $. In
other words, (\ref{pf.thm.F.gW.FtoE.indass}) is satisfied for $P=Q$.

Thus, we have constructed a $y_{Q}\in A$ such that
(\ref{pf.thm.F.gW.FtoE.indass}) is satisfied for $P=Q$. This completes a step
of our recursive construction of the family $\left(  y_{P}\right)  _{P\in N}$.
This family therefore exists. In other words, Assertion $\mathcal{E}_{1}$
holds. Thus, the implication $\mathcal{F}_{1}\Longrightarrow\mathcal{E}_{1}$
is proven.

We have now proven the two implications $\mathcal{E}_{1}\Longrightarrow
\mathcal{F}_{1}$ and $\mathcal{F}_{1}\Longrightarrow\mathcal{E}_{1}$.
Combining them, we obtain the equivalence $\mathcal{E}_{1}\Longleftrightarrow
\mathcal{F}_{1}$.

Let us define one more notation: For every $P\in N$ and every monic divisor
$D$ of $P$, we define an element $g_{P,D}$ of $A$ by $g_{P,D}=\varphi
_{D}\left(  b_{P/D}\right)  $. (This is well-defined\footnote{\textit{Proof.}
Let $P\in N$, and let $D$ be a monic divisor of $P$. Since $D$ is a monic
divisor of $P\in N$, we have $D\in N$ (since $N$ is a $q$-nest). Hence,
$\varphi_{D}$ is well-defined. Also, $P/D\in\mathbb{F}_{q}\left[  T\right]  $
(since $D\mid P$). The polynomial $P/D$ is monic (since $P$ and $D$ are
monic), and thus is a monic divisor of $P\in N$. Hence, $P/D\in N$ (since $N$
is a $q$-nest). Thus, $b_{P/D}$ is well-defined. Therefore, $\varphi
_{D}\left(  b_{P/D}\right)  $ is well-defined (since $\varphi_{D}$ is
well-defined). Qed.}.)

Next, let us introduce two more assertions:

\begin{statement}
\textit{Assertion }$\mathcal{L}$\textit{:} Every $P\in N$ and every monic
divisor $E$ of $P$ satisfy%
\[
E\sum_{\substack{D\mid P;\\DE\mid P}}\mu\left(  D\right)  g_{P,DE}\in PA.
\]

\end{statement}

\begin{statement}
\textit{Assertion }$\mathcal{M}$\textit{:} Every $P\in N$ and every monic
divisor $E$ of $P$ satisfy%
\[
E\sum_{\substack{D\mid P;\\DE\mid P}}\varphi_{C}\left(  D\right)  g_{P,DE}\in
PA.
\]

\end{statement}

Lemma \ref{lem.F.gW.F-G-gen} shows that these two Assertions $\mathcal{L}$ and
$\mathcal{M}$ are equivalent. In other words, we have the equivalence
$\mathcal{L}\Longleftrightarrow\mathcal{M}$.

We shall now prove the implications $\mathcal{F}_{1}\Longrightarrow
\mathcal{L}$, $\mathcal{L}\Longrightarrow\mathcal{F}_{1}$, $\mathcal{G}%
_{1}\Longrightarrow\mathcal{M}$ and $\mathcal{M}\Longrightarrow\mathcal{G}%
_{1}$:

\textit{Proof of the implication }$\mathcal{F}_{1}\Longrightarrow\mathcal{L}%
$\textit{:} Assume that Assertion $\mathcal{F}_{1}$ holds. That is, every
$P\in N$ satisfies%
\begin{equation}
\sum_{D\mid P}\mu\left(  D\right)  \varphi_{D}\left(  b_{P/D}\right)  \in PA.
\label{pf.thm.F.gW.FtoL.ass}%
\end{equation}


Now we need to prove that Assertion $\mathcal{L}$ holds, i.e., that every
$P\in N$ and every monic divisor $E$ of $P$ satisfy%
\begin{equation}
E\sum_{\substack{D\mid P;\\DE\mid P}}\mu\left(  D\right)  g_{P,DE}\in PA.
\label{pf.thm.F.gW.FtoL.goal}%
\end{equation}


Let $P\in N$. Let $E$ be a monic divisor of $P$. Thus, $E\mid P$, so that
$P/E\in\mathbb{F}_{q}\left[  T\right]  $. Moreover, the polynomial $P/E$ is
monic (since $P$ and $E$ are monic). Hence, $P/E$ is a monic divisor of $P\in
N$. Thus, $P/E\in N$ (since $N$ is a $q$-nest). Hence,
(\ref{pf.thm.F.gW.FtoL.ass}) (applied to $P/E$ instead of $P$) yields%
\begin{equation}
\sum_{D\mid P/E}\mu\left(  D\right)  \varphi_{D}\left(  b_{\left(  P/E\right)
/D}\right)  \in\left(  P/E\right)  A. \label{pf.thm.F.gW.FtoL.2}%
\end{equation}


But the map $\varphi_{E}$ is $\mathcal{F}$-linear (by (\ref{pf.thm.F.gW.Flin}%
), applied to $E$ instead of $P$). Furthermore, we have%
\begin{equation}
\varphi_{E}\circ\varphi_{D}=\varphi_{DE}\ \ \ \ \ \ \ \ \ \ \text{for every
monic divisor }D\text{ of }P/E \label{pf.thm.F.gW.FtoL.4}%
\end{equation}
\footnote{\textit{Proof of (\ref{pf.thm.F.gW.FtoL.4}):} Let $D$ be a monic
divisor of $P/E$.
\par
We have $D\mid P/E$, thus $\dfrac{P/E}{D}\in\mathbb{F}_{q}\left[  T\right]  $.
Also, the polynomial $DE$ is monic (since $D$ and $E$ are monic) and divides
$P$ (since $\dfrac{P}{DE}=\dfrac{P/E}{D}\in\mathbb{F}_{q}\left[  T\right]  $).
Thus, $DE$ is a monic divisor of $P\in N$. Hence, $DE\in N$ (since $N$ is a
$q$-nest). Thus, (\ref{pf.thm.F.gW.compositionDE}) (applied to $E$ and $D$
instead of $D$ and $E$) shows that $\varphi_{E}\circ\varphi_{D}=\varphi
_{ED}=\varphi_{DE}$. This proves (\ref{pf.thm.F.gW.FtoL.4}).}.

Applying the map $\varphi_{E}$ to both sides of the relation
(\ref{pf.thm.F.gW.FtoL.2}), we obtain%
\[
\varphi_{E}\left(  \sum_{D\mid P/E}\mu\left(  D\right)  \varphi_{D}\left(
b_{\left(  P/E\right)  /D}\right)  \right)  \in\varphi_{E}\left(  \left(
P/E\right)  A\right)  \subseteq\left(  P/E\right)  \varphi_{E}\left(
A\right)
\]
(since the map $\varphi_{E}$ is $\mathcal{F}$-linear). In view of%
\begin{align*}
&  \varphi_{E}\left(  \sum_{D\mid P/E}\mu\left(  D\right)  \varphi_{D}\left(
b_{\left(  P/E\right)  /D}\right)  \right) \\
&  =\sum_{D\mid P/E}\mu\left(  D\right)  \underbrace{\varphi_{E}\left(
\varphi_{D}\left(  b_{\left(  P/E\right)  /D}\right)  \right)  }_{=\left(
\varphi_{E}\circ\varphi_{D}\right)  \left(  b_{\left(  P/E\right)  /D}\right)
}\ \ \ \ \ \ \ \ \ \ \left(  \text{since the map }\varphi_{E}\text{ is
}\mathcal{F}\text{-linear}\right) \\
&  =\sum_{D\mid P/E}\mu\left(  D\right)  \underbrace{\left(  \varphi_{E}%
\circ\varphi_{D}\right)  }_{\substack{=\varphi_{DE}\\\text{(by
(\ref{pf.thm.F.gW.FtoL.4}))}}}\left(  \underbrace{b_{\left(  P/E\right)  /D}%
}_{\substack{=b_{P/\left(  DE\right)  }\\\text{(since }\left(  P/E\right)
/D=P/\left(  DE\right)  \text{)}}}\right) \\
&  =\underbrace{\sum_{D\mid P/E}}_{\substack{=\sum_{\substack{D\mid P;\\DE\mid
P}}\\\text{(since the monic divisors }D\text{ of }P/E\\\text{are exactly the
monic divisors }D\text{ of }P\\\text{satisfying }DE\mid P\text{ (since }E\mid
P\text{))}}}\mu\left(  D\right)  \underbrace{\varphi_{DE}\left(  b_{P/\left(
DE\right)  }\right)  }_{\substack{=g_{P,DE}\\\text{(since }g_{P,DE}%
=\varphi_{DE}\left(  b_{P/\left(  DE\right)  }\right)  \\\text{(by the
definition of }g_{P,DE}\text{))}}}\\
&  =\sum_{\substack{D\mid P;\\DE\mid P}}\mu\left(  D\right)  g_{P,DE},
\end{align*}
this rewrites as $\sum_{\substack{D\mid P;\\DE\mid P}}\mu\left(  D\right)
g_{P,DE}\in\left(  P/E\right)  \varphi_{E}\left(  A\right)  $. Hence,%
\[
E\underbrace{\sum_{\substack{D\mid P;\\DE\mid P}}\mu\left(  D\right)
g_{P,DE}}_{\in\left(  P/E\right)  \varphi_{E}\left(  A\right)  }%
\in\underbrace{E\left(  P/E\right)  }_{=P}\underbrace{\varphi_{E}\left(
A\right)  }_{\subseteq A}\subseteq PA.
\]
In other words, (\ref{pf.thm.F.gW.FtoL.goal}) holds. Thus, Assertion
$\mathcal{L}$ holds. We have thus proven the implication $\mathcal{F}%
_{1}\Longrightarrow\mathcal{L}$.

\textit{Proof of the implication }$\mathcal{F}_{1}\Longrightarrow\mathcal{L}%
$\textit{:} Assume that Assertion $\mathcal{L}$ holds. That is, every $P\in N$
and every monic divisor $E$ of $P$ satisfy%
\begin{equation}
E\sum_{\substack{D\mid P;\\DE\mid P}}\mu\left(  D\right)  g_{P,DE}\in PA.
\label{pf.thm.F.gW.LtoF.ass}%
\end{equation}


Now we need to prove that Assertion $\mathcal{F}_{1}$ holds, i.e., that every
$P\in N$ satisfies%
\begin{equation}
\sum_{D\mid P}\mu\left(  D\right)  \varphi_{D}\left(  b_{P/D}\right)  \in PA.
\label{pf.thm.F.gW.LtoF.goal}%
\end{equation}


Let $P\in N$. Then, $1$ is a monic divisor of $P$. Hence,
(\ref{pf.thm.F.gW.LtoF.ass}) (applied to $E=1$) yields%
\[
1\sum_{\substack{D\mid P;\\D\cdot1\mid P}}\mu\left(  D\right)  g_{P,D\cdot
1}\in PA.
\]
In view of%
\begin{align*}
&  1\underbrace{\sum_{\substack{D\mid P;\\D\cdot1\mid P}}}_{=\sum
_{\substack{D\mid P;\\D\mid P}}=\sum_{D\mid P}}\mu\left(  D\right)
\underbrace{g_{P,D\cdot1}}_{\substack{=g_{P,D}=\varphi_{D}\left(
b_{P/D}\right)  \\\text{(by the definition of }g_{P,D}\text{)}}}\\
&  =1\sum_{D\mid P}\mu\left(  D\right)  \varphi_{D}\left(  b_{P/D}\right)
=\sum_{D\mid P}\mu\left(  D\right)  \varphi_{D}\left(  b_{P/D}\right)  ,
\end{align*}
this rewrites as $\sum_{D\mid P}\mu\left(  D\right)  \varphi_{D}\left(
b_{P/D}\right)  \in PA$. In other words, (\ref{pf.thm.F.gW.LtoF.goal}) holds.
Thus, Assertion $\mathcal{F}_{1}$ holds. We have thus proven the implication
$\mathcal{L}\Longrightarrow\mathcal{F}_{1}$.

\textit{Proof of the implication }$\mathcal{G}_{1}\Longrightarrow\mathcal{M}%
$\textit{:} The implication $\mathcal{G}_{1}\Longrightarrow\mathcal{M}$ can be
proven in exactly the same way as the implication $\mathcal{F}_{1}%
\Longrightarrow\mathcal{L}$ (except that every appearance of \textquotedblleft%
$\mu$\textquotedblright\ must be replaced by \textquotedblleft$\varphi_{C}%
$\textquotedblright).

\textit{Proof of the implication }$\mathcal{M}\Longrightarrow\mathcal{G}_{1}%
$\textit{:} The implication $\mathcal{M}\Longrightarrow\mathcal{G}_{1}$ can be
proven in exactly the same way as the implication $\mathcal{L}\Longrightarrow
\mathcal{F}_{1}$ (except that every appearance of \textquotedblleft$\mu
$\textquotedblright\ must be replaced by \textquotedblleft$\varphi_{C}%
$\textquotedblright).

We now have proven the four implications $\mathcal{F}_{1}\Longrightarrow
\mathcal{L}$, $\mathcal{L}\Longrightarrow\mathcal{F}_{1}$, $\mathcal{G}%
_{1}\Longrightarrow\mathcal{M}$ and $\mathcal{M}\Longrightarrow\mathcal{G}%
_{1}$. Combining them, we obtain the two equivalences $\mathcal{F}%
_{1}\Longleftrightarrow\mathcal{L}$ and $\mathcal{G}_{1}\Longleftrightarrow
\mathcal{M}$.

Finally, let us prove the equivalence $\mathcal{F}_{1}\Longleftrightarrow
\mathcal{G}_{2}$:

\textit{Proof of the equivalence }$\mathcal{F}_{1}\Longleftrightarrow
\mathcal{G}_{2}$\textit{:} For every $P\in N$ and $D\in\mathbb{F}_{q}\left[
T\right]  _{+}$, we have
\[
\underbrace{\varphi\left(  D\right)  }_{\substack{=\mu\left(  D\right)  \text{
in }\mathbb{F}_{q}\\\text{(by Proposition \ref{prop.phi-Q.formula}
\textbf{(d)},}\\\text{applied to }M=D\text{)}}}\varphi_{D}\left(
b_{P/D}\right)  =\mu\left(  D\right)  \varphi_{D}\left(  b_{P/D}\right)  .
\]
Therefore, Assertion $\mathcal{G}_{2}$ is equivalent to $\mathcal{F}_{1}$. In
other words, we obtain the equivalence $\mathcal{F}_{1}\Longleftrightarrow
\mathcal{G}_{2}$.

We now have obtained the following equivalences:%
\begin{align*}
&  \mathcal{C}_{1}\Longleftrightarrow\mathcal{D}_{1}\Longleftrightarrow
\mathcal{D}_{2}\Longleftrightarrow\mathcal{E}_{1}%
,\ \ \ \ \ \ \ \ \ \ \mathcal{E}_{1}\Longleftrightarrow\mathcal{F}%
_{1},\ \ \ \ \ \ \ \ \ \ \mathcal{L}\Longleftrightarrow\mathcal{M},\\
&  \mathcal{F}_{1}\Longleftrightarrow\mathcal{L}%
,\ \ \ \ \ \ \ \ \ \ \mathcal{G}_{1}\Longleftrightarrow\mathcal{M}%
,\ \ \ \ \ \ \ \ \ \ \mathcal{F}_{1}\Longleftrightarrow\mathcal{G}_{2}.
\end{align*}
Combining them all, we obtain the chain of equivalences%
\[
\mathcal{C}_{1}\Longleftrightarrow\mathcal{D}_{1}\Longleftrightarrow
\mathcal{D}_{2}\Longleftrightarrow\mathcal{E}_{1}\Longleftrightarrow
\mathcal{F}_{1}\Longleftrightarrow\mathcal{L}\Longleftrightarrow
\mathcal{M}\Longleftrightarrow\mathcal{G}_{1}\Longleftrightarrow
\mathcal{G}_{2}.
\]
In particular, the assertions $\mathcal{C}_{1}$, $\mathcal{D}_{1}$,
$\mathcal{D}_{2}$, $\mathcal{E}_{1}$, $\mathcal{F}_{1}$, $\mathcal{G}_{1}$,
and $\mathcal{G}_{2}$ are equivalent. This proves Theorem \ref{thm.F.gW}.
\end{proof}

\subsection{Examples: \textquotedblleft Necklace congruences\textquotedblright%
\ for $\mathbb{F}_{q}\left[  T\right]  $}

Theorem \ref{thm.F.gW} shows the equivalence of several assertions, but we
have yet to see a situation in which these assertions hold. Let us now explore
a few such situations. We begin with the simplest ones:

\begin{proposition}
\label{prop.F.gW.example1}Let $N$ be the $q$-nest $\mathbb{F}_{q}\left[
T\right]  _{+}$. Let $A=\mathbb{F}_{q}\left[  T\right]  $. Notice that $A$ is
a commutative $\mathbb{F}_{q}\left[  T\right]  $-algebra, and thus an
$\mathcal{F}$-module (according to Convention \ref{conv.F.acts-on-commalg}).

For every $P\in N$, define an endomorphism $\varphi_{P}$ of the $\mathbb{F}%
_{q}$-vector space $A$ by $\varphi_{P}=\operatorname*{id}$.

Fix a polynomial $Q\in\mathbb{F}_{q}\left[  T\right]  $.

\textbf{(a)} The three Assumptions 1, 2 and 3 of Theorem \ref{thm.F.gW} are satisfied.

\textbf{(b)} The assertions $\mathcal{C}_{1}$, $\mathcal{D}_{1}$,
$\mathcal{D}_{2}$, $\mathcal{E}_{1}$, $\mathcal{F}_{1}$, $\mathcal{G}_{1}$,
and $\mathcal{G}_{2}$ of Theorem \ref{thm.F.gW} are satisfied for the family
$\left(  b_{P}\right)  _{P\in N}=\left(  F^{\deg P}Q\right)  _{P\in N}\in
A^{N}$.

\textbf{(c)} The assertions $\mathcal{C}_{1}$, $\mathcal{D}_{1}$,
$\mathcal{D}_{2}$, $\mathcal{E}_{1}$, $\mathcal{F}_{1}$, $\mathcal{G}_{1}$,
and $\mathcal{G}_{2}$ of Theorem \ref{thm.F.gW} are satisfied for the family
$\left(  b_{P}\right)  _{P\in N}=\left(  \left(  \operatorname*{Carl}P\right)
Q\right)  _{P\in N}\in A^{N}$.

\textbf{(d)} The assertions $\mathcal{C}_{1}$, $\mathcal{D}_{1}$,
$\mathcal{D}_{2}$, $\mathcal{E}_{1}$, $\mathcal{F}_{1}$, $\mathcal{G}_{1}$,
and $\mathcal{G}_{2}$ of Theorem \ref{thm.F.gW} are satisfied for the family
$\left(  b_{P}\right)  _{P\in N}=\left(  Q\right)  _{P\in N}\in A^{N}$.
\end{proposition}

Before we prove this proposition, let us get two simple lemmas out of our way:

\begin{lemma}
\label{lem.F.gW.example1.lem.dumbed-down}Let $\pi$ be a monic irreducible
polynomial in $\mathbb{F}_{q}\left[  T\right]  $. Set $d=\deg\pi$. Let
$P\in\mathbb{F}_{q}\left[  T\right]  $. Then, $P^{q^{d}}\equiv
P\operatorname{mod}\pi\mathbb{F}_{q}\left[  T\right]  $.
\end{lemma}

\begin{proof}
[Proof of Lemma \ref{lem.F.gW.example1.lem.dumbed-down}.]Let $\mathbb{F}_{\pi
}$ denote the field $\mathbb{F}_{q}\left[  T\right]  /\pi\mathbb{F}_{q}\left[
T\right]  $. This is a field extension of $\mathbb{F}_{q}$. Furthermore, it is
well-known that $\mathbb{F}_{\pi}=\mathbb{F}_{q}\left[  T\right]
/\pi\mathbb{F}_{q}\left[  T\right]  $ is an $\mathbb{F}_{q}$-vector space of
dimension $\deg\pi=d$. Hence, $\left\vert \mathbb{F}_{\pi}\right\vert
=\left\vert \mathbb{F}_{q}\right\vert ^{d}=q^{d}$ (since $\left\vert
\mathbb{F}_{q}\right\vert =q$). In particular, $\mathbb{F}_{\pi}$ is a finite field.

If $Q$ is any element of $\mathbb{F}_{q}\left[  T\right]  $, then we let
$\overline{Q}$ denote the residue class of $Q\in\mathbb{F}_{q}\left[
T\right]  $ modulo the ideal $\pi\mathbb{F}_{q}\left[  T\right]  $. This
residue class $\overline{Q}$ lies in $\mathbb{F}_{q}\left[  T\right]
/\pi\mathbb{F}_{q}\left[  T\right]  =\mathbb{F}_{\pi}$. Applying this to
$Q=P$, we conclude that $\overline{P}$ lies in $\mathbb{F}_{\pi}$. In other
words, $\overline{P}\in\mathbb{F}_{\pi}$.

But another known fact says that if $L$ is a finite field, then every $a\in L$
satisfies $a^{\left\vert L\right\vert }=a$. Applying this to $L=\mathbb{F}%
_{\pi}$ and $a=\overline{P}$, we obtain $\overline{P}^{\left\vert
\mathbb{F}_{\pi}\right\vert }=\overline{P}$. Since $q^{d}=\left\vert
\mathbb{F}_{\pi}\right\vert $, we have $\overline{P^{q^{d}}}=\overline
{P^{\left\vert \mathbb{F}_{\pi}\right\vert }}=\overline{P}^{\left\vert
\mathbb{F}_{\pi}\right\vert }=\overline{P}$. In other words, $P^{q^{d}}\equiv
P\operatorname{mod}\pi\mathbb{F}_{q}\left[  T\right]  $ (because if $Q$ is any
element of $\mathbb{F}_{q}\left[  T\right]  $, then $\overline{Q}$ denotes the
residue class of $Q\in\mathbb{F}_{q}\left[  T\right]  $ modulo the ideal
$\pi\mathbb{F}_{q}\left[  T\right]  $). This proves Lemma
\ref{lem.F.gW.example1.lem.dumbed-down}.
\end{proof}

\begin{lemma}
\label{lem.F.gW.example1.lem}Let $A=\mathbb{F}_{q}\left[  T\right]  $. Notice
that $A$ is a commutative $\mathbb{F}_{q}\left[  T\right]  $-algebra, and thus
an $\mathcal{F}$-module (according to Convention \ref{conv.F.acts-on-commalg}%
). Let $\pi$ be a monic irreducible polynomial in $\mathbb{F}_{q}\left[
T\right]  $. Let $P\in A$.

\textbf{(a)} We have $\left(  \operatorname*{Carl}\pi\right)  P\equiv
P\operatorname{mod}\pi A$. Here, $\left(  \operatorname*{Carl}\pi\right)  P$
denotes the image of $P$ under the action of $\operatorname*{Carl}\pi
\in\mathcal{F}$ on the $\mathcal{F}$-module $A$.

\textbf{(b)} We have $F^{\deg\pi}P\equiv P\operatorname{mod}\pi A$.
\end{lemma}

\begin{proof}
[Proof of Lemma \ref{lem.F.gW.example1.lem}.]\textbf{(b)} Set $d=\deg\pi$.
Observe that $P\in A=\mathbb{F}_{q}\left[  T\right]  $. Thus, Lemma
\ref{lem.F.gW.example1.lem.dumbed-down} yields $P^{q^{d}}\equiv
P\operatorname{mod}\pi\mathbb{F}_{q}\left[  T\right]  $. In other words,
$P^{q^{d}}\equiv P\operatorname{mod}\pi A$ (because $\mathbb{F}_{q}\left[
T\right]  =A$).

Now, (\ref{eq.prop.F.acts-on-commalg.Fk.eq}) (applied to $k=d$ and $m=P$)
yields $F^{d}\cdot P=P^{q^{d}}\equiv P\operatorname{mod}\pi A$. Since
$d=\deg\pi$, this rewrites as $F^{\deg\pi}\cdot P\equiv P\operatorname{mod}\pi
A$. In other words, $F^{\deg\pi}P\equiv P\operatorname{mod}\pi A$. This proves
Lemma \ref{lem.F.gW.example1.lem} \textbf{(b)}.

\textbf{(a)} Corollary \ref{cor.F.u(pi).mod} (applied to $a=P$) yields
$\left(  \operatorname*{Carl}\pi\right)  P\equiv F^{\deg\pi}P\equiv
P\operatorname{mod}\pi A$ (by Lemma \ref{lem.F.gW.example1.lem} \textbf{(b)}).
Lemma \ref{lem.F.gW.example1.lem} \textbf{(a)} is thus proven.
\end{proof}

\begin{proof}
[Proof of Proposition \ref{prop.F.gW.example1}.]\textbf{(a)} Assumptions 1 and
3 of Theorem \ref{thm.F.gW} are clearly satisfied (since $\varphi
_{P}=\operatorname*{id}$ for each $P\in N$). It thus remains to prove that
Assumption 2 of Theorem \ref{thm.F.gW} is satisfied.

\textit{Proof of Assumption 2 of Theorem \ref{thm.F.gW}:} Let $a\in A$. Let
$\pi\in N$ be monic irreducible. We must prove that $\varphi_{\pi}\left(
a\right)  \equiv\left(  \operatorname*{Carl}\pi\right)  a\operatorname{mod}\pi
A$. Here, $\left(  \operatorname*{Carl}\pi\right)  a$ denotes the image of $a$
under the action of $\operatorname*{Carl}\pi\in\mathcal{F}$ on the
$\mathcal{F}$-module $A$.

Proposition \ref{prop.F.u(pi)} shows that there exists a unique $u\left(
\pi\right)  \in\mathcal{F}$ such that $\operatorname*{Carl}\pi=F^{\deg\pi}%
+\pi\cdot u\left(  \pi\right)  $. Consider this $u\left(  \pi\right)  $. We
have%
\[
\underbrace{\left(  \operatorname*{Carl}\pi\right)  }_{=F^{\deg\pi}+\pi\cdot
u\left(  \pi\right)  }a=\left(  F^{\deg\pi}+\pi\cdot u\left(  \pi\right)
\right)  a=F^{\deg\pi}a+\pi\cdot\underbrace{u\left(  \pi\right)  a}_{\in A}\in
F^{\deg\pi}a+\pi A.
\]
In other words, $\left(  \operatorname*{Carl}\pi\right)  a\equiv F^{\deg\pi
}a\operatorname{mod}\pi A$. Thus,%
\begin{equation}
\left(  \operatorname*{Carl}\pi\right)  a\equiv F^{\deg\pi}a\equiv
a\operatorname{mod}\pi A \label{pf.prop.F.gW.example1.a.1}%
\end{equation}
(by Lemma \ref{lem.F.gW.example1.lem} \textbf{(b)}, applied to $P=a$).

But $\varphi_{\pi}=\operatorname*{id}$ (by the definition of $\varphi_{\pi}$),
and thus $\varphi_{\pi}\left(  a\right)  =\operatorname*{id}\left(  a\right)
=a\equiv\left(  \operatorname*{Carl}\pi\right)  a\operatorname{mod}\pi A$ (by
(\ref{pf.prop.F.gW.example1.a.1})). This completes our proof of Assumption 2
of Theorem \ref{thm.F.gW}.

Thus, all three Assumptions 1, 2 and 3 of Theorem \ref{thm.F.gW} are
satisfied. This proves Proposition \ref{prop.F.gW.example1} \textbf{(a)}.

\textbf{(b)} Define a family $\left(  b_{P}\right)  _{P\in N}\in A^{N}$ by
$\left(  b_{P}\right)  _{P\in N}=\left(  F^{\deg P}Q\right)  _{P\in N}$. Thus,%
\begin{equation}
b_{P}=F^{\deg P}Q\ \ \ \ \ \ \ \ \ \ \text{for every }P\in N.
\label{pf.prop.F.gW.example1.b.bP=}%
\end{equation}
We now must prove that the assertions $\mathcal{C}_{1}$, $\mathcal{D}_{1}$,
$\mathcal{D}_{2}$, $\mathcal{E}_{1}$, $\mathcal{F}_{1}$, $\mathcal{G}_{1}$,
and $\mathcal{G}_{2}$ of Theorem \ref{thm.F.gW} are satisfied for this family.

We shall first show that Assertion $\mathcal{C}_{1}$ is satisfied:

\textit{Proof of Assertion }$\mathcal{C}_{1}$\textit{:} Let $P\in N$ and
$\pi\in\operatorname*{PF}P$. We must prove that $\varphi_{\pi}\left(
b_{P/\pi}\right)  \equiv b_{P}\operatorname{mod}\pi^{v_{\pi}\left(  P\right)
}A$.

We have $\pi\in\operatorname*{PF}P$, thus $P/\pi\in\mathbb{F}_{q}\left[
T\right]  $. The polynomial $P/\pi$ is monic (since $P$ and $\pi$ are monic),
and thus belongs to $\mathbb{F}_{q}\left[  T\right]  _{+}=N$. Hence, the
equality (\ref{pf.prop.F.gW.example1.b.bP=}) (applied to $P/\pi$ instead of
$P$) yields $b_{P/\pi}=F^{\deg\left(  P/\pi\right)  }Q$. But $\varphi_{\pi
}=\operatorname*{id}$ (by the definition of $\varphi_{\pi}$), and thus%
\begin{equation}
\varphi_{\pi}\left(  b_{P/\pi}\right)  =\operatorname*{id}\left(  b_{P/\pi
}\right)  =b_{P/\pi}=F^{\deg\left(  P/\pi\right)  }Q.
\label{pf.prop.F.gW.example1.b.1}%
\end{equation}


Lemma \ref{lem.F.gW.example1.lem} \textbf{(b)} (applied to $Q$ instead of $P$)
yields $F^{\deg\pi}Q\equiv Q\operatorname{mod}\pi A$. Thus, Corollary
\ref{cor.F.lift.lift-all} \textbf{(a)} (applied to $P/\pi$, $F^{\deg\pi}Q$ and
$Q$ instead of $N$, $a$ and $b$) yields%
\[
F^{\deg\left(  P/\pi\right)  }F^{\deg\pi}Q\equiv F^{\deg\left(  P/\pi\right)
}Q\operatorname{mod}\pi^{v_{\pi}\left(  P/\pi\right)  +1}A.
\]
Since%
\begin{align*}
F^{\deg\left(  P/\pi\right)  }F^{\deg\pi}  &  =F^{\deg\left(  P/\pi\right)
+\deg\pi}=F^{\deg P}\\
&  \ \ \ \ \ \ \ \ \ \ \left(  \text{since }\deg\left(  P/\pi\right)  +\deg
\pi=\deg\underbrace{\left(  \left(  P/\pi\right)  \pi\right)  }_{=P}=\deg
P\right)
\end{align*}
and
\[
\underbrace{v_{\pi}\left(  P/\pi\right)  }_{=v_{\pi}\left(  P\right)  -v_{\pi
}\left(  \pi\right)  }+1=v_{\pi}\left(  P\right)  -\underbrace{v_{\pi}\left(
\pi\right)  }_{=1}+1=v_{\pi}\left(  P\right)  -1+1=v_{\pi}\left(  P\right)  ,
\]
this rewrites as
\[
F^{\deg P}Q\equiv F^{\deg\left(  P/\pi\right)  }Q\operatorname{mod}\pi
^{v_{\pi}\left(  P\right)  }A.
\]
Now, (\ref{pf.prop.F.gW.example1.b.bP=}) becomes%
\[
b_{P}=F^{\deg P}Q\equiv F^{\deg\left(  P/\pi\right)  }Q=\varphi_{\pi}\left(
b_{P/\pi}\right)  \operatorname{mod}\pi^{v_{\pi}\left(  P\right)  }A
\]
(by (\ref{pf.prop.F.gW.example1.b.1})). In other words, $\varphi_{\pi}\left(
b_{P/\pi}\right)  \equiv b_{P}\operatorname{mod}\pi^{v_{\pi}\left(  P\right)
}A$. Thus, Assertion $\mathcal{C}_{1}$ is proven.

We now have shown that Assertion $\mathcal{C}_{1}$ is satisfied. Thus, all the
assertions $\mathcal{C}_{1}$, $\mathcal{D}_{1}$, $\mathcal{D}_{2}$,
$\mathcal{E}_{1}$, $\mathcal{F}_{1}$, $\mathcal{G}_{1}$, and $\mathcal{G}_{2}$
of Theorem \ref{thm.F.gW} are satisfied (since Theorem \ref{thm.F.gW} says
that these assertions are equivalent). This proves Proposition
\ref{prop.F.gW.example1} \textbf{(b)}.

\textbf{(c)} Define a family $\left(  b_{P}\right)  _{P\in N}\in A^{N}$ by
$\left(  b_{P}\right)  _{P\in N}=\left(  \left(  \operatorname*{Carl}P\right)
Q\right)  _{P\in N}$. Thus,%
\begin{equation}
b_{P}=\left(  \operatorname*{Carl}P\right)  Q\ \ \ \ \ \ \ \ \ \ \text{for
every }P\in N. \label{pf.prop.F.gW.example1.c.bP=}%
\end{equation}
We now must prove that the assertions $\mathcal{C}_{1}$, $\mathcal{D}_{1}$,
$\mathcal{D}_{2}$, $\mathcal{E}_{1}$, $\mathcal{F}_{1}$, $\mathcal{G}_{1}$,
and $\mathcal{G}_{2}$ of Theorem \ref{thm.F.gW} are satisfied for this family.

We shall first show that Assertion $\mathcal{C}_{1}$ is satisfied:

\textit{Proof of Assertion }$\mathcal{C}_{1}$\textit{:} Let $P\in N$ and
$\pi\in\operatorname*{PF}P$. We must prove that $\varphi_{\pi}\left(
b_{P/\pi}\right)  \equiv b_{P}\operatorname{mod}\pi^{v_{\pi}\left(  P\right)
}A$.

We have $\pi\in\operatorname*{PF}P$, thus $P/\pi\in\mathbb{F}_{q}\left[
T\right]  $. The polynomial $P/\pi$ is monic (since $P$ and $\pi$ are monic),
and thus belongs to $\mathbb{F}_{q}\left[  T\right]  _{+}=N$. Hence, the
equality (\ref{pf.prop.F.gW.example1.c.bP=}) (applied to $P/\pi$ instead of
$P$) yields $b_{P/\pi}=\left(  \operatorname*{Carl}\left(  P/\pi\right)
\right)  Q$. But $\varphi_{\pi}=\operatorname*{id}$ (by the definition of
$\varphi_{\pi}$), and thus%
\begin{equation}
\varphi_{\pi}\left(  b_{P/\pi}\right)  =\operatorname*{id}\left(  b_{P/\pi
}\right)  =b_{P/\pi}=\left(  \operatorname*{Carl}\left(  P/\pi\right)
\right)  Q. \label{pf.prop.F.gW.example1.c.1}%
\end{equation}


Lemma \ref{lem.F.gW.example1.lem} \textbf{(a)} (applied to $Q$ instead of $P$)
yields $\left(  \operatorname*{Carl}\pi\right)  Q\equiv Q\operatorname{mod}\pi
A$. Thus, Corollary \ref{cor.F.lift.lift-all} \textbf{(b)} (applied to $P/\pi
$, $\left(  \operatorname*{Carl}\pi\right)  Q$ and $Q$ instead of $N$, $a$ and
$b$) yields%
\[
\left(  \operatorname*{Carl}\left(  P/\pi\right)  \right)  \left(
\operatorname*{Carl}\pi\right)  Q\equiv\left(  \operatorname*{Carl}\left(
P/\pi\right)  \right)  Q\operatorname{mod}\pi^{v_{\pi}\left(  P/\pi\right)
+1}A.
\]
Since%
\begin{align*}
\left(  \operatorname*{Carl}\left(  P/\pi\right)  \right)  \left(
\operatorname*{Carl}\pi\right)   &  =\operatorname*{Carl}\left(
\underbrace{\left(  P/\pi\right)  \pi}_{=P}\right)
\ \ \ \ \ \ \ \ \ \ \left(
\begin{array}
[c]{c}%
\text{since }\operatorname*{Carl}\text{ is an }\mathbb{F}_{q}\text{-algebra}\\
\text{homomorphism}%
\end{array}
\right) \\
&  =\operatorname*{Carl}P
\end{align*}
and
\[
\underbrace{v_{\pi}\left(  P/\pi\right)  }_{=v_{\pi}\left(  P\right)  -v_{\pi
}\left(  \pi\right)  }+1=v_{\pi}\left(  P\right)  -\underbrace{v_{\pi}\left(
\pi\right)  }_{=1}+1=v_{\pi}\left(  P\right)  -1+1=v_{\pi}\left(  P\right)  ,
\]
this rewrites as
\[
\left(  \operatorname*{Carl}P\right)  Q\equiv\left(  \operatorname*{Carl}%
\left(  P/\pi\right)  \right)  Q\operatorname{mod}\pi^{v_{\pi}\left(
P\right)  }A.
\]
Now, (\ref{pf.prop.F.gW.example1.c.bP=}) becomes%
\[
b_{P}=\left(  \operatorname*{Carl}P\right)  Q\equiv\left(
\operatorname*{Carl}\left(  P/\pi\right)  \right)  Q=\varphi_{\pi}\left(
b_{P/\pi}\right)  \operatorname{mod}\pi^{v_{\pi}\left(  P\right)  }A
\]
(by (\ref{pf.prop.F.gW.example1.c.1})). In other words, $\varphi_{\pi}\left(
b_{P/\pi}\right)  \equiv b_{P}\operatorname{mod}\pi^{v_{\pi}\left(  P\right)
}A$. Thus, Assertion $\mathcal{C}_{1}$ is proven.

We now have shown that Assertion $\mathcal{C}_{1}$ is satisfied. Thus, all the
assertions $\mathcal{C}_{1}$, $\mathcal{D}_{1}$, $\mathcal{D}_{2}$,
$\mathcal{E}_{1}$, $\mathcal{F}_{1}$, $\mathcal{G}_{1}$, and $\mathcal{G}_{2}$
of Theorem \ref{thm.F.gW} are satisfied (since Theorem \ref{thm.F.gW} says
that these assertions are equivalent). This proves Proposition
\ref{prop.F.gW.example1} \textbf{(c)}.

\textbf{(d)} Define a family $\left(  b_{P}\right)  _{P\in N}\in A^{N}$ by
$\left(  b_{P}\right)  _{P\in N}=\left(  Q\right)  _{P\in N}$. Thus,%
\begin{equation}
b_{P}=Q\ \ \ \ \ \ \ \ \ \ \text{for every }P\in N.
\label{pf.prop.F.gW.example1.d.bP=}%
\end{equation}
We now must prove that the assertions $\mathcal{C}_{1}$, $\mathcal{D}_{1}$,
$\mathcal{D}_{2}$, $\mathcal{E}_{1}$, $\mathcal{F}_{1}$, $\mathcal{G}_{1}$,
and $\mathcal{G}_{2}$ of Theorem \ref{thm.F.gW} are satisfied for this family.

We shall first show that Assertion $\mathcal{C}_{1}$ is satisfied:

\textit{Proof of Assertion }$\mathcal{C}_{1}$\textit{:} Let $P\in N$ and
$\pi\in\operatorname*{PF}P$. We must prove that $\varphi_{\pi}\left(
b_{P/\pi}\right)  \equiv b_{P}\operatorname{mod}\pi^{v_{\pi}\left(  P\right)
}A$.

We have $\pi\in\operatorname*{PF}P$, thus $P/\pi\in\mathbb{F}_{q}\left[
T\right]  $. The polynomial $P/\pi$ is monic (since $P$ and $\pi$ are monic),
and thus belongs to $\mathbb{F}_{q}\left[  T\right]  _{+}=N$. Hence, the
equality (\ref{pf.prop.F.gW.example1.b.bP=}) (applied to $P/\pi$ instead of
$P$) yields $b_{P/\pi}=Q$. But $\varphi_{\pi}=\operatorname*{id}$ (by the
definition of $\varphi_{\pi}$), and thus%
\begin{equation}
\varphi_{\pi}\left(  b_{P/\pi}\right)  =\operatorname*{id}\left(  b_{P/\pi
}\right)  =b_{P/\pi}=Q. \label{pf.prop.F.gW.example1.d.1}%
\end{equation}


Now, (\ref{pf.prop.F.gW.example1.b.bP=}) becomes $b_{P}=Q=\varphi_{\pi}\left(
b_{P/\pi}\right)  $ (by (\ref{pf.prop.F.gW.example1.d.1})). Hence,%
\[
b_{P}\equiv\varphi_{\pi}\left(  b_{P/\pi}\right)  \operatorname{mod}%
\pi^{v_{\pi}\left(  P\right)  }A.
\]
In other words, $\varphi_{\pi}\left(  b_{P/\pi}\right)  \equiv b_{P}%
\operatorname{mod}\pi^{v_{\pi}\left(  P\right)  }A$. Thus, Assertion
$\mathcal{C}_{1}$ is proven.

We now have shown that Assertion $\mathcal{C}_{1}$ is satisfied. Thus, all the
assertions $\mathcal{C}_{1}$, $\mathcal{D}_{1}$, $\mathcal{D}_{2}$,
$\mathcal{E}_{1}$, $\mathcal{F}_{1}$, $\mathcal{G}_{1}$, and $\mathcal{G}_{2}$
of Theorem \ref{thm.F.gW} are satisfied (since Theorem \ref{thm.F.gW} says
that these assertions are equivalent). This proves Proposition
\ref{prop.F.gW.example1} \textbf{(d)}.
\end{proof}

Spelling out the claims of Theorem \ref{thm.F.gW} in basic terms provides a
plethora of congruences between polynomials in $\mathbb{F}_{q}\left[
T\right]  $. We will not list of all them, but only give one example,
conjectured by the math.stackexchange user \textquotedblleft
Levent\textquotedblright\ in \cite{levent}:

\begin{corollary}
\label{cor.F.gW.example1.levent}Let $Q\in\mathbb{F}_{q}\left[  T\right]  $.
Then,%
\[
P\mid\sum_{D\mid P}\varphi\left(  \dfrac{P}{D}\right)  Q^{q^{\deg D}%
}\ \ \ \ \ \ \ \ \ \ \text{for every }P\in\mathbb{F}_{q}\left[  T\right]
_{+}.
\]

\end{corollary}

\begin{proof}
[First proof of Corollary \ref{cor.F.gW.example1.levent}.]Define $N$, $A$ and
$\varphi_{P}$ (for all $P\in N$) as in Proposition \ref{prop.F.gW.example1}.
Define a family $\left(  b_{P}\right)  _{P\in N}\in A^{N}$ by $\left(
b_{P}\right)  _{P\in N}=\left(  F^{\deg P}Q\right)  _{P\in N}$. Then, every
$P\in N$ satisfies%
\begin{equation}
b_{P}=F^{\deg P}Q=F^{\deg P}\cdot Q=Q^{q^{\deg P}}
\label{pf.cor.F.gW.example1.levent.bP=1}%
\end{equation}
(by (\ref{eq.prop.F.acts-on-commalg.Fk.eq}), applied to $k=\deg P$ and $m=Q$).

Proposition \ref{prop.F.gW.example1} \textbf{(b)} shows that the assertions
$\mathcal{C}_{1}$, $\mathcal{D}_{1}$, $\mathcal{D}_{2}$, $\mathcal{E}_{1}$,
$\mathcal{F}_{1}$, $\mathcal{G}_{1}$, and $\mathcal{G}_{2}$ of Theorem
\ref{thm.F.gW} are satisfied for this family $\left(  b_{P}\right)  _{P\in
N}=\left(  F^{\deg P}Q\right)  _{P\in N}$. In particular, Assertion
$\mathcal{G}_{2}$ is satisfied. In other words, every $P\in N$ satisfies%
\begin{equation}
\sum_{D\mid P}\varphi\left(  D\right)  \varphi_{D}\left(  b_{P/D}\right)  \in
PA. \label{pf.cor.F.gW.example1.levent.2}%
\end{equation}


Now, let $P\in\mathbb{F}_{q}\left[  T\right]  _{+}$. Thus, $P\in\mathbb{F}%
_{q}\left[  T\right]  _{+}=N$ (since $N$ was defined to be $\mathbb{F}%
_{q}\left[  T\right]  _{+}$).

But the polynomial $P$ is monic. Hence, the map%
\begin{align*}
\left(  \text{the set of all monic divisors of }P\right)   &  \rightarrow
\left(  \text{the set of all monic divisors of }P\right)  ,\\
D  &  \mapsto P/D
\end{align*}
is well-defined and a bijection (actually, it is an involution). Thus, we can
substitute $P/D$ for $D$ in the sum $\sum_{D\mid P}\varphi\left(  D\right)
\varphi_{D}\left(  b_{P/D}\right)  $. We thus obtain%
\begin{align*}
&  \sum_{D\mid P}\varphi\left(  D\right)  \varphi_{D}\left(  b_{P/D}\right) \\
&  =\sum_{D\mid P}\varphi\left(  \underbrace{P/D}_{=\dfrac{P}{D}}\right)
\underbrace{\varphi_{P/D}}_{\substack{=\operatorname*{id}\\\text{(by the
definition of }\varphi_{P/D}\text{)}}}\left(  \underbrace{b_{P/\left(
P/D\right)  }}_{\substack{=b_{D}=Q^{q^{\deg D}}\\\text{(by
(\ref{pf.cor.F.gW.example1.levent.bP=1}), applied}\\\text{to }D\text{ instead
of }P}}\right) \\
&  =\sum_{D\mid P}\varphi\left(  \dfrac{P}{D}\right)
\underbrace{\operatorname*{id}\left(  Q^{q^{\deg D}}\right)  }_{=Q^{q^{\deg
D}}}=\sum_{D\mid P}\varphi\left(  \dfrac{P}{D}\right)  Q^{q^{\deg D}}.
\end{align*}
Hence,%
\[
\sum_{D\mid P}\varphi\left(  \dfrac{P}{D}\right)  Q^{q^{\deg D}}=\sum_{D\mid
P}\varphi\left(  D\right)  \varphi_{D}\left(  b_{P/D}\right)  \in PA
\]
(by (\ref{pf.cor.F.gW.example1.levent.2})). In other words, $P\mid\sum_{D\mid
P}\varphi\left(  \dfrac{P}{D}\right)  Q^{q^{\deg D}}$. This proves Corollary
\ref{cor.F.gW.example1.levent}.
\end{proof}

This said, it is not much harder to prove Corollary
\ref{cor.F.gW.example1.levent} without any reference to Theorem \ref{thm.F.gW}%
, using just the results of Subsection \ref{subsect.proofs.numthefuns}:

\begin{proof}
[Second proof of Corollary \ref{cor.F.gW.example1.levent}.]Let
$\operatorname*{Frob}$ denote the Frobenius endomorphism of the $\mathbb{F}%
_{q}$-algebra $\mathbb{F}_{q}\left[  T\right]  $. This is the map
$\mathbb{F}_{q}\left[  T\right]  \rightarrow\mathbb{F}_{q}\left[  T\right]  $
that sends each $P\in\mathbb{F}_{q}\left[  T\right]  $ to $P^{q}$. It is
well-known that $\operatorname*{Frob}$ is an $\mathbb{F}_{q}$-algebra
endomorphism of $\mathbb{F}_{q}\left[  T\right]  $.

We make a few auxiliary observations:

\begin{statement}
\textit{Observation 1:} Let $u\in\mathbb{N}$, $a\in\mathbb{F}_{q}\left[
T\right]  $ and $b\in\mathbb{F}_{q}\left[  T\right]  $. Then, $a^{q^{u}%
}-b^{q^{u}}=\left(  a-b\right)  ^{q^{u}}$.
\end{statement}

[\textit{Proof of Observation 1:} We have%
\begin{equation}
\operatorname*{Frob}\nolimits^{k}c=c^{q^{k}}\ \ \ \ \ \ \ \ \ \ \text{for
every }k\in\mathbb{N}\text{ and }c\in\mathbb{F}_{q}\left[  T\right]  .
\label{pf.cor.F.gW.example1.levent.2nd.ob1.pf.1}%
\end{equation}
(Indeed, this is easy to prove by induction over $k$, using the definition of
$\operatorname*{Frob}$.)

Now, recall that $\operatorname*{Frob}$ is an $\mathbb{F}_{q}$-algebra
endomorphism of $\mathbb{F}_{q}\left[  T\right]  $. Hence, so is its $u$-th
power $\operatorname*{Frob}\nolimits^{u}$. Thus,%
\[
\operatorname*{Frob}\nolimits^{u}\left(  a-b\right)
=\underbrace{\operatorname*{Frob}\nolimits^{u}a}_{\substack{=a^{q^{u}%
}\\\text{(by (\ref{pf.cor.F.gW.example1.levent.2nd.ob1.pf.1}), applied to
}c=a\text{)}}}-\underbrace{\operatorname*{Frob}\nolimits^{u}b}%
_{\substack{=b^{q^{u}}\\\text{(by
(\ref{pf.cor.F.gW.example1.levent.2nd.ob1.pf.1}), applied to }c=b\text{)}%
}}=a^{q^{u}}-b^{q^{u}}.
\]
Thus,%
\[
a^{q^{u}}-b^{q^{u}}=\operatorname*{Frob}\nolimits^{u}\left(  a-b\right)
=\left(  a-b\right)  ^{q^{u}}%
\]
(by (\ref{pf.cor.F.gW.example1.levent.2nd.ob1.pf.1}), applied to $c=a-b$).
This proves Observation 1.]

\begin{statement}
\textit{Observation 2:} Let $\pi$ be a monic irreducible polynomial in
$\mathbb{F}_{q}\left[  T\right]  $. Let $a$ and $b$ be two elements of
$\mathbb{F}_{q}\left[  T\right]  $ such that $a\equiv b\operatorname{mod}%
\pi\mathbb{F}_{q}\left[  T\right]  $. Let $N\in\mathbb{F}_{q}\left[  T\right]
$ be nonzero. Then, $a^{q^{\deg N}}\equiv b^{q^{\deg N}}\operatorname{mod}%
\pi^{v_{\pi}\left(  N\right)  +1}\mathbb{F}_{q}\left[  T\right]  $.
\end{statement}

[\textit{Proof of Observation 2:} We can regard Observation 2 as a particular
case of Corollary \ref{cor.F.lift.lift-all} \textbf{(a)} (applied to
$A=\mathbb{F}_{q}\left[  T\right]  $). But let us give a self-contained proof instead.

We have $a-b\in\pi\mathbb{F}_{q}\left[  T\right]  $ (since $a\equiv
b\operatorname{mod}\pi\mathbb{F}_{q}\left[  T\right]  $). In other words,
$a-b=\pi c$ for some $c\in\mathbb{F}_{q}\left[  T\right]  $. Consider this
$c$. Now, define $u\in\mathbb{N}$ by $u=\deg N$.

But every nonnegative integer $m$ satisfies $2^{m}\geq m+1$ (this is easy to
prove). Applying this to $m=u$, we find $2^{u}\geq u+1$. But $\pi^{v_{\pi
}\left(  N\right)  }\mid N$ and thus $\deg\left(  \pi^{v_{\pi}\left(
N\right)  }\right)  \leq\deg N=u$. Hence, $u\geq\deg\left(  \pi^{v_{\pi
}\left(  N\right)  }\right)  =v_{\pi}\left(  N\right)  \underbrace{\deg\pi
}_{\geq1}\geq v_{\pi}\left(  N\right)  $. But $q\geq2$ and thus $q^{u}%
\geq2^{u}\geq\underbrace{u}_{\geq v_{\pi}\left(  N\right)  }+1\geq v_{\pi
}\left(  N\right)  +1$.

But Observation 1 yields $a^{q^{u}}-b^{q^{u}}=\left(  \underbrace{a-b}_{=\pi
c}\right)  ^{q^{u}}=\left(  \pi c\right)  ^{q^{u}}=\pi^{q^{u}}c^{q^{u}}$.
Hence, $\pi^{q^{u}}\mid a^{q^{u}}-b^{q^{u}}$ in $\mathbb{F}_{q}\left[
T\right]  $. But $q^{u}\geq v_{\pi}\left(  N\right)  +1$, and thus
$\pi^{v_{\pi}\left(  N\right)  +1}\mid\pi^{q^{u}}\mid a^{q^{u}}-b^{q^{u}}$. In
other words, $a^{q^{u}}\equiv b^{q^{u}}\operatorname{mod}\pi^{v_{\pi}\left(
N\right)  +1}\mathbb{F}_{q}\left[  T\right]  $. Since $u=\deg N$, this
rewrites as $a^{q^{\deg N}}\equiv b^{q^{\deg N}}\operatorname{mod}\pi^{v_{\pi
}\left(  N\right)  +1}\mathbb{F}_{q}\left[  T\right]  $. Thus, Observation 2
is proven.]

Next, fix $P\in\mathbb{F}_{q}\left[  T\right]  _{+}$. Let $\mathbf{S}$ be the
set of all squarefree monic divisors of $P$.

\begin{statement}
\textit{Observation 3:} We have%
\[
\sum_{D\mid P}\varphi\left(  \dfrac{P}{D}\right)  Q^{q^{\deg D}}=\sum
_{D\in\mathbf{S}}\mu\left(  D\right)  Q^{q^{\deg\left(  P/D\right)  }}.
\]

\end{statement}

[\textit{Proof of Observation 3:} Let $\mathbf{D}$ be the set of all monic
divisors of $P$. Then, the map%
\[
\mathbf{D}\rightarrow\mathbf{D},\ \ \ \ \ \ \ \ \ \ D\mapsto P/D
\]
is well-defined (since $P$ itself is monic) and invertible (since it is its
own inverse). Thus, this map is a bijection. Hence, we can substitute $P/D$
for $D$ in the sum $\sum_{D\in\mathbf{D}}\varphi\left(  \dfrac{P}{D}\right)
Q^{q^{\deg D}}$. We thus obtain
\begin{align*}
&  \sum_{D\in\mathbf{D}}\varphi\left(  \dfrac{P}{D}\right)  Q^{q^{\deg D}}\\
&  =\sum_{D\in\mathbf{D}}\varphi\left(  \underbrace{\dfrac{P}{P/D}}%
_{=D}\right)  Q^{q^{\deg\left(  P/D\right)  }}=\sum_{D\in\mathbf{D}%
}\underbrace{\varphi\left(  D\right)  }_{\substack{=\mu\left(  D\right)
\text{ in }\mathbb{F}_{q}\\\text{(by Proposition \ref{prop.phi-Q.formula}
\textbf{(d)}}\\\text{(applied to }D\text{ instead of }M\text{))}}%
}Q^{q^{\deg\left(  P/D\right)  }}\\
&  =\underbrace{\sum_{D\in\mathbf{D}}}_{\substack{=\sum_{\substack{D\mid
P}}\\\text{(since }\mathbf{D}\text{ is the set of all}\\\text{monic divisors
of }P\text{)}}}\mu\left(  D\right)  Q^{q^{\deg\left(  P/D\right)  }}%
=\sum_{\substack{D\mid P}}\mu\left(  D\right)  Q^{q^{\deg\left(  P/D\right)
}}\\
&  =\underbrace{\sum_{\substack{D\mid P;\\D\text{ is squarefree}}%
}}_{\substack{=\sum_{D\in\mathbf{S}}\\\text{(since }\mathbf{S}\text{ is the
set}\\\text{of all squarefree}\\\text{monic divisors of }P\text{)}}}\mu\left(
D\right)  Q^{q^{\deg\left(  P/D\right)  }}+\sum_{\substack{D\mid P;\\D\text{
is not squarefree}}}\underbrace{\mu\left(  D\right)  }%
_{\substack{=0\\\text{(by the definition}\\\text{of }\mu\text{, since
}D\\\text{is not squarefree)}}}Q^{q^{\deg\left(  P/D\right)  }}\\
&  =\sum_{D\in\mathbf{S}}\mu\left(  D\right)  Q^{q^{\deg\left(  P/D\right)  }%
}+\underbrace{\sum_{\substack{D\mid P;\\D\text{ is not squarefree}%
}}0Q^{q^{\deg\left(  P/D\right)  }}}_{=0}=\sum_{D\in\mathbf{S}}\mu\left(
D\right)  Q^{q^{\deg\left(  P/D\right)  }}.
\end{align*}
Comparing this with%
\[
\underbrace{\sum_{D\in\mathbf{D}}}_{\substack{=\sum_{D\mid P}\\\text{(since
}\mathbf{D}\text{ is the set of all}\\\text{monic divisors of }P\text{)}%
}}\varphi\left(  \dfrac{P}{D}\right)  Q^{q^{\deg D}}=\sum_{D\mid P}%
\varphi\left(  \dfrac{P}{D}\right)  Q^{q^{\deg D}},
\]
this yields%
\[
\sum_{D\mid P}\varphi\left(  \dfrac{P}{D}\right)  Q^{q^{\deg D}}=\sum
_{D\in\mathbf{S}}\mu\left(  D\right)  Q^{q^{\deg\left(  P/D\right)  }}.
\]
This proves Observation 3.]

\begin{statement}
\textit{Observation 4:} Let $P\in\mathbb{F}_{q}\left[  T\right]  _{+}$. Let
$\pi\in\operatorname*{PF}P$. Let $D$ be a monic divisor of $P$ such that
$\pi\nmid D$. Then,
\[
Q^{q^{\deg\left(  P/D\right)  }}\equiv Q^{q^{\deg\left(  P/\left(  \pi
D\right)  \right)  }}\operatorname{mod}\pi^{v_{\pi}\left(  P\right)
}\mathbb{F}_{q}\left[  T\right]  .
\]

\end{statement}

[\textit{Proof of Observation 4:} Observe that $P/D\in\mathbb{F}_{q}\left[
T\right]  $ (since $D$ is a divisor of $P$). Also, $\pi\nmid D$ and thus
$v_{\pi}\left(  D\right)  =0$. But $\pi\in\operatorname*{PF}P$, so that
$\pi\mid P$ and thus $v_{\pi}\left(  P\right)  >0$. Now, $v_{\pi}\left(
P/D\right)  =v_{\pi}\left(  P\right)  -\underbrace{v_{\pi}\left(  D\right)
}_{=0}=v_{\pi}\left(  P\right)  >0$. In other words, $\pi\mid P/D$. Hence,
$\left(  P/D\right)  /\pi\in\mathbb{F}_{q}\left[  T\right]  $.

Set $d=\deg\pi$. Lemma \ref{lem.F.gW.example1.lem.dumbed-down} (applied to $Q$
instead of $P$) yields%
\[
Q^{q^{d}}\equiv Q\operatorname{mod}\pi\mathbb{F}_{q}\left[  T\right]  .
\]
Hence, Observation 2 (applied to $a=Q^{q^{d}}$, $b=Q$ and $N=\left(
P/D\right)  /\pi$) yields
\[
\left(  Q^{q^{d}}\right)  ^{q^{\deg\left(  \left(  P/D\right)  /\pi\right)  }%
}\equiv Q^{q^{\deg\left(  \left(  P/D\right)  /\pi\right)  }}%
\operatorname{mod}\pi^{v_{\pi}\left(  \left(  P/D\right)  /\pi\right)
+1}\mathbb{F}_{q}\left[  T\right]  .
\]
Since%
\begin{align*}
\left(  Q^{q^{d}}\right)  ^{q^{\deg\left(  \left(  P/D\right)  /\pi\right)
}}  &  =Q^{q^{d}q^{\deg\left(  \left(  P/D\right)  /\pi\right)  }%
}=Q^{q^{d+\deg\left(  \left(  P/D\right)  /\pi\right)  }}\\
&  \ \ \ \ \ \ \ \ \ \ \left(  \text{since }q^{d}q^{\deg\left(  \left(
P/D\right)  /\pi\right)  }=q^{d+\deg\left(  \left(  P/D\right)  /\pi\right)
}\right)
\end{align*}
and%
\[
\underbrace{v_{\pi}\left(  \left(  P/D\right)  /\pi\right)  }_{=v_{\pi}\left(
P/D\right)  -v_{\pi}\left(  \pi\right)  }+1=\underbrace{v_{\pi}\left(
P/D\right)  }_{=v_{\pi}\left(  P\right)  }-\underbrace{v_{\pi}\left(
\pi\right)  }_{=1}+1=v_{\pi}\left(  P\right)  -1+1=v_{\pi}\left(  P\right)  ,
\]
this rewrites as%
\[
Q^{q^{d+\deg\left(  \left(  P/D\right)  /\pi\right)  }}\equiv Q^{q^{\deg
\left(  \left(  P/D\right)  /\pi\right)  }}\operatorname{mod}\pi^{v_{\pi
}\left(  P\right)  }\mathbb{F}_{q}\left[  T\right]  .
\]
Since%
\begin{align*}
\underbrace{d}_{=\deg\pi}+\deg\left(  \left(  P/D\right)  /\pi\right)   &
=\deg\pi+\deg\left(  \left(  P/D\right)  /\pi\right) \\
&  =\deg\left(  \underbrace{\pi\cdot\left(  \left(  P/D\right)  /\pi\right)
}_{=P/D}\right)  =\deg\left(  P/D\right)
\end{align*}
and%
\[
\deg\left(  \underbrace{\left(  P/D\right)  /\pi}_{=P/\left(  \pi D\right)
}\right)  =\deg\left(  P/\left(  \pi D\right)  \right)  ,
\]
this rewrites as
\[
Q^{q^{\deg\left(  P/D\right)  }}\equiv Q^{q^{\deg\left(  P/\left(  \pi
D\right)  \right)  }}\operatorname{mod}\pi^{v_{\pi}\left(  P\right)
}\mathbb{F}_{q}\left[  T\right]  .
\]
This proves Observation 4.]

Recall that $\mathbf{S}$ is the set of all squarefree monic divisors of $P$.
Each of these squarefree monic divisors has the form $\prod_{\eta\in I}\eta$
for some subset $I$ of $\operatorname*{PF}P$. More precisely, the map%
\begin{align}
\left\{  I\subseteq\operatorname*{PF}P\right\}   &  \rightarrow\mathbf{S}%
,\nonumber\\
I  &  \mapsto\prod_{\eta\in I}\eta\label{pf.cor.F.gW.example1.levent.2nd.bij}%
\end{align}
is a bijection. Moreover, every subset $I$ of $\operatorname*{PF}P$ satisfies
\begin{align}
\mu\left(  \prod_{\eta\in I}\eta\right)   &  =\left(  -1\right)  ^{\left\vert
\operatorname*{PF}\left(  \prod_{\eta\in I}\eta\right)  \right\vert
}\ \ \ \ \ \ \ \ \ \ \left(  \text{since }\prod_{\eta\in I}\eta\text{ is
squarefree}\right) \nonumber\\
&  =\left(  -1\right)  ^{\left\vert I\right\vert }\ \ \ \ \ \ \ \ \ \ \left(
\text{since }\operatorname*{PF}\left(  \prod_{\eta\in I}\eta\right)
=I\right)  . \label{pf.cor.F.gW.example1.levent.2nd.2}%
\end{align}

Now, we claim the following:

\begin{statement}
\textit{Observation 5:} Let $\pi\in\operatorname*{PF}P$. Let $I\subseteq
\operatorname*{PF}P$ be such that $\pi\notin I$. Then,
\[
Q^{q^{\deg\left(  P/\prod_{\eta\in I}\eta\right)  }}\equiv Q^{q^{\deg\left(
P/\left(  \prod_{\eta\in I\cup\left\{  \pi\right\}  }\eta\right)  \right)  }%
}\operatorname{mod}\pi^{v_{\pi}\left(  P\right)  }\mathbb{F}_{q}\left[
T\right]  .
\]

\end{statement}

[\textit{Proof of Observation 5:} From $\pi\notin I$, we obtain%
\begin{equation}
\prod_{\eta\in I\cup\left\{  \pi\right\}  }\eta=\pi\prod_{\eta\in I}\eta.
\label{pf.cor.F.gW.example1.levent.2nd.ob5.pf.1}%
\end{equation}


We have $I\subseteq\operatorname*{PF}P$. Thus, the elements of $I$ are monic
irreducible divisors of $P$. In particular, the elements of $I$ are monic
irreducible polynomials in $\mathbb{F}_{q}\left[  T\right]  $. These monic
irreducible polynomials are all distinct from $\pi$ (since $\pi\notin I$), and
therefore coprime to $\pi$ (since $\pi$ is irreducible). Hence, the elements
of $I$ are polynomials coprime to $\pi$. Therefore, $\prod_{\eta\in I}\eta$ is
a product of polynomials coprime to $\pi$. Thus, $\prod_{\eta\in I}\eta$
itself is coprime to $\pi$. Consequently, $\pi\nmid\prod_{\eta\in I}\eta$.

But $\prod_{\eta\in I}\eta\in\mathbf{S}$ (since $\prod_{\eta\in I}\eta$ is the
image of $I$ under the bijection (\ref{pf.cor.F.gW.example1.levent.2nd.bij})).
In other words, $\prod_{\eta\in I}\eta$ is a squarefree monic divisor of $P$.
Hence, Observation 4 (applied to $D=\prod_{\eta\in I}\eta$) yields
\[
Q^{q^{\deg\left(  P/\prod_{\eta\in I}\eta\right)  }}\equiv Q^{q^{\deg\left(
P/\left(  \pi\prod_{\eta\in I}\eta\right)  \right)  }}\operatorname{mod}%
\pi^{v_{\pi}\left(  P\right)  }\mathbb{F}_{q}\left[  T\right]  .
\]
In view of (\ref{pf.cor.F.gW.example1.levent.2nd.ob5.pf.1}), this rewrites as%
\[
Q^{q^{\deg\left(  P/\prod_{\eta\in I}\eta\right)  }}\equiv Q^{q^{\deg\left(
P/\left(  \prod_{\eta\in I\cup\left\{  \pi\right\}  }\eta\right)  \right)  }%
}\operatorname{mod}\pi^{v_{\pi}\left(  P\right)  }\mathbb{F}_{q}\left[
T\right]  .
\]
This proves Observation 5.]

\begin{statement}
\textit{Observation 6:} Let $\pi\in\operatorname*{PF}P$. Then,
\[
\sum_{D\in\mathbf{S}}\mu\left(  D\right)  Q^{q^{\deg\left(  P/D\right)  }%
}\equiv0\operatorname{mod}\pi^{v_{\pi}\left(  P\right)  }\mathbb{F}_{q}\left[
T\right]  .
\]

\end{statement}

[\textit{Proof of Observation 6:} Recall that
(\ref{pf.cor.F.gW.example1.levent.2nd.bij}) is a bijection. Thus, we can
substitute $\prod_{\eta\in I}\eta$ for $D$ in the sum $\sum_{D\in\mathbf{S}%
}\mu\left(  D\right)  Q^{q^{\deg\left(  P/D\right)  }}$. Thus, we obtain%
\begin{align}
&  \sum_{D\in\mathbf{S}}\mu\left(  D\right)  Q^{q^{\deg\left(  P/D\right)  }%
}\nonumber\\
&  =\sum_{I\subseteq\operatorname*{PF}P}\underbrace{\mu\left(  \prod_{\eta\in
I}\eta\right)  }_{\substack{=\left(  -1\right)  ^{\left\vert I\right\vert
}\\\text{(by (\ref{pf.cor.F.gW.example1.levent.2nd.2}))}}}Q^{q^{\deg\left(
P/\prod_{\eta\in I}\eta\right)  }}=\sum_{I\subseteq\operatorname*{PF}P}\left(
-1\right)  ^{\left\vert I\right\vert }Q^{q^{\deg\left(  P/\prod_{\eta\in
I}\eta\right)  }}\nonumber\\
&  =\sum_{\substack{I\subseteq\operatorname*{PF}P;\\\pi\in I}}\left(
-1\right)  ^{\left\vert I\right\vert }Q^{q^{\deg\left(  P/\prod_{\eta\in
I}\eta\right)  }}+\sum_{\substack{I\subseteq\operatorname*{PF}P;\\\pi\notin
I}}\left(  -1\right)  ^{\left\vert I\right\vert }Q^{q^{\deg\left(
P/\prod_{\eta\in I}\eta\right)  }}
\label{pf.cor.F.gW.example1.levent.2nd.obs6.pf.1}%
\end{align}
(since every $I\subseteq\operatorname*{PF}P$ satisfies either $\pi\in I$ or
$\pi\notin I$ (but not both)).

But we have $\pi\in\operatorname*{PF}P$. Hence, the map%
\begin{align*}
\left\{  I\subseteq\operatorname*{PF}P\ \mid\ \pi\notin I\right\}   &
\rightarrow\left\{  I\subseteq\operatorname*{PF}P\ \mid\ \pi\in I\right\}  ,\\
J  &  \mapsto J\cup\left\{  \pi\right\}
\end{align*}
is well-defined and a bijection\footnote{This is a particular case (obtained
by setting $G=\operatorname*{PF}P$ and $g=\pi$) of the following fact:
\par
Let $G$ be a set. Let $g\in G$. Then, the map%
\begin{align*}
\left\{  I\subseteq G\ \mid\ g\notin I\right\}   &  \rightarrow\left\{
I\subseteq G\ \mid\ g\in I\right\}  ,\\
J  &  \mapsto J\cup\left\{  g\right\}
\end{align*}
is well-defined and a bijection. (Its inverse is the map
\begin{align*}
\left\{  I\subseteq G\ \mid\ g\in I\right\}   &  \rightarrow\left\{
I\subseteq G\ \mid\ g\notin I\right\}  ,\\
J  &  \mapsto J\setminus\left\{  g\right\}  .
\end{align*}
This is all straightforward to check.)}. Hence, we can substitute
$J\cup\left\{  \pi\right\}  $ for $I$ in the sum $\sum_{\substack{I\subseteq
\operatorname*{PF}P;\\\pi\in I}}\left(  -1\right)  ^{\left\vert I\right\vert
}Q^{q^{\deg\left(  P/\prod_{\eta\in I}\eta\right)  }}$. We thus obtain%
\begin{align}
&  \sum_{\substack{I\subseteq\operatorname*{PF}P;\\\pi\in I}}\left(
-1\right)  ^{\left\vert I\right\vert }Q^{q^{\deg\left(  P/\prod_{\eta\in
I}\eta\right)  }}\nonumber\\
&  =\sum_{\substack{J\subseteq\operatorname*{PF}P;\\\pi\notin J}%
}\underbrace{\left(  -1\right)  ^{\left\vert J\cup\left\{  \pi\right\}
\right\vert }}_{\substack{=-\left(  -1\right)  ^{\left\vert J\right\vert
}\\\text{(since }\left\vert J\cup\left\{  \pi\right\}  \right\vert =\left\vert
J\right\vert +1\\\text{(since }\pi\notin J\text{))}}}Q^{q^{\deg\left(
P/\prod_{\eta\in J\cup\left\{  \pi\right\}  }\eta\right)  }}=-\sum
_{\substack{J\subseteq\operatorname*{PF}P;\\\pi\notin J}}\left(  -1\right)
^{\left\vert J\right\vert }Q^{q^{\deg\left(  P/\prod_{\eta\in J\cup\left\{
\pi\right\}  }\eta\right)  }}\nonumber\\
&  =-\sum_{\substack{I\subseteq\operatorname*{PF}P;\\\pi\notin I}}\left(
-1\right)  ^{\left\vert I\right\vert }Q^{q^{\deg\left(  P/\prod_{\eta\in
I\cup\left\{  \pi\right\}  }\eta\right)  }}
\label{pf.cor.F.gW.example1.levent.2nd.obs6.pf.3}%
\end{align}
(here, we have renamed the summation index $J$ as $I$).

Now, (\ref{pf.cor.F.gW.example1.levent.2nd.obs6.pf.1}) becomes%
\begin{align*}
&  \sum_{D\in\mathbf{S}}\mu\left(  D\right)  Q^{q^{\deg\left(  P/D\right)  }%
}\\
&  =\underbrace{\sum_{\substack{I\subseteq\operatorname*{PF}P;\\\pi\in
I}}\left(  -1\right)  ^{\left\vert I\right\vert }Q^{q^{\deg\left(
P/\prod_{\eta\in I}\eta\right)  }}}_{\substack{=-\sum_{\substack{I\subseteq
\operatorname*{PF}P;\\\pi\notin I}}\left(  -1\right)  ^{\left\vert
I\right\vert }Q^{q^{\deg\left(  P/\prod_{\eta\in I\cup\left\{  \pi\right\}
}\eta\right)  }}\\\text{(by (\ref{pf.cor.F.gW.example1.levent.2nd.obs6.pf.3}%
))}}}+\sum_{\substack{I\subseteq\operatorname*{PF}P;\\\pi\notin I}}\left(
-1\right)  ^{\left\vert I\right\vert }\underbrace{Q^{q^{\deg\left(
P/\prod_{\eta\in I}\eta\right)  }}}_{\substack{\equiv Q^{q^{\deg\left(
P/\left(  \prod_{\eta\in I\cup\left\{  \pi\right\}  }\eta\right)  \right)  }%
}\operatorname{mod}\pi^{v_{\pi}\left(  P\right)  }\mathbb{F}_{q}\left[
T\right]  \\\text{(by Observation 5)}}}\\
&  \equiv-\sum_{\substack{I\subseteq\operatorname*{PF}P;\\\pi\notin I}}\left(
-1\right)  ^{\left\vert I\right\vert }Q^{q^{\deg\left(  P/\prod_{\eta\in
I\cup\left\{  \pi\right\}  }\eta\right)  }}+\sum_{\substack{I\subseteq
\operatorname*{PF}P;\\\pi\notin I}}\left(  -1\right)  ^{\left\vert
I\right\vert }Q^{q^{\deg\left(  P/\left(  \prod_{\eta\in I\cup\left\{
\pi\right\}  }\eta\right)  \right)  }}\\
&  =0\operatorname{mod}\pi^{v_{\pi}\left(  P\right)  }\mathbb{F}_{q}\left[
T\right]  .
\end{align*}
Thus, Observation 6 is proven.]

Recall that $P$ is a monic polynomial. Hence, $\prod_{\pi\in\operatorname*{PF}%
P}\pi^{v_{\pi}\left(  P\right)  }$ is the factorization of $P$ into monic
irreducible factors. Thus, $\prod_{\pi\in\operatorname*{PF}P}\pi^{v_{\pi
}\left(  P\right)  }=P$.

But the polynomials $\pi^{v_{\pi}\left(  P\right)  }$ for distinct $\pi
\in\operatorname*{PF}P$ are mutually coprime. Hence, their least common
multiple is their product. In other words, the least common multiple of the
polynomials $\pi^{v_{\pi}\left(  P\right)  }$ (where $\pi$ ranges over
$\operatorname*{PF}P$) is $\prod_{\pi\in\operatorname*{PF}P}\pi^{v_{\pi
}\left(  P\right)  }=P$.

Now, define a polynomial $Z\in\mathbb{F}_{q}\left[  T\right]  $ by%
\[
Z=\sum_{D\mid P}\varphi\left(  \dfrac{P}{D}\right)  Q^{q^{\deg D}}.
\]
Then, for every $\pi\in\operatorname*{PF}P$, we have%
\begin{align*}
Z &  =\sum_{D\mid P}\varphi\left(  \dfrac{P}{D}\right)  Q^{q^{\deg D}}%
=\sum_{D\in\mathbf{S}}\mu\left(  D\right)  Q^{q^{\deg\left(  P/D\right)  }%
}\ \ \ \ \ \ \ \ \ \ \left(  \text{by Observation 3}\right)  \\
&  \equiv0\operatorname{mod}\pi^{v_{\pi}\left(  P\right)  }\mathbb{F}%
_{q}\left[  T\right]  \ \ \ \ \ \ \ \ \ \ \left(  \text{by Observation
6}\right)  ,
\end{align*}
and thus $\pi^{v_{\pi}\left(  P\right)  }\mid Z$. Therefore, the least common
multiple of the polynomials $\pi^{v_{\pi}\left(  P\right)  }$ (where $\pi$
ranges over $\operatorname*{PF}P$) divides $Z$. In other words, $P$ divides
$Z$ (since the least common multiple of the polynomials $\pi^{v_{\pi}\left(
P\right)  }$ (where $\pi$ ranges over $\operatorname*{PF}P$) is $P$). Thus,%
\[
P\mid Z=\sum_{D\mid P}\varphi\left(  \dfrac{P}{D}\right)  Q^{q^{\deg D}}.
\]
This proves Corollary \ref{cor.F.gW.example1.levent} again.
\end{proof}

\subsection{(More sections to be added here!)}

[...]

XTODO: Conclude torsionfreeness in two ways.

XTODO: polynomial ring example.

[...]

\section{\label{sect.tinfoil}Speculations}

\subsection{So what is $\Lambda_{\operatorname*{Carl}}$ ?}

So what is the Carlitz analogue of the ring of symmetric functions?

I'm still groping in the dark here. But at least I'm seeing some hints of why
this isn't as simple as in the classical case (although I guess the theory of
symmetric functions can only be called ``simple'' with the wisdom of hindsight
anyway). After Subsection \ref{subsect.F} it appears to me that the
multiplication isn't crucial to the functor $W_{N}$, but rather an extra
structure that gets carried along (whatever this means).\footnote{What about
Lie algebras? What properties should a Lie algebra structure on an
$\mathcal{F}$-module $A$ satisfy so that $W_{N}\left(  A\right)  $ also is a
Lie algebra? Will $W_{N}\left(  A\right)  $ then also share these properties?}
This suggests that I shouldn't be looking at the representing object of the
functor $W_{N}:\mathbf{CRing}_{\mathbb{F}_{q}\left[  T\right]  }%
\rightarrow\mathbf{CRing}_{\mathbb{F}_{q}\left[  T\right]  }$, but at the
representing object of the functor $W_{N}:\left.  _{\mathcal{F}}%
\mathbf{Mod}\right.  \rightarrow\left.  _{\mathcal{F}}\mathbf{Mod}\right.  $,
or at least that the latter is more fundamental than the former. To begin
with, it's smaller.

A representing object of a functor $\left.  _{\mathcal{F}}\mathbf{Mod}\right.
\rightarrow\left.  _{\mathcal{F}}\mathbf{Mod}\right.  $ is the same as an
$\mathcal{F}$-$\mathcal{F}$-bimodule\footnote{This is a particular case of the
following general fact: If $A$ and $B$ are two algebras, then any $A$%
-$B$-bimodule $M$ gives rise to a representable functor $\operatorname*{Hom}%
\nolimits_{_{A}\mathbf{Mod}}\left(  _{A}M,-\right)  :\left.  _{A}%
\mathbf{Mod}\right.  \rightarrow\left.  _{B}\mathbf{Mod}\right.  $.}. The
$\mathcal{F}$-$\mathcal{F}$-bimodule which represents the functor
$W_{N}:\left.  _{\mathcal{F}}\mathbf{Mod}\right.  \rightarrow\left.
_{\mathcal{F}}\mathbf{Mod}\right.  $ is the free left $\mathcal{F}$-module
$\Lambda_{\mathcal{F}}$ with basis $\left(  x_{P}\right)  _{P\in N}$, and with
right $\mathcal{F}$-module structure defined as follows: Let $p_{P}%
=\sum\limits_{D\mid P}D\left[  \dfrac{P}{D}\right]  \left(  x_{D}\right)  $
for every $P\in N$. (The intuition is that $x_{P}$ are analogues of the
\textquotedblleft Witt vector coordinates\textquotedblright\ of $\Lambda
$\ \ \ \ \footnote{These are the symmetric functions $w_{n}$ in \cite[Exercise
2.80]{reiner-hopf}. Their name stems from their relation to the Witt vectors;
from a combinatorial viewpoint, they are a rather exotic family.} and $p_{P}$
are \textquotedblleft power sum symmetric functions\textquotedblright.) Then,
set $p_{P}f=fp_{P}$ for every $P\in N$ and $f\in\mathcal{F}$. This uniquely
determines a right $\mathcal{F}$-module structure (since it has to commute
with the left one), although its existence is not really obvious. Thus
$\Lambda_{\mathcal{F}}$ is defined.

When $N$ is the whole set $\mathbb{F}_{q}\left[  T\right]  _{+}$, the
$\mathcal{F}$-$\mathcal{F}$-bimodule $\Lambda_{\mathcal{F}}$ has some claims
to be the Carlitz analogue of the ring of symmetric functions, although it is
an $\mathcal{F}$-$\mathcal{F}$-bimodule rather than a ring. Nevertheless, I
don't feel able to realize it as an actual set of symmetric power series. The
Carlitz structure is way too additive for that. In some sense, what made the
power sums algebraically independent over the integers was the fact that
$\left(  x+y\right)  ^{2}\neq x^{2}+y^{2}$ etc.; but in the Carlitz case,
$\left[  P\right]  $ is additive and even $\mathbb{F}_{q}$-linear for every
$P\in\mathbb{F}_{q}\left[  T\right]  $, so that if we would define the
``$P$-th power sum polynomial'' in some variables $\xi_{i}$ to mean
$\sum\limits_{i}\left[  P\right]  \left(  \xi_{i}\right)  $, then all these
polynomials would be linearly dependent over $\mathcal{F}$ simply because
$\sum\limits_{i}\left[  P\right]  \left(  \xi_{i}\right)  =\left[  P\right]
\left(  \sum\limits_{i}\xi_{i}\right)  =\left(  \operatorname*{Carl}\left(
P\right)  \right)  \left(  \sum\limits_{i}\xi_{i}\right)  $.

The absence of multiplicative structure makes it hard to even guess what
``elementary symmetric functions'' or ``complete homogeneous symmetric
functions'' would be in the Carlitz situation. But Carlitz exponential and
Carlitz logarithm are well-defined on every left $\mathcal{F}$-module on which
$\mathbb{F}_{q}\left[  T\right]  $ acts invertibly (i. e., whose
$\mathbb{F}_{q}\left[  T\right]  $-module structure extends to an
$\mathbb{F}_{q}\left(  T\right)  $-module structure) and which has appropriate
closure properties. We might try to use them to construct the ``elementary
symmetric functions'' by some analogue of the classical $\sum\limits_{n\in
\mathbb{N}}\left(  -1\right)  ^{n}e_{n}T^{n}=\exp\left(  -\sum\limits_{n\geq
1}\dfrac{1}{n}p_{n}T^{n}\right)  $ formula from the theory of symmetric
functions.\footnote{Another suggestion by James Borger.} The problem is that
this is an identity in power series, and we would first have to find out what
the right analogue of power series is in this context.

There is other stuff to do as well. One can look for explicit formulas for the
right $\mathcal{F}$-action on the $x_{P}$ in $\Lambda_{\mathcal{F}}$. And one
can try to define the analogue of plethysm (which, as far as I understand,
should be an $\mathcal{F}$-$\mathcal{F}$-bilinear map from $\Lambda
_{\mathcal{F}}\otimes_{\mathcal{F}}\Lambda_{\mathcal{F}}$ to $\Lambda
_{\mathcal{F}}$ making $\Lambda_{\mathcal{F}}$ into what would be an
$\mathcal{F}$-algebra if it were commutative?).

\subsection{Some computations in $\Lambda_{\mathcal{F}}$}

Let me see if I'm able to get something concrete out of the above reveries.
How about computing the right $\mathcal{F}$-action on concrete basis elements
of $\Lambda_{\mathcal{F}}$ ?

Assume that $N$ is the whole $\mathbb{F}_{q}\left[  T\right]  _{+}$.

By definition, $p_{1}=x_{1}$, so that \fbox{$x_{1}f=fx_{1}$ for every
$f\in\mathcal{F}$} (since $p_{1}f=fp_{1}$ for every $f\in\mathcal{F}$). That
is, $x_{1}$ is central with respect to the two $\mathcal{F}$-actions. Nothing
to see here.

By definition, $p_{T}=\underbrace{\left[  T\right]  \left(  x_{1}\right)
}_{=\left(  F+T\right)  x_{1}}+Tx_{T}=\left(  F+T\right)  x_{1}+Tx_{T}$. Now,
$p_{T}f=fp_{T}$ for every $f\in\mathcal{F}$. Apply this to $f=T$ and
substitute $p_{T}=\left(  F+T\right)  x_{1}+Tx_{T}$; you obtain%
\[
\left(  \left(  F+T\right)  x_{1}+Tx_{T}\right)  T=T\left(  \left(
F+T\right)  x_{1}+Tx_{T}\right)  .
\]
Since%
\begin{align*}
\left(  \left(  F+T\right)  x_{1}+Tx_{T}\right)  T  &  =\left(  F+T\right)
\underbrace{x_{1}T}_{\substack{=Tx_{1}\\\text{(since }x_{1}\text{ is
central)}}}+Tx_{T}T=\underbrace{\left(  F+T\right)  Tx_{1}}_{=T\left(
T^{q-1}F+T\right)  x_{1}}+Tx_{T}T\\
&  =T\left(  \left(  T^{q-1}F+T\right)  x_{1}+x_{T}T\right)  ,
\end{align*}
this rewrites as $T\left(  \left(  T^{q-1}F+T\right)  x_{1}+x_{T}T\right)
=T\left(  \left(  F+T\right)  x_{1}+Tx_{T}\right)  $. Since $T$ is a left
non-zero-divisor in $\mathcal{F}$ and thus also in $\Lambda_{\mathcal{F}}$ (as
$\Lambda_{\mathcal{F}}$ is a free left $\mathcal{F}$-module), we can cancel
the $T$ out of this, and obtain $\left(  T^{q-1}F+T\right)  x_{1}%
+x_{T}T=\left(  F+T\right)  x_{1}+Tx_{T}$. Hence, $x_{T}T=\left(  F+T\right)
x_{1}+Tx_{T}-\left(  T^{q-1}F+T\right)  x_{1}$. This simplifies to
\newline\fbox{$x_{T}T=Tx_{T}-\left(  T^{q-1}-1\right)  Fx_{1}$}.

Let's do $x_{T}F$. Apply $p_{T}f=fp_{T}$ to $f=F$, and substitute
$p_{T}=\left(  F+T\right)  x_{1}+Tx_{T}$ again; the result is%
\[
\left(  \left(  F+T\right)  x_{1}+Tx_{T}\right)  F=F\left(  \left(
F+T\right)  x_{1}+Tx_{T}\right)  .
\]
Subtraction of $\left(  F+T\right)  x_{1}F$ turns this into
\begin{align*}
Tx_{T}F  &  =F\left(  \left(  F+T\right)  x_{1}+Tx_{T}\right)  -\left(
F+T\right)  x_{1}F\\
&  =FFx_{1}+\underbrace{FT}_{=T^{q}F}x_{1}+\underbrace{FT}_{=T^{q}F}%
x_{T}-F\underbrace{x_{1}T}_{\substack{=Fx_{1}\\\text{(since }x_{1}\text{ is
central)}}}-Tx_{1}F\\
&  =FFx_{1}+T^{q}Fx_{1}+T^{q}Fx_{T}-FFx_{1}-Tx_{1}F=T^{q}Fx_{1}+T^{q}%
Fx_{T}-Tx_{1}F\\
&  =T\left(  T^{q-1}Fx_{1}+T^{q-1}Fx_{T}-x_{1}F\right)  .
\end{align*}
Cancelling $T$, we obtain%
\[
x_{T}F=T^{q-1}Fx_{1}+T^{q-1}Fx_{T}-\underbrace{x_{1}F}_{\substack{=Fx_{1}%
\\\text{(since }x_{1}\text{ is central)}}}T^{q-1}Fx_{1}+T^{q-1}Fx_{T}-Fx_{1}.
\]
This simplifies to \fbox{$x_{T}F=\left(  T^{q-1}-1\right)  Fx_{1}%
+T^{q-1}Fx_{T}$}.

Let's be more bold and try a general irreducible polynomial, just to see how
far we can simplify. Let $\pi\in\mathbb{F}_{q}\left[  T\right]  _{+}$ be
irreducible. What is $x_{\pi}T$ ? As usual, $p_{\pi}=\left(
\operatorname*{Carl}\pi\right)  x_{1}+\pi x_{\pi}$ satisfies $p_{\pi}%
f=fp_{\pi}$ for every $f\in\mathcal{F}$. Applying this to $f=T$ and
substituting $p_{\pi}=\left(  \operatorname*{Carl}\pi\right)  x_{1}+\pi
x_{\pi}$, we get%
\[
\left(  \left(  \operatorname*{Carl}\pi\right)  x_{1}+\pi x_{\pi}\right)
T=T\left(  \left(  \operatorname*{Carl}\pi\right)  x_{1}+\pi x_{\pi}\right)
.
\]
Subtracting $\left(  \operatorname*{Carl}\pi\right)  x_{1}T$ from here, we get%
\begin{align*}
\pi x_{\pi}T  &  =T\left(  \left(  \operatorname*{Carl}\pi\right)  x_{1}+\pi
x_{\pi}\right)  -\left(  \operatorname*{Carl}\pi\right)  x_{1}T\\
&  =T\left(  \operatorname*{Carl}\pi\right)  x_{1}+T\pi x_{\pi}-\left(
\operatorname*{Carl}\pi\right)  \underbrace{x_{1}T}_{\substack{=Tx_{1}%
\\\text{(since }x_{1}\text{ is central)}}}\\
&  =T\left(  \operatorname*{Carl}\pi\right)  x_{1}+T\pi x_{\pi}-\left(
\operatorname*{Carl}\pi\right)  Tx_{1}\\
&  =T\pi x_{\pi}+\left[  T,\operatorname*{Carl}\pi\right]  x_{1}.
\end{align*}
Thus, $\left[  T,\operatorname*{Carl}\pi\right]  $ must lie in $\pi
\mathcal{F}$, and an explicit formula for the quotient would be very useful.
Well, the fact that $\left[  T,\operatorname*{Carl}\pi\right]  $ lies in
$\pi\mathcal{F}$ is easily derived from (\ref{carl.pi}), but there seems to be
no way to write the quotient in finite terms. Let us rather introduce a
notation for it: Let $\eth_{T}\left(  \pi\right)  $ denote the (unique)
$f\in\mathcal{F}$ satisfying $\left[  T,\operatorname*{Carl}\pi\right]  =\pi
f$ (for $\pi$ irreducible monic). In more elementary (and commutative) terms,
$\eth_{T}\left(  \pi\right)  =\dfrac{T\left[  \pi\right]  \left(  X\right)
-\left[  \pi\right]  \left(  TX\right)  }{\pi}$. Now,%
\[
\pi x_{\pi}T=\underbrace{T\pi}_{=\pi T}x_{\pi}+\underbrace{\left[
T,\operatorname*{Carl}\pi\right]  }_{=\pi\eth_{T}\left(  \pi\right)  }%
x_{1}=\pi Tx_{\pi}+\pi\eth_{T}\left(  \pi\right)  x_{1}.
\]
Cancelling $\pi$, we obtain \fbox{$x_{\pi}T=Tx_{\pi}+\eth_{T}\left(
\pi\right)  x_{1}$}.

The question is: Do we get $x_{\pi}F$ explicitly using $\eth_{T}\left(
\pi\right)  $, or will we have to introduce another new operator? Apply
$p_{\pi}f=fp_{\pi}$ to $f=F$ and substitute $p_{\pi}=\left(
\operatorname*{Carl}\pi\right)  x_{1}+\pi x_{\pi}$. The result is%
\[
\left(  \left(  \operatorname*{Carl}\pi\right)  x_{1}+\pi x_{\pi}\right)
F=F\left(  \left(  \operatorname*{Carl}\pi\right)  x_{1}+\pi x_{\pi}\right)
.
\]
Subtracting $\left(  \operatorname*{Carl}\pi\right)  x_{1}F$ from here, we get%
\begin{align*}
\pi x_{\pi}F  &  =F\left(  \left(  \operatorname*{Carl}\pi\right)  x_{1}+\pi
x_{\pi}\right)  -\left(  \operatorname*{Carl}\pi\right)  x_{1}F\\
&  =F\left(  \operatorname*{Carl}\pi\right)  x_{1}+F\pi x_{\pi}-\left(
\operatorname*{Carl}\pi\right)  \underbrace{x_{1}F}_{\substack{=Fx_{1}%
\\\text{(since }x_{1}\text{ is central)}}}\\
&  =F\left(  \operatorname*{Carl}\pi\right)  x_{1}+F\pi x_{\pi}-\left(
\operatorname*{Carl}\pi\right)  Fx_{1}\\
&  =F\pi x_{\pi}+\left[  F,\operatorname*{Carl}\pi\right]  x_{1}.
\end{align*}
Oh, but $\left[  F,\operatorname*{Carl}\pi\right]  +\left[
T,\operatorname*{Carl}\pi\right]  =\left[  \underbrace{F+T}%
_{=\operatorname*{Carl}T},\operatorname*{Carl}\pi\right]  =\left[
\operatorname*{Carl}T,\operatorname*{Carl}\pi\right]  =\operatorname*{Carl}%
\underbrace{\left[  T,\pi\right]  }_{=0}=0$, so that $\left[
F,\operatorname*{Carl}\pi\right]  =-\underbrace{\left[  T,\operatorname*{Carl}%
\pi\right]  }_{=\pi\eth_{T}\left(  \pi\right)  }=-\pi\eth_{T}\left(
\pi\right)  $. Hence,%
\[
\pi x_{\pi}F=F\pi x_{\pi}+\underbrace{\left[  F,\operatorname*{Carl}%
\pi\right]  }_{=-\pi\eth_{T}\left(  \pi\right)  }x_{1}=\underbrace{F\pi}%
_{=\pi^{q}F}x_{\pi}-\pi\eth_{T}\left(  \pi\right)  x_{1}=\pi^{q}Fx_{\pi}%
-\pi\eth_{T}\left(  \pi\right)  x_{1}.
\]
Cancelling $\pi$, we obtain \fbox{$x_{\pi}F=\pi^{q-1}Fx_{\pi}-\eth_{T}\left(
\pi\right)  x_{1}$}.

\section{\label{sect.log}The logarithm series}

Here is my result on the logarithm series, which so far has not found any application.

\begin{theorem}
\label{thm.carlitzlog}Let $q$ be a prime power. Consider the Carlitz logarithm
$\log_{C}\in\mathbb{F}_{q}\left(  T\right)  \left[  \left[  X\right]  \right]
$ defined in \cite[Section 7]{kc-carlitz} (but with $q$ instead of $p$). Then,
in the power series ring $\mathbb{F}_{q}\left(  T\right)  \left[  \left[
X,S\right]  \right]  $, we have%
\begin{equation}
\log_{C}\left(  SX\right)  =\sum\limits_{N\in\mathbb{F}_{q}\left[  T\right]
_{+}}\left(  -1\right)  ^{\deg N}S^{q^{\deg N}}\dfrac{\left[  N\right]
\left(  X\right)  }{N}. \label{thm.carlitzlog.1}%
\end{equation}
(The right hand side of this converges in the usual topology on $\mathbb{F}%
_{q}\left[  \left[  X,S\right]  \right]  $.)
\end{theorem}

Let us recall the definition of $\log_{C}$ for the sake of completeness: For
every $j\in\mathbb{N}$, let $L_{j}$ be the polynomial $\left(  T^{q^{j}%
}-T\right)  \left(  T^{q^{j-1}}-T\right)  ...\left(  T^{q^{1}}-T\right)
\in\mathbb{F}_{q}\left[  T\right]  $. Then, $\log_{C}\in\mathbb{F}_{q}\left(
T\right)  \left[  \left[  X\right]  \right]  $ is defined by%
\begin{equation}
\log_{C}\left(  X\right)  =\sum\limits_{j\in\mathbb{N}}\left(  -1\right)
^{j}\dfrac{X^{q^{j}}}{L_{j}}. \label{carlitzlog}%
\end{equation}


It should be noticed that it is possible to specialize $S$ to $1$ in
(\ref{thm.carlitzlog.1}), but then the right hand side will only be convergent
in a rather weak sense (it will only converge if all terms with $N$ having a
given degree are first added up, and then the sums are being summed over the
degree rather than the single terms).

In contrast to the preceding results, Theorem \ref{thm.carlitzlog} seems to be
neither straightforward nor provable by translating some classical argument.
So let me sketch a proof (which is rather roundabout and hopefully
simplifiable). First, I need an auxiliary result which itself seems rather interesting:

\begin{proposition}
\label{prop.carlitzlog.lem}Let $q$ be a prime power. Let $A$ be a commutative
$\mathbb{F}_{q}$-algebra. Let $n\in\mathbb{N}$. Let $P\in A\left[  X\right]  $
be a polynomial such that $\deg P<q^{n}-1$. Let $e_{1}$, $e_{2}$, $...$,
$e_{n}$ be $n$ elements of $A$. Then,%
\[
\sum\limits_{\left(  \lambda_{1},\lambda_{2},...,\lambda_{n}\right)
\in\mathbb{F}_{q}^{n}}P\left(  \lambda_{1}e_{1}+\lambda_{2}e_{2}%
+...+\lambda_{n}e_{n}\right)  =0.
\]

\end{proposition}

\textit{Proof of Proposition \ref{prop.carlitzlog.lem} (sketch).} We can WLOG
assume that $P=X^{k}$ for some $k\in\left\{  0,1,...,q^{n}-2\right\}  $.
Assume this and consider this $k$. Since $k<q^{n}-1$, we can write $k$ in the
form $k=k_{n-1}q^{n-1}+k_{n-2}q^{n-2}+...+k_{0}q^{0}$ with $k_{i}<q$ and with
$k_{0}+k_{1}+...+k_{n-1}\leq n\left(  q-1\right)  -1$. Thus,%
\[
P=X^{k}=X^{k_{n-1}q^{n-1}+k_{n-2}q^{n-2}+...+k_{0}q^{0}}=\prod\limits_{i=0}%
^{n-1}X^{k_{i}q^{i}}=\prod\limits_{i=0}^{n-1}\left(  X^{q^{i}}\right)
^{k_{i}}.
\]
Hence,%
\begin{align*}
&  \sum\limits_{\left(  \lambda_{1},\lambda_{2},...,\lambda_{n}\right)
\in\mathbb{F}_{q}^{n}}P\left(  \lambda_{1}e_{1}+\lambda_{2}e_{2}%
+...+\lambda_{n}e_{n}\right) \\
&  =\sum\limits_{\left(  \lambda_{1},\lambda_{2},...,\lambda_{n}\right)
\in\mathbb{F}_{q}^{n}}\prod\limits_{i=0}^{n-1}\left(  \underbrace{\left(
\lambda_{1}e_{1}+\lambda_{2}e_{2}+...+\lambda_{n}e_{n}\right)  ^{q^{i}}%
}_{\substack{=\lambda_{1}e_{1}^{q^{i}}+\lambda_{2}e_{2}^{q^{i}}+...+\lambda
_{n}e_{n}^{q^{i}}\\\text{(since we are over }\mathbb{F}_{q}\text{)}}}\right)
^{k_{i}}\\
&  =\sum\limits_{\left(  \lambda_{1},\lambda_{2},...,\lambda_{n}\right)
\in\mathbb{F}_{q}^{n}}\prod\limits_{i=0}^{n-1}\left(  \lambda_{1}e_{1}^{q^{i}%
}+\lambda_{2}e_{2}^{q^{i}}+...+\lambda_{n}e_{n}^{q^{i}}\right)  ^{k_{i}}.
\end{align*}
Now, consider the product $\prod\limits_{i=0}^{n-1}\left(  \lambda_{1}%
e_{1}^{q^{i}}+\lambda_{2}e_{2}^{q^{i}}+...+\lambda_{n}e_{n}^{q^{i}}\right)
^{k_{i}}$ \textbf{as a polynomial (over }$A$\textbf{) in the variables
}$\lambda_{1}$, $\lambda_{2}$, $...$, $\lambda_{n}$. Then, it is a polynomial
of degree $k_{0}+k_{1}+...+k_{n-1}\leq n\left(  q-1\right)  -1$. It is
well-known (e. g., from the proof of the Chevalley-Warning theorem) that any
such polynomial yields $0$ when summed over all $\left(  \lambda_{1}%
,\lambda_{2},...,\lambda_{n}\right)  \in\mathbb{F}_{q}^{n}$ (because each of
its monomials has at least one exponent $<q-1$, and then summing the variable
which has this exponent over $\mathbb{F}_{q}$ already gives $0$ with all other
variables remaining fixed). This proves Proposition \ref{prop.carlitzlog.lem}.

Another auxiliary result:

\begin{proposition}
\label{prop.carlitzlog.lem2}Let $q$ be a prime power. Let $L$ be a field
extension of $\mathbb{F}_{q}$. Let $V$ be a finite $\mathbb{F}_{q}$-vector
subspace of $L$. Let $t\in L\setminus V$. Then,%
\[
\sum\limits_{v\in V}\dfrac{1}{t+v}=\left(  \prod\limits_{v\in V}\dfrac{1}%
{t+v}\right)  \cdot\left(  \prod\limits_{v\in V\setminus0}v\right)  .
\]

\end{proposition}

\textit{Proof of Proposition \ref{prop.carlitzlog.lem2} (sketched).} Let $W$
be the polynomial $\prod\limits_{v\in V}\left(  X+v\right)  \in L\left[
X\right]  $. This polynomial is a $q$-polynomial (indeed, Theorem
\ref{thm.mac1.subspace} (applied to $L=A$) shows that $f_{V}$ is a
$q$-polynomial, but clearly $f_{V}=W$); hence, its derivative equals its
coefficient in front of $X^{1}$ (because the derivative of any $q$-polynomial
in characteristic $p\mid q$ equals its coefficient in front of $X^{1}$). But
this coefficient is $\prod\limits_{v\in V\setminus0}v$. Thus, we know that the
derivative of $W$ equals $\prod\limits_{v\in V\setminus0}v$. Hence,
$W^{\prime}\left(  t\right)  =\prod\limits_{v\in V\setminus0}v$.

On the other hand, since $W=\prod\limits_{v\in V}\left(  X+v\right)  $, the
Leibniz formula yields%
\begin{align*}
W^{\prime}  &  =\sum\limits_{w\in V}\underbrace{\left(  X+w\right)  ^{\prime}%
}_{=1}\cdot\prod\limits_{\substack{v\in V;\\v\neq w}}\left(  X+v\right)
=\sum\limits_{w\in V}\prod\limits_{\substack{v\in V;\\v\neq w}}\left(
X+v\right)  =\sum\limits_{w\in V}\dfrac{\prod\limits_{v\in V}\left(
X+v\right)  }{X+w}\\
&  =\left(  \prod\limits_{v\in V}\left(  X+v\right)  \right)  \cdot\left(
\sum\limits_{w\in V}\dfrac{1}{X+w}\right)  .
\end{align*}
Applying this to $X=t$, we obtain%
\[
W^{\prime}\left(  t\right)  =\left(  \prod\limits_{v\in V}\left(  t+v\right)
\right)  \cdot\left(  \sum\limits_{w\in V}\dfrac{1}{t+w}\right)  ,
\]
so that%
\begin{align*}
\sum\limits_{w\in V}\dfrac{1}{t+w}  &  =\dfrac{1}{\prod\limits_{v\in V}\left(
t+v\right)  }\cdot\underbrace{W^{\prime}\left(  t\right)  }_{=\prod
\limits_{v\in V\setminus0}v}=\dfrac{1}{\prod\limits_{v\in V}\left(
t+v\right)  }\cdot\left(  \prod\limits_{v\in V\setminus0}v\right) \\
&  =\left(  \prod\limits_{v\in V}\dfrac{1}{t+v}\right)  \cdot\left(
\prod\limits_{v\in V\setminus0}v\right)  .
\end{align*}
Rename the index $w$ as $v$ and obtain the claim of Proposition
\ref{prop.carlitzlog.lem2}.

\textit{Proof of Theorem \ref{thm.carlitzlog} (sketched).} By
(\ref{carlitzlog}), we have%
\[
\log_{C}\left(  SX\right)  =\sum\limits_{j\in\mathbb{N}}\left(  -1\right)
^{j}\dfrac{\left(  SX\right)  ^{q^{j}}}{L_{j}}=\sum\limits_{j\in\mathbb{N}%
}\left(  -1\right)  ^{j}S^{q^{j}}\dfrac{X^{q^{j}}}{L_{j}}.
\]
Hence, it is clearly enough to show that every $m\in\mathbb{N}$ satisfies%
\begin{equation}
\dfrac{X^{q^{m}}}{L_{m}}=\sum\limits_{\substack{N\in\mathbb{F}_{q}\left[
T\right]  _{+};\\\deg N=m}}\dfrac{\left[  N\right]  \left(  X\right)  }{N}.
\label{pf.carlitzlog.1}%
\end{equation}


So let $m\in\mathbb{N}$. Introduce the polynomials $E_{j}\left(  Y\right)
\in\mathbb{F}_{q}\left(  T\right)  \left[  Y\right]  $ for all $j\in
\mathbb{N}$ as in \cite[Section 7]{kc-carlitz}, but with $q$ instead of $p$.
Let's spell out their definition: With $e_{C}$ denoting the Carlitz
exponential, the power series $e_{C}\left(  Y\log_{C}X\right)  \in
\mathbb{F}_{q}\left(  T\right)  \left[  \left[  X,Y\right]  \right]  $ is a
$q$-power series, i. e., its coefficient before $X^{\alpha}Y^{\beta}$ can only
be nonzero if both $\alpha$ and $\beta$ are powers of $q$. Now, for every
$j\in\mathbb{N}$, define $E_{j}\left(  Y\right)  $ to be the coefficient of
this power series $e_{C}\left(  Y\log_{C}X\right)  $, \textbf{regarded as a
power series in }$X$ \textbf{over }$\mathbb{F}_{q}\left(  T\right)  \left[
Y\right]  $, before $X^{q^{j}}$. Of course, this $E_{j}\left(  Y\right)  $ is
a $q$-polynomial in $\mathbb{F}_{q}\left(  T\right)  \left[  Y\right]  $.
Moreover, $\deg\left(  E_{j}\right)  =q^{j}$ and $E_{j}\left(  0\right)  =0$
for all $j\in\mathbb{N}$. Furthermore, $E_{j}\left(  M\right)  =0$ for every
$M\in\mathbb{F}_{q}\left[  T\right]  $ satisfying $\deg M<j$. Finally,
$E_{j}\left(  M\right)  =1$ for every $M\in\mathbb{F}_{q}\left[  T\right]  $
satisfying $\deg M=j$. But most importantly, $\left[  M\right]  \left(
X\right)  =\sum\limits_{j\in\mathbb{N}}E_{j}\left(  M\right)  X^{q^{j}}$ in
$\mathbb{F}_{q}\left(  T\right)  \left[  X\right]  $ for every $M\in
\mathbb{F}_{q}\left[  T\right]  $. Hence, for every nonzero $M\in
\mathbb{F}_{q}\left(  T\right)  \left[  X\right]  $, we have%
\begin{align}
\dfrac{\left[  M\right]  \left(  X\right)  }{M}  &  =\dfrac{\sum
\limits_{j\in\mathbb{N}}E_{j}\left(  M\right)  X^{q^{j}}}{M}=\sum
\limits_{j\in\mathbb{N}}\dfrac{E_{j}\left(  M\right)  }{M}X^{q^{j}}%
=\sum\limits_{j=0}^{\deg M}\dfrac{E_{j}\left(  M\right)  }{M}X^{q^{j}%
}\nonumber\\
&  \ \ \ \ \ \ \ \ \ \ \left(  \text{since }E_{j}\left(  M\right)  =0\text{
whenever }\deg M<j\right) \nonumber\\
&  =\sum\limits_{j=0}^{\deg M-1}\dfrac{E_{j}\left(  M\right)  }{M}X^{q^{j}%
}+\underbrace{\dfrac{E_{\deg M}\left(  M\right)  }{M}}_{\substack{=\dfrac
{1}{M}\\\text{(since }E_{j}\left(  M\right)  =1\text{ whenever }\deg
M=j\text{)}}}X^{q^{\deg M}}\nonumber\\
&  =\sum\limits_{j=0}^{\deg M-1}\dfrac{E_{j}\left(  M\right)  }{M}X^{q^{j}%
}+\dfrac{1}{M}X^{q^{\deg M}} \label{pf.carlitzlog.lem2.5}%
\end{align}
But since $E_{j}\left(  0\right)  =0$ for all $j\in\mathbb{N}$, we know that
for every $j\in\mathbb{N}$, the polynomial $E_{j}\left(  Y\right)  $ is
divisible by $Y$. Thus, $\dfrac{E_{j}\left(  Y\right)  }{Y}$ is a polynomial
of degree $q^{j}-1$ for every $j\in\mathbb{N}$ (since $\deg\left(
E_{j}\right)  =q^{j}$). Renaming $Y$ as $X$, we see that $\dfrac{E_{j}\left(
X\right)  }{X}$ is a polynomial of degree $q^{j}-1$ for every $j\in\mathbb{N}%
$. Hence, $\dfrac{E_{j}\left(  X+T^{m}\right)  }{X+T^{m}}\in\mathbb{F}%
_{q}\left(  T\right)  \left[  X\right]  $ also is a polynomial of degree
$q^{j}-1$ for every $j\in\mathbb{N}$. Hence, for every $j\in\left\{
0,1,...,m-1\right\}  $, we can apply Proposition \ref{prop.carlitzlog.lem} to
$A=\mathbb{F}_{q}\left(  T\right)  $, $n=m$, $P=\dfrac{E_{j}\left(
X+T^{m}\right)  }{X+T^{m}}$ and $e_{i}=T^{i-1}$, and conclude that
\[
\sum\limits_{\left(  \lambda_{1},\lambda_{2},...,\lambda_{m}\right)
\in\mathbb{F}_{q}}\dfrac{E_{j}\left(  \lambda_{1}T^{0}+\lambda_{2}%
T^{1}+...+\lambda_{m}T^{m-1}+T^{m}\right)  }{\lambda_{1}T^{0}+\lambda_{2}%
T^{1}+...+\lambda_{m}T^{m-1}+T^{m}}=0
\]
(since $j<m$ and thus $q^{j}-1<q^{m}-1$). Since the sums of the form
$\lambda_{1}T^{0}+\lambda_{2}T^{1}+...+\lambda_{m}T^{m-1}+T^{m}$ with $\left(
\lambda_{1},\lambda_{2},...,\lambda_{m}\right)  \in\mathbb{F}_{q}$ are
precisely the monic polynomials in $\mathbb{F}_{q}\left[  T\right]  $ with
degree $m$ (each appearing exactly once), this rewrites as%
\begin{equation}
\sum\limits_{\substack{N\in\mathbb{F}_{q}\left[  T\right]  _{+};\\\deg
N=m}}\dfrac{E_{j}\left(  N\right)  }{N}=0\ \ \ \ \ \ \ \ \ \ \text{for every
}j\in\left\{  0,1,...,m-1\right\}  \text{.} \label{pf.carlitzlog.lem2.8}%
\end{equation}


Now,%
\begin{align*}
&  \sum\limits_{\substack{N\in\mathbb{F}_{q}\left[  T\right]  _{+};\\\deg
N=m}}\dfrac{\left[  N\right]  \left(  X\right)  }{N}\\
&  =\sum\limits_{\substack{N\in\mathbb{F}_{q}\left[  T\right]  _{+};\\\deg
N=m}}\left(  \sum\limits_{j=0}^{\deg N-1}\dfrac{E_{j}\left(  N\right)  }%
{N}X^{q^{j}}+\dfrac{1}{N}X^{q^{\deg N}}\right) \\
&  \ \ \ \ \ \ \ \ \ \ \left(  \text{here we applied
(\ref{pf.carlitzlog.lem2.5}) to }M=N\right) \\
&  =\sum\limits_{j=0}^{m-1}\underbrace{\sum\limits_{\substack{N\in
\mathbb{F}_{q}\left[  T\right]  _{+};\\\deg N=m}}\dfrac{E_{j}\left(  N\right)
}{N}}_{\substack{=0\\\text{(by (\ref{pf.carlitzlog.lem2.8}))}}}X^{q^{j}}%
+\sum\limits_{\substack{N\in\mathbb{F}_{q}\left[  T\right]  _{+};\\\deg
N=m}}\dfrac{1}{N}X^{q^{m}}\\
&  =\sum\limits_{\substack{N\in\mathbb{F}_{q}\left[  T\right]  _{+};\\\deg
N=m}}\dfrac{1}{N}X^{q^{m}}=\sum\limits_{\substack{v\in\mathbb{F}_{q}\left[
T\right]  ;\\\deg v<m}}\dfrac{1}{T^{m}+v}X^{q^{m}}\\
&  \ \ \ \ \ \ \ \ \ \ \left(
\begin{array}
[c]{c}%
\text{since the monic polynomials in }\mathbb{F}_{q}\left[  T\right]  \text{
of degree }m\text{ are exactly}\\
\text{the sums of the form }T^{m}+v\text{ with }v\text{ being a polynomial
in}\\
\mathbb{F}_{q}\left[  T\right]  \text{ of degree }<m
\end{array}
\right) \\
&  =\left(  \prod\limits_{\substack{v\in\mathbb{F}_{q}\left[  T\right]
;\\\deg v<m}}\dfrac{1}{T^{m}+v}\right)  \cdot\left(  \prod
\limits_{\substack{v\in\mathbb{F}_{q}\left[  T\right]  ;\\\deg v<m;\\v\neq
0}}v\right)  X^{q^{m}}\\
&  \ \ \ \ \ \ \ \ \ \ \left(
\begin{array}
[c]{c}%
\text{by Proposition \ref{prop.carlitzlog.lem2}, applied to }L=\mathbb{F}%
_{q}\left(  T\right)  \text{, }t=T^{m}\\
\text{and }V=\left\{  v\in\mathbb{F}_{q}\left[  T\right]  \ \mid\ \deg
v<m\right\}
\end{array}
\right) \\
&  =\underbrace{\left(  \prod\limits_{\substack{N\in\mathbb{F}_{q}\left[
T\right]  _{+};\\\deg N=m}}\dfrac{1}{N}\right)  \cdot\left(  \prod
\limits_{\substack{v\in\mathbb{F}_{q}\left[  T\right]  ;\\\deg v<m;\\v\neq
0}}v\right)  }_{\substack{=\dfrac{1}{L_{m}}\\\text{(this is relatively
straightforward to prove}\\\text{using standard results on finite fields)}%
}}X^{q^{m}}=\dfrac{X^{q^{m}}}{L_{m}}.
\end{align*}
This proves (\ref{pf.carlitzlog.1}) and thus Theorem \ref{thm.carlitzlog}.

I hope there is a better proof.

\begin{thebibliography}{99}                                                                                               %


\bibitem {jb-bg1}James Borger, \textit{The basic geometry of Witt vectors, I:
The affine case}, Algebra \& Number Theory 5 (2011), no. 2, pp 231--285. Also
available as preprint arXiv:0801.1691v6.\newline\url{http://arxiv.org/abs/0801.1691v6}

\bibitem {bw-pa}James Borger, Ben Wieland, \textit{Plethystic algebra},
arXiv:math/0407227v1.\newline\url{http://arxiv.org/abs/math/0407227v1}

\bibitem {kc-carlitz}Keith Conrad, \textit{Carlitz extensions}.\newline\url{http://www.math.uconn.edu/~kconrad/blurbs/gradnumthy/carlitz.pdf}

\bibitem {dhns}Tom Denton, Florent Hivert, Anne Schilling, Nicolas M.
Thi\'{e}ry, \textit{On the representation theory of finite }$\mathcal{J}%
$\textit{-trivial monoids}, arXiv:1010.3455v3. \url{http://arxiv.org/abs/1010.3455v3}

\bibitem {dyckerhoff}Tobias Dyckerhoff, \textit{Hall Algebras - Bonn,
Wintersemester 14/15}, lecture notes, February 5, 2015.\newline\url{http://www.math.uni-bonn.de/people/dyckerho/notes.pdf}

\bibitem {reiner-hopf}Darij Grinberg, Victor Reiner, \textit{Hopf Algebras in
Combinatorics}, arXiv:1409.8356v4, August 22, 2016.\newline\url{http://www.math.umn.edu/~reiner/Classes/HopfComb.pdf}

\bibitem {dg-witt5}Darij Grinberg, \textit{Witt\#5: Around the integrality
criterion 9.93}, sidenote to Michiel Hazewinkel's \textquotedblleft Witt
vectors. Part 1\textquotedblright.\newline\url{http://mit.edu/~darij/www/algebra/witt5.pdf}

\bibitem {dg-witt5c}Darij Grinberg, \textit{Witt\#5c: The Chinese Remainder
Theorem for Modules}, sidenote to Michiel Hazewinkel's \textquotedblleft Witt
vectors. Part 1\textquotedblright.\newline\url{http://mit.edu/~darij/www/algebra/witt5c.pdf}

\bibitem {dg-witt5f}Darij Grinberg, \textit{Witt\#5f: Ghost-Witt integrality
for binomial rings}, sidenote to Michiel Hazewinkel's \textquotedblleft Witt
vectors. Part 1\textquotedblright.\newline\url{http://mit.edu/~darij/www/algebra/witt5f.pdf}

\bibitem {hw-witt1}Michiel Hazewinkel, \textit{Witt vectors. Part 1},
arXiv:0804.3888.\newline\url{http://arxiv.org/abs/0804.3888v1}

\bibitem {hesselholt-drw}Lars Hesselholt, \textit{The big de Rham-Witt
complex}, Acta Math. 214 (2015), pp. 135--207.\newline A preprint is also
available as arXiv:1006.3125v3: \url{http://arxiv.org/abs/1006.3125v3}

\bibitem {hesselholt-witt}Lars Hesselholt, \textit{Lecture notes on Witt
vectors}, MIT, Cambridge, Massachusetts, USA, 2005.\newline\url{http://www.math.nagoya-u.ac.jp/~larsh/papers/s03/wittsurvey.pdf}

\bibitem {levent}Levent, \textit{math.stackexchange post \#1824797
(\textquotedblleft Show that }$\sum_{d\mid f}\varphi\left(  \dfrac{f}%
{d}\right)  a^{\left\vert d\right\vert }\equiv0\operatorname{mod}%
f$\textit{\textquotedblright)}. \url{http://math.stackexchange.com/q/1824797}

\bibitem {jacobson-rl}Nathan Jacobson, \textit{Restricted Lie algebras of
characteristic }$p$, Trans. Amer. Math. Soc. \textbf{50} (1941), pp.
15--25.\newline\url{http://www.ams.org/journals/tran/1941-050-01/S0002-9947-1941-0005118-0/home.html}

\bibitem {mac-schurvar}I. G. Macdonald, \textit{Schur functions: Theme and
variations}, S\'{e}minaire Lotharingien de Combinatoire 28, B28a
(1992).\newline\url{http://www.emis.de/journals/SLC/opapers/s28macdonald.html}

\bibitem {ore-pp1}Oystein Ore, \textit{On a Special Class of Polynomials},
Trans. Amer. Math. Soc. \textbf{35} (1933), pp. 559--584.\newline\url{http://www.ams.org/journals/tran/1933-035-03/S0002-9947-1933-1501703-0/}

\bibitem {ore-pp2}Oystein Ore, \textit{Errata in my paper: \textquotedblleft
On a special class of polynomials\textquotedblright\ [Trans. Amer. Math. Soc.
35 (1933), no. 3, 559-584; 1501703]}, Trans. Amer. Math. Soc. \textbf{36}
(1934), p. 275.\newline\url{http://www.ams.org/journals/tran/1934-036-02/S0002-9947-1934-1501741-9/}

\bibitem {rabinoff-witt}Joseph Rabinoff, \textit{The Theory of Witt
Vectors}.\newline\url{http://www.math.harvard.edu/~rabinoff/misc/witt.pdf}

\bibitem {stanley-ec2}Richard Stanley, \textit{Enumerative Combinatorics,
volume 2}, Cambridge University Press 2001.
\end{thebibliography}


\end{document}